%StartInfo%%%%%%%%%%%%%%%%%%%%%%%%%%%%%%%%%%%%%%%%%%%%%%%%%%%%%%%%%%%%%%%%%%%%
%  Autor:
%  Titel:
%  File:
%  Ref:
%  Mod:
%EndInfo%%%%%%%%%%%%%%%%%%%%%%%%%%%%%%%%%%%%%%%%%%%%%%%%%%%%%%%%%%%%%%%%%%%%%%
%\poemtitle{Fröhlich soll mein Herze springen}
\begin{multicols}{2}
\settowidth{\versewidth}{Meine Schuld kann mich nicht drücken,}
\begin{verse}[\versewidth]
 
\flagverse{1.} Fröhlich soll mein Herze springen\\
dieser Zeit,\\
da vor Freud\\
alle Engel singen.\\
HöRt, hört, wie mit vollen Choren\\
alle Luft\\
laute ruft:\\
Christus Ist geboren.
 
\flagverse{2.} Heute geht aus seiner Kammer\\
gottes Held,\\
der die Welt\\
reißt aus allem Jammer.\\
Gott Wird Mensch, dir Mensch zugute;\\
Gottes Kind,\\
das verbind't\\
sich mit unserm Blute.
 
\flagverse{3.} Sollt uns Gott nun können hassen,\\
der uns gibt,\\
was er liebt\\
über alle Maßen?\\
Gott Gibt, unserm Leid zu wehren,\\
seinen Sohn\\
aus dem Thron\\
seiner Macht und Ehren.
 
\flagverse{4.} Sollte von uns sein gekehret,\\
der sein Reich\\
und zugleich\\
sich selbst uns verehret?\\
Sollt Uns Gottes Sohn nicht lieben\\
der jetzt kömmt,\\
von uns nimmt,\\
was uns will betrüben?
 
\flagverse{5.} Hätte für der Menschen Orden\\
unser Heil\\
einen Greul,\\
wär er nicht Mensch worden;\\
hätt er Lust zu unserm Schaden,\\
ei, so würd\\
unsre Bürd\\
er nicht auf sich laden.
 
\flagverse{6.} Er nimmt auf sich, was auf Erden\\
wir getan,\\
gibt sich an,\\
unser Lamm zu werden,\\
unser Lamm, das für uns stirbet\\
und bei Gott\\
für den Tod\\
Gnad und Fried erwirbet.

\begin{verbatim}










\end{verbatim}

\flagverse{7.} Nun er liegt in seiner Krippen,\\
ruft zu sich\\
mich und dich,\\
spricht mit süßen Lippen:\\
Lasset Fahrn, o lieben Brüder,\\
was euch quält,\\
was euch fehlt;\\
ich bring alles wieder.
 
\flagverse{8.} Ei, so kommt und laßt uns laufen;\\
stellt euch ein,\\
groß und klein,\\
eilt mit großen Haufen;\\
liebt den, der vor Liebe brennet,\\
schaut den Stern,\\
der euch gern\\
Licht und Labsal gönnet.
 
\flagverse{9.} Die ihr schwebt in großem Leiden,\\
sehet, hier\\
ist die Tür\\
zu der wahren Freuden.\\
Faßt ihn wohl, er wird euch führen\\
an den Ort,\\
da hinfort\\
euch kein Kreuz wird rühren.
 
\flagverse{10.} Wer sich fühlt beschwert im Herzen,\\
wer empfind't\\
seine Sünd\\
und Gewissensschmerzen,\\
sei getrost, hier wird gefunden,\\
der in Eil\\
machet heil\\
die vergift'ten Wunden.
 
\flagverse{11.} Die ihr arm seid und elende,\\
kommt herbei,\\
füllet frei\\
eures Glaubens Hände!\\
Hier Sind alle guten Gaben\\
und das Gold,\\
da ihr sollt\\
euer Herz mit laben.
 
\flagverse{12.} Süßes Heil, laß dich umfangen,\\
laß mich dir,\\
meine Zier,\\
unverrückt anhangen.\\
Du Bist meines Lebens Leben;\\
nun kann ich\\
mich durch dich\\
wohl zufrieden geben.
\pagebreak

\flagverse{13.} Meine Schuld kann mich nicht drücken,\\
denn du hast\\
meine Last\\
all auf deinem Rücken.\\
Kein Fleck Ist an mir zu finden,\\
ich bin gar\\
rein und klar\\
aller meiner Sünden.
 
\flagverse{14.} Ich bin rein um deinetwillen,\\
du gibst gnug\\
ehr und Schmuck,\\
mich darein zu hüllen.\\
Ich Will dich ins Herze schließen;\\
o mein Ruhm.\\
Edle Blum,\\
laß dich recht genießen.

\end{verse}
\end{multicols}

\begin{center}
\settowidth{\versewidth}{Ich will dich mit Fleiß bewahren,}
\begin{verse}[\versewidth]

\flagverse{15.} Ich will dich mit Fleiß bewahren,\\ % Choral WO
ich will dir\\
leben hier,\\
dir will ich abfahren.\\
Mit Dir will ich endlich schweben\\
voller Freud,\\
ohne Zeit\\
dort im andern Leben.
  
\end{verse}
\end{center}
 

%\attrib{\small{Letzte Strophe  Choral Weihnachtsoratorium}


