%StartInfo%%%%%%%%%%%%%%%%%%%%%%%%%%%%%%%%%%%%%%%%%%%%%%%%%%%%%%%%%%%%%%%%%%%%
%  Autor:
%  Titel:
%  File:
%  Ref:
%  Mod:
%EndInfo%%%%%%%%%%%%%%%%%%%%%%%%%%%%%%%%%%%%%%%%%%%%%%%%%%%%%%%%%%%%%%%%%%%%%%
%\poemtitle{Geh aus, mein Herz, und suche Freud}
\begin{multicols}{2}
\settowidth{\versewidth}{Die Lerche schwingt sich in die Luft,}
\begin{verse}[\versewidth]

\flagverse{1.} Geh aus, mein Herz, und suche Freud\\
in dieser lieben Sommerzeit\\
an deines Gottes Gaben;\\
schau an der schönen Gärten Zier\\
und siehe, wie sie mir und dir\\
sich ausgeschmücket haben.

\flagverse{2.} Die Bäume stehen voller Laub,\\
das Erdreich decket seinen Staub\\
mit einem grünen Kleide;\\
Narzissus und die Tulipan,\\
die ziehen sich viel schöner an\\
als Salomonis Seide.

\flagverse{3.} Die Lerche schwingt sich in die Luft,\\
das Täublein fleugt aus seiner Kluft\\
und macht sich in die Wälder;\\
die hochbegabte Nachtigall\\
ergötzt und füllt mit ihrem Schall\\
Berg, Hügel, Tal und Felder.

\flagverse{4.} Die Glucke führt ihr Völklein aus,\\
der Storch baut und bewohnt sein Haus,\\
das Schwälblein speist die Jungen;\\
der schnelle Hirsch, das leichte Reh\\
ist froh und kommt aus seiner Höh\\
ins tiefe Gras gesprungen.

\flagverse{5.} Die Bächlein rauschen in dem Sand\\
und malen sich in ihrem Rand\\
mit schattenreichen Myrten;\\
die Wiesen liegen hart dabei\\
und klingen ganz von Lustgeschrei\\
der Schaf und ihrer Hirten.

\flagverse{6.} Die unverdroßne Bienenschar\\
fleucht hin und her, sucht hie und dar\\
ihr edle Honigspeise.\\
Des süßen Weinstocks starker Saft\\
bringt täglich neue Stärk und Kraft\\
in seinem schwachen Reise.

\flagverse{7.} Der Weizen wächset mit Gewalt,\\
darüber jauchzet Jung und Alt\\
und rühmt die große Güte\\
des, der so überflüssig labt\\
und mit so manchem Gut begabt\\
das menschliche Gemüte.

\flagverse{8.} Ich selbsten kann und mag nicht ruhn;\\
des großen Gottes großes Tun\\
erweckt mir alle Sinnen;\\
ich singe mit, wenn alles singt,\\
und lasse, was dem Höchsten klingt,\\
aus meinem Herzen rinnen.

\flagverse{9.} Ach, denk ich, bist du hier so schön\\
und läßt du uns so lieblich gehn\\
auf dieser armen Erden,\\
was will doch wohl nach dieser Welt\\
dort in dem festen Himmelszelt\\
und güldnen Schlosse werden!

\flagverse{10.} Welch hohe Lust, welch heller Schein\\
wird wohl in Christi Garten sein!\\
Wie Muß es da wohl klingen,\\
da so viel tausend Seraphim\\
mit eingestimmtem Mund und Stimm\\
ihr Halleluja singen!

\flagverse{11.} O wär ich da, o stünd ich schon,\\
ach, süßer Gott, vor deinem Thron\\
und trüge meine Palmen,\\
so wollt ich nach der Engel Weis\\
erhöhen deines Namens Preis\\
mit tausend schönen Psalmen!

\flagverse{12.} Doch gleichwohl will ich, weil ich noch\\
hier trage dieses Leibes Joch,\\
auch nicht gar stille schweigen;\\
mein Herze soll sich fort und fort\\
an diesem und an allem Ort\\
zu deinem Lobe neigen.

\flagverse{13.} Hilf mir und segne meinen Geist\\
mit Segen, der vom Himmel fleußt,\\
daß ich dir stetig blühe!\\
Gib, daß der Sommer deiner Gnad\\
in meiner Seelen früh und spat\\
viel Glaubensfrücht erziehe!

\flagverse{14.} Mach in mir deinem Geiste Raum,\\
daß ich dir werd ein guter Baum,\\
und laß mich wohl bekleiben;\\
verleihe, daß zu deinem Ruhm\\
ich deines Gartens schöne Blum\\
und Pflanze möge bleiben!

\end{verse}
\end{multicols}

\begin{center}
\settowidth{\versewidth}{Der, vor dem die Welt erschrickt,}
\begin{verse}[\versewidth]

\flagverse{15.} Erwähle mich zum Paradeis\\
und laß mich bis zur letzten Reis\\
an Leib und Seele grünen;\\
so will ich dir und deiner Ehr\\
allein und sonsten keinem mehr\\
hier und dort ewig dienen.
  
\end{verse}
\end{center}


%\attrib{\small{THZE}}
