%StartInfo%%%%%%%%%%%%%%%%%%%%%%%%%%%%%%%%%%%%%%%%%%%%%%%%%%%%%%%%%%%%%%%%%%%%
%  Autor:
%  Titel:
%  File:
%  Ref:
%  Mod:
%EndInfo%%%%%%%%%%%%%%%%%%%%%%%%%%%%%%%%%%%%%%%%%%%%%%%%%%%%%%%%%%%%%%%%%%%%%%
%\poemtitle{pt}
\begin{multicols}{2}
\settowidth{\versewidth}{Schwing dich auf zu deinem Gott,}
\begin{verse}[\versewidth]
%schwing dich auf zu deinem Gott

\flagverse{1.} Schwing dich auf zu deinem Gott,\\
du betrübte Seele!\\
Warum liegst du, Gott zum Spott,\\
in der Schwermutshöhle?\\
Merkst du nicht des Satans List?\\
Er will durch sein Kämpfen\\
deinen Trost, den Jesus Christ\\
dir erworben, dämpfen.

\flagverse{2.} Schüttle deinen Kopf und sprich:\\
Fleuch, du alte Schlange!\\
Was erneust du deinen Stich,\\
machst mir angst und bange?\\
Ist dir doch der Kopf zerknickt,\\
und ich bin durchs Leiden\\
meines Heilands dir entzückt\\
in den Saal der Freuden.

\flagverse{3.} Wirfst du mir mein Sünd'gen für?\\
Wo hat Gott befohlen,\\
daß mein Urteil über mir\\
ich bei dir soll holen?\\
Wer hat dir die Macht geschenkt,\\
andre zu verdammen,\\
der du selbst doch liegst versenkt\\
in der Höllen Flammen?

\flagverse{4.} Hab ich was nicht recht getan,\\
ist mirs leid von Herzen;\\
dahingegen nehm ich an\\
Christi Blut und Schmerzen.\\
Denn das ist die Ranzion\\
meiner Missetaten.\\
Bring ich dies vor Gottes Thron,\\
ist mir wohl geraten.

\flagverse{5.} Christi Unschuld ist mein Ruhm,\\
sein Recht meine Krone,\\
sein Verdienst mein Eigentum,\\
da ich frei in wohne\\
als in einem festen Schloß,\\
das kein Feind kann fällen,\\
brächt er gleich davor Geschoß\\
und Gewalt der Höllen.

\flagverse{6.} Stürme, Teufel und du Tod,\\
was könnt ihr mir schaden?\\
Deckt mich doch in meiner Not\\
Gott mit seiner Gnaden.\\
Der Gott, der mir seinen Sohn\\
selbst verehrt aus Liebe,\\
daß der ewge Spott und Hohn\\
mich nicht dort betrübe.

\flagverse{7.} Schreie, tolle Welt, es sei\\
mir Gott nicht gewogen,\\
es ist lauter Täuscherei\\
und im Grund erlogen.\\
Wäre Gott mir gram und feind,\\
würd er seine Gaben,\\
die mein eigen worden seind,\\
wohl behalten haben.

\flagverse{8.} Denn was ist im Himmelszelt,\\
was im tiefen Meere,\\
was ist Gutes in der Welt,\\
das nicht mir gut wäre?\\
Weme brennt das Sternenlicht?\\
Wozu ist gegeben\\
luft und Wasser? Dient es nicht\\
mir und meinem Leben?

\flagverse{9.} Weme wird das Erdreich naß\\
von dem Tau und Regen?\\
Weme grünet Laub und Gras?\\
Weme füllt der Segen\\
berg und Tale, Feld und Wald?\\
Wahrlich, mir zur Freude,\\
daß ich meinen Aufenthalt\\
hab und Leibesweide.

\flagverse{10.} Meine Seele lebt in mir\\
durch die süßen Lehren,\\
so die Christen mit Begier\\
alle Tage hören.\\
Gott eröffnet früh und spat\\
meinen Geist und Sinnen,\\
daß sie seines Geistes Gnad\\
in sich ziehen können.

\flagverse{11.} Was sind der Propheten Wort\\
und Apostel Schreiben\\
als ein Licht am dunklen Ort,\\
fackeln, die vertreiben\\
meines Herzens Finsternis\\
und in Glaubenssachen\\
das Gewissen fein gewiß\\
und recht grundfest machen?

\flagverse{12.} Nun, auf diesen heilgen Grund\\
bau ich mein Gemüte,\\
sehe, wie der Höllenhund\\
zwar dawider wüte;\\
gleichwohl muß er lassen stehn,\\
was Gott aufgerichtet,\\
aber schändlich muß vergehn,\\
was er selber dichtet.

\flagverse{13.} Ich bin Gottes, Gott ist mein:\\
Wer ist, der uns scheide?\\
Dringt das liebe Kreuz herein\\
mit dem bittern Leide,\\
laß es dringen, kommt es doch\\
von geliebten Händen,\\
bricht und kriegt geschwind ein Loch,\\
wenn es Gott will wenden.

\flagverse{14.} Kinder, die der Vater soll\\
ziehn zu allem Guten,\\
die gedeihen selten wohl\\
ohne Zucht und Ruten.\\
Bin ich denn nun Gottes Kind,\\
warum will ich fliehen,\\
wenn er mich von meiner Sünd\\
auf was Guts will ziehen?

\flagverse{15.} Es ist herzlich gut gemeint\\
mit der Christen Plagen:\\
Wer hier zeitlich wohl geweint,\\
darf nicht ewig klagen,\\
sondern hat vollkommne Lust\\
dort in Christi Garten\\
(dem er einig recht bewußt)\\
endlich zu gewarten.

\flagverse{16.} Gottes Kinder säen zwar\\
traurig und mit Tränen,\\
aber endlich bringt das Jahr,\\
wonach sie sich sehnen;\\
denn es kommt die Erntezeit,\\
da sie Garben machen,\\
da wird all ihr Gram und Leid\\
lauter Freud und Lachen.

\end{verse}
\end{multicols}

\begin{center}
\settowidth{\versewidth}{Der, vor dem die Welt erschrickt,}
\begin{verse}[\versewidth]



\flagverse{17.} Ei, so faß, o Christenherz,\\
alle deine Schmerzen,\\
wirf sie fröhlich hinterwärts,\\
laß des Trostes Kerzen\\
dich entzünden mehr und mehr,\\
gib dem großen Namen\\
deines Gottes Preis und Ehr,\\
er wird helfen. Amen.

  
\end{verse}
\end{center}



%\attrib{\small{THZE}}
