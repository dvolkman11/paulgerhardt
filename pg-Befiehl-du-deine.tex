%StartInfo%%%%%%%%%%%%%%%%%%%%%%%%%%%%%%%%%%%%%%%%%%%%%%%%%%%%%%%%%%%%%%%%%%%%
%  Desc:  Befiehl du deine Wege - Paul Gerhard
%  Desc:  Include 
%  Desc:  Zweispaltiger Satz mit \verse
%  Tags:  GERHARD VERSE EG361 INCLUDE
%  File:  pg-befiehl-du-deine.tex
%  Autor: dv
%  Ref:   
%  Mod:   06.10.2016/dv/initial
%  Mod:   
%EndInfo%%%%%%%%%%%%%%%%%%%%%%%%%%%%%%%%%%%%%%%%%%%%%%%%%%%%%%%%%%%%%%%%%%%%%%



%\poemtitle{Befiehl du deine Wege}
\begin{multicols}{2}
\settowidth{\versewidth}{Befiehl du deine Wege und was}                                                  
\begin{verse}[\versewidth]                                                                                              
  \flagverse{1.} \emph{Befiehl} du deine Wege\\
  und was dein Herze kränkt\\
  der allertreusten Pflege des,\\
  der den Himmel lenkt.\\
  Der Wolken, Luft Und Winden\\
  gibt Wege Lauf und Bahn,\\
  der wird auch Wege finden,\\
  die dein Fuß gehen kann.
  
  \flagverse{2.} \emph{Dem Herren} mußt du trauen,\\
  wenn dir's soll wohlergehn;\\
  auf sein Werk mußt du schauen,\\
  wenn dein Werk soll bestehn.\\
  Mit Sorgen und mit Grämen\\
  und mit selbsteigner Pein\\
  läßt Gott sich gar nichts nehmen,\\
  es muß erbeten sein.

  
  \flagverse{3.} \emph{Dein} ewge Treu und Gnade,\\
  o Vater,  weiß und sieht,\\
  was gut sei oder schade\\
  dem sterblichen Geblüt;\\
  und was du dann erlesen,\\
  das treibst du, starker Held\\
  und bringst zum Stand und Wesen,\\
  was deinem Rat gefällt.
  
  \flagverse{4.} \emph{Weg} hast du allerwegen,\\
  an Mitteln fehlt dir's nicht\\
  dein Tun ist lauter Segen,\\
  dein Gang ist lauter Licht;\\
  dein Werk kann niemand hindern,\\
  dein Arbeit darf nicht ruhn,\\
  wenn du, was deinen Kindern\\
  ersprießlich ist, willst tun.

  \flagverse{5.} \emph{Und} ob gleich alle Teufel\\
  hier wollten widerstehn,\\
  so wird doch ohne Zweifel\\
  Gott nicht zurücke gehn;\\
  was er sich vorgenommen\\
  und was er haben will,\\
  das muß doch endlich kommen\\
  zu seinem Zweck und Ziel.

  \flagverse{6.} \emph{Hoff}, o du arme Seele,\\
  hoff und sei unverzagt!\\
  Gott Wird dich aus der Höhle,\\
  da dich der Kummern plagt,\\
  mit großen Gnaden rücken;\\
  erwarte nur die Zeit,\\
  so wirst du schon erblicken\\
  die Sonn der schönsten Freud.

  \flagverse{7.} \emph{Auf}, auf, gib deinem Schmerze\\
  und Sorgen gute Nacht,\\
  laß fahren, was das Herze\\
  betrübt und traurig macht;\\
  bist du doch nicht Regente,\\
  der alles führen soll,\\
  Gott sitzt im Regimente\\
  und führet alles wohl.

  \flagverse{8.} \emph{Ihn,} ihn laß tun und walten,\\
  er ist ein weiser Fürst\\
  und wird sich so verhalten,\\
  daß du dich wundern wirst,\\
  wenn er, wie ihm gebühret,\\
  mit wunderbaren Rat\\
  das Werk hinausgeführet,\\
  das dich bekümmert hat.

  \flagverse{9.} \emph{Er} wird zwar eine Weile\\
  mit seinem Trost verziehn\\
  und tun an seinem Teile,\\
  als hätt in seinem Sinn\\
  er deiner sich begeben\\
  und sollt'st du für und für\\
  in Angst und Nöten schweben,\\
  als frag er nichts nach dir.

  \flagverse{10.} \emph{Wird's} aber sich befinden,\\
  daß du ihm treu verbleibst,\\
  so wird er dich entbinden,\\
  da du's am mindsten glaubst;\\
  er wird dein Herze lösen\\
  von der so schweren Last,\\
  die du zu keinem Bösen\\
  bisher getragen hast.

  \flagverse{11.} \emph{Wohl} dir, du Kind der Treue,\\
  du hast und trägst davon\\
  mit Ruhm und Dankgeschreie\\
  den Sieg und Ehrenkron;\\
  Gott gibt dir selbst die Palmen\\
  in deine rechte Hand,\\
  und du singst Freudenpsalmen\\
  dem, der dein Leid gewandt.

  \flagverse{12.} \emph{Mach En}d, o Herr, mach Ende\\
  mit aller unsrer Not;\\
  stärk unsre Füß und Hände\\
  und laß bis in den Tod\\
  uns allzeit deiner Pflege\\
  und Treu empfohlen sein,\\
  so gehen unsre Wege\\
  gewiß zum Himmel ein.
  
  

\end{verse}
\end{multicols}
%\attrib{\small{Paul Gerhard 1653}}

  
