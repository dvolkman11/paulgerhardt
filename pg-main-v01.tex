%StartInfo%%%%%%%%%%%%%%%%%%%%%%%%%%%%%%%%%%%%%%%%%%%%%%%%%%%%%%%%%%%%%%%%%%%%
%  Desc:  Paul Gerhard
%  Desc:  Gedichte 
%  Desc:  Zweispaltiger Satz mit \verse
%  Tags:  GERHARD VERSE 
%  File:  pg-main-v01.tex
%  Autor: dv
%  Ref:   
%  Mod:   
%  Mod:   06.10.2016/dv/initial
%EndInfo%%%%%%%%%%%%%%%%%%%%%%%%%%%%%%%%%%%%%%%%%%%%%%%%%%%%%%%%%%%%%%%%%%%%%%

%----------------------------------------------------------------------------------------
%	PACKAGES AND OTHER DOCUMENT CONFIGURATIONS
%----------------------------------------------------------------------------------------

\documentclass{book}
\usepackage[T1]{fontenc}
\usepackage[utf8]{inputenc} % set input encoding (not needed with XeLaTeX)
\usepackage[ngerman]{babel}

%%% PAGE DIMENSIONS ---------------------------------------------------------------------
\usepackage[centering,includeheadfoot,margin=2cm]{geometry}
%\usepackage[a4paper, top=3cm, bottom=3cm]{geometry}
\geometry{a4paper}       % or letterpaper (US) or a5paper or....
% \geometry{margins=2in} % for example, change the margins to 2 inches all round
% \geometry{landscape}   % set up the page for landscape
%                           read geometry.pdf for detailed page layout information

%\usepackage{graphicx} % support the \includegraphics command and options

% \usepackage[parfill]{parskip} % Activate to begin paragraphs with an empty line rather than an indent

%%% PACKAGES ---------------------------------------------------------------------------
%\usepackage{booktabs} % for much better looking tables
%\usepackage{array} % for better arrays (eg matrices) in maths
%\usepackage{paralist} % very flexible & customisable lists (eg. enumerate/itemize, etc.)
\usepackage{verbatim} % adds environment for commenting out blocks of text & for better verbatim
%\usepackage{subfig} % make it possible to include more than one captioned figure/table in a single float
% These packages are all incorporated in the memoir class to one degree or another...


\usepackage{lmodern}
\usepackage{multicol}
\usepackage{verse}
\usepackage{attrib}                                                                                                      

\usepackage{makeidx}
\makeindex

%%% dv: Bis zu 12 Strophen in zwei Spalten auf eine Seite zu setzen: --------------------
\parindent0em
\setlength{\voffset}{-1.5cm}
\setlength{\hoffset}{-0.5cm}
\setlength{\textheight}{27cm}
\setlength{\textwidth}{17cm}
\setlength{\columnwidth}{7cm}
\setlength{\columnsep}{1cm}
\setlength{\vleftskip}{0.1cm}
\setlength{\oddsidemargin}{0.1cm}
\setlength{\vgap}{0.1cm}
%\setlength{\stanzaskip}{1.3cm}

%----------------------------------------------------------------------------------------
%	TITLE PAGE
%----------------------------------------------------------------------------------------
%\newcommand*{\plogo}{\fbox{X} % Generic publisher logo
\newcommand*{\titleGP}{\begingroup % Create the command for including the title page in the document
\centering % Center all text
\vspace*{\baselineskip} % White space at the top of the page

\rule{\textwidth}{1.6pt}\vspace*{-\baselineskip}\vspace*{2pt} % Thick horizontal line
\rule{\textwidth}{0.4pt}\\[\baselineskip] % Thin horizontal line

{\LARGE PAUL GERHARDT\\[0.9\baselineskip] Frisch auf, getrost und unverzagt!}\\[0.2\baselineskip] % Title

\rule{\textwidth}{0.4pt}\vspace*{-\baselineskip}\vspace{3.2pt} % Thin horizontal line
\rule{\textwidth}{1.6pt}\\[\baselineskip] % Thick horizontal line

\scshape % Small caps
Gedichte und Lieder  \\ % Tagline(s) or further description
%für Kerstin zum 60'sten Geburtstag \\%[\baselineskip] % Tagline(s) or further description
%Siegersleben\\ 2016\par % Location and year

\vspace*{25\baselineskip} % Whitespace between location/year and editors

%Gesetzt \\[\baselineskip]
%von \\[\baselineskip]
%{\Large Dietmar Volkmann\par} % Editor list
%{\itshape The University of California \\ Berkeley\par} % Editor affiliation

\vfill % Whitespace between editor names and publisher logo

%\plogo \\[0.9\baselineskip] % Publisher logo
{\scshape Siegersleben}\par % Location and year
{\scshape 2017} \\[0.9\baselineskip] % Year published
{\large S.D.G.}\par % Publisher

\thispagestyle{empty}%dv
\endgroup}%


%----------------------------------------------------------------------------------------
%	 DOCUMENT
%----------------------------------------------------------------------------------------
\pagestyle{empty}
%dv: Layout des Inhaltsverzeichnisses definieren:
\AtBeginDocument{%
  \renewcommand\contentsname{%
    \centerline{INHALT}
    \rule{\textwidth}{0.2pt}\vspace*{-\baselineskip}\vspace{3.2pt} 
    \rule{\textwidth}{1.2pt}\\[\baselineskip]
  }
}


\begin{document} 


\titleGP % This command includes the title page

\frontmatter% ---------------------------------------------------------------------------

\newpage

\centering % Center all text
\vspace*{\baselineskip} % White space at the top of the page

\rule{\textwidth}{1.6pt}\vspace*{-\baselineskip}\vspace*{2pt} % Thick horizontal line
\rule{\textwidth}{0.4pt}\\[\baselineskip] % Thin horizontal line

%{\LARGE PAUL GERHARDT\\[0.9\baselineskip] Frisch auf, getrost und unverzagt!}\\[0.2\baselineskip] % Title

\rule{\textwidth}{0.4pt}\vspace*{-\baselineskip}\vspace{3.2pt} % Thin horizontal line
\rule{\textwidth}{1.6pt}\\[\baselineskip] % Thick horizontal line

%\scshape % Small caps
%Die Gedichte Paul Gerhardts  \\ % Tagline(s) or further description
%für Kerstin zum 60'sten Geburtstag \\%[\baselineskip] % Tagline(s) or further description
%Siegersleben\\ 2016\par % Location and year

\vspace*{25\baselineskip} % Whitespace between location/year and editors

%Gesetzt \\[\baselineskip]
%von \\[\baselineskip]
%{\Large Dietmar Volkmann\par} % Editor list
%{\itshape The University of California \\ Berkeley\par} % Editor affiliation

\vfill % Whitespace between editor names and publisher logo

%\plogo \\[0.9\baselineskip] % Publisher logo
{\scshape * * *}\par % Location and year
\newpage

\newpage

\centering % Center all text
\vspace*{\baselineskip} % White space at the top of the page

\rule{\textwidth}{1.6pt}\vspace*{-\baselineskip}\vspace*{2pt} % Thick horizontal line
\rule{\textwidth}{0.4pt}\\[\baselineskip] % Thin horizontal line

{\LARGE Für Kerstin}\\[\baselineskip]

\rule{\textwidth}{0.4pt}\vspace*{-\baselineskip}\vspace{3.2pt} % Thin horizontal line
%\rule{\textwidth}{1.6pt}\\[\baselineskip] % Thick horizontal line

%\scshape % Small caps
%Die Gedichte Paul Gerhardts  \\ % Tagline(s) or further description
%für Kerstin zum 60'sten Geburtstag \\%[\baselineskip] % Tagline(s) or further description
%Siegersleben\\ 2016\par % Location and year

\vspace*{25\baselineskip} % Whitespace between location/year and editors

%Gesetzt \\[\baselineskip]
%von \\[\baselineskip]
%{\Large Dietmar Volkmann\par} % Editor list
%{\itshape The University of California \\ Berkeley\par} % Editor affiliation

\vfill % Whitespace between editor names and publisher logo

%\plogo \\[0.9\baselineskip] % Publisher logo
{\scshape * * *}\par % Location and year
\newpage


\pagestyle{plain}


\tableofcontents
\clearpage
\newpage

\centering % Center all text
\vspace*{\baselineskip} % White space at the top of the page

\rule{\textwidth}{1.6pt}\vspace*{-\baselineskip}\vspace*{2pt} % Thick horizontal line
\rule{\textwidth}{0.4pt}\\[\baselineskip] % Thin horizontal line

%{\LARGE PAUL GERHARDT\\[0.9\baselineskip] Frisch auf, getrost und unverzagt!}\\[0.2\baselineskip] % Title

\rule{\textwidth}{0.4pt}\vspace*{-\baselineskip}\vspace{3.2pt} % Thin horizontal line
\rule{\textwidth}{1.6pt}\\[\baselineskip] % Thick horizontal line

%\scshape % Small caps
%Die Gedichte Paul Gerhardts  \\ % Tagline(s) or further description
%für Kerstin zum 60'sten Geburtstag \\%[\baselineskip] % Tagline(s) or further description
%Siegersleben\\ 2016\par % Location and year

\vspace*{25\baselineskip} % Whitespace between location/year and editors

%Gesetzt \\[\baselineskip]
%von \\[\baselineskip]
%{\Large Dietmar Volkmann\par} % Editor list
%{\itshape The University of California \\ Berkeley\par} % Editor affiliation

\vfill % Whitespace between editor names and publisher logo

%\plogo \\[0.9\baselineskip] % Publisher logo
{\scshape * * *}\par % Location and year
\newpage


\mainmatter% ----------------------------------------------------------------------------

\section*{\centerline{\LARGE ADVENT}}
\addcontentsline{toc}{section}{ADVENT}
\rule{\textwidth}{0.2pt}\vspace*{-\baselineskip}\vspace{3.2pt} % Thin horizontal line
\rule{\textwidth}{1.2pt}\\[\baselineskip] % Thick horizontal line


\index{ Gedichte zum Advent}
%      ^ Das Lehrzeichen am Anfang bring diese Einträge an den Anfang des Index!

\centerline{\scshape Die Zeit ist nunmehr nah}    
\vspace*{2\baselineskip}
\centerline{\scshape Warum willst du draußen stehen}
\vspace*{2\baselineskip}
\centerline{\scshape Wie soll ich dich empfangen}
%\vspace*{55\baselineskip}

\newpage
%
\subsection*{\centerline{Die Zeit ist nunmehr nah}}
\addcontentsline{toc}{subsection}{Die Zeit ist nunmehr nah}
%StartInfo%%%%%%%%%%%%%%%%%%%%%%%%%%%%%%%%%%%%%%%%%%%%%%%%%%%%%%%%%%%%%%%%%%%%
%  Autor:
%  Titel:
%  File:
%  Ref:
%  Mod:
%EndInfo%%%%%%%%%%%%%%%%%%%%%%%%%%%%%%%%%%%%%%%%%%%%%%%%%%%%%%%%%%%%%%%%%%%%%%
%\poemtitle{Die Zeit ist nunmehr nah}
\begin{multicols}{2}
\settowidth{\versewidth}{Werd ich denn auch vor Freud}
\begin{verse}[\versewidth]

%Veranlaßt durch den Kometen des Jahres 1652
%
\flagverse{1.} Die Zeit ist nunmehr nah,\\
Herr Jesu, du bist da.\\
Die Wunder, die den Leuten\\
dein Ankunft sollen deuten,\\
die sind, wie wir gesehen,\\
in großer Zahl geschehen.

\flagverse{2.} Was soll ich denn nun tun?\\
Ich soll auf dem beruhn,\\
was du mir hast verheißen,\\
daß du mich wollest reißen\\
aus meines Grabes Kammer\\
und allem andern Jammer.

\flagverse{3.} Ach Jesu, wie so schön\\
wird mir's alsdann ergehn!\\
Du wirst mit tausend Blicken\\
mich durch und durch erquicken,\\
wenn ich hier von der Erde\\
mich zu dir schwingen werde.

\flagverse{4.} Ach, was wird doch dein Wort,\\
o süßer Seelenhort,\\
was wird doch sein dein Sprechen,\\
wenn dein Herz aus wird brechen\\
zu mir und meinen Brüdern\\
als deinen Leibesgliedern.

\flagverse{5.} Werd ich denn auch vor Freud\\
in solcher Gnadenzeit\\
den Augen ihre Zähren\\
und Tränen können wehren,\\
daß sie mir nicht mit Haufen\\
auf meine Wangen laufen?

\flagverse{6.} Was für ein schönes Licht\\
wird mir dein Angesicht,\\
das ich in jenem Leben\\
werd erstmal sehen, geben!\\
Wie wird mir deine Güte\\
entzücken mein Gemüte!

\flagverse{7.} Dein Augen, deinen Mund,\\
den Leib, der noch verwundt,\\
da wir so fest auf trauen,\\
das werd ich alles schauen,\\
auch innig herzlich grüßen\\
die Mal an Händ und Füßen.

\flagverse{8.} Dir ist allein bewußt\\
die ungefälschte Lust\\
und edle Seelenspeise\\
in deinem Paradeise.\\
Die kannst du wohl beschreiben,\\
ich kann nichts mehr als gläuben.

\flagverse{9.} Doch was ich hie gegläubt,\\
das steht gewiß und bleibt\\
mein Teil, dem gar nicht gleichen\\
die Güter aller Reichen;\\
all andres Gut vergehet,\\
mein Erbteil, das bestehet.

\flagverse{10.} Ach Herr, mein schönstes Gut,\\
wie wird sich all mein Blut\\
in allen Adern freuen\\
und auf das Neu erneuen,\\
wenn du mir wirst mit Lachen\\
die Himmelstür aufmachen!

\flagverse{11.} Komm her, komm und empfind,\\
o auserwähltes Kind,\\
komm, schmecke, was für Gaben\\
ich und mein Vater haben,\\
komm, wirst du sagen, weide\\
dein Herz in ewger Freude!

\flagverse{12.} Ach, du so arme Welt,\\
was ist dein Gold und Geld\\
hier gegen diese Kronen\\
und mehr als güldnen Thronen,\\
die Christus hingestellet\\
dem Volk, das ihm gefället.

\flagverse{13.} Hie ist der Engel Land,\\
der selgen Seelen Stand;\\
hie hör ich nichts als singen,\\
hie seh ich nichts als springen,\\
hie ist kein Kreuz, kein Leiden,\\
kein Tod, kein bittres Scheiden.

\flagverse{14.} Halt ein, mein schwacher Sinn,\\
halt ein! Wo denkst du hin?\\
Willst du, was grundlos, gründen?\\
Was unbegreiflich, finden?\\
Hier muß der Witz sich neigen\\
und alle Redner schweigen.

\flagverse{15.} Dich aber, meine Zier,\\
dich laß ich nicht von mir;\\
dein will ich stets gedenken,\\
Herr, der du mir wirst schenken\\
mehr als mit meiner Seelen\\
ich wünschen kann und zählen.

\flagverse{16.} Ach, wie ist mir so weh,\\
eh ich dich aus der Höh,\\
Herr, sehe zu uns kommen!\\
Ach, daß zum Heil der Frommen\\
du meinen Wunsch und Willen\\
noch möchtest heut erfüllen!

\flagverse{17.} Doch du weißt deine Zeit.\\
Mir ziemt nur, stets bereit\\
und fertig dazustehen\\
und so zum Herren zu gehen,\\
daß alle Stund und Tage\\
mein Herz mich zu dir trage.

\flagverse{18.} Dies gib, Herr, und verleih,\\
auf daß dein Huld und Treu\\
ohn Unterlaß mich wecke,\\
daß mich dein Tag nicht schrecke,\\
da unser Schreck auf Erden\\
soll Fried und Freude werden.

\end{verse}
\end{multicols}
%\attrib{\small{THZE}}

\index{Die Zeit ist nunmehr nah}
\newpage
\subsection*{\centerline{Warum willst du draußen stehen}}
\addcontentsline{toc}{subsection}{Warum willst du draußen stehen}
%StartInfo%%%%%%%%%%%%%%%%%%%%%%%%%%%%%%%%%%%%%%%%%%%%%%%%%%%%%%%%%%%%%%%%%%%%
%  Desc:  Warum willst du draußen stehen - Paul Gerhard
%  Desc:  Include 
%  Desc:  Zweispaltiger Satz mit \verse
%  Tags:  GERHARD VERSE EG361 INCLUDE
%  File:  pg-warum-willst-du-ich.tex
%  Autor: dv
%  Ref:   
%  Mod:   15.11.2016/dv/initial
%  Mod:   
%EndInfo%%%%%%%%%%%%%%%%%%%%%%%%%%%%%%%%%%%%%%%%%%%%%%%%%%%%%%%%%%%%%%%%%%%%%%

%\poemtitle{Warum willst du daraußen stehen}
\begin{multicols}{2}
\settowidth{\versewidth}{Warum willst du draußen stehen,}                                                  
\begin{verse}[\versewidth]
  
  \flagverse{1.} Warum willst du draußen stehen,\\
  du Gesegneter des Herrn?\\
  Laß dir bei mir einzugehen\\
  wohl gefallen, du mein Stern!\\
  Du, mein Jesu, meine Freud,\\
  Helfer in der rechten Zeit,\\
  hilf, o Heiland, meinem Herzen\\
  von den Wunden, die mich schmerzen.

  \flagverse{2.} Meine Wunden sind der Jammer,\\
  welchen oftmals Tag und Nacht\\
  des Gesetzes starker Hammer\\
  mir mit seinem Schrecken macht.\\
  O der schweren Donnerstimm,\\
  die mir Gottes Zorn und Grimm\\
  also tief ins Herze schläget,\\
  daß sich all mein Blut beweget.

  \flagverse{3.} Dazu kommt des Teufels Lügen,\\
  der mir alle Gnad absagt,\\
  als müßt ich nun ewig liegen\\
  in der Höllen, die ihn plagt;\\
  ja auch, was noch ärger ist,\\
  so zermartert und zerfrißt\\
  mich mein eigenes Gewissen\\
  mit vergift'ten Schlangenbissen.


  \flagverse{4.} Will ich dann mein Elend lindern\\
  und erleichtern meine Not\\
  bei der Welt und ihren Kindern,\\
  fall ich vollends in den Kot:\\
  Da ist Trost, der mich betrübt,\\
  Freude, die mein Unglück liebt,\\
  Helfer, die mir Herzleid machen,\\
  gute Freunde, die mein lachen.

  \flagverse{5.} In der Welt ist alles nichtig,\\
  nichts ist, das nicht kraftlos wär:\\
  Hab ich Hoheit, die ist flüchtig!\\
  Hab ich Reichtum, was ist's mehr\\
  als ein Stücklein armer Erd?\\
  Hab ich Lust, was ist sie wert?\\
  Was ist's, das mich heut erfreuet,\\
  das mich morgen nicht gereuet?

  \flagverse{6.} Aller Trost und alle Freude\\
  ruht in dir, Herr Jesu Christ;\\
  dein Erfreuen ist die Weide,\\
  da man sich recht fröhlich ißt.\\
  Leuchte mir, o Freudenlicht,\\
  ehe mir mein Herze bricht;\\
  laß mich, Herr, an dir erquicken;\\
  Jesu, komm, laß dich erblicken!

  \flagverse{7.} Freu dich, Herz, du bist erhöret,\\
  jetzo zeucht er bei dir ein,\\
  sein Gang ist zu dir gekehret,\\
  heiß ihn nur willkommen sein\\
  und bereite dich ihm zu,\\
  gib dich ganz zu seiner Ruh,\\
  öffne dein Gemüt und Seele,\\
  klag ihm, was dich drückt und quäle.

  \flagverse{8.} Siehst du, wie sich alles setzet,\\
  was dir vor zuwider stund?\\
  Hörst du, wie er dich ergötzet\\
  mit dem zuckersüßen Mund?\\
  Ei, wie läßt der große Drach\\
  all sein Tun und Toben nach!\\
  Er muß aus dem Vorteil ziehen\\
  und in seinen Abgrund fliehen.

  \flagverse{9.} Nun, du hast ein süßes Leben;\\
  alles, was du willst, ist dein.\\
  Christus, der sich dir ergeben,\\
  legt sein Reichtum bei dir ein.\\
  Seine Gnad ist deine Kron\\
  und du bist sein Hütt' und Thron.\\
  Er hat dich in sich geschlossen,\\
  nennt dich seinen Hausgenossen.

  \flagverse{10.} Seines Himmels güldne Decke\\
  spannt er um dich ringsherum,\\
  daß dich fort nicht mehr erschrecke\\
  deines Feindes Ungestüm.\\
  Seine Engel stellen sich\\
  dir zur Seiten, wann du dich\\
  hier willst oder dorthin wenden,\\
  tragen sie dich auf den Händen.

  \flagverse{11.} Was du Böses hast begangen,\\
  das ist alles abgeschafft.\\
  Gottes Liebe nimmt gefangen\\
  deiner Sünde Macht und Kraft.\\
  Christi Sieg behält das Feld,\\
  und was Böses in der Welt\\
  sich will wider dich erregen,\\
  wird zu lauter Glück und Segen.

  \flagverse{12.} Alles dient zu deinem Frommen,\\
  was dir bös und schädlich scheint,\\
  weil dich Christus angenommen\\
  und es treulich mit dir meint.\\
  Bleibst du deme wieder treu,\\
  ist's gewiß und bleibt dabei,\\
  daß du mit den Engeln droben\\
  ihn dort ewig werdest loben.

\end{verse}
\end{multicols}
%\attrib{\small{1653}}

\index{Warum willst du draußen stehen}
\newpage
\subsection*{\centerline{Wie soll ich dich empfangen}}
\addcontentsline{toc}{subsection}{Wie soll ich dich empfangen}
%StartInfo%%%%%%%%%%%%%%%%%%%%%%%%%%%%%%%%%%%%%%%%%%%%%%%%%%%%%%%%%%%%%%%%%%%%
%  Desc:  Wie soll ich dich empfangen - Paul Gerhard
%  Desc:  Include 
%  Desc:  Zweispaltiger Satz mit \verse
%  Tags:  GERHARD VERSE EG361 INCLUDE
%  File:  pg-wie-soll-ich.tex
%  Autor: dv
%  Ref:   
%  Mod:   06.10.2016/dv/initial
%  Mod:   
%EndInfo%%%%%%%%%%%%%%%%%%%%%%%%%%%%%%%%%%%%%%%%%%%%%%%%%%%%%%%%%%%%%%%%%%%%%%

%\poemtitle{Wie soll ich dich empfangen}
\begin{multicols}{2}
\settowidth{\versewidth}{Wie soll ich dich empfangen}                                                  
\begin{verse}[\versewidth]                                                                                              
  \flagverse{1.} Wie soll ich dich empfangen\\
  und wie begegn' ich dir?\\
  O Aller Welt Verlangen!\\
  O Meiner Seelen Zier!\\
  O Jesu, Jesu setze\\
  mir selbst die Fackel bei,\\
  damit, was dich ergötze,\\
  mir kund und wissend sei.

  \flagverse{2.} Dein Zion streut dir Palmen\\
  und grüne Zweige hin,\\
  und ich will dir mit Psalmen\\
  ermuntern meinen Sinn.\\
  Mein Herze soll dir grünen\\
  in stetem Lob und Preis\\
  und deinem Namen dienen,\\
  so gut es kann und weiß.
  
  \flagverse{3.} Was hast du unterlassen\\
  zu meinem Trost und Freud'?\\
  Als Leib und Seele saßen,\\
  in ihrem größten Leid,\\
  als mir das Reich genommen,\\
  da Fried' und Freude lacht,\\
  da bist du, mein Heil, kommen,\\
  und hast mich froh gemacht.
  
  \flagverse{4.} Ich lag in schweren Banden,\\
  du kommst und machst mich los;\\
  ich stand in Spott und Schanden,\\
  du kommst und machst mich groß,\\
  und hebst mich hoch zu Ehren\\
  und schenkst mir großes Gut,\\
  das sich nicht läßt verzehren,\\
  wie irdisch Reichtum tut.

  \flagverse{5.} Nichts, nichts hat dich getrieben\\
  zu mir vom Himmelszelt\\
  als das geliebte Lieben,\\
  damit du alle Welt\\
  in ihren tausend Plagen\\
  und großen Jammerlast,\\
  die kein Mund kann aussagen,\\
  so fest umfangen hast.

  \flagverse{6.} Das schreib dir in dein Herze,\\
  du hochbetrübtes Heer,\\
  bei denen Gram und Schmerze\\
  sich häuft je mehr und mehr;\\
  seid unverzagt, ihr habet\\
  die Hilfe vor der Tür!\\
  Der eure Herzen labet\\
  und tröstet, steht allhier.

  \flagverse{7.} Ihr dürft euch nicht bemühen\\
  noch sorgen Tag und Nacht,\\
  wie ihr ihn wollet ziehen\\
  mit eures Armes Macht.\\
  Er kommt, er kommt mit Willen,\\
  ist voller Lieb und Lust,\\
  all Angst und Not zu stillen,\\
  die ihm an euch bewußt.

  \flagverse{8.} Auch dürft ihr nicht erschrecken\\
  vor eurer Sündenschuld.\\
  Nein! Jesus will sie decken\\
  mit seiner Lieb und Huld!\\
  Er kommt, er kommt den Sündern\\
  zum Trost und wahren Heil,\\
  schafft, daß bei Gottes Kindern\\
  verbleib ihr Erb und Teil.

  \flagverse{9.} Was fragt ihr nach dem Schreien\\
  der Feind und ihrer Tück?\\
  Der Herr wird sie zerstreuen\\%EG Der Herr (?)
  in einem Augenblick.\\
  Er kommt, er kommt, ein König,\\
  dem wahrlich alle Feind'\\
  auf Erden viel zu wenig\\
  zum Widerstande seind.

  \flagverse{10.} Er kommt zum Weltgerichte,\\
  zum Fluch dem, der ihm flucht,\\
  mit Gnad und süßem Lichte\\
  dem, der ihn liebt und sucht.\\
  Ach komm, ach komm, o Sonne\\%im EG nach Sonne ein Komma (?)
  und hol uns allzumahl\\
  zum ewgen Licht und Wonne\\
  in deinen Freudensaal.

\end{verse}
\end{multicols}
%\attrib{\small{1653}}

\index{Wie soll ich dich empfangen}

%\newpage
\newpage

\centering % Center all text
\vspace*{\baselineskip} % White space at the top of the page

\rule{\textwidth}{1.6pt}\vspace*{-\baselineskip}\vspace*{2pt} % Thick horizontal line
\rule{\textwidth}{0.4pt}\\[\baselineskip] % Thin horizontal line

%{\LARGE PAUL GERHARDT\\[0.9\baselineskip] Frisch auf, getrost und unverzagt!}\\[0.2\baselineskip] % Title

\rule{\textwidth}{0.4pt}\vspace*{-\baselineskip}\vspace{3.2pt} % Thin horizontal line
\rule{\textwidth}{1.6pt}\\[\baselineskip] % Thick horizontal line

%\scshape % Small caps
%Die Gedichte Paul Gerhardts  \\ % Tagline(s) or further description
%für Kerstin zum 60'sten Geburtstag \\%[\baselineskip] % Tagline(s) or further description
%Siegersleben\\ 2016\par % Location and year

\vspace*{25\baselineskip} % Whitespace between location/year and editors

%Gesetzt \\[\baselineskip]
%von \\[\baselineskip]
%{\Large Dietmar Volkmann\par} % Editor list
%{\itshape The University of California \\ Berkeley\par} % Editor affiliation

\vfill % Whitespace between editor names and publisher logo

%\plogo \\[0.9\baselineskip] % Publisher logo
{\scshape * * *}\par % Location and year
\newpage

%\newpage

\section*{\centerline{\LARGE WEIHNACHTEN}}
\addcontentsline{toc}{section}{WEIHNACHTEN}
\rule{\textwidth}{0.2pt}\vspace*{-\baselineskip}\vspace{3.2pt}
\rule{\textwidth}{1.2pt}\\[\baselineskip]

\index{ Gedichte zu Weihnachten}

\centerline{\scshape Alle, die ihr, Gott zu Ehren }
\vspace*{2\baselineskip}
\centerline{\scshape Fröhlich soll mein Herze springen }
\vspace*{2\baselineskip}
\centerline{\scshape Ich steh an deiner Krippen hier }
\vspace*{2\baselineskip}
\centerline{\scshape Kommt und laßt uns Christum ehren}
\vspace*{2\baselineskip}
\centerline{\scshape O Jesu Christ, dein Kripplein ist }
\vspace*{2\baselineskip}
\centerline{\scshape Schaut, schaut, was ist für Wunder dar? }
\vspace*{2\baselineskip}
\centerline{\scshape Wir singen dir, Immanuel }

\newpage

\subsection*{\centerline{Alle, die ihr, Gott zu Ehren}}
\addcontentsline{toc}{subsection}{Alle die ihr Gott zu Ehren}
%StartInfo%%%%%%%%%%%%%%%%%%%%%%%%%%%%%%%%%%%%%%%%%%%%%%%%%%%%%%%%%%%%%%%%%%%%
%  Autor:
%  Titel:
%  File:
%  Ref:
%  Mod:
%EndInfo%%%%%%%%%%%%%%%%%%%%%%%%%%%%%%%%%%%%%%%%%%%%%%%%%%%%%%%%%%%%%%%%%%%%%%
%\poemtitle{Alle, die ihr, Gott zu ehren}
\begin{multicols}{2}
\settowidth{\versewidth}{Schlaf, mein Krönlein! Licht und Leben,}
\begin{verse}[\versewidth]
 
\flagverse{1.} Alle, die ihr, Gott zu ehren,\\
unsre Christlust wollt vermehren,\\
eya, eya,\\
steht und hört vor allen Dingen\\
Gottes Mutter fröhlich singen\\
bei dem Kripplein ihres Sohns:\\
Eya, eya,\\
schlaf und ruhe,\\
schlaf, schlaf, liebes Jesulein!
 
\flagverse{2.} Schlaf, du großer Weltberater,\\
Bräutgam, Sohn und selbst auch Vater,\\
eya, eya,\\
Bett und Lager, das dich träget,\\
hab ich dir zurecht geleget,\\
schlaf, du schönstes Kindelein!\\
Eya, eya,\\
schlaf und ruhe,\\
schlaf, schlaf, trautes Herzelein!
 
\flagverse{3.} Schlaf, mein Krönlein! Licht und Leben,\\
was dir lieb, will ich dir geben,\\
eya, eya,\\
schlaf, du Ausbund aller Gaben,\\
laß dich speisen, laß dich laben\\
bei der armen Krippen hier!\\
Eya, eya,\\
schlaf und ruhe,\\
schlaf, schlaf, du mein Ehr und Ruhm!
 
\flagverse{4.} Schlaf, o bestes aller Güter,\\
schlaf, o Perle der Gemüter,\\
eya, eya,\\
schlaf mein Trost, dem nichts zu gleichen,\\
Milch und Honig muß dir weichen,\\
schlaf, du edler Herzensgast!\\
Eya, eya,\\
schlaf und ruhe,\\
schlaf, schlaf, werte Lilienblum!
 
\flagverse{5.} Schlaf, o Kind, den Gott erkoren,\\
schlaf, o Schatz, den ich geboren,\\
eya, eya,\\
schlaf, du frommer Seelen Weide,\\
schlaf, du frommer Herzen Freude,\\
schlaf, du meines Leibes Frucht!\\
Eya, eya,\\
schlaf und ruhe,\\
schlaf, schlaf, allersüß'stes Lieb!
 
\flagverse{6.} Ich will dir dein Bettlein zieren,\\
ganz mit Blumen überführen,\\
eya, eya,\\
schlaf, du Lust, die wir erwählen,\\
schlaf, du Paradies der Seelen,\\
schlaf, du wahres Himmelsbrot!\\
Eya, eya,\\
schlaf und ruhe,\\
schlaf, schlaf, Heiland aller Welt!

\end{verse}
\end{multicols}
%\attrib{\small{THZE}}

\index{Alle, die ihr, Gott zu Ehren}
\newpage
\subsection*{\centerline{Fröhlich soll mein Herze springen}}
\addcontentsline{toc}{subsection}{Fröhlich soll mein Herze springen}
%StartInfo%%%%%%%%%%%%%%%%%%%%%%%%%%%%%%%%%%%%%%%%%%%%%%%%%%%%%%%%%%%%%%%%%%%%
%  Autor:
%  Titel:
%  File:
%  Ref:
%  Mod:
%EndInfo%%%%%%%%%%%%%%%%%%%%%%%%%%%%%%%%%%%%%%%%%%%%%%%%%%%%%%%%%%%%%%%%%%%%%%
%\poemtitle{Fröhlich soll mein Herze springen}
\begin{multicols}{2}
\settowidth{\versewidth}{Meine Schuld kann mich nicht drücken,}
\begin{verse}[\versewidth]
 
\flagverse{1.} Fröhlich soll mein Herze springen\\
dieser Zeit, da vor Freud\\
alle Engel singen.\\
Hört, hört, wie mit vollen Choren\\
alle Luft laute ruft:\\
Christus ist geboren.
 
\flagverse{2.} Heute geht aus seiner Kammer\\
Gottes Held, der die Welt\\
reißt aus allem Jammer.\\
Gott wird Mensch, dir Mensch zugute;\\
Gottes Kind, das verbind't\\
sich mit unserm Blute.
 
\flagverse{3.} Sollt uns Gott nun können hassen,\\
der uns gibt, was er liebt\\
über alle Maßen?\\
Gott gibt, unserm Leid zu wehren,\\
seinen Sohn aus dem Thron\\
seiner Macht und Ehren.
 
\flagverse{4.} Sollte von uns sein gekehret,\\
der sein Reich und zugleich\\
sich selbst uns verehret?\\
Sollt uns Gottes Sohn nicht lieben\\
der jetzt kömmt, von uns nimmt,\\
was uns will betrüben?
 
\flagverse{5.} Hätte für der Menschen Orden\\
unser Heil einen Greul,\\
wär er nicht Mensch worden;\\
hätt er Lust zu unserm Schaden,\\
ei, so würd unsre Bürd\\
er nicht auf sich laden.
 
\flagverse{6.} Er nimmt auf sich, was auf Erden\\
wir getan, gibt sich an,\\
unser Lamm zu werden,\\
unser Lamm, das für uns stirbet\\
und bei Gott für den Tod\\
Gnad und Fried erwirbet.

\flagverse{7.} Nun er liegt in seiner Krippen,\\
ruft zu sich mich und dich,\\
spricht mit süßen Lippen:\\
Lasset fahrn, o liebe Brüder,\\
was euch quält, was euch fehlt;\\
ich bring alles wieder.
 
\flagverse{8.} Ei, so kommt und laßt uns laufen;\\
stellt euch ein, groß und klein,\\
eilt mit großen Haufen;\\
liebt den, der vor Liebe brennet,\\
schaut den Stern, der euch gern\\
Licht und Labsal gönnet.
 
\flagverse{9.} Die ihr schwebt in großem Leiden,\\
sehet, hier ist die Tür\\
zu der wahren Freuden.\\
Faßt ihn wohl, er wird euch führen\\
an den Ort, da hinfort\\
euch kein Kreuz wird rühren.
 
\flagverse{10.} Wer sich fühlt beschwert im Herzen,\\
wer empfind't seine Sünd\\
und Gewissensschmerzen,\\
sei getrost, hier wird gefunden,\\
der in Eil machet heil\\
die vergift'ten Wunden.
 
\flagverse{11.} Die ihr arm seid und elende,\\
kommt herbei, füllet frei\\
eures Glaubens Hände!\\
Hier sind alle guten Gaben\\
und das Gold, da ihr sollt\\
euer Herz mit laben.
 
\flagverse{12.} Süßes Heil, laß dich umfangen,\\
laß mich dir, meine Zier,\\
unverrückt anhangen.\\
Du bist meines Lebens Leben;\\
nun kann ich mich durch dich\\
wohl zufrieden geben.

\flagverse{13.} Meine Schuld kann mich nicht drücken,\\
denn du hast meine Last\\
all auf deinem Rücken.\\
Kein Fleck ist an mir zu finden,\\
ich bin gar rein und klar\\
aller meiner Sünden.
 
\flagverse{14.} Ich bin rein um deinetwillen,\\
du gibst gnug Ehr und Schmuck,\\
mich darein zu hüllen.\\
Ich will dich ins Herze schließen;\\
o mein Ruhm. Edle Blum,\\
laß dich recht genießen.

\end{verse}
\end{multicols}

\begin{center}
\settowidth{\versewidth}{Ich will dich mit Fleiß bewahren,}
\begin{verse}[\versewidth]

\flagverse{15.} Ich will dich mit Fleiß bewahren,\\ % Choral WO
ich will dir leben hier,\\
dir will ich abfahren.\\
Mit dir will ich endlich schweben\\
voller Freud, ohne Zeit\\
dort im andern Leben.
  
\end{verse}
\end{center}
 

%\attrib{\small{Letzte Strophe  Choral Weihnachtsoratorium}



\index{Fröhlich soll mein Herze springen}
\newpage
\subsection*{\centerline{Ich steh an deiner Krippen hier}}
\addcontentsline{toc}{subsection}{Ich steh an deiner Krippen hier}
%StartInfo%%%%%%%%%%%%%%%%%%%%%%%%%%%%%%%%%%%%%%%%%%%%%%%%%%%%%%%%%%%%%%%%%%%%
%  Autor:
%  Titel:
%  File:
%  Ref:
%  Mod:
%EndInfo%%%%%%%%%%%%%%%%%%%%%%%%%%%%%%%%%%%%%%%%%%%%%%%%%%%%%%%%%%%%%%%%%%%%%%
%\poemtitle{pt}
\begin{multicols}{2}
\settowidth{\versewidth}{Nehmt weg das Stroh, nehmt weg das Heu,}
\begin{verse}[\versewidth]
 
\flagverse{1.} Ich steh an deiner Krippen hier,\\
o Jesulein, mein Leben;\\
ich komme, bring und schenke dir,\\
was du mir hast gegeben.\\
Nimm hin, es ist mein Geist und Sinn,\\
Herz, Seel und Mut, nimm alles hin\\
und laß dir's wohlgefallen.
 
\flagverse{2.} Du hast mit deiner Lieb erfüllt\\
mein Adern und Geblüte,\\
dein schöner Glanz, dein süßes Bild\\
liegt mir ganz im Gemüte.\\
Und wie mag es auch anders sein:\\
Wie könnt ich dich, mein Herzelein,\\
aus meinem Herzen lassen!
 
\flagverse{3.} Da ich noch nicht geboren war,\\
da bist du mir geboren\\
und hast mich dir zu eigen gar,\\
eh ich dich kannt, erkoren.\\
Eh ich durch deine Hand gemacht,\\
da hast du schon bei dir bedacht,\\
wie du mein wolltest werden.
 
\flagverse{4.} Ich lag in tiefster Todesnacht,\\
du warest meine Sonne,\\
die Sonne, die mir zugebracht\\
Licht, Leben, Freud und Wonne.\\
O Sonne, die das werte Licht\\
des Glaubens in mir zugericht't,\\
wie schön sind deine Strahlen!
 
\flagverse{5.} Ich sehe dich mit Freuden an\\
und kann mich nicht satt sehen,\\
und weil ich nun nicht weiter kann,\\
so tu ich, was geschehen.\\
O daß mein Sinn ein Abgrund wär\\
und meine Seel ein weites Meer,\\
daß ich dich möchte fassen!
 
\flagverse{6.} Vergönne mir, o Jesulein,\\
daß ich dein Mündlein küsse,\\
das Mündlein, das den süßen Wein,\\
auch Milch und Honigflüsse\\
weit übertrifft in seiner Kraft;\\
es ist voll Labsal, Stärk und Saft,\\
der Mark und Bein erquicket.
 
\flagverse{7.} Wenn oft mein Herz im Leibe weint\\
und keinen Trost kann finden,\\
da ruft mir's zu: Ich bin dein Freund,\\
ein Tilger deiner Sünden!\\
Was trauerst du, mein Brüderlein?\\
Du sollst ja guter Dinge sein,\\
ich zahle deine Schulden.
 
\flagverse{8.} Wer ist der Meister, der allhier\\
nach Würdigkeit ausstreichet\\
die Händlein, so dies Kindlein mir\\
anlachende zureichet?\\
Der Schnee ist hell, die Milch ist weiß,\\
verlieren doch beid ihren Preis,\\
wann diese Händlein blicken.
 
\flagverse{9.} Wo nehm ich Weisheit und Verstand,\\
mit Lobe zu erhöhen\\
die Äuglein, die so unverwandt\\
nach mir gerichtet stehen?\\
Der volle Mond ist schön und klar,\\
schön ist der güldnen Sterne Schar,\\
dies' Äuglein sind viel schöner.
 
\flagverse{10.} O daß doch ein so lieber Stern\\
soll in der Krippen liegen!\\
Für edle Kinder großer Herrn\\
gehören güldne Wiegen.\\
Ach, Heu und Stroh ist viel zu schlecht,\\
Samt, Seide, Purpur wären recht,\\
dies Kindlein drauf zu legen.
 
\flagverse{11.} Nehmt weg das Stroh, nehmt weg das Heu,\\
ich will mir Blumen holen,\\
daß meines Heilands Lager sei\\
auf lieblichen Violen.\\
Mit Rosen, Nelken, Rosmarin\\
aus schönen Gärten will ich ihn\\
von obenher bestreuen.
 
\flagverse{12.} Zur Seiten will ich hier und dar\\
viel weißer Lilien stecken,\\
die sollen seiner Äuglein Paar\\
im Schlafe sanft bedecken.\\
Doch liebt viel mehr das dürre Gras\\
dies Kindelein, als alles das,\\
was ich hier nenn und denke.
 
\flagverse{13.} Du fragest nicht nach Lust der Welt\\
noch nach des Leibes Freuden,\\
du hast dich bei uns eingestellt,\\
an unsrer Statt zu leiden,\\
suchst meiner Seelen Herrlichkeit,\\
durch dein selbsteignes Herzeleid,\\
das will ich dir nicht wehren.
 
\flagverse{14.} Eins aber, hoff ich, wirst du mir,\\
mein Heiland, nicht versagen:\\
Daß ich dich möge für und für\\
in, bei und an mir tragen.\\
So laß mich doch dein Kripplein sein;\\
komm, komm und lege bei mir ein\\
dich und all deine Freuden.

\end{verse}
\end{multicols}


\begin{center}
\settowidth{\versewidth}{Zwar sollt ich denken, wie gering,}
\begin{verse}[\versewidth]

\flagverse{15.} Zwar sollt ich denken, wie gering\\
ich dich bewirten werde,\\
du bist der Schöpfer aller Ding,\\
ich bin nur Staub und Erde.\\
Doch bist du so ein frommer Gast,\\
daß du noch nie verschmähest hast\\
den, der dich gerne siehet.

  
\end{verse}
\end{center}



%\attrib{\small{THZE}}

\index{Ich steh an deiner Krippen hier}
\newpage
\subsection*{\centerline{Kommt und laßt uns Christum ehren}}
\addcontentsline{toc}{subsection}{Kommt und laßt uns Christum ehren}
%StartInfo%%%%%%%%%%%%%%%%%%%%%%%%%%%%%%%%%%%%%%%%%%%%%%%%%%%%%%%%%%%%%%%%%%%%
%  Autor:
%  Titel:
%  File:
%  Ref:
%  Mod:
%EndInfo%%%%%%%%%%%%%%%%%%%%%%%%%%%%%%%%%%%%%%%%%%%%%%%%%%%%%%%%%%%%%%%%%%%%%%
%\poemtitle{pt}
\begin{multicols}{2}
\settowidth{\versewidth}{Kommt und laßt uns Christum ehren,}
\begin{verse}[\versewidth]
 
\flagverse{1.} Kommt und laßt uns Christum ehren,\\
Herz und Sinnen zu ihm kehren:\\
Singet fröhlich, laßt euch hören,\\
wertes Volk der Christenheit!
 
\flagverse{2.} Sünd und Hölle mag sich grämen,\\
Tod und Teufel mag sich schämen,\\
wir, die unser Heil annehmen,\\
werfen allen Kummer hin.
 
\flagverse{3.} Sehet, was hat Gott gegeben!\\
Seinen Sohn zum ewgen Leben.\\
Dieser kann und will uns heben\\
aus dem Leid ins Himmels Freud.
 
\flagverse{4.} Seine Seel ist uns gewogen,\\
lieb und Gunst hat ihn gezogen\\
uns, die Satanas betrogen,\\
zu besuchen aus der Höh.
 
\flagverse{5.} Jakobs Stern ist aufgegangen,\\
stillt das sehnliche Verlangen,\\
bricht den Kopf der alten Schlangen\\
und zerstört der Höllen Reich.
 
\flagverse{6.} Unser Kerker, da wir saßen\\
und mit Sorgen ohne Maßen\\
uns das Herze selbst abfraßen,\\
ist entzwei und wir sind frei.
 
\flagverse{7.} O du hochgesegnte Stunde,\\
da wir das von Herzensgrunde\\
glauben und mit unserm Munde\\
danken dir, o Jesulein!
 
\flagverse{8.} Schönstes Kindlein in dem Stalle,\\
sei uns freundlich, bring uns alle\\
dahin, da mit süßem Schalle\\
dich der Engel Heer erhöht.

\end{verse}
\end{multicols}
%\attrib{\small{THZE}}

\index{Kommt und laßt uns Christum ehren}
\newpage
\subsection*{\centerline{O Jesu Christ, dein Kripplein ist}}
\addcontentsline{toc}{subsection}{O Jesu Christ dein Kripplein ist}
%StartInfo%%%%%%%%%%%%%%%%%%%%%%%%%%%%%%%%%%%%%%%%%%%%%%%%%%%%%%%%%%%%%%%%%%%%
%  Autor:
%  Titel:
%  File:
%  Ref:
%  Mod: Korrektur 31.10.2017
%EndInfo%%%%%%%%%%%%%%%%%%%%%%%%%%%%%%%%%%%%%%%%%%%%%%%%%%%%%%%%%%%%%%%%%%%%%%
%\poemtitle{pt}
\begin{multicols}{2}
\settowidth{\versewidth}{ Schweig arger Feind! Da sitzt mein Freund,}
\begin{verse}[\versewidth]
%o Jesu Christ, dein Kripplein ist mein Paradies

\flagverse{1.} O Jesu Christ, dein Kripplein ist\\
mein Paradies,\\
da meine Seele weidet!\\
Hier ist der Ort,\\
hier liegt das Wort,\\
mit unserm Fleisch\\
persönlich angekleidet.

\flagverse{2.} Dem Meer und Wind gehorsam sind,\\
gibt's sich zum Dienst\\
und wird ein Knecht der Sünder.\\
Du, Gottes Sohn,\\
wirst Erd und Ton,\\
gering und schwach\\
wie wir und unsre Kinder.

\flagverse{3.} Du, höchstes Gut, hebst unser Blut\\
in deinen Thron\\
hoch über alle Höhen.\\
Du, ewge Kraft,\\
machst Brüderschaft\\
mit uns, die wie ein Dampf\\
und Rauch vergehen.

\flagverse{4.} Was will uns nun zuwider tun\\
der Seelenfeind\\
mit allem Gift und Gallen?\\
Was wirft er mir\\
und andern für,\\
daß Adam ist,\\
und wir mit ihm, gefallen?

\flagverse{5.} Schweig arger Feind! Da sitzt mein Freund,\\
mein Fleisch Blut,\\
hoch in dem Himmel droben;\\
was du gefällt,\\
das hat der Held\\
aus Jakobs Stamm\\
zu großer Ehr erhoben.

\flagverse{6.} Sein Licht und Heil macht alles heil;\\
der Himmelsschatz\\
bringt allen Schaden wieder.\\
Der Freudenquell\\
Immanuel\\
schlägt Teufel,\\
Höll und all ihr Reich darnieder.

\flagverse{7.} Drum frommer Christ, wer du auch bist,\\
sei gutes Muts\\
und laß dich nicht betrüben;\\
weil Gottes Kind\\
dich ihm verbind't,\\
so kanns nicht anders sein,\\
Gott muß dich lieben.

\flagverse{8.} Gedenke doch, wie herrlich hoch\\
er über alle Jammer\\
dich geführet!\\
Der Engel Heer\\
ist selbst nicht mehr\\
als eben du\\
mit Seligkeit gezieret.

\flagverse{9.} Du siehest ja vor Augen da\\
dein Fleisch und Blut\\
die Luft und Wolken lenken;\\
was will doch sich\\
– ich frage dich –\\
erheben, dich in Angst\\
und Furcht zu senken?

\flagverse{10.} Dein blöder Sinn geht oft dahin,\\
ruft Ach und Weh,\\
läßt allen Trost verschwinden.\\
Komm her und richt\\
dein Angesicht\\
zum Kripplein Christi,\\
da, da wirst du's finden.

\flagverse{11.} Wirst du geplagt? Ei, unverzagt!\\
Dein Bruder wird\\
dein Unglück nicht verschmähen;\\
sein Herz ist weich\\
und gnadenreich,\\
kann unser Leid\\
nicht ohne Tränen sehen.

\flagverse{12.} Tritt zu ihm zu! Such Hilf und Ruh!\\
Er wird's so machen,\\
daß du ihm wirst danken.\\
Er weiß und kennt\\
was beißt und brennt,\\
versteht wohl,\\
wie zu Mute sei dem Kranken.

\flagverse{13.} Denn eben drum hat er den Grimm\\
des Kreuzes\\
auch am Leibe wollen tragen,\\
daß seine Pein\\
ihm möge sein\\
ein unverrückt Erinnrung\\
unsrer Plagen.

\flagverse{14.} Mit einem Wort: Er ist die Pfort\\
zu dieses und des andern Lebens\\
Freuden;\\
er macht behend\\
ein seligs End\\
an alle dem,\\
was fromme Herzen leiden.

\end{verse}
\end{multicols}

\begin{center}
\settowidth{\versewidth}{Laß aller Welt ihr Gut und Geld}
\begin{verse}[\versewidth]

\flagverse{15.} Laß aller Welt ihr Gut und Geld\\
und siehe nur,\\
daß dieser Schatz dir bleibe!\\
Wer den hier fest hält\\
und nicht läßt,\\
den ehrt und krönt er\\
dort an Seel und Leibe.

\end{verse}
\end{center}



%\attrib{\small{THZE}}

\index{O Jesu Christ, dein Kripplein ist}
\newpage
\subsection*{\centerline{Schaut, schaut, was ist für Wunder dar?}}
\addcontentsline{toc}{subsection}{Schaut schaut was ist für Wunder dar?}
%StartInfo%%%%%%%%%%%%%%%%%%%%%%%%%%%%%%%%%%%%%%%%%%%%%%%%%%%%%%%%%%%%%%%%%%%%
%  Autor:
%  Titel:
%  File:
%  Ref:
%  Mod:
%EndInfo%%%%%%%%%%%%%%%%%%%%%%%%%%%%%%%%%%%%%%%%%%%%%%%%%%%%%%%%%%%%%%%%%%%%%%
%\poemtitle{pt}
\begin{multicols}{2}
\settowidth{\versewidth}{Schaut, schaut, was ist für Wunder dar?}
\begin{verse}[\versewidth]
 
\flagverse{1.} Schaut, schaut, was ist für Wunder dar?\\
Die schwarze Nacht wird hell und klar,\\
ein großes Licht bricht dort herein,\\
ihm weichet aller Sterne Schein.
 
\flagverse{2.} Es ist ein rechtes Wunderlicht\\
und gar die alte Sonne nicht,\\
weils, wider die Natur, die Nacht\\
zu einem hellen Tage macht.
 
\flagverse{3.} Was wird hierdurch uns zeigen an\\
der die Natur so ändern kann?\\
Es muß ein großes Werk geschehn,\\
wie wir aus solchem Zeichen sehn.
 
\flagverse{4.} Sollt auch erscheinen dieser Zeit\\
die Sonne der Gerechtigkeit,\\
der helle Stern aus Jakobs Stamm,\\
der Heiden Licht, des Weibes Sam?
 
\flagverse{5.} Es ist also. Des Himmels Heer,\\
das bringt uns jetzt die Freudenmär,\\
wie sich nunmehr hab eingestellt\\
zu Bethlehem das Heil der Welt.
 
\flagverse{6.} O Gütigkeit! Was lange Jahr\\
ihm hat der frommen Väter Schar\\
gewünscht und sehnlich oft begehrt,\\
des werden wir von Gott gewährt.
 
\flagverse{7.} Drum auf, ihr Menschenkinder, auf!\\
Auf, auf, und nehmet euren Lauf\\
mit mir hin zu der Stell und Ort,\\
davon gemeld't der Engel Wort.
 
\flagverse{8.} Schaut hin, dort liegt im finstern Stall,\\
des Herrschaft gehet überall!\\
Da Speise vormals sucht ein Rind,\\
da ruht jetzt der Jungfrauen Kind.
 
\flagverse{9.} O Menschenkind, betracht es recht\\
und strauchle nicht, dieweil so schlecht,\\
so elend scheint dies Kindelein;\\
es ist und soll auch uns groß sein.
 
\flagverse{10.} Es wird im Fleisch hier vorgestellt,\\
der alles schuf und noch erhält.\\
Das Wort, so bald im Anfang war\\
bei Gott, selbst Gott, das lieget dar.
 
\flagverse{11.} Es ist der eingeborne Sohn\\
des Vaters, unser Gnadenthron,\\
das A und O, der große Gott,\\
der Siegsfürst, der Herr Zebaoth.
 
\flagverse{12.} Denn weil die Zeit nunmehr erfüllt,\\
da Gottes Zorn muß sein gestillt,\\
wird sein Sohn Mensch, trägt unsre Schuld,\\
wirbt uns durch sein Blut Gottes Huld.
 
\flagverse{13.} Dies ist die rechte Freudenzeit.\\
Weg Trauern, weg, weg alles Leid!\\
Trotz dem, der ferner uns verhöhnt!\\
Gott selbst ist Mensch. Wir sind versöhnt.
 
\flagverse{14.} Der Sünden Büßer ist nun hier,\\
den Schlangentreter haben wir,\\
der Höllen Pest, des Todes Gift,\\
des Lebens Fürsten man hier trifft.
 
\flagverse{15.} Es hat mit uns nun keine Not,\\
weil Sünde, Teufel, Höll und Tod\\
zu Spott und Schanden sind gemacht\\
in dieser großen Wundernacht.
 
\flagverse{16.} O selig, selig alle Welt,\\
die sich an dieses Kindlein hält!\\
Wohl dem, der dieses recht erkennt\\
und gläubig seinen Heiland nennt!
 
\flagverse{17.} Es danke Gott, wer danken kann,\\
der unser sich so hoch nimmt an\\
und sendet aus des Himmels Thron\\
uns, seinen Feinden, seinen Sohn.
 
\flagverse{18.} Drum stimmt an mit der Engel Heer:\\
Gott in der Höhe sei nun Ehr!\\
Auf Erden Frieden jederzeit!\\
Den Menschen Wonn und Fröhlichkeit!
   
\end{verse}
\end{multicols}
%\attrib{\small{THZE}}

\index{Schaut, was für ein Wunder}
\newpage
\subsection*{\centerline{Wir singen dir, Immanuel}}
\addcontentsline{toc}{subsection}{Wir singen dir Immanuel}
%StartInfo%%%%%%%%%%%%%%%%%%%%%%%%%%%%%%%%%%%%%%%%%%%%%%%%%%%%%%%%%%%%%%%%%%%%
%  Desc:  Wir singen dir, Immanuel - Paul Gerhard
%  Desc:  Input 
%  Desc:  Zweispaltiger Satz mit \verse
%  Tags:  GERHARD VERSE INCLUDE
%  File:  pg-wir-singen-dir.tex
%  Autor: dv
%  Ref:   
%  Mod:   16.11.2016/dv/initial
%  Mod:   
%EndInfo%%%%%%%%%%%%%%%%%%%%%%%%%%%%%%%%%%%%%%%%%%%%%%%%%%%%%%%%%%%%%%%%%%%%%%

%\poemtitle{Wir singen dir, Immanuel}
\begin{multicols}{2}
\settowidth{\versewidth}{Hast du doch selbst dich schwach gemacht,}                                                  
\begin{verse}[\versewidth]
  
  \flagverse{1.} Wir singen dir, Immanuel,\\
  du Lebensfürst und Gnadenquell,\\
  du Himmelsblum und Morgenstern,\\
  du Jungfraunsohn, Herr aller Herrn!\\
  Halleluja!

  \flagverse{2.} Wir singen dir in deinem Heer\\
  aus aller Kraft Lob, Preis und Ehr,\\
  daß du, o lang gewünschter Gast,\\
  dich nunmehr eingestellet hast.\\
  Halleluja!

  \flagverse{3.} Vom Anfang, da die Welt gemacht,\\
  hat so manch Herz nach dir gewacht;\\
  dich hat gehofft so lange Jahr\\
  der Väter und Propheten Schar.\\
  Halleluja!

  \flagverse{4.} Vor andern hat dein hoch begehrt\\
  der Hirt und König deiner Herd,\\
  der Mann, der dir so wohl gefiel,\\
  wann er dir sang auf Saitenspiel.\\
  Halleluja!

  \flagverse{5.} Ach, daß der Herr aus Zion käm\\
  und unsre Bande von uns nähm!\\
  Ach, Daß die Hilfe bräch herein,\\
  so würde Jakob fröhlich sein.\\
  Halleluja!

  \flagverse{6.} Nun du bist hier, da liegest du,\\
  hältst in dem Kripplein deine Ruh;\\
  bist klein und machst doch alles groß,\\
  bekleidst die Welt und kommst doch bloß.\\
  Halleluja!
  
  \flagverse{7.} Du kehrst in fremder Hausung ein,\\
  und sind doch alle Himmel dein;\\
  trinkst Milch aus deiner Mutter Brust\\
  und bist doch selbst der Engel Lust.\\
  Halleluja!

  \flagverse{8.} Du hast dem Meer sein Ziel gesteckt\\
  und wirst mit Windeln zugedeckt;\\
  bist Gott und liegst auf Heu und Stroh,\\
  wirst Mensch und bist doch A und O.\\
  Halleluja!

  \flagverse{9.} Du bist der Ursprung aller Freud\\
  und duldest so viel Herzeleid;\\
  bist aller Heiden Trost und Licht,\\
  suchst selber Trost und findst ihn nicht.\\
  Halleluja!

  \flagverse{10.} Du bist der süße Menschenfreund,\\
  doch sind dir so viel Menschen feind;\\
  Herodis Heer hält dich für Greul\\
  und bist doch nichts als lauter Heil.\\
  Halleluja!

  \flagverse{11.} Ich aber, dein geringster Knecht,\\
  ich sag es frei und mein es recht:\\
  Ich liebe dich, doch nicht so viel,\\
  als ich dich gerne lieben will.\\
  Halleluja!

  \flagverse{12.} Der Will ist da, die Kraft ist klein;\\
  doch wird dir nicht zuwider sein\\
  mein armes Herz, und was es kann,\\
  wirst du in Gnaden nehmen an.\\
  Halleluja!

  \flagverse{13.} Hast du doch selbst dich schwach gemacht,\\
  erwähltest, was die Welt veracht't;\\
  warst arm und dürftig, nahmst vorlieb\\
  da, wo der Mangel dich hintrieb.\\
  Halleluja!

  \flagverse{14.} Du schliefst ja auf der Erden Schoß;\\
  so war das Kripplein auch nicht groß;\\
  der Stall, das Heu, das dich umfing,\\
  war alles schlecht und sehr gering.\\
  Halleluja!

  \flagverse{15.} Darum so hab ich guten Mut:\\
  Du Wirst auch halten mich für gut.\\
  O Jesulein, Dein frommer Sinn\\
  macht, daß ich so voll Trostes bin.\\
  Halleluja!

  \flagverse{16.}Bin ich gleich sünd- und lastervoll,\\
  hab ich gelebt nicht, wie ich soll,\\
  ei, kommst du doch deswegen her,\\
  daß sich der Sünder zu dir kehr.\\
  Halleluja!

  \flagverse{17.} Hätt ich nicht auf mir Sündenschuld,\\
  hätt ich kein Teil an deiner Huld;\\
  vergeblich wärst du mir geborn,\\
  wenn ich nicht wär in Gottes Zorn.\\
  Halleluja!

  \flagverse{18.} So faß ich dich nun ohne Scheu,\\
  du machst mich alles Jammers frei;\\
  du trägst den Zorn, du würgst den Tod,\\
  verkehrst in Freud all Angst und Not.\\
  Halleluja!

  \flagverse{19.} Du bist mein Haupt, hinwiederum\\
  bin ich dein Glied und Eigentum\\
  und will, so viel dein Geist mir gibt,\\
  stets dienen dir, wie dir's beliebt.\\
  Halleluja!

  \flagverse{20.} Ich will dein Halleluja hier\\
  mit Freuden singen für und für\\
  und dort in deinem Ehrensaal\\
  solls schallen ohne Zeit und Zahl.\\
  Halleluja!

\end{verse}
\end{multicols}
%\attrib{\small{1653}}

\index{Wir singen dir, Immanuel}
\newpage

%\newpage

\section*{\centerline{\LARGE NEUJAHR}}
\addcontentsline{toc}{section}{NEUJAHR}
\rule{\textwidth}{0.2pt}\vspace*{-\baselineskip}\vspace{3.2pt}
\rule{\textwidth}{1.2pt}\\[\baselineskip]

\index{ Gedichte zu Neujahr}

\centerline{\scshape Nun laßt uns gehn und treten }
\vspace*{2\baselineskip}
\centerline{\scshape Warum machet solche Schmerzen }

\newpage
  
\subsection*{\centerline{Nun laßt uns gehn und treten}}
\addcontentsline{toc}{subsection}{Nun laßt uns gehn und treten}
%StartInfo%%%%%%%%%%%%%%%%%%%%%%%%%%%%%%%%%%%%%%%%%%%%%%%%%%%%%%%%%%%%%%%%%%%%
%  Autor:
%  Titel:
%  File:
%  Ref:
%  Mod:
%EndInfo%%%%%%%%%%%%%%%%%%%%%%%%%%%%%%%%%%%%%%%%%%%%%%%%%%%%%%%%%%%%%%%%%%%%%%
%\poemtitle{pt}
\begin{multicols}{2}
\settowidth{\versewidth}{Durch soviel Angst und Plagen,}
\begin{verse}[\versewidth]
 
\flagverse{1.} Nun laß uns gehn und treten\\
mit Singen und mit Beten\\
zum Herrn, der unserm Leben\\
bis hierher Kraft gegeben.
 
\flagverse{2.} Wir gehn dahin und wandern\\
von einem Jahr zum andern,\\
wir leben und gedeihen\\
vom Alten bis zum Neuen;
 
\flagverse{3.} Durch soviel Angst und Plagen,\\
durch Zittern und durch Zagen,\\
durch Krieg und große Schrecken,\\
die alle Welt bedecken.
 
\flagverse{4.} Denn wie von treuen Müttern\\
in schweren Ungewittern\\
die Kindlein hier auf Erden\\
mit Fleiß bewahret werden:
 
\flagverse{5.} Also auch nichts minder\\
läßt Gott ihm seine Kinder,\\
wenn Not und Trübsal blitzen,\\
in seinem Schoße sitzen.
 
\flagverse{6.} Ach Hüter unsers Lebens,\\
fürwahr, es ist vergebens\\
mit unserm Tun und Machen,\\
wo nicht dein Augen wachen.
 
\flagverse{7.} Gelobt sei deine Treue,\\
die alle Morgen neue,\\
Lob sei den starken Händen,\\
die alles Herzleid wenden.
 
\flagverse{8.} Laß ferner dich erbitten,\\
o Vater, und bleib mitten\\
in unserm Kreuz und Leiden\\
ein Brunnen unsrer Freuden.
 
\flagverse{9.} Gib mir und allen denen,\\
die sich von Herzen sehnen\\
nach dir und deiner Hulde,\\
ein Herz, das sich gedulde.
 
\flagverse{10.} Schleuß zu die Jammerpforten\\
und laß an allen Orten\\
auf so viel Blutvergießen\\
die Freudenströme fließen.
 
\flagverse{11.} Sprich deinen milden Segen\\
zu allen unsern Wegen,\\
laß Großen und auch Kleinen\\
die Gnadensonne scheinen.
 
\flagverse{12.} Sei der Verlaßnen Vater,\\
der Irrenden Berater,\\
der Unversorgten Gabe,\\
der Armen Gut und Habe.
 
\flagverse{13.} Hilf gnädig allen Kranken,\\
gib fröhliche Gedanken\\
den hochbetrübten Seelen,\\
die sich mit Schwermut quälen.
 
\flagverse{14.} Und endlich, was das Meiste,\\
füll uns mit deinem Geiste,\\
der uns hier herrlich ziere\\
und dort zum Himmel führe.

\end{verse}
\end{multicols}
%\attrib{\small{THZE}}

\begin{center}
\settowidth{\versewidth}{Der, vor dem die Welt erschrickt,}
\begin{verse}[\versewidth]

\flagverse{15.} Das alles wollst du geben,\\
o meines Lebens Leben,\\
mir und der Christen Schare\\
zum selgen neuen Jahre.
  
\end{verse}
\end{center}

\index{Nun laßt uns gehn und treten}
\newpage
\subsection*{\centerline{Warum machet solche Schmerzen}}
\addcontentsline{toc}{subsection}{Warum machet solche Schmerzen}
%StartInfo%%%%%%%%%%%%%%%%%%%%%%%%%%%%%%%%%%%%%%%%%%%%%%%%%%%%%%%%%%%%%%%%%%%%
%  Autor:
%  Titel:
%  File:
%  Ref:
%  Mod:
%EndInfo%%%%%%%%%%%%%%%%%%%%%%%%%%%%%%%%%%%%%%%%%%%%%%%%%%%%%%%%%%%%%%%%%%%%%%
%\poemtitle{pt}
\begin{multicols}{2}
\settowidth{\versewidth}{Für dich darfst du dies nicht dulden,}
\begin{verse}[\versewidth]
 
\flagverse{1.} Warum machet solche Schmerzen,\\
warum machet solche Pein,\\
der von unbeschnittnem Herzen,\\
dir, herzliebstes Jesulein,\\
mit Beschneidung, da du doch\\
frei von des Gesetzes Joch.\\
Weil du, einem Menschenkinde\\
zwar gleich, doch ganz ohne Sünde?
 
\flagverse{2.} Für dich darfst du dies nicht dulden,\\
du bist ja des Bundes Herr,\\
unsre, unsre große Schulden,\\
die so grausam, die so schwer\\
auf uns liegen, daß es dich\\
jammert herz- und inniglich,\\
die trägst du ab, uns zu retten,\\
die sonst nichts zu zahlen hätten.
 
\flagverse{3.} Freut, ihr Schuldner, euch deswegen,\\
ja, sei fröhlich alle Welt,\\
weil heut anhebt zu erlegen\\
gottes Sohn das Lösegeld;\\
das Gesetz wird heut erfüllt,\\
heut wird Gottes Zorn gestillt.\\
Heut macht uns, so sollten sterben,\\
Gottes Blut zu Gottes Erben.
 
\flagverse{4.} Wer mag recht die Gnad erkennen?\\
Wer kann dafür dankbar sein?\\
Herz und Mund soll stets dich nennen\\
unsern Heiland, Jesulein!\\
Deine Güte wollen wir\\
nach Vermögen preisen hier,\\
weil wir in der Schwachheit wallen;\\
dort soll baß dein Lob erschallen.

\end{verse}
\end{multicols}
%\attrib{\small{THZE}}

\index{Warum machet solche Schmerzen}
\newpage

\newpage

\section*{\centerline{\LARGE PASSION}}
\addcontentsline{toc}{section}{PASSION}
\rule{\textwidth}{0.2pt}\vspace*{-\baselineskip}\vspace{3.2pt}
\rule{\textwidth}{1.2pt}\\[\baselineskip]

\index{ Gedichte zur Passion}

\centerline{\scshape Ein Lämmlein geht und trägt die Schuld }
\vspace*{2\baselineskip}
\centerline{\scshape Gegrüßet seist du, Gott mein Heil }
\vspace*{2\baselineskip}
\centerline{\scshape Gegrüßet seist du, meine Kron }
\vspace*{2\baselineskip}
\centerline{\scshape Hör an, mein Herz, die sieben Wort }
\vspace*{2\baselineskip}
\centerline{\scshape Ich grüße dich, du frömmster Mann }
\vspace*{2\baselineskip}
\centerline{\scshape O Haupt voll Blut und Wunden }
\vspace*{2\baselineskip}
\centerline{\scshape O Herz des Königs aller Welt }
\vspace*{2\baselineskip}
\centerline{\scshape O Mensch, beweine deine Sünd }
\vspace*{2\baselineskip}
\centerline{\scshape O Welt, sieh hier dein Leben }
\vspace*{2\baselineskip}
\centerline{\scshape Sei mir tausendmal gegrüßet }
\vspace*{2\baselineskip}
\centerline{\scshape Sei wohl gegrüßet, guter Hirt}
\vspace*{2\baselineskip}
\centerline{\scshape Siehe, mein getreuer Knecht }
\vspace*{2\baselineskip}
\centerline{\scshape Als Gottes Lamm und Leue }

\newpage

\subsection*{\centerline{Ein Lämmlein geht und trägt die Schuld}}
\addcontentsline{toc}{subsection}{Ein Lämmlein geht und trägt die Schuld}
%StartInfo%%%%%%%%%%%%%%%%%%%%%%%%%%%%%%%%%%%%%%%%%%%%%%%%%%%%%%%%%%%%%%%%%%%%
%  Autor:
%  Titel:
%  File:
%  Ref:
%  Mod:
%EndInfo%%%%%%%%%%%%%%%%%%%%%%%%%%%%%%%%%%%%%%%%%%%%%%%%%%%%%%%%%%%%%%%%%%%%%%
%\poemtitle{Ein Lämmlein geht und trägt die Schuld}
\begin{multicols}{2}
\settowidth{\versewidth}{Ein Lämmlein geht und trägt die Schuld}
\begin{verse}[\versewidth]
 
\flagverse{1.} Ein Lämmlein geht und trägt die Schuld\\
der Welt und ihrer Kinder;\\
es geht und büßet in Geduld\\
die Sünden aller Sünder.\\
Es geht dahin, wird matt und krank,\\
ergibt sich auf die Würgebank,\\
verzeiht sich allen Freuden;\\
es nimmet an Schmach, Hohn und Spott,\\
Angst, Wunden, Striemen, Kreuz und Tod\\
und spricht: Ich wills gern leiden.
 
\flagverse{2.} Das Lämmlein ist der große Freund\\
und Heiland meiner Seelen;\\
den, den hat Gott zum Sündenfeind\\
und Sühner wollen wählen.\\
\flqq Geh hin, mein Kind, und nimm dich an\\
der Kinder, die ich ausgetan\\
zur Straf und Zornesruten;\\
die Straf ist schwer, der Zorn ist groß;\\
du kannst und sollst sie machen los\\
durch Sterben und durch Bluten.\frqq
 
\flagverse{3.} \flqq Ja, Vater, ja von Herzensgrund,\\
leg auf, ich will dirs tragen.\\
Mein Wollen hängt an deinem Mund;\\
mein Wirken ist dein Sagen.\frqq\\
O Wunder Lieb, o Liebesmacht,\\%                        O Wunder Lieb ... ?
du kannst, was nie kein Mensch gedacht,\\
Gott seinen Sohn abzwingen.\\
O Liebe, Liebe, Du bist stark,\\
du strecktest den ins Grab und Sarg,\\
vor dem die Felsen springen.
%                                       Die Liebe ist stärker als Gott?
 
\flagverse{4.} Du marterst ihn am Kreuzesstamm\\
mit Nägeln und mit Spießen;\\
du schlachtest ihn als wie ein Lamm,\\
machst Herz und Adern fließen:\\
Das Herze mit der Seufzer Kraft,\\
die Adern mit dem edlen Saft\\
des purpurroten Blutes.\\
O süßes Lamm, was soll ich dir\\
erweisen dafür, daß du mir\\
erweisest so viel Gutes?
 
\flagverse{5.} Mein Lebetage will ich dich\\
aus meinem Sinn nicht lassen;\\
dich will ich stets, gleich wie du mich,\\
mit Liebesarmen fassen.\\
Du sollst sein meines Herzens Licht,\\
und wenn mein Herz in Stücken bricht,\\
sollst du mein Herze bleiben.\\
Ich will mich dir, mein höchster Ruhm,\\
hiermit zu deinem Eigentum\\
beständiglich verschreiben.
 
\flagverse{6.} Ich will von deiner Lieblichkeit\\
bei Nacht und Tage singen,\\
mich selbst auch dir nach Möglichkeit\\
zum Freudenopfer bringen.\\
Mein Bach des Lebens soll sich dir\\
und deinem Namen für und für\\
in Dankbarkeit ergießen;\\
und was du mir zu gut getan,\\
das will ich stets, so tief ich kann,\\
in mein Gedächtnis schließen.
 
\flagverse{7.} Erweitre dich, mein Herzensschrein,\\
du sollst ein Schatzhaus werden\\
der Schätze, die viel größer sein\\
als Himmel, Meer und Erden.\\
Weg mit dem Gold Arabia!\\
Weg Kalmus, Myrrhen, Kassia!\\
Ich hab ein Bessers funden:\\
Mein großer Schatz, Herr Jesu Christ,\\
ist dieses, was geflossen ist\\
aus deines Leibes Wunden.
 
\flagverse{8.} Das soll und will ich mir zu nutz\\
zu allen Zeiten machen;\\
im Streite soll es sein mein Schutz,\\
in Traurigkeit mein Lachen,\\
in Fröhlichkeit mein Saitenspiel,\\
und wenn mir nichts mehr schmecken will,\\
soll mich dies Manna speisen.\\
Im Durst solls sein mein Wasserquell,\\
in Einsamkeit mein Sprachgesell\\
zu Haus und auch auf Reisen.
 
\flagverse{9.} Was schadet mir des Todes Gift?\\
Dein Blut, Das ist mein Leben.\\
Wenn mich der Sonnen Hitze trifft,\\
so kann mirs Schatten geben.\\
Setzt mir der Wehmut Schmerzen zu,\\
so find ich bei dir meine Ruh\\
als auf dem Bett ein Kranker.\\
Und wenn des Kreuzes Ungestüm\\
mein Schifflein treibet üm und üm,\\
so bist du dann mein Anker.
 
\flagverse{10}. Wenn endlich ich soll treten ein\\
in deines Reiches Freuden,\\
so soll dies Blut mein Purpur sein,\\
ich will mich darin kleiden;\\
es soll sein meines Hauptes Kron,\\
in welcher ich will vor dem Thron\\
des höchsten Vaters gehen\\
und dir, dem er mich anvertraut,\\
als eine wohlgeschmückte Braut\\
an deiner Seite stehen.

\end{verse}
\end{multicols}
%\attrib{\small{THZE}}

\index{Ein Lämmlein geht}
\newpage
\subsection*{\centerline{Gegrüßet seist du, Gott mein Heil}}
\addcontentsline{toc}{subsection}{Gegrüßet seist du Gott mein Heil}
%StartInfo%%%%%%%%%%%%%%%%%%%%%%%%%%%%%%%%%%%%%%%%%%%%%%%%%%%%%%%%%%%%%%%%%%%%
%  Autor:
%  Titel:
%  File:
%  Ref:
%  Mod:
%EndInfo%%%%%%%%%%%%%%%%%%%%%%%%%%%%%%%%%%%%%%%%%%%%%%%%%%%%%%%%%%%%%%%%%%%%%%
%\poemtitle{Gegrüßet seist du, Gott mein Heil}
\begin{multicols}{2}
\settowidth{\versewidth}{Mach, Herr, durch deines Herzens Quell}
\begin{verse}[\versewidth]

%{5.} An die Brust (Salve, salus mea, Deus) Gegrüßet seist du, Gott mein Heil

\flagverse{1.} Gegrüßet seist du, Gott mein Heil,\\
mein Auge, Lieb und schönstes Teil,\\
gegrüßet seist du, werte Brust,\\
du Gottessohn, du Menschenlust,\\
du Träger aller Bürd und Last,\\
du aller Müden Ruh und Rast.

\flagverse{2.} Mein Jesu, neige dich zu mir\\
mit deiner Brust, damit von dir\\
mein Herz in deiner Lieb entbrenn\\
und von der ganzen Welt sich trenn.\\
Halt Herz Und Brust in Andacht reich\\
und mich ganz deinem Willen gleich.

\flagverse{3.} Mach, Herr, durch deines Herzens Quell\\
mein Herz von Unflat rein und hell!\\
Der Du bist Gottes Glanz und Bild\\
und aller Armen Trost und Schild,\\
teil aus den Schätzen deiner Gnad\\
auch mir mit Gnade, Rat und Tat.

\flagverse{4.} O süße Brust, tu mir die Gunst\\
und fülle mich mit deiner Brunst!\\
Du Bist der Weisheit tiefer Grund,\\
dich lobt und singt der Engel Mund,\\
aus dir entspringt die edle Frucht,\\
die dein Johannes bei dir sucht.

\end{verse}
\end{multicols}

\begin{center}
\settowidth{\versewidth}{Der, vor dem die Welt erschrickt,}
\begin{verse}[\versewidth]
  
\flagverse{5.} In dir wohnt alle Gottesfüll,\\
hast alles, was ich wünsch und will,\\
du bist das rechte Gotteshaus,\\
drum, wann zur Welt ich muß hinaus,\\
so schließ mich treulich in dir ein.
  
\end{verse}
\end{center}

%\attrib{\small{THZE}}

\index{Gegrüßet seist du, Gott mein Heil}
\newpage
\subsection*{\centerline{Gegrüßet seist du, meine Kron}}
\addcontentsline{toc}{subsection}{Gegrüßet seist du meine Kron}
%StartInfo%%%%%%%%%%%%%%%%%%%%%%%%%%%%%%%%%%%%%%%%%%%%%%%%%%%%%%%%%%%%%%%%%%%%
%  Autor:
%  Titel:
%  File:
%  Ref:
%  Mod:
%EndInfo%%%%%%%%%%%%%%%%%%%%%%%%%%%%%%%%%%%%%%%%%%%%%%%%%%%%%%%%%%%%%%%%%%%%%%
%\poemtitle{Gegrüßet seist du, meine Kron}
\begin{multicols}{2}
\settowidth{\versewidth}{Was soll ich dir doch immermehr,}
\begin{verse}[\versewidth]

%{2.} An die Knie (Salve Jesu, rex sanctorum) Gegrüßet seist du, meine Kron

\flagverse{1.} Gegrüßet seist du, meine Kron\\
und König aller Frommen,\\
der du zum Trost von deinem Thron\\
uns armen Sündern kommen!\\
O wahrer Mensch, o wahrer Gott,\\
o Helfer, voller Hohn und Spott,\\
den du doch nicht verschuldest!\\
Ach, wie so arm, wie nackt und bloß\\
hängst du am Kreuz,\\
wie schwer und groß\\
ist dein Schmerz, den du duldest.

\flagverse{2.} Es fleußet deines Blutes Bach\\
mit ganzem vollem Haufen,\\
dein Leib ist auch mit Ungemach\\
ganz durch und durch belaufen.\\
O ungeschränkte Majestät,\\
wie kommts, daß dirs so kläglich geht?\\
Das macht dein Huld und Treue.\\
Wer dankt dir des? Wo ist der Mann,\\
der sich, wie du für uns getan,\\
für dich zu sterben freue?

\flagverse{3.} Was soll ich dir doch immermehr,\\
o Liebster, dafür geben,\\
daß dein Herz sich so hoch und sehr\\
bemüht hat um mein Leben?\\
Du rettest mich durch deinen Tod\\
von mehr als eines Todes Not\\
und machst mich sicher wohnen.\\
Laß HöLl und Teufel böse sein,\\
was schad'ts? Sie müssen dennoch mein\\
und meiner Seele schonen.

\flagverse{4.} Vor großer Lieb und heilger Lust,\\
damit du mich erfüllet,\\
so wird mein Leid gestillet,\\
drück ich dich an mein Herz und Brust,\\
das deinen Augen wohlbekannt.\\
Und das ist dir ja keine Schand,\\
ein krankes Herz zu laben.\\
Ach bleib mir hold und gutes Muts,\\
bis mich die Ströme deines Bluts\\
ganz rein gewaschen haben.

\end{verse}
\end{multicols}

\begin{center}
\settowidth{\versewidth}{Sei du mein Schatz und höchste Freud,}
\begin{verse}[\versewidth]




\flagverse{5.} Sei du mein Schatz und höchste Freud,\\
ich will dein Diener bleiben,\\
und deines Kreuzes Herzeleid\\
will ich in mein Herz schreiben.\\
Verleihe Du nur Kraft und Macht,\\
damit, was ich bei mir bedacht,\\
ich mög ins Werk auch setzen;\\
so wirst du, Schönster, meinen Sinn\\
und alles, was ich hab und bin,\\
ohn Unterlaß ergötzen.

  
\end{verse}
\end{center}

%\attrib{\small{THZE}}

\index{Gegrüßet seist du, meine Kron}
\newpage
\subsection*{\centerline{Hör an, mein Herz, die sieben Wort}}
\addcontentsline{toc}{subsection}{Hör an mein Herz die sieben Wort}
%StartInfo%%%%%%%%%%%%%%%%%%%%%%%%%%%%%%%%%%%%%%%%%%%%%%%%%%%%%%%%%%%%%%%%%%%%
%  Autor:
%  Titel:
%  File:
%  Ref:
%  Mod:
%EndInfo%%%%%%%%%%%%%%%%%%%%%%%%%%%%%%%%%%%%%%%%%%%%%%%%%%%%%%%%%%%%%%%%%%%%%%
%\poemtitle{Hör an, mein Herz, die sieben Wort,}
\begin{multicols}{2}
\settowidth{\versewidth}{Hör an, mein Herz, die sieben Wort}
\begin{verse}[\versewidth]
%die sieben Worte Christi am Kreuz\\
%hör an, mein Herz, die sieben Wort

\flagverse{1.} Hör an, mein Herz, die sieben Wort,\\
die Jesus ausgesprochen,\\
da ihm durch Qual und blutgen Mord\\
sein Herz am Kreuz gebrochen.\\
Tu auf den Schrein und schleuß sie ein\\
als edle hohe Gaben,\\
so wirst du Freud in schwerem Leid\\
und Trost im Kreuze haben.

\flagverse{2.} Sein allererste Sorge war,\\
zu schützen, die ihn hassen,\\
bat, daß sein Gott der bösen Schar\\
wollt ihre Sünd erlassen.\\
Vergib, vergib, sprach er aus Lieb,\\
o Vater, ihnen allen!\\
Ihr keiner ist, der säh und wüßt,\\
in was für Tat sie fallen.

\flagverse{3.} Lehrt uns hiemit, wie schön es sei,\\
die lieben, die uns kränken,\\
und ihnen ohne Heuchelei\\
all ihre Fehler schenken.\\
Er zeigt zugleich, wie gnadenreich\\
und fromm sei sein Gemüte,\\
daß auch sein Feind, der's böse meint,\\
bei ihm nichts find als Güte.

\flagverse{4.} Drauf spricht er seine Mutter an,\\
die bei Johanne stunde,\\
tröst't sie am Kreuz, so gut er kann,\\
mit seinem schwachen Munde:\\
Sieh hier dein Sohn! Weib, der wird schon\\
mein Amt bei dir verwalten.\\
Und, Jünger, sieh, hie stehet, die\\
du sollst als Mutter halten.

\flagverse{5.} Ach, treues Herz, so sorgest du\\
für alle deine Frommen.\\
Du siehst und schauest fleißig zu,\\
wie sie in Trübsal kommen,\\
trittst auch mit Rat und treuer Tat\\
zu ihnen auf die Seiten;\\
du bringst sie fort, gibst ihnen Ort\\
und Raum bei guten Leuten.

\flagverse{6.} Die dritte Red hast du getan\\
dem, der dich, Herr, gebeten:\\
Gedenk und nimm dich meiner an,\\
wenn du nun wirst eintreten\\
in deinen Thron und Ehr und Kron\\
als Himmelsfürst aufsetzen!\\
Ich will gewiß im Paradies,\\
sprachst du, dich heut ergötzen.

\vfill\null
\columnbreak


\flagverse{7.} O süßes Wort, o Freudenstimm!\\
Was will uns nun erschrecken?\\
Laß gleich den Tod mit großem Grimm\\
hergehn aus allen Ecken;\\
stürmt er gleich sehr, was kann er mehr,\\
als Leib und Seele scheiden?\\
Indessen schwing ich mich und spring\\
ins Paradies der Freuden.

\flagverse{8.} Nun wohl der Schächer wird mit Freud\\
aus Christi Wort erfüllet,\\
er aber selbst fängt an und schreit,\\
gleich als ein Leue brüllet:\\
Eli, mein Gott! Welch Angst und Not\\
muß ich, dein Kind, ausstehen!\\
Ich ruf, und du schweigst still dazu,\\
läß'st mich zu Grunde gehen.

\flagverse{9.} Nimm dies zur Folge, frommes Kind,\\
wann Gott sich grausam stellet,\\
schau, daß du, wenn sich Trübsal find't,\\
nicht werdest umgefället.\\
Halt steif und fest: Der dich jetzt läßt,\\
wird dich gar bald erfreuen,\\
sei du nur treu und halt dabei\\
stark an mit gläubgem Schreien.

\flagverse{10.} Der Herr fährt fort, ruft laut und hell,\\
klagt, wie ihn heftig dürste:\\
Mich dürstet, sprach der ewge Quell\\
und edle Lebensfürste.\\
Was meint er hier? Er zeiget dir,\\
wie matt er sich getragen\\
an deiner Last, die du ihm hast\\
gemacht in Sündentagen.

\flagverse{11.} Er deutet auch darneben an,\\
wie ihn so hoch verlange,\\
daß dies sein Kreuz bei jedermann\\
frucht bring und wohl verfange.\\
Das merk mit Fleiß, wer sich im Schweiß\\
der Seelenangst muß quälen:\\
Das ewge Licht schleußt keinen nicht\\
vom Teil und Heil der Seelen.

\flagverse{12.} Als nun des Todes finstre Nacht\\
begunnt hereinzudringen,\\
sprach Gottes Sohn: Es ist vollbracht\\
das, was ich soll vollbringen.\\
Was hier und dar die heilge Schar\\
der Väter und Propheten\\
hat aufgesetzt, wie man zuletzt\\
mich kreuzgen würd und töten.

\vfill\null
\columnbreak

\flagverse{13.} Ist's dann vollbracht, was willst du nun\\
dich so vergeblich plagen,\\
als müßt ein Mensch mit seinem Tun\\
die Sündenschuld abtragen?\\
Es ist vollbracht! Das nimm in Acht,\\
du darfst hie nichts zu geben,\\
als daß du gläubst und gläubig bleibst\\
in deinem ganzen Leben.

\flagverse{14.} Nun endlich redt er noch einmal,\\
schreit auf ohn alle Maßen:\\
Mein Vater, nimm in deinen Saal\\
das, was ich jetzt muß lassen:\\
Nimm meinen Geist, der hier sich reißt\\
aus meinem kalten Herzen!\\
Und hiermit wird der große Hirt\\
entbunden aller Schmerzen.

\end{verse}
\end{multicols}


\begin{center}
\settowidth{\versewidth}{Der, vor dem die Welt erschrickt,}
\begin{verse}[\versewidth]
\begin{verbatim}

\end{verbatim}

\flagverse{15.} O wollte Gott, daß ich mein End\\
auch also möchte enden\\
und meinen Geist in Gottes Händ\\
und treuen Schoß hinsenden!\\
Ach laß, mein Hort, dein letztes Wort\\
mein letztes Wort auch werden!\\
So werd ich schön und selig gehn\\
zum Vater von der Erden.
  
\end{verse}
\end{center}




%\attrib{\small{THZE}}

\index{Hör an mein Herz, die sieben Wort}
\newpage
\subsection*{\centerline{Ich grüße dich, du frömmster Mann}}
\addcontentsline{toc}{subsection}{Ich grüße dich du frömmster Mann}
%StartInfo%%%%%%%%%%%%%%%%%%%%%%%%%%%%%%%%%%%%%%%%%%%%%%%%%%%%%%%%%%%%%%%%%%%%
%  Autor:
%  Titel:
%  File:
%  Ref:
%  Mod:
%EndInfo%%%%%%%%%%%%%%%%%%%%%%%%%%%%%%%%%%%%%%%%%%%%%%%%%%%%%%%%%%%%%%%%%%%%%%
%\poemtitle{pt}
\begin{multicols}{2}
\settowidth{\versewidth}{Du zahlst mit beiden Händen dar}
\begin{verse}[\versewidth]

% An die Hände (Salve Jesu, pastor bone) Sei wohl gegrüßet, guter Hirt

\flagverse{1.} Sei wohl gegrüßet, guter Hirt,\\
und ihr, o heilgen Hände\\
voll Rosen, die man preisen wird\\
bis an des Himmels Ende.\\
Die Rosen, die\\
ich mein allhie,\\
sind deine Mal und Plagen,\\
die dir am End\\
in deine Händ\\
am Kreuze sind geschlagen.

\flagverse{2.} Du zahlst mit beiden Händen dar\\
die edlen roten Gulden\\
und bringst die ganze Menschenschar\\
dadurch aus allen Schulden.\\
Ach laß von mir,\\
o Liebster, dir\\
dies' Hände herzlich drücken\\
und mit dem Blut,\\
das mir zugut\\
vergossen, mich erquicken.

\flagverse{3.} Wie freundlich tust du dich doch zu\\
und greifst mit beiden Armen\\
nach aller Welt, in Lieb und Ruh\\
uns ewig zu erwarmen.\\
Ach Herr, sieh hier,\\
mit was Begier\\
ich Armer zu dir trete!\\
Sei mir bereit\\
und gib mir Freud\\
und Trost, darum ich bete.

\flagverse{4.} Zeuch allen meinen Geist und Sinn\\
nach dir und deiner Höhe!\\
Gib, daß mein Herz nur immerhin\\
nach deinem Kreuze stehe,\\
ja daß ich mich\\
selbst williglich\\
mit dir ans Kreuze binde!\\
Und mehr und mehr\\
töt und zerstör\\
in mir des Fleisches Sünde.

\end{verse}
\end{multicols}

\begin{center}
\settowidth{\versewidth}{Der, vor dem die Welt erschrickt,}
\begin{verse}[\versewidth]

  

\flagverse{5.} Ich herz und küsse wiederum\\
aus rechtem treuen Herzen,\\
Herr, deine Händ und sage Ruhm\\
und Dank für ihren Schmerzen;\\
darneben geb\\
ich, weil ich leb,\\
in diese deine Hände\\
Herz, Seel und Leib,\\
und also bleib.

\end{verse}
\end{center}



%\attrib{\small{THZE}}

\index{Ich grüße dich, du frömmster Mann}
\newpage
\subsection*{\centerline{O Haupt voll Blut und Wunden}}
\addcontentsline{toc}{subsection}{O Haupt voll Blut und Wunden}
%StartInfo%%%%%%%%%%%%%%%%%%%%%%%%%%%%%%%%%%%%%%%%%%%%%%%%%%%%%%%%%%%%%%%%%%%%
%  Autor:
%  Titel:
%  File:
%  Ref:
%  Mod:
%EndInfo%%%%%%%%%%%%%%%%%%%%%%%%%%%%%%%%%%%%%%%%%%%%%%%%%%%%%%%%%%%%%%%%%%%%%%
%\poemtitle{pt}
\begin{multicols}{2}
\settowidth{\versewidth}{O Haupt voll Blut und Wunden,}
\begin{verse}[\versewidth]

%\{7.} An das Angesicht (Salve caput cruentatum) O Haupt voll Blut und Wunden

\flagverse{1.} O Haupt voll Blut und Wunden,\\
voll Schmerz und voller Hohn,\\
o Haupt, zu Spott gebunden\\
mit einer Dornenkron!\\
O Haupt, sonst schön gezieret\\
mit höchster Ehr und Zier,\\
jetzt aber hoch schimpfieret,\\
gegrüßet seist du mir!

\flagverse{2.} Du edles Angesichte,\\
davor sonst schrickt und scheut\\
das große Weltgewichte,\\
wie bist du so bespeit,\\
wie bist du so erbleichet,\\
wer hat dein Augenlicht,\\
dem sonst kein Licht nicht gleichet,\\
so schändlich zugericht?

\flagverse{3.} Die Farbe deiner Wangen,\\
der roten Lippen Pracht\\
ist hin und ganz vergangen,\\
des blassen Todes Macht\\
hat alles hingenommen,\\
hat alles hingerafft,\\
und daher bist du kommen\\
von deines Leibes Kraft.

\flagverse{4.} Nun, was du, Herr, erduldet,\\
ist alles meine Last,\\
ich hab es selbst verschuldet,\\
was du getragen hast!\\
Schau her, hier steh ich Armer,\\
der Zorn verdienet hat,\\
gib mir, o mein Erbarmer,\\
den Anblick deiner Gnad.

\flagverse{5.} Erkenne mich, mein Hüter;\\
mein Hirte, nimm mich an!\\
Von dir, Quell aller Güter,\\
ist mir viel Guts getan.\\
Dein Mund hat mich gelabet\\
mit Milch und süßer Kost;\\
dein Geist hat mich begabet\\
mit mancher Himmelslust.

\flagverse{6.} Ich will hier bei dir stehen,\\
verachte mich doch nicht!\\
Von dir will ich nicht gehen,\\
wann dir dein Herze bricht.\\
Wann dein Haupt wird erblassen\\
im letzten Todesstoß,\\
alsdann will ich dich fassen\\
in meinen Arm und Schoß.

\flagverse{7.} Es dient zu meinen Freuden\\
und kommt mir herzlich wohl,\\
wenn ich in deinem Leiden,\\
mein Heil, mich finden soll.\\
Ach, möcht ich, o mein Leben,\\
an deinem Kreuze hier\\
mein Leben von mir geben,\\
wie wohl geschähe mir!

\flagverse{8.} Ich danke dir von Herzen,\\
o Jesu, liebster Freund,\\
für deines Todes Schmerzen,\\
da du's so gut gemeint.\\
Ach gib, daß ich mich halte\\
zu dir und deiner Treu\\
und, wenn ich nun erkalte,\\
in dir mein Ende sei.

\flagverse{9.} Wenn ich einmal soll scheiden,\\
so scheide nicht von mir;\\
wenn ich den Tod soll leiden,\\
so tritt du dann herfür.\\
Wenn mir am allerbängsten\\
wird um das Herze sein,\\
so reiß mich aus den Ängsten\\
kraft deiner Angst und Pein.

\flagverse{10.} Erscheine mir zum Schilde,\\
zum Trost in meinem Tod\\
und laß mich sehn dein Bilde\\
in deiner Kreuzesnot.\\
Da will ich nach dir blicken,\\
da will ich glaubensvoll\\
dich fest an mein Herz drücken:\\
Wer so stirbt, der stirbt wohl.
      
\end{verse}
\end{multicols}
%\attrib{\small{THZE}}

\index{O Haupt voll Blut und Wunden}
\newpage
\subsection*{\centerline{O Herz des Königs aller Welt}}
\addcontentsline{toc}{subsection}{O Herz des Königs aller Welt}
%StartInfo%%%%%%%%%%%%%%%%%%%%%%%%%%%%%%%%%%%%%%%%%%%%%%%%%%%%%%%%%%%%%%%%%%%%
%  Autor:
%  Titel:
%  File:
%  Ref:
%  Mod:
%EndInfo%%%%%%%%%%%%%%%%%%%%%%%%%%%%%%%%%%%%%%%%%%%%%%%%%%%%%%%%%%%%%%%%%%%%%%
%\poemtitle{pt}
\begin{multicols}{2}
\settowidth{\versewidth}{Ach, wie bezwang und drang dich doch}
\begin{verse}[\versewidth]

%{6.} An das Herz (Summi regis cor aveto) O Herz des Königs aller Welt

\flagverse{1.} O Herz des Königs aller Welt,\\
des Herrschers in dem Himmelszelt,\\
dich grüßt mein Herz in Freuden.\\
Mein Herze, wie dir wohl bewußt,\\
hat seine größt und höchste Lust\\
an dir und deinem Leiden.\\
Ach, wie bezwang und drang dich doch\\
dein edle Lieb, ins bittre Joch\\
der Schmerzen dich zu geben,\\
da du dich neigtest in den Tod,\\
zu retten aus der Todesnot\\
mich und mein armes Leben.

\flagverse{2.} O Tod, du fremder Erdengast,\\
wie warst du so ein herbe Last\\
dem allersüß'sten Herzen!\\
Dich hat ein Weib der Welt gebracht,\\
und machst dem, der die Welt gemacht,\\
so unerhörte Schmerzen!\\
Du meines Herzens Herz und Sinn,\\
du brichst und fällst und stirbst dahin,\\
wollst mir ein Wort gewähren:\\
Ergreif mein Herz und schleuß es ein\\
in dir und deiner Liebe Schrein.\\
Mehr will ich nicht begehren.

\flagverse{3.} Mein Herz ist kalt, hart und betört\\
von allem, was zur Welt gehört,\\
fragt nur nach eitlen Sachen,\\
drum, herzes Herze, bitt ich dich,\\
du wollest dies mein Herz und mich\\
warm, weich und sauber machen.\\
Laß deine Flamm und starke Glut\\
durch all mein Herze, Geist und Mut\\
mit allen Kräften dringen;\\
laß deine Lieb und Freundlichkeit\\
zur Gegenlieb und Dankbarkeit\\
mich armen Sünder bringen.

\flagverse{4.} Erweitre dich, mach alles voll!\\
Sei meine Ros und riech mir wohl,\\
bring Herz und Herz zusammen,\\
entzünde mich durch dich und laß\\
mein Herz ohn End und alle Maß\\
in deiner Liebe flammen!\\
Wer dieses hat, wie wohl ist dem;\\
in dir beruhn ist angenehm,\\
ach, niemand kanns gnug sagen.\\
Wer dich recht liebt, ergibt sich frei,\\
in deiner Lieb und süßen Treu\\
auch wohl den Tod zu tragen.

\flagverse{5.} Ich ruf aus aller Herzensmacht\\
dich, Herz, in dem mein Herz erwacht,\\
ach laß dich doch errufen!\\
Komm, beug und neige dich zu mir\\
an meines Herzens arme Tür\\
und zeuch mich auf die Stufen\\
der Andacht und der Freudigkeit,\\
gib, daß mein Herz in Lieb und Leid\\
dein eigen sei und bleibe,\\
daß dir es dien an allem Ort\\
und dir zu Ehren immerfort\\
all seine Zeit vertreibe.

\flagverse{6.} O Herzensros', o schönste Blum!\\
Ach, wie so köstlich ist dein Ruhm,\\
du bist nicht auszupreisen.\\
Eröffne dich, laß deinen Saft\\
und des Geruchs erhöhte Kraft\\
mein Herz und Seele speisen!\\
Dein Herz, Herr Jesu, ist verwundt,\\
ach tritt zu mir in meinen Bund\\
und gib mir deinen Orden!\\
Verwund auch mich, o süßes Heil,\\
und triff mein Herz mit deinem Pfeil,\\
wie du verwundet worden.

\end{verse}
\end{multicols}

\begin{center}
\settowidth{\versewidth}{Der, vor dem die Welt erschrickt,}
\begin{verse}[\versewidth]

  


\flagverse{7.} Nimm mein Herz, o mein höchstes Gut,\\
und leg es hin, wo dein Herz ruht,\\
da ists wohl aufgehoben.\\
Da gehts mit dir gleich als zum Tanz,\\
da lobt es deines Hauses Glanz\\
und kanns doch nicht gnug loben.\\
Hier setzt sichs, hier gefällts ihm wohl,\\
hier freut sichs, daß es bleiben soll.\\
Erfüll, Herr, meinen Willen!\\
Und weil mein Herz dein Herze liebt,\\
so laß auch, wie dein Recht es gibt,

%\attrib{\small{THZE}}
\end{verse}
\end{center}







\index{O Herz des Königs aller Welt}
\newpage
\subsection*{\centerline{O Mensch, beweine deine Sünd}}
\addcontentsline{toc}{subsection}{O Mensch beweine deine Sünd}
%StartInfo%%%%%%%%%%%%%%%%%%%%%%%%%%%%%%%%%%%%%%%%%%%%%%%%%%%%%%%%%%%%%%%%%%%%
%  Autor:
%  Titel:
%  File:
%  Ref:
%  Mod:
%EndInfo%%%%%%%%%%%%%%%%%%%%%%%%%%%%%%%%%%%%%%%%%%%%%%%%%%%%%%%%%%%%%%%%%%%%%%
%\poemtitle{pt}
\begin{center}
\settowidth{\versewidth}{O Mensch beweine deine Sünd, um welcher willen Gottes Kind ein Mensche mußte werden;}
\begin{verse}[\versewidth]
%o Mensch, beweine deine Sünd\\
%die Passion nach Sebaldus Heyd

\flagverse{1.} O Mensch beweine deine Sünd, um welcher willen Gottes Kind ein Mensche mußte werden;\\
er kam von seines Vaters Thron, ward einer armen Jungfrau Sohn, tat große Ding auf Erden.\\
Die Kranken macht er frisch und stark und risse, was schon lag im Sarg, dem Tod aus seinem Rachen;\\
bis daß er selbst durch Feindes Händ am Kreuze seines Lebens End in Schmerzen mußte machen.

\flagverse{2.} Denn als nun wieder Ostern war, nahm er zu sich der Zwölfe Schar und sprach mit treuem Munde:\\
\flqq Nach zweien Tagen kommt die Nacht, da man das Osterlämmlein schlacht't;\\
dann ist auch meine Stunde.\frqq \\
Da ging die ganze Klerisei zu Rat, wie sie ihm kämen bei, hingegen die ihn liebte,\\
salbt ihn gar schön in Simons Haus, der Herr strich diese Tat heraus, schalt den, der sie betrübte.

\flagverse{3.} Das war der bös Ischarioth, der seinen Herrn der bösen Rott geschworen zu verraten.\\
Das fromme Lamm, der Heiland, kam, aß süßes Brot und Osterlamm, wie andre Juden taten.\\
Drauf stiftet er sein Fleisch und Blut, des Neuen Testamentes Gut, zu trinken und zu essen,\\
und stund hernach von seinem Ort, wusch seine Jünger, redt'te Wort, die nimmer zu vergessen.

\flagverse{4.} Er kam zum heilgen Öleberg; da, da ging an das hohe Werk mit Zittern und mit Zagen.\\
Die Erde nahm den Blutschweiß an, der häufig aus ihm drang und rann, der Himmel hört ihn sagen:\\
\flqq O Vaterherz, gefällt es dir, so gehe dieser Kelch von mir; wo nicht, gescheh dein Wille!\frqq \\
Und täte das zum dritten Mal, indessen lag der Jünger Zahl in Schlaf und süßer Stille.

\flagverse{5.} \flqq Ach\frqq, sprach das liebe treue Herz, \flqq ihr liegt und schlaft;\\
mich hat der Schmerz und Todesangst umfangen.\\
Ach, wacht und betet, betet, wacht! Damit ihr von des Feindes Macht nicht werdet hintergangen.\\
Nun ist mein Stündlein vor der Tür, steht auf! Da kommet her zu mir mein Jünger und Verräter!\frqq \\
Er hatte kaum gehöret auf, umringt ihn Judas und sein Hauf als einen Übeltäter.

\flagverse{6.} Der Führer küßt ihn mit dem Mund, und war doch nichts im Herzensgrund\\
als bittres Gift und Fluchen,\\
doch trat der Heiland frei dahin, sprach klar und deutlich: \flqq Seht, ich bin, den eure Augen suchen.\\
Sucht ihr denn mich, so lasset gehn, die ihr hier bei mir sehet stehn.\frqq Meint hiermit seine Jünger.\\
Und als des Petri strenger Sinn den Malchum schluge, heilt er ihn am Ohr mit seinem Finger.

\flagverse{7.} \flqq Steck ein das Schwert\frqq,  sprach unser Licht, \flqq solch Arbeit dienet hieher nicht,\\
mein Kelch muß sein getrunken.\frqq \\
Drauf ist der Richter aller Welt den Hohepriestern dargestellt; und da ist auch gesunken\\
des Petri Herz und Leuenmut, nicht zwar durch Schwert und Feuersglut, nur durch ein bloßes Fragen,\\
ob er nicht Jesu Jünger sei? Da fällt sein Glaube, Lieb und Treu, weiß nichts als Nein zu sagen.

\flagverse{8.} Auf diesen Fall kam große Reu, er fing an, da der Hahne schrei, sehr bitterlich zu weinen.\\
Das Auge, das die Herzen sieht, tät einen Blick, ließ Gnad und Güt dem armen Petro scheinen.\\
Die falschen Zeugen traten dar und red'ten viel, so nimmer wahr, auch niemals wird geschehen;\\
drum auch der Herr unnötig schätzt, daß er sein Wort dagegen setzt, läßts durch den Wind zerwehen.

\flagverse{9.} Dem aber, dem er ward verklagt, antwortet er, da er ihn fragt, ob er von Gott geboren:\\
\flqq Ja, ich bin Mensch und Gottes Sohn, der Welt zum Heil, zur Freud und Kron vom Vater auserkoren;\\
ihr werdet meine Herrlichkeit zur Rechten Gottes mit der Zeit hoch in den Wolken sehen.\frqq \\
Das nennt der Lästrer Lästerwort, da schrie ein jeder: \flqq Tod und Mord!\frqq \\
Da ging es an ein Schmähen.

\flagverse{10.} Man schlug, man spie ihm ins Gesicht. O Wunder, Wunder, daß hier nicht die Erde sich zerrissen!\\
O Wunder, daß nicht Gottes Grimm mit seiner starken Donnerstimm vom Himmel drein geschmissen!\\
Sie bunden ihm die Augen zu und hatten weder Maß noch Ruh im Höhnen und im Schlagen;\\
denn wenn sie schlugen, fragten sie: \flqq Sag an, wer tats? Du kannst es je als ein Prophete sagen!\frqq

\flagverse{11.} Und damit war es noch nicht aus. Am Morgen ward er in das Haus Pilati hingeführet.\\
Der Judas dacht den Sachen nach, sein frecher Sinn sank hin und brach, sein Herze ward gerühret;\\
es ward ihm leid, er hatte Reu, weil aber war kein Trost dabei, ging Seel und Leib zugrunde.\\
Er nahm ein grausam schrecklich End, er und sein Name bleibt geschänd't noch bis auf diese Stunde.

\flagverse{12.} Da Jesu vor Pilato stund, war sehr viel Klag und gar kein Grund; das meiste, das man triebe\\
war, daß er nichts mehr tu und lehr, als was die Untertanen kehr vons Kaisers Pflicht und Liebe,\\
dieweil er sich zum Könge macht. Pilatus ward dahin gebracht, daß er den Herren fragte,\\
ob er der Juden König wär? Der Herr sprach: Ja, zu Gottes Ehr, er wäre, was er sagte.

\bigskip

\flagverse{13.} Weil nun Herodes, dessen Hand sonst herrscht im Galiläerland, gleich damals war zugegen,\\
schickt ihm Pilatus Christum hin. Des freut er sich in seinem Sinn, ließ ihn zum Spott anlegen\\
ein weißes Kleid, ein arme Tracht, und da man seiner gnug gelacht, da schickt er ihn zurücke\\
Pilato heim; der ging zu Rat und fand ihn rein von arger Tat, unschuldig aller Tücke.

\flagverse{14.} Er nahm den Mörder Barrabam, dem jedermann sonst war sehr gram, den stellt er in die Mitten:\\
\flqq Hier sind der Übeltäter zwei\frqq, sprach er zum Volk,\flqq es steht euch frei, ihr möget einen bitten.\frqq \\
\flqq Halt Jesum\frqq, schrie die tolle Schar,\flqq laß Barrabam, wie er vor war, frei ledig in das Seine.\frqq \\
\flqq Was fang ich denn mit Jesu an?\frqq \flqq Ans Kreuz, ans Kreuz mit diesem Mann!\frqq antwortet die Gemeine.

\flagverse{15.} Da gab Pilatus Jesum hin dem Kriegesvolk, das geißelt ihn ohn alle Gnad und Schonen.\\
Der freche Haufe trat zuhauf und setzen unserm Könge auf von Dornen eine Kronen.\\
Er ward gehandelt als ein Tor; sie äfften ihn mit einem Rohr und schlugen ihn nicht wenig.\\
\flqq Du bist ein König\frqq, sagten sie, \flqq drum beugen wir dir unsre Knie, Glück zu, o Judenkönig!\frqq

\flagverse{16.} Als er nun übel zugericht't, führt ihn Pilatus ins Gesicht des Volks und sprach darneben:\\
\flqq Seht, seht doch, welch ein armer Wurm!\\
Nun wird sich euer Grimm und Sturm einmal zufrieden geben.\frqq \\
\flqq Nein, nein\frqq, sprach die vergallte Rott, \flqq zum Kreuz, zum Kreuz! Nur immer tot!\frqq\\
Pilatus wusch die Hände und wollt im Kote reine sein;\\
dem aber, der in allem rein, bestimmt er Tod und Ende.

\flagverse{17.} Das Leben ging zum bittern Tod und mußte seine letzte Not mit eignen Schultern tragen.\\
Er trug sein Kreuz und unsern Schmerz, darüber führt manch Mutterherz ein hochbetrübtes Klagen.\\
\flqq Weint nicht\frqq, sprach Christus, \flqq über mich, ein jeder weine über sich und über seine Sünde!\frqq \\
Es kommt die Zeit, da selig wird gepreiset die, so nicht gebiert und gar nicht weiß vom Kinde.

\flagverse{18.} Da man nun kam zur Schädelstatt, da ward, ders nicht verdienet hat, bis in den Tod gekränket.\\
Zwar also, daß ein Mörderpaar zur Seiten wurde hier und dar er mitten ein gehenket.\\
Man nahm ihm Leben, Ehr und Blut; den sanften Sinn, den frommen Mut, den mußten sie ihm lassen.\\
Er liebte, die ihm weh getan, rief seinen Vater für die an, die ihm sein Herz zerfraßen.

\flagverse{19.} Pilatus heftet oben an ein Überschrift, die jedermann, der bei dem Kreuz gewesen,\\
Hebräer, Römer, Griechenland und wer Vernunft hat und Verstand, gar wohl hat können lesen.\\
Die Krieger nehmen ihm sein Kleid und teilen sich diese Beut, der Rock bleibt unzerstücket;\\
er wird dem Los anheimgestellt, des soll er sein, wem jenes fällt; laßt sehen, wem es glücket.

\flagverse{20.} Maria voller Lieb und Treu stund an dem Kreuz, und auch dabei, den unser Heiland liebte.\\
\flqq Sieh hier\frqq, sprach Jesus, \flqq Weib, dein Sohn! Und Jünger, siehe deine Kron und Mutter, die betrübte;\\
die laß dir ja befohlen sein!\frqq Dies Wort, das drang ins Herz hinein Johanni, dem geliebten.\\
Er nahm die auf und tat ihr wohl, die andern machten Jammers voll durch Bosheit, die sie übten.

\flagverse{21.} Viel Lästrer red'ten böse Ding, auch einer, der zur Seiten hing, goß auf ihn seinen Geifer.\\
Der aber an dem andern Ort straft ihn und seine Lästerwort mit großem Ernst und Eifer,\\
sprach Jesum an: \flqq O Himmelsfürst, gedenke meiner, wenn du wirst nun in dein Reich eingehen!\frqq \\
\flqq Fürwahr, fürwahr, ich sage dir\frqq, sprach Jesus, \flqq du wirst heut bei mir im Paradiese stehen.\frqq

\flagverse{22.} Der Mittag kam und war doch Nacht, die Sonn, die alles fröhlich macht, war selbst mit Leid erfüllet.\\
Des Lichtes Schöpfer fühlet Pein, drum mußt mit finstern Schatten sein das schönste Licht verhüllet.\\
\flqq Eli!\frqq rief Jesus, \flqq Gott, mein Gott, wie läßt du mich in meiner Not und Angst so gar alleine?\frqq \\
Und bald darauf: \flqq Mich dürstet sehr!\frqq Das alles hört der Juden Heer und weiß nicht, was er meine.

\flagverse{23.} Sie sind vom Zorne taub und blind, hart wie ein Stein, der nichts empfindt, auch gar nicht zu erweichen.\\
Sie nehmen aus dem Essigfaß und machen einen Schwamm mit naß, den lassen sie ihm reichen.\\
Ihr Herz ist voller Bitterkeit, und damit sind sie auch bereit, den, der jetzt stirbt, zu laben.\\
Viel machen aus dem Ernst ein Spiel und sprechen: \flqq  Halt, laß sehn, er will Eliä Hilfe haben.\frqq

\flagverse{24.} Er aber sprach: Es ist vollbracht! Und darauf ward er von der Macht des Todes überfallen.\\
Er neigte sich zur sanften Ruh, er schloß die schwachen Augen zu und schrie mit großem Schallen:\\
\flqq Nimm auf, nimm auf, Herr, meinen Geist, du, mein herzliebster Vater, weißt,\\
wie du ihn sollst bewahren!\frqq \\
Und also ist der große Held, der Himmel, Erd und alles hält, von dieser Welt gefahren.

\flagverse{25.} Er fuhr dahin. Im Augenblick zerriß der Vorhang in zwei Stück, die Erd erschrak und bebte.\\
Die Felsen sprangen in die Luft, auch öffnet sich der Gräber Gruft und was darinnen lebte.\\
Der Juden Herzen blieben hart, allein der Hauptmann, dem da ward die Wach am Kreuz befohlen,\\
der glaubt, und mit ihm sein Gesind, es wäre Jesus Gottes Kind und sagtens unverhohlen.

\flagverse{26.} Man brach den Schächern ihre Bein, mein und dein Heiland blieb allein an Beinen ungebrochen.\\
Das aber ist wahr und gewiß, daß ein Soldat mit seinem Spieß die Seiten ihm zerstochen,\\
aus welcher Wund ein edle Flut von Blut und Wasser uns zugut alsbald herausgeflossen.\\
Zuletzt ward er vom Kreuz gebracht und, wohl beschickt,\\
noch vor der Nacht in Josephs Grab geschlossen.

\flagverse{27.} Die Juden hatten wohl gehört, er würde, wie er selbst gelehrt, von Toten auferstehen;\\
das halten sie für unwahr sein, sie bilden ihnen aber ein, es möchte List ergehen.\\
Drum siegeln sie des Grabes Tür und legten starke Wache für; umsonst und gar vergebens!\\
Der Herr dringt durch, kein Fels und Stein,\\
kein Wächter mag zu mächtig sein dem Fürsten unsres Lebens.

\flagverse{28.} Nun seh und lern ein jedermann, wie sehr viel Gutes uns getan der Bräutgam unsrer Seelen:\\
Er nahm auf sich all unser Schuld und ließ aus treuer Lieb und Huld sich unserthalben quälen.\\
Zerknirschtes Herz, betrübter Geist, den seine Sünde nagt und beißt, laß Sorg und Kummer fallen,\\
weil unser Heiland Jesus Christ ein Sündenopfer worden ist, dir und uns Menschen allen!

\flagverse{29.} Du aber, der du sicher stehst, und ohne Buße täglich gehst in ungescheute Sünden,\\
betrachte, was für Straf und Last, wenn du dein Maß gefüllet hast, dich endlich werde finden!\\
Denn tut man das am grünen Baum, so denke, was für Ort und Raum der dürre werd erlangen.\\
O Jesu, gibt uns deinen Sinn und bring uns alle, wo du hin durch deinen Tod gegangen!
  
\end{verse}
\end{center}




%\attrib{\small{THZE}}

\index{O Mensch beweine deine Sünd}
\newpage
\subsection*{\centerline{O Welt, sieh hier dein Leben}}
\addcontentsline{toc}{subsection}{O Welt sieh hier dein Leben}
%StartInfo%%%%%%%%%%%%%%%%%%%%%%%%%%%%%%%%%%%%%%%%%%%%%%%%%%%%%%%%%%%%%%%%%%%%
%  Autor:
%  Titel:
%  File:
%  Ref:
%  Mod:
%EndInfo%%%%%%%%%%%%%%%%%%%%%%%%%%%%%%%%%%%%%%%%%%%%%%%%%%%%%%%%%%%%%%%%%%%%%%
%\poemtitle{pt}
\begin{multicols}{2}
\settowidth{\versewidth}{Tritt her und schau mit Fleiße:}
\begin{verse}[\versewidth]
%o Welt, sieh hier dein Leben

\flagverse{1.} O Welt, sieh hier dein Leben\\
am Stamm des Kreuzes schweben!\\
Dein Heil sinkt in den Tod!\\
Der große Fürst der Ehren\\
läßt willig sich beschweren\\
mit Schlägen, Hohn und großem Spott.

\flagverse{2.} Tritt her und schau mit Fleiße:\\
Sein Leib ist ganz mit Schweiße\\
des Blutes überfüllt;\\
aus seinem edlen Herzen\\
vor unerschöpften Schmerzen\\
ein Seufzer nach dem andern quillt.

\flagverse{3.} Wer hat dich so geschlagen,\\
mein Heil, und dich mit Plagen\\
so übel zugericht't?\\
Du bist ja nicht ein Sünder\\
wie wir und unsre Kinder,\\
von Übeltaten weißt du nicht.

\flagverse{4.} Ich, ich und meine Sünden,\\
die sich wie Körnlein finden\\
des Sandes an dem Meer,\\
die haben dir erreget\\
das Elend, das dich schläget,\\
und das betrübte Marterheer.

\flagverse{5.} Ich bins, ich sollte büßen,\\
an Händen und an Füßen\\
gebunden, in der Höll;\\
die Geißeln und die Banden\\
und was du ausgestanden,\\
das hat verdienet meine Seel.

\flagverse{6.} Du nimmst auf deinen Rücken\\
die Lasten, die mich drücken\\
viel sehrer als ein Stein.\\
Du wirst ein Fluch, dagegen\\
verehrst du mir den Segen;\\
dein Schmerzen muß mein Labsal sein.

\flagverse{7.} Du setzest dich zum Bürgen,\\
ja lässest dich gar würgen\\
für mich und meine Schuld;\\
mir lässest du dich krönen\\
mit Dornen, die dich höhnen,\\
und leidest alles mit Geduld.

\flagverse{8.} Du springst ins Todes Rachen,\\
mich frei und los zu machen\\
von solchem Ungeheur.\\
Mein Sterben nimmst du abe,\\
vergräbst es in dem Grabe,\\
o unerhörtes Liebesfeur!

\flagverse{9.} Ich bin, mein Heil, verbunden\\
all Augenblick und Stunden\\
dir überhoch und sehr.\\
Was Leib und Seel vermögen,\\
das soll ich billig legen\\
allzeit an deinen Dienst und Ehr.

\flagverse{10.} Nun, ich kann nicht viel geben\\
in diesem armen Leben;\\
eins aber will ich tun:\\
Es soll dein Tod und Leiden\\
bis Leib und Seele scheiden,\\
mir stets in meinem Herzen ruhn.

\flagverse{11.} Ich wills vor Augen setzen,\\
mich stets daran ergötzen,\\
ich sei auch, wo ich sei;\\
es soll mir sein ein Spiegel\\
der Unschuld und ein Siegel\\
der Lieb und unverfälschten Treu.

\flagverse{12.} Wie heftig unsre Sünden\\
den frommen Gott entzünden,\\
wie Rach und Eifer gehn,\\
wie grausam seine Ruten,\\
wie zornig seine Fluten,\\
will ich aus diesem Leiden sehn.

\flagverse{13.} Ich will daraus studieren,\\
wie ich mein Herz soll zieren\\
mit stillem, sanften Mut,\\
und wie ich die soll lieben,\\
die mich doch sehr betrüben\\
mit Werken, so die Bosheit tut.

\flagverse{14.} Wenn böse Zungen stechen,\\
mir Glimpf und Namen brechen.\\
So will ich zähmen mich;\\
das Unrecht will ich dulden,\\
dem Nächsten seine Schulden\\
verzeihen gern und williglich.

\flagverse{15.} Ich will mich mit dir schlagen\\
ans Kreuz und dem absagen,\\
was meinem Fleisch gelüst't.\\
Was deine Augen hassen,\\
das will ich fliehn und lassen,\\
so viel mir immer möglich ist.

\flagverse{16.} Dein Seufzen und dein Stöhnen\\
und die viel tausend Tränen,\\
die dir geflossen zu,\\
die sollen mich am Ende\\
in deinen Schoß und Hände\\
begleiten zu der ewgen Ruh.

\end{verse}
\end{multicols}
%\attrib{\small{THZE}}

\index{O Welt, sieh hier dein Leben}
\newpage
\subsection*{\centerline{Sei mir tausendmal gegrüßet}}
\addcontentsline{toc}{subsection}{Sei mir tausendmal gegrüßet}
%StartInfo%%%%%%%%%%%%%%%%%%%%%%%%%%%%%%%%%%%%%%%%%%%%%%%%%%%%%%%%%%%%%%%%%%%%
%  Autor:
%  Titel:
%  File:
%  Ref:
%  Mod:
%EndInfo%%%%%%%%%%%%%%%%%%%%%%%%%%%%%%%%%%%%%%%%%%%%%%%%%%%%%%%%%%%%%%%%%%%%%%
%\poemtitle{pt}
\begin{multicols}{2}
\settowidth{\versewidth}{Heile mich, o Heil der Seelen,}
\begin{verse}[\versewidth]

%{1.} An die Füße (Salve, mundi salutare) Sei mir tausendmal gegrüßet

\flagverse{1.} Sei mir tausendmal gegrüßet,\\
der mich je und je geliebt,\\
jesu, der du selbst gebüßet\\
das, womit ich dich betrübt.\\
Ach, wie ist mir doch so wohl,\\
wenn ich knien und liegen soll\\
an dem Kreuze, da du stirbest\\
und um meine Seele wirbest.

\flagverse{2.} Ich umfange, herz und küsse\\
der gekränkten Wunden Zahl\\
und die purpurroten Flüsse,\\
deine Füß' und Nägelmal.\\
O, wer kann doch, schönster Fürst,\\
den so hoch nach uns gedürst't,\\
deinen Durst und Liebsverlangen\\
völlig fassen und umfangen?

\flagverse{3.} Heile mich, o Heil der Seelen,\\
wo ich krank und traurig bin;\\
nimm die Schmerzen, die mich quälen,\\
und den ganzen Schaden hin,\\
den mir Adams Fall gebracht\\
und ich selbsten mir gemacht.\\
Wird, o Arzt, dein Blut mich netzen,\\
wird sich all mein Jammer setzen.

\flagverse{4.} Schreibe deine blutgen Wunden\\
mir, Herr, in das Herz hinein,\\
daß sie mögen alle Stunden\\
bei mir unvergessen sein.\\
Du bist doch mein liebstes Gut,\\
da mein ganzes Herze ruht.\\
Laß mich hie zu deinen Füßen\\
deiner Lieb und Gunst genießen.

\end{verse}
\end{multicols}

\begin{center}
\settowidth{\versewidth}{Der, vor dem die Welt erschrickt,}
\begin{verse}[\versewidth]

\flagverse{5.} Diese Füße will ich halten\\
auf das best ich immer kann;\\
schaue meiner Hände Falten\\
und mich selbsten freundlich an\\
von dem hohen Kreuzesbaum\\
und gib meiner Bitte Raum,\\
sprich: Laß all dein Trauern schwinden,\\
ich, ich tilg all deine Sünden.

\end{verse}
\end{center}
%\attrib{\small{THZE}}

\index{Sei mir gegrüßet tausendmal}
\newpage
\subsection*{\centerline{Sei wohl gegrüßet, guter Hirt}}
\addcontentsline{toc}{subsection}{Sei wohl gegrüßet guter Hirt}
%StartInfo%%%%%%%%%%%%%%%%%%%%%%%%%%%%%%%%%%%%%%%%%%%%%%%%%%%%%%%%%%%%%%%%%%%%
%  Autor:
%  Titel:
%  File:
%  Ref:
%  Mod:
%EndInfo%%%%%%%%%%%%%%%%%%%%%%%%%%%%%%%%%%%%%%%%%%%%%%%%%%%%%%%%%%%%%%%%%%%%%%
%\poemtitle{pt}
\begin{multicols}{2}
\settowidth{\versewidth}{Wie freundlich tust du dich doch zu}
\begin{verse}[\versewidth]

%{3.} An die Hände (Salve Jesu, pastor bone) Sei wohl gegrüßet, guter Hirt

\flagverse{1.} Sei wohl gegrüßet, guter Hirt,\\
und ihr, o heilgen Hände\\
voll Rosen, die man preisen wird\\
bis an des Himmels Ende.\\
Die Rosen, die\\
ich mein allhie,\\
sind deine Mal und Plagen,\\
die dir am End\\
in deine Händ\\
am Kreuze sind geschlagen.

\flagverse{2.} Du zahlst mit beiden Händen dar\\
die edlen roten Gulden\\
und bringst die ganze Menschenschar\\
dadurch aus allen Schulden.\\
Ach laß von mir,\\
o Liebster, dir\\
dies' Hände herzlich drücken\\
und mit dem Blut,\\
das mir zugut\\
vergossen, mich erquicken.

\flagverse{3.} Wie freundlich tust du dich doch zu\\
und greifst mit beiden Armen\\
nach aller Welt, in Lieb und Ruh\\
uns ewig zu erwarmen.\\
Ach Herr, sieh hier,\\
mit was Begier\\
ich Armer zu dir trete!\\
Sei mir bereit\\
und gib mir Freud\\
und Trost, darum ich bete.

\flagverse{4.} Zeuch allen meinen Geist und Sinn\\
nach dir und deiner Höhe!\\
Gib, daß mein Herz nur immerhin\\
nach deinem Kreuze stehe,\\
ja daß ich mich\\
selbst williglich\\
mit dir ans Kreuze binde!\\
Und mehr und mehr\\
töt und zerstör\\
in mir des Fleisches Sünde.
\end{verse}
\end{multicols}

\begin{center}
\settowidth{\versewidth}{Der, vor dem die Welt erschrickt,}
\begin{verse}[\versewidth]

\flagverse{5.} Ich herz und küsse wiederum\\
aus rechtem treuen Herzen,\\
Herr, deine Händ und sage Ruhm\\
und Dank für ihren Schmerzen;\\
darneben geb\\
ich, weil ich leb,\\
in diese deine Hände\\
Herz, Seel und Leib,\\
und also bleib.

\end{verse}
\end{center}


%\attrib{\small{THZE}}

\index{Sei wohl gegrüßet, guter Hirt}
\newpage
\subsection*{\centerline{Siehe, mein getreuer Knecht}}               % witt:1851:geliebter
\addcontentsline{toc}{subsection}{Siehe mein getreuer Knecht}
%StartInfo%%%%%%%%%%%%%%%%%%%%%%%%%%%%%%%%%%%%%%%%%%%%%%%%%%%%%%%%%%%%%%%%%%%%
%  Autor:
%  Titel:
%  File:
%  Ref:
%  Mod:
%EndInfo%%%%%%%%%%%%%%%%%%%%%%%%%%%%%%%%%%%%%%%%%%%%%%%%%%%%%%%%%%%%%%%%%%%%%%
%\poemtitle{pt}
\begin{multicols}{2}
\settowidth{\versewidth}{Siehe, mein getreuer Knecht,}
\begin{verse}[\versewidth]
%siehe, mein getreuer Knecht\\
%(Jes. 52, 13 ff. u. 53)

\flagverse{1.} Siehe, mein getreuer Knecht,\\
der wird weislich handeln,\\
ohne Tadel, schlecht und recht\\
auf der Erden wandeln;\\
sein getreuer, frommer Sinn\\
wird in Einfalt gehen,\\
und noch dennoch wird man ihn\\
an das Kreuz erhöhen.

\flagverse{2.} Hoch am Kreuze wird mein Sohn\\
große Marter leiden,\\
und viel werden ihn mit Hohn\\
als ein Scheusal meiden.\\
Aber also wird sein Blut\\
auf die Heiden springen\\
und das ewge wahre Gut\\
in ihr Herze bringen.

\flagverse{3.} Kön'ge werden ihren Mund\\
gegen ihn verhalten.\\
Und aus innerm Herzensgrund\\
ihre Hände falten.\\
Das verblend'te taube Heer\\
wird ihn sehn und hören\\
und mit Lust zu seiner Ehr\\
ihren Glauben mehren.

\flagverse{4.} Aber da, wo Gottes Licht\\
reichlich wird gespüret,\\
hält man sich mit nichten nicht\\
wie es sich gebühret:\\
Denn wer glaubt im Judenland\\
unsrer Predigt Worten?\\
Wem wird Gottes Arm bekannt\\
in Israels Orten?

\flagverse{5.} Niemand will fast seinen Preis\\
ihm hie lassen werden,\\
denn er schießt auf wie ein Reis\\
aus der dürren Erden,\\
krank, verdorret, ungestalt,\\
voller Blut und Schmerzen,\\
daher scheut ihn jung und alt\\
mit verwandtem Herzen.

\flagverse{6.} Ei, was hat er denn getan?\\
Was sind seine Schulden,\\
daß er da für jedermann\\
solche Schmach muß dulden?\\
Hat er etwann Gott betrübt\\
bei gesunden Tagen,\\
daß er ihm anitzo gibt\\
seinen Lohn mit Plagen?

\flagverse{7.} Nein, fürwahr! Wahrhaftig nein!\\
Er ist ohne Sünden.\\
Sondern was der Mensch für Pein\\
billig sollt empfinden,\\
was für Krankheit, Angst und Weh\\
uns von Recht gebühret,\\
das ist's was ihn in die Höh\\
an das Kreuz geführet.

\flagverse{8.} Daß ihn Gott so heftig schlägt,\\
tut er unsertwillen,\\
daß er solche Bürden trägt,\\
damit will er stillen\\
gottes Zorn und großen Grimm,\\
daß wir Frieden haben\\
durch sein Leiden und in ihm\\
leib und Seele laben.

\flagverse{9.} Wir sinds, die wir in der Irr\\
als die Schafe gingen\\
und noch stets zur Höllentür\\
als die Tollen dringen.\\
Aber Gott, der fromm und treu,\\
nimmt, was wir verdienen\\
und legts seinem Sohne bei,\\
der muß uns versühnen.

\flagverse{10.} Nun, er tut es herzlich gern,\\
ach, des frommen Herzens!\\
Er nimmt an den Zorn des Herrn\\
mit viel tausend Schmerzen\\
und ist allzeit voll Geduld,\\
läßt kein Wörtlein hören\\
wider die, so ohne Schuld\\
ihn so hoch beschweren.

\flagverse{11.} Wie ein Lämmlein sich dahin\\
läßt zur Schlachtbank leiten\\
und hat in dem frommen Sinn\\
gar kein Widerstreiten,\\
läßt sich handeln, wie man will,\\
fangen, binden, zähmen\\
und dazu in großer Still\\
auch sein Leben nehmen.

\flagverse{12.} Also läßt auch Gottes Lamm\\
ohne Widersprechen\\
ihm sein Herz am Kreuzesstamm\\
unsertwegen brechen.\\
Er sinkt in den Tod hinab,\\
den er selbst doch bindet,\\
weil er sterbend Tod und Grab\\
mächtig überwindet.

\flagverse{13.} Er wird aus der Angst und Qual\\
endlich ausgerissen,\\
tritt den Feinden allzumal\\
ihren Kopf mit Füßen.\\
Wer will seines Lebens Läng\\
immer mehr ausrechnen?\\
Seiner Tag und Jahre Meng\\
ist nicht auszusprechen.

\flagverse{14.} Doch ist er wahrhaftig hier\\
für sein Volk gestorben\\
und hat völlig mir und dir\\
heil und Gnad erworben,\\
kommt auch in das Grab hinein\\
herrlich eingehüllet,\\
wie die, so mit Reichtum sein\\
in der Welt erfüllet.

\flagverse{15.} Er wird als ein böser Mann\\
vor der Welt geplaget,\\
da er doch noch nie getan,\\
auch noch nie gesaget,\\
was da bös und unrecht wär;\\
er hat nie betrogen,\\
nie verletzet Gottes Ehr,\\
sein Mund nie gelogen.

\flagverse{16.} Ach, er ist für fremde Sünd\\
in den Tod gegeben,\\
auf daß du, o Menschenkind,\\
durch ihn möchtest leben,\\
daß er mehrte sein Geschlecht,\\
den gerechten Samen,\\
der Gott dient und Opfer brächt\\
seinem heilgen Namen.

\flagverse{17.} Denn das ist sein höchste Freud\\
und des Vaters Wille,\\
daß den Erdkreis weit und breit\\
sein Erkenntnis fülle,\\
damit der gerechte Knecht,\\
der vollkommne Sühner,\\
gläubig mach und recht gerecht\\
alle Sündendiener.

\flagverse{18.} Große Menge wird ihm Gott\\
zur Verehrung schenken,\\
darum, daß er sich mit Spott\\
für uns lassen kränken,\\
da er denen gleich gesetzt,\\
die sehr übertreten,\\
auch die, so ihn hoch verletzt,\\
bei Gott selbst verbeten.

\end{verse}
\end{multicols}
%\attrib{\small{THZE}}

\index{Siehe, mein getreuer Knecht}
\newpage
\subsection*{\centerline{Als Gottes Lamm und Leue}}                  %witt:1851:Auf das Begräbnis Jesu.
\addcontentsline{toc}{subsection}{Als Gottes Lamm und Leue}
%StartInfo%%%%%%%%%%%%%%%%%%%%%%%%%%%%%%%%%%%%%%%%%%%%%%%%%%%%%%%%%%%%%%%%%%%%
%  Autor:
%  Titel:
%  File:
%  Ref:
%  Mod:
%EndInfo%%%%%%%%%%%%%%%%%%%%%%%%%%%%%%%%%%%%%%%%%%%%%%%%%%%%%%%%%%%%%%%%%%%%%%
%ANM:\poemtitle{Als Gottes Lamm und Leue}
\begin{multicols}{2}
\settowidth{\versewidth}{und wenn die Unschuld gnug geschändt,}
\begin{verse}[\versewidth]
%ANM:die Grablegung Christi\\
%ANM:als Gottes Lamm und Leue entschlafen\\
\flagverse{1.} Als Gottes Lamm und Leue\\
entschlafen und verschieden,\\
erwacht in Lieb und Treue\\
ein Paar recht frommer Jüden.\\
Die Machten sich zum Kreuz hinzu,\\
dich, o du unser ewge Ruh,\\
zu deiner Ruh zu bringen.

\flagverse{2.} Also weiß Gott die Seinen\\
am Kreuz in Acht zu nehmen\\
und, die es böse meinen,\\
zu rechter Zeit zu zähmen.\\
Das Wüten nimmt zuletzt ein End,\\
und wenn die Unschuld gnug geschändt,\\
so findt sich, der sie ehre.

\flagverse{3.} Dann einer aus dem Rate,\\
Joseph, der fromme Reiche,\\
der wagt es, ging und bate\\
Pilatum um die Leiche.\\
Pilatus war bereit und gab\\
Befehl, daß man sie nähm herab\\
und Joseph übergäbe.

\flagverse{4.} Gesegnet sei dein Wille,\\
Joseph, und dein Begehren,\\
Gott wolle dir die Fülle\\
der Freuden dort gewähren,\\
daß du, den meine Seele liebt,\\
vom Kreuze, da man ihn betrübt,\\
so freudig losgebeten.

\flagverse{5.} Hierzu hat sich auch funden\\
des Nicodemi Treue,\\
der bringt bei hundert Pfunden\\
der besten Spezereie,\\
die Myrrhen samt der Aloe\\
zu salben den, der aus der Höh\\
uns salbt mit seinem Geiste.

\flagverse{6.} Da siehst du, wie die Schwachen\\
zuletzt gestärket werden.\\
Gott kann zu Helden machen,\\
was blöd ist hie auf Erden.\\
Der Glaube, der im Finstern lag,\\
bricht endlich an den hellen Tag\\
und leuchtet wie die Sonne.

\flagverse{7.} Nun, diese beiden Frommen\\
ergreifen mit viel Weinen\\
den, der vom Kreuz genommen,\\
und wickeln ihn in Leinen,\\
verwahren ihn zugleich dabei\\
mit edler teurer Spezerei,\\
wie in Judäa bräuchlich.

\flagverse{8.} So soll man Christum ehren,\\
wann er nun liegt darnieder.\\
Wir sollen balsamieren\\
ihn und sein arme Glieder,\\
die Unbekleid'ten wickeln ein\\
und die, so ganz verlassen sein,\\
mit unsrer Hilf annehmen.

\flagverse{9.} Es war nicht weit von hinnen,\\
wo Christus starb, zu schauen\\
ein Garten und darinnen\\
des Josephs Grab, gehauen\\
gar neu in einem Felsenstein,\\
da legten ihren Schatz hinein\\
die zwei geliebten Herzen.

\flagverse{10.} Ach Jesu, dessen Schmerzen\\
mir all mein Heil erworben,\\
komm, ruh in meinem Herzen,\\
das in der Sünd erstorben!\\
Laß dirs gefallen, ich will dir\\
dein Grab bereiten in mir hier,\\
so leb und sterb ich selig.

\end{verse}
\end{multicols}
%\attrib{\small{THZE}}

\index{Als Gottes Lamm und Leue}
\newpage

\newpage

\section*{\centerline{\LARGE OSTERN - PFINGSTEN - TRINITATIS}}
\addcontentsline{toc}{section}{OSTERN - PFINGSTEN - TRINITATIS}
\rule{\textwidth}{0.2pt}\vspace*{-\baselineskip}\vspace{3.2pt}
\rule{\textwidth}{1.2pt}\\[\baselineskip]

\index{ Gedichte zu Ostern und Pfingsten}

\centerline{\scshape Auf auf, mein Herz, mit Freuden }
\vspace*{2\baselineskip}
\centerline{\scshape Nun freut euch hier und überall }
\vspace*{2\baselineskip}
\centerline{\scshape Sei fröhlich alles weit und breit}
\vspace*{2\baselineskip}
\centerline{\scshape Gott Vater, sende deinen Geist }
\vspace*{2\baselineskip}
\centerline{\scshape O du allersüß'ste Freude }
\vspace*{2\baselineskip}
\centerline{\scshape Zeuch ein zu deinen Toren }
\vspace*{2\baselineskip}
\centerline{\scshape Was alle Weisheit in der Welt }

\newpage

\subsection*{\centerline{Auf auf, mein Herz, mit Freuden}}
\addcontentsline{toc}{subsection}{Auf auf mein Herz mit Freuden}
%StartInfo%%%%%%%%%%%%%%%%%%%%%%%%%%%%%%%%%%%%%%%%%%%%%%%%%%%%%%%%%%%%%%%%%%%%
%  Autor:
%  Titel:
%  File:
%  Ref:
%  Mod:
%EndInfo%%%%%%%%%%%%%%%%%%%%%%%%%%%%%%%%%%%%%%%%%%%%%%%%%%%%%%%%%%%%%%%%%%%%%%
%ANM:\poemtitle{Auf auf, mein Herz, mit Freuden}

\begin{multicols}{2}
\settowidth{\versewidth}{Ich hang und bleib auch hangen}
\begin{verse}[\versewidth]
\flagverse{1.} Auf auf, mein Herz, mit Freuden\\
nimm wahr, was heut geschicht!\\
Wie kommt nach großem Leiden\\
nun ein so großes Licht!\\
Mein Heiland war gelegt\\
da, wo man uns hinträgt,\\
wenn von uns unser Geist\\
gen Himmel ist gereist.

\flagverse{2.} Er war ins Grab gesenket,\\
der Feind trieb groß Geschrei.\\
Eh ers vermeint und denket,\\
ist Christus wieder frei\\
und ruft Victoria!\\
Schwingt fröhlich hie und da\\
sein Fähnlein als ein Held,\\
der Feld und Mut behält.

\flagverse{3.} Der Held steht auf dem Grabe\\
und sieht sich munter um,\\
der Feind liegt und legt abe\\
Gift, Gall und Ungestüm,\\
er wirft zu Christi Fuß\\
sein Höllenreich und muß\\
selbst in des Siegers Band\\
ergeben Fuß und Hand.

\flagverse{4.} Das ist mir anzuschauen\\
ein rechtes Freudenspiel,\\
nun soll mir nicht mehr grauen\\
vor allem, was mir will\\
entnehmen meinen Mut\\
zusamt dem edlen Gut,\\
so mir durch Jesum Christ\\
aus Lieb erworben ist.

\flagverse{5.} Die Höll und ihre Rotten,\\
die krümmen mir kein Haar,\\
der Sünden kann ich spotten,\\
bleib allzeit ohn Gefahr.\\
Der Tod mit seiner Macht\\
wird nichts bei mir geacht't,\\
er bleibt ein totes Bild,\\
und wär er noch so wild.

\flagverse{6.} Die Welt ist mir ein Lachen\\
mit ihrem großen Zorn,\\
sie zürnt und kann nichts machen,\\
all Arbeit ist verlorn.\\
Die Trübsal trübt mir nicht\\
mein Herz und Angesicht,\\
das Unglück ist mein Glück,\\
die Nacht mein Sonnenblick.

\flagverse{7.} Ich hang und bleib auch hangen\\
an Christo als ein Glied,\\
wo mein Haupt durch ist gangen,\\
da nimmt er mich auch mit.\\
Er reißet durch den Tod,\\
durch Welt, durch Sünd, durch Not,\\
er reißet durch die Höll:\\
Ich bin stets sein Gesell.

\flagverse{8.} Er dringt zum Saal der Ehren,\\
ich folg ihm immer nach\\
und darf mich gar nicht kehren\\
an einzig Ungemach.\\
Es tobe, was da kann,\\
mein Haupt nimmt sich mein an,\\
mein Heiland ist mein Schild,\\
der alles Toben stillt.

\end{verse}
\end{multicols}

\begin{center}
\settowidth{\versewidth}{Ich hang und bleib auch hangen}
\begin{verse}[\versewidth]
  
\flagverse{9.} Er bringt mich an die Pforten,\\
die in den Himmel führt,\\
daran mit güldnen Worten\\
der Reim gelesen wird:\\
Wer dort wird mit verhöhnt,\\
wird hier auch mit gekrönt,\\
wer dort mit sterben geht,\\
wird hier auch mit erhöht.

\end{verse}
\end{center}
%\attrib{\small{THZE}}

\index{Auf auf, mein Herz mit Freuden}
\newpage
\subsection*{\centerline{Nun freut euch hier und überall}}
\addcontentsline{toc}{subsection}{Nun freut euch hier und überall}
%StartInfo%%%%%%%%%%%%%%%%%%%%%%%%%%%%%%%%%%%%%%%%%%%%%%%%%%%%%%%%%%%%%%%%%%%%
%  Autor:
%  Titel:
%  File:
%  Ref:
%  Mod:
%EndInfo%%%%%%%%%%%%%%%%%%%%%%%%%%%%%%%%%%%%%%%%%%%%%%%%%%%%%%%%%%%%%%%%%%%%%%
%\poemtitle{pt}
\begin{multicols}{2}
\settowidth{\versewidth}{der sprach: Habt Freud und Trost und seid}
\begin{verse}[\versewidth]

\flagverse{1.} Nun freut euch hier und überall,\\
ihr Christen, lieben Brüder!\\
Das Heil, das durch den Todesfall\\
gesunken, stehet wieder.\\
Des Lebens Leben lebet noch,\\
sein Arm hat aller Feinde Joch\\
mit aller Macht zerbrochen.

\flagverse{2.} Der Held, der alles hält, er lag\\
im Grab als überwunden,\\
er lag, bis daß der dritte Tag\\
sich in die Welt gefunden;\\
da dieser kam, kam auch die Zeit,\\
da, der uns in dem Tod erfreut,\\
sich aus dem Tod erhube.

\flagverse{3.} Die Morgenröte war noch nicht\\
mit ihrem Licht vorhanden,\\
und siehe, da war schon das Licht,\\
das ewig leucht', erstanden;\\
die Sonne war noch nicht erwacht,\\
da wacht und ging in voller Macht\\
die unerschaffne Sonne.

\flagverse{4.} Das wußte nicht die fromme Schar,\\
die Christo angehangen,\\
drum als nunmehr der Sabbat war\\
zum End hinabgegangen,\\
begunnt Maria Magdalen\\
und andre mit ihr auszugehn\\
und Spezerei zu kaufen.

\flagverse{5.} Ihr Herz und Hand ist hoch bemüht,\\
ein Salböl darzugeben\\
für Jesu, dessen teure Güt\\
uns salbt zum ewgen Leben.\\
Ach, liebes Herz, der seinen Geist\\
vom Himmel in die Herzen geußt,\\
darf keines Öls noch Salben.

\flagverse{6.} Ja du, o heilger Jungfrausohn,\\
bist schon gnug balsamieret\\
als König, der im Himmelsthron\\
und überall regieret!\\
Dein Balsam ist die ewge Kraft,\\
dadurch Gott Erd und Himmel schafft,\\
die läßt dich nicht verwesen.

\flagverse{7.} Doch geht die fromme Einfalt hin\\
bald in dem frühsten Morgen,\\
sie gehn, und plötzlich wird ihr Sinn\\
voll großer schwerer Sorgen.\\
Ei, sprechen sie, wer wälzt den Stein\\
vons Grabes Tür und läßt uns ein\\
zum Leichnam unsres Herren? –

\flagverse{8.} So sorgten sie zur selben Zeit\\
für das, was schon bestellet,\\
es war der Stein ja allbereit\\
erhoben und gefället\\
durch einen, der des Erdreichs Wucht\\
erbeben macht und in die Flucht\\
des Grabes Hüter jagte.

\flagverse{9.} Das war ein Diener aus der Höh,\\
von denen, die uns schützen,\\
sein Kleid war weißer als der Schnee,\\
sein Ansehn gleich den Blitzen,\\
der hat das fest verschlossne Grab\\
eröffnet und den Stein herab\\
vons Grabes Tür gewälzet.

\flagverse{10.} Das Weiberhäuflein kam und ging\\
hinein ohn alle Mühe.\\
Hör aber, was für Wunderding\\
sich da begab! Denn siehe,\\
das, was sie suchten, findt sich nicht\\
und wo ihr Herz nicht hingericht,\\
das ist allda zur Stelle.

\flagverse{11.} Sie suchten ihrer Seelen Hort\\
und finden sein Gesinde,\\
sie hören aus der Engel Wort\\
wies gar viel anders stünde,\\
als ihr betrübtes Herz gemeint:\\
Daß billig wer bisher geweint,\\
nun jauchzen soll und lachen.

\flagverse{12.} Sie sehn das Grab entledigt stehn,\\
und als sie das gesehen,\\
da läuft Maria Magdalen,\\
zu sagen, was geschehen.\\
Die andre Schar ist Kummers voll\\
und weiß nicht, was sie machen soll,\\
verharret bei dem Grabe.

\flagverse{13.} Da stellen sich in heller Zier\\
zween edle Himmelsboten,\\
die sprechen: Ei, was suchet ihr\\
das Leben bei den Toten?\\
Der Heiland lebt! Er ist nicht hie!\\
Heut ist er, glaubt uns, heute früh\\
ist er vom Tod erstanden.

\flagverse{14.} Gedenkt und sinnt ein wenig nach\\
den Reden, die er triebe,\\
da er so klar und deutlich sprach,\\
wie er zwar würd aus Liebe\\
den Tod ausstehn und große Plag,\\
jedennoch an dem dritten Tag\\
er herrlich triumphieren.

\flagverse{15.} Da dachten sie an Christi Wort\\
und gingen von dem Grabe\\
hin zu der elf Apostel Ort\\
und sagten, was sich habe\\
erzeigt in ihrem Angesicht;\\
man hielt es aber anders nicht,\\
als ob es Märlein wären.

\flagverse{16.} Maria, die betrübt', sich gibt\\
in schnelles Abescheiden,\\
findt Petrum und den Jesus liebt,\\
erzählet allen beiden:\\
Ach, spricht sie, unser Herr ist hin,\\
und niemand ist, der, wo man ihn\\
hab hingelegt, will wissen.

\flagverse{17.} Der Hochgeliebte läuft geschwind\\
und kommt zuerst zum Grabe;\\
er guckt, und da er nichts mehr findt\\
als Leinen, weicht er abe.\\
Da aber Simon Petrus kömmt,\\
geht er ins Grab hinein und nimmt\\
das Werk recht in die Augen.

\flagverse{18.} Er sieht die Leinen für sich dar,\\
zu voraus, wie mit Fleiße\\
gelegt und eingewickelt war\\
das Haupttuch zu dem Schweiße:\\
Da ging auch, der am ersten kam,\\
hinein, wie Petrus tät, und nahm,\\
was er da sah ins Herze.

\flagverse{19.} Da glauben sie nun dem Bericht,\\
weil sie mit Augen schauen,\\
was sie zuvor als ein Gedicht\\
gehöret von den Frauen;\\
doch werden sie Verwunderns voll,\\
denn keiner weiß, daß Christus soll\\
von Toten auferwachen.

\flagverse{20.} Maria steht vorm Grab und weint,\\
und plötzlich wird sie innen,\\
daß zween in weißen Kleidern seind\\
vor ihr im Grabe drinnen,\\
die sprechen: Weib, was weinest du?\\
Sie haben meines Herzens Ruh,\\
sprach sie, hinweggenommen.

\flagverse{21.} Mein Herr ist weg, und ich weiß nicht,\\
wo ich soll suchen gehen.\\
Indessen wendt sie ihr Gesicht\\
und siehet Jesum stehen.\\
Der spricht: O Weib, was fehlet dir?\\
Was weinest du, was suchst du hier? –\\
sie meint, der Gärtner rede.

\flagverse{22.} Ach, spricht sie, Herr, hast du's getan,\\
so sag es unverhohlen,\\
wo liegt mein Herr? wo komm ich an?\\
So will ich mir ihn holen.\\
Der Herr spricht mit gewohnter Stimm:\\
Maria! – Da wendt sie sich um\\
und spricht: Sieh da, Rabbuni!

\flagverse{23.} Rühr mich nicht an! Ich bin noch nicht\\
zum Vater aufgefahren,\\
geh aber hin, sprach unser Licht,\\
sags meiner Brüder Scharen:\\
Ich fahr als eures Todes Tod\\
zu meinem und zu eurem Gott\\
und unser aller Vater.

\flagverse{24.} Maria ist das arme Weib,\\
von welcher unser Meister,\\
der starke Helfer, vormals treib\\
auf einmal sieben Geister.\\
Die, die ists, welcher Jesus Christ\\
am ersten Mal erschienen ist\\
am heilgen Ostertage.

% weiter in der anderen Spalte:
\vfill\null
\columnbreak
 
\flagverse{25.} Nun, sie ging hin, täts denen kund,\\
die mit ihr Jesum liebten\\
und über ihn von Herzensgrund\\
sich grämten und betrübten.\\
Kein einzger aber fiel ihr bei,\\
ein jeder hielts für Fantasei,\\
und wollt es niemand glauben.

\flagverse{26.} Es gingen auch ins Grab hinein\\
die andre Schar der Frauen,\\
da gab sich ihrem Augenschein\\
ein Jüngling anzuschauen\\
in einem langen weißen Kleid,\\
der sprach: Habt Freud und Trost und seid\\
ohn alle Furcht und Schrecken.

\flagverse{27.} Ihr sucht den Held von Nazareth,\\
der doch hie nicht vorhanden;\\
seht, das ist seines Lagers Stätt,\\
von der er auferstanden.\\
Geht schnell, sagts Petro und der Zahl\\
der andern Jünger allzumal:\\
Ihr Herr und Meister lebe.

\flagverse{28.} Die Weiber eilen schnell davon,\\
den Jüngern Post zu bringen,\\
und siehe da, die Freudensonn,\\
nach der sie alle gingen,\\
die geht daher, und sehen sie\\
im Leben, den sie also früh\\
als einen Toten suchten.

\flagverse{29.} Sein süßer Mund macht all ihr Leid\\
mit seinem Grüßen süße,\\
sie treten zu mit großer Freud\\
und greifen seine Füße.\\
Er aber spricht: Seid guten Muts!\\
Geht hin, sagt meinen Brüdern Guts,\\
verrichtet, was ihr sahet.

\flagverse{30.} Sprecht, daß sie nunmehr also fort\\
in Galiläum gehen,\\
allda will ich, kraft meiner Wort,\\
vor ihren Augen stehen. –\\
und hiemit schloß er sein Gebot.\\
Die Weiber gehn und loben Gott,\\
verrichten, was befohlen.

\flagverse{31.} O Lebensfürst, o starker Leu\\
aus Judä Stamm erstanden,\\
so bist du nun wahrhaftig frei\\
von Todes Strick und Banden.\\
Du hast gesiegt und trägst zu Lohn\\
ein allzeit unverwelkte Kron\\
als Herr all deiner Feinde.

\flagverse{32.} Was fragst du nach des Teufels Spott\\
und ungereimten Klagen!\\
Man hat, spricht er und seine Rott,\\
ihn heimlich weggetragen.\\
Die Jünger haben ihn bei Nacht\\
gestohlen und bei Seit gebracht,\\
indem wir feste schliefen.

% weiter in der anderen Spalte:
\vfill\null
\columnbreak


\flagverse{33.} O Bosheit! War dein Schlaf so fest,\\
wie hast du können sehen?\\
Ist denn dein Auge wach gewest,\\
wie läßt du's so geschehen,\\
daß durch der Jünger schwache Hand\\
der Stein und seines Siegels Band\\
werd auf- und abgelöset?

\flagverse{34.} Es ist dein hart verstockter Sinn,\\
der dich zum Lügen leitet,\\
so fahr auch nun zum Abgrund hin,\\
da dir dein Lohn bereitet!\\
Ich aber will, Herr Jesu Christ,\\
so lang ein Leben in mir ist,\\
bekennen, daß du lebest.

\flagverse{35.} Ich will dich rühmen, wie du seist\\
die Pest und Gift der Höllen,\\
ich will auch, Herr, durch deinen Geist\\
mich dir zur Seiten stellen\\
und mit dir sterben, wie du stirbst,\\
und was du in dem Sieg erwirbst,\\
soll meine Beute bleiben.

\flagverse{36.} Ich will von Sünden auferstehn,\\
wie du vom Grab aufstehest:\\
Ich will zum andern Leben gehn,\\
wie du zum Himmel gehest.\\
Dies Leben ist doch lauter Tod,\\
drum komm und reiß aus aller Not\\
uns in das rechte Leben!

\end{verse}
\end{multicols}
%\attrib{\small{THZE}}

\index{Nun freut euch hier und überall}
\newpage
\subsection*{\centerline{Sei fröhlich alles weit und breit}}
\addcontentsline{toc}{subsection}{Sei fröhlich alles weit und breit}
%StartInfo%%%%%%%%%%%%%%%%%%%%%%%%%%%%%%%%%%%%%%%%%%%%%%%%%%%%%%%%%%%%%%%%%%%%
%  Autor:
%  Titel:
%  File:
%  Ref:
%  Mod:
%EndInfo%%%%%%%%%%%%%%%%%%%%%%%%%%%%%%%%%%%%%%%%%%%%%%%%%%%%%%%%%%%%%%%%%%%%%%
%\poemtitle{pt}
\begin{multicols}{2}
\settowidth{\versewidth}{Nein, nein! Er trägt sein Haupt empor,}
\begin{verse}[\versewidth]

\flagverse{1.} Sei fröhlich alles weit und breit,\\
was vormals war verloren,\\
weil heut der Herr der Herrlichkeit,\\
den Gott selbst auserkoren\\
zum Sündenbüßer, der sein Blut\\
am Kreuz vergossen uns zu gut,\\
vom Tod ist auferstanden.

\flagverse{2.} Wie schön hast du durch deine Macht,\\
du wilder Feind des Lebens,\\
den Lebensfürsten umgebracht:\\
Dein Stachel ist vergebens\\
durch ihn geschossen, schnöder Feind,\\
du hättest wahrlich wohl gemeint,\\
er würd im Staube bleiben.

\flagverse{3.} Nein, nein! Er trägt sein Haupt empor,\\
ist mächtig durchgedrungen\\
durch deine Bande, durch dein Tor,\\
ja hat im Sieg verschlungen\\
dich selbst, daß, wer an ihn nur gläubt,\\
von dir jetzt ein Gespötte treibt\\
und spricht: wo ist dein Stachel?

\flagverse{4.} Denn deine Macht, die ist dahin\\
und keinen Schaden bringet\\
dem, der sich stets mit Herz und Sinn\\
zu diesem Fürsten schwinget,\\
der fröhlich spricht: Ich leb, und ihr\\
sollt mit mir leben für und für,\\
weil ich es euch erworben.

\flagverse{5.} Der Tod hat keine Kraft nicht mehr,\\
ihr dürfet ihn nicht scheuen,\\
ich bin sein Siegsfürst und sein Herr,\\
des sollt ihr euch erfreuen.\\
Dazu so bin ich euer Haupt,\\
drum werdet ihr, wenn ihr mir glaubt,\\
als Glieder mit mir leben.

\flagverse{6.} Der Höllen Sieg, der ist auch mein,\\
ich habe sie zerstöret,\\
es darf nicht fürchten ihre Pein,\\
wer mich und mein Wort höret.\\
Und weil des Teufels Macht und List\\
gedämpft, sein Kopf zertreten ist,\\
mag er ihm auch nicht schaden.
\end{verse}
\end{multicols}

\begin{center}
\settowidth{\versewidth}{Der, vor dem die Welt erschrickt,}
\begin{verse}[\versewidth]

\flagverse{7.} Nun Gott sei Dank, der uns den Sieg\\
durch Jesum hat gegeben\\
und uns den Frieden für den Krieg\\
und für den Tod das Leben\\
erworben, der die Sünd und Tod,\\
Welt, Teufel, Höll und was in Not.
   
\end{verse}
\end{center}

%\attrib{\small{THZE}}

\index{Sei fröhlich alles weit und breit}
\newpage
\subsection*{\centerline{Gott Vater, sende deinen Geist}}
\addcontentsline{toc}{subsection}{Gott Vater sende deinen Geist}
%StartInfo%%%%%%%%%%%%%%%%%%%%%%%%%%%%%%%%%%%%%%%%%%%%%%%%%%%%%%%%%%%%%%%%%%%%
%  Autor:
%  Titel:
%  File:
%  Ref:
%  Mod:
%EndInfo%%%%%%%%%%%%%%%%%%%%%%%%%%%%%%%%%%%%%%%%%%%%%%%%%%%%%%%%%%%%%%%%%%%%%%
%\poemtitle{Gott Vater, sende deinen Geist}
\begin{multicols}{2}
\settowidth{\versewidth}{Kein Menschenkind hier auf der Erd}
\begin{verse}[\versewidth]

\flagverse{1.} Gott Vater, sende deinen Geist,\\
den uns dein Sohn erbitten heißt,\\
aus deines Himmels Höhen.\\
Wir bitten, wie er uns gelehrt:\\
Laß uns doch ja nicht unerhört\\
von deinem Throne gehen!

\flagverse{2.} Kein Menschenkind hier auf der Erd\\
ist dieser edlen Gabe wert,\\
bei uns ist kein Verdienen.\\
Hier gilt gar nichts als Lieb und Gnad,\\
die Christus uns verdienet hat\\
mit Büßen und Versühnen.

\flagverse{3.} Es jammert deinen Vatersinn\\
der große Jammer, da wir hin\\
durch Adams Fall gefallen.\\
Durch dieses Fallen ist die Macht\\
des bösen Geistes leider bracht\\
auf ihn und auf uns allen.

\flagverse{4.} Wir halten, Herr, an unserm Heil\\
und sind gewiß, daß wir dein Teil\\
in Christo werden bleiben,\\
die wir durch seinen Tod und Blut\\
des Himmels Erb und höchstes Gut\\
zu haben treulich gläuben.

\flagverse{5.} Und das ist auch ein Gnadenwerk\\
und deines heilgen Geistes Stärk,\\
in uns ist kein Vermögen.\\
Wie bald würd unser Glaub und Treu,\\
Herr, wo du uns nicht stündest bei,\\
sich in die Aschen legen.

\flagverse{6.} Dein Geist hält unsres Glaubens Licht,\\
wenn alle Welt dawider ficht\\
mit Sturm und vielen Waffen,\\
und wenn auch gleich der Fürst der Welt\\
selbst wider uns sich legt ins Feld,\\
so kann er doch nichts schaffen.

\flagverse{7.} Wo Gottes Geist ist, da ist Sieg,\\
wo dieser hilft, da wird der Krieg\\
gewißlich wohl ablaufen.\\
Was ist doch Satans Reich und Stand?\\
Wann Gottes Geist erhebt die Hand,\\
fällt alles übern Haufen.

\flagverse{8.} Er reißt der Höllen Band entzwei,\\
er tröst't und macht das Herze frei,\\
von allem, was uns kränket;\\
wenn uns des Unglücks Wetter schreckt,\\
so ist ers, der uns schützt und deckt\\
viel besser, als man denket.

\flagverse{9.} Er macht das bittre Kreuze süß,\\
ist unser Licht in Finsternis,\\
führt uns als seine Schafe,\\
hält über uns sein Schild und wacht,\\
daß seine Herd in tiefer Nacht\\
mit Ruh und Frieden schlafe.

\flagverse{10.} Den Geist, den Gott vom Himmel gibt,\\
der leitet alles, was ihn liebt,\\
auf wohlgebahnten Wegen,\\
er setzt und richtet unsern Fuß,\\
daß er nicht anders treten muß,\\
als wo man findt den Segen.

\flagverse{11.} Er macht geschickt und rüstet aus\\
die Diener, die des Herren Haus\\
in diesem Leben bauen;\\
er ziert ihr Herz, Mund und Verstand,\\
läßt ihnen, was uns unbekannt,\\
zu unserm Besten schauen.

\flagverse{12.} Er öffnet unsers Herzens Tor,\\
wenn sie sein Wort in unser Ohr\\
als edlen Samen streuen,\\
er gibet Kraft demselben Wort,\\
und wenn es fället, bringt ers fort\\
und lässets wohl gedeihen.

\flagverse{13.} Er lehret uns die Furcht des Herrn,\\
liebt Reinigkeit und wohnet gern\\
in frommen keuschen Seelen.\\
Was niedrig ist, was Tugend ehrt,\\
was Buße tut und sich bekehrt,\\
das pflegt er zu erwählen.

\flagverse{14.} Er ist und bleibet stets getreu,\\
er steht uns auch im Tode bei,\\
wenn alle Ding abstehen;\\
er lindert unsre letzte Qual,\\
läßt uns hindurch ins Himmels Saal\\
getrost und fröhlich gehen.

\flagverse{15.} O selig, wer in dieser Welt\\
läßt diesem Gaste Haus und Zelt\\
in seiner Seel aufschlagen!\\
Wer ihn aufnimmt in dieser Zeit,\\
den wird er dort zur ewgen Freud\\
in Gottes Hütte tragen.

\flagverse{16.} Nun, Herr und Vater aller Güt,\\
hör unsern Wunsch: Geuß ins Gemüt\\
uns allen diese Gabe!\\
Gib deinen Geist, der uns allhier\\
regiere und dort für und für\\
im ewgen Leben labe!

\end{verse}
\end{multicols}
%\attrib{\small{THZE}}

\index{Gott Vater, sende deinen Geist}
\newpage
\subsection*{\centerline{O du allersüß'ste Freude}}
\addcontentsline{toc}{subsection}{O du allersüß'ste Freude}
%StartInfo%%%%%%%%%%%%%%%%%%%%%%%%%%%%%%%%%%%%%%%%%%%%%%%%%%%%%%%%%%%%%%%%%%%%
%  Autor:
%  Titel:
%  File:
%  Ref:
%  Mod:
%EndInfo%%%%%%%%%%%%%%%%%%%%%%%%%%%%%%%%%%%%%%%%%%%%%%%%%%%%%%%%%%%%%%%%%%%%%%
%\poemtitle{pt}
\begin{multicols}{2}
\settowidth{\versewidth}{Du wirst aus des Himmels Throne}
\begin{verse}[\versewidth]

\flagverse{1.} O du allersüß'ste Freude!\\
O du allerschönstes Licht!\\
Der du uns in Lieb und Leide\\
unbesuchet lässest nicht,\\
Geist des Höchsten! Höchster Fürst,\\
der du hältst und halten wirst\\
ohn Aufhören alle Dinge,\\
höre, höre, was ich singe!

\flagverse{2.} Du bist ja die beste Gabe,\\
die ein Mensche nennen kann;\\
wenn ich dich erwünsch und habe,\\
geb ich alles Wünschen an.\\
Ach, ergib dich, komm zu mir\\
in mein Herze, das du dir,\\
da ich in die Welt geboren,\\
selbst zum Tempel auserkoren.

\flagverse{3.} Du wirst aus des Himmels Throne\\
wie ein Regen ausgeschütt,\\
bringst vom Vater und vom Sohne\\
nichts als lauter Segen mit;\\
laß doch, o du werter Gast,\\
Gottes Segen, den du hast\\
und verwaltst nach deinem Willen,\\
mich an Leib und Seele füllen.

\flagverse{4.} Du bist weis und voll Verstandes,\\
was geheim ist, ist dir kund,\\
zählst den Staub des kleinen Sandes,\\
gründst des tiefen Meeres Grund.\\
Nun, du weißt auch zweifelsfrei,\\
wie verderbt und blind ich sei;\\
drum gib Weisheit und vor allen,\\
wie ich möge Gott gefallen.

\flagverse{5.} Du bist heilig, läßt dich finden,\\
wo man rein und sauber ist,\\
fleuchst hingegen Schand und Sünden,\\
wie die Tauben Stank und Mist.\\
Mache mich, o Gnadenquell,\\
durch dein Waschen rein und hell;\\
laß mich fliehen, was du fliehest,\\
gib mir, was du gerne siehest.

\flagverse{6.} Du bist, wie ein Schäflein pfleget,\\
frommes Herzens, sanftes Muts,\\
bleibst im Lieben unbeweget,\\
tust uns Bösen alles Guts.\\
Ach, verleih und gib mir auch\\
diesen edlen Sinn und Brauch,\\
daß ich Freund und Feinde liebe,\\
keinen, den du liebst, betrübe.

\flagverse{7.} Mein Hort, ich bin wohl zufrieden,\\
wenn du mich nur nicht verstößt,\\
bleib ich von dir ungeschieden,\\
ei, so bin ich gnug getröst.\\
Laß mich sein dein Eigentum,\\
ich versprech hinwiederum,\\
hier und dort all mein Vermögen\\
dir zu Ehren anzulegen.

\flagverse{8.} Ich entsage alle deme,\\
was dir deinen Ruhm benimmt,\\
ich will, daß mein Herz annehme\\
nun allein, was von dir kömmt.\\
Was der Satan will und sucht,\\
will ich halten als verflucht,\\
ich will seinen schnöden Wegen\\
mich mit Ernst zuwiderlegen.

\flagverse{9.} Nur allein daß du mich stärkest\\
und mir treulich stehest bei;\\
hilf, mein Helfer, wo du merkest,\\
daß mir Hilfe nötig sei.\\
Brich des bösen Fleisches Sinn,\\
nimm den alten Willen hin,\\
mach ihn allerdinge neue,\\
daß sich mein Gott meiner freue.

\flagverse{10.} Sei mein Retter! Halt mich eben;\\
wenn ich sinke, sei mein Stab!\\
Wenn ich sterbe, sein mein Leben,\\
wenn ich liege, sei mein Grab!\\
Wenn ich wieder aufersteh,\\
ei, so hilf mir, daß ich geh\\
hin, da du in ewgen Freuden\\
wirst dein' Auserwählten weiden.

\end{verse}
\end{multicols}
%\attrib{\small{THZE}}

\index{O du allersüßte Freude}
\newpage
\subsection*{\centerline{Zeuch ein zu deinen Toren}}
\addcontentsline{toc}{subsection}{Zeuch ein zu deinen Toren}
%StartInfo%%%%%%%%%%%%%%%%%%%%%%%%%%%%%%%%%%%%%%%%%%%%%%%%%%%%%%%%%%%%%%%%%%%%
%  Autor:
%  Titel:
%  File:
%  Ref:
%  Mod:
%EndInfo%%%%%%%%%%%%%%%%%%%%%%%%%%%%%%%%%%%%%%%%%%%%%%%%%%%%%%%%%%%%%%%%%%%%%%
%\poemtitle{pt}
\begin{multicols}{2}
\settowidth{\versewidth}{Zeuch ein, laß mich empfinden}
\begin{verse}[\versewidth]

\flagverse{1.} Zeuch ein zu deinen Toren,\\
sei meines Herzens Gast,\\
der du, da ich geboren,\\
mich neu geboren hast,\\
o hochgeliebter Geist\\
des Vaters und des Sohnes,\\
mit beiden gleichen Thrones,\\
mit beiden gleich gespeist.

\flagverse{2.} Zeuch ein, laß mich empfinden\\
und schmecken deine Kraft,\\
die Kraft, die uns von Sünden\\
hilf und Errettung schafft.\\
Entsündge meinen Sinn,\\
daß ich mit reinem Geiste\\
dir Ehr und Dienste leiste,\\
die ich dir schuldig bin.

\flagverse{3.} Ich war ein wilder Reben,\\
du hast mich gut gemacht,\\
der Tod durchdrang mein Leben,\\
du hast ihn umgebracht\\
und in der Tauf erstickt,\\
als wie in einer Flute,\\
mit dessen Tod und Blute,\\
der uns im Tod erquickt.

\flagverse{4.} Du bist das heilig Öle,\\
dadurch gesalbet ist\\
mein Leib und meine Seele\\
dem Herren Jesu Christ\\
zum wahren Eigentum,\\
zum Priester und Propheten,\\
zum Könge, den in Nöten\\
Gott schützt vom Heiligtum.

\flagverse{5.} Du bist ein Geist, der lehret,\\
wie man recht beten soll,\\
dein Beten wird erhöret,\\
dein Singen klinget wohl.\\
Es steigt zum Himmel an,\\
es steigt und läßt nicht abe,\\
bis der geholfen habe,\\
der allen helfen kann.

\flagverse{6.} Du bist ein Geist der Freuden,\\
von Trauern hältst du nicht,\\
erleuchtest uns im Leiden\\
mit deines Trostes Licht.\\
Ach ja, wie manchesmal\\
hast du mit süßen Worten\\
mir aufgetan die Pforten\\
zum güldnen Freudensaal.

\flagverse{7.} Du bist ein Geist der Liebe,\\
ein Freund der Freundlichkeit,\\
willst nicht, daß uns betrübe\\
Zorn, Zank, Haß, Neid und Streit;\\
der Feindschaft bist du feind,\\
willst, daß durch Liebesflammen\\
sich wieder tun zusammen\\
die voller Zwietracht seind.

\flagverse{8.} Du, Herr, hast selbst in Händen\\
die ganze weite Welt,\\
kannst Menschenherzen wenden,\\
wie es dir wohlgefällt:\\
So gib doch deine Gnad\\
zum Fried und Liebesbanden,\\
verknüpf in allen Landen,\\
was sich getrennet hat.

\flagverse{9.} Ach, edle Friedensquelle,\\
schleuß deinen Abgrund auf\\
und gib dem Frieden schnelle\\
hier wieder seinen Lauf.\\
Halt ein die große Flut,\\
die Flut, die eingerissen\\
so, daß man siehet fließen,\\
wie Wasser, Menschenblut.

\flagverse{10.} Laß deinem Volk erkennen\\
die Vielheit seiner Sünd,\\
auch Gottes Grimm so brennen,\\
daß er bei uns entzünd\\
den ernsten bittern Schmerz\\
und Buße, die bereuet,\\
des sich zuerst gefreuet\\
ein weltergebnes Herz.

\flagverse{11.} Auf Buße folgt der Gnaden,\\
auf Reu der Freuden Blick,\\
sich bessern heilt den Schaden,\\
fromm werden bringet Glück.\\
Herr, tus zu deiner Ehr,\\
erweiche Stahl und Steine,\\
auf daß das Herze weine,\\
das böse sich bekehr.

\flagverse{12.} Erhebe dich und steure\\
dem Herzleid auf der Erd,\\
bring wieder und erneuere\\
die Wohlfahrt deiner Herd!\\
Laß blühen wie zuvorn\\
die Länder, so verheeret,\\
die Kirchen, so zerstöret\\
durch Krieg und Feuerszorn.

\flagverse{13.} Beschirm die Polizeien,\\
bau unsers Fürsten Thron,\\
daß er und wir gedeihen,\\
schmück, als mit einer Kron,\\
die Alten mit Verstand,\\
mit Frömmigkeit die Jugend,\\
mit Gottesfurcht und Tugend\\
das Volk im ganzen Land.

\flagverse{14.} Erfülle die Gemüter\\
mit reiner Glaubenszier,\\
die Häuser und die Güter\\
mit Segen für und für.\\
Vertreib den bösen Geist,\\
der dir sich widersetzet\\
und was dein Herz ergötzet,\\
aus unserm Herzen reißt.

\flagverse{15.} Gib Freudigkeit und Stärke,\\
zu stehen in dem Streit,\\
den Satans Reich und Werke\\
uns täglich anerbeut,\\
hilf kämpfen ritterlich,\\
damit wir überwinden\\
und ja zum Dienst der Sünden\\
kein Christ ergebe sich.

\flagverse{16.} Richt unser ganzes Leben\\
allzeit nach deinem Sinn,\\
und wenn wirs sollen geben\\
in Todes Rachen hin,\\
wenns mit uns hie wird aus,\\
so hilf uns fröhlich sterben\\
und nach dem Tod ererben\\
des ew'gen Lebens Haus.

\end{verse}
\end{multicols}
%\attrib{\small{THZE}}

\index{Zeuch ein zu deinen Toren}
\newpage
\subsection*{\centerline{Was alle Weisheit in der Welt}}
\addcontentsline{toc}{subsection}{Was alle Weisheit in der Welt}
%StartInfo%%%%%%%%%%%%%%%%%%%%%%%%%%%%%%%%%%%%%%%%%%%%%%%%%%%%%%%%%%%%%%%%%%%%
%  Autor:
%  Titel:
%  File:
%  Ref:
%  Mod:
%EndInfo%%%%%%%%%%%%%%%%%%%%%%%%%%%%%%%%%%%%%%%%%%%%%%%%%%%%%%%%%%%%%%%%%%%%%%
%\poemtitle{pt}
\begin{multicols}{2}
\settowidth{\versewidth}{und treu der Frommen Schutz und Retter,}
\begin{verse}[\versewidth]

\flagverse{1.} Was alle Weisheit in der Welt\\
bei uns hier kann kaum lallen,\\
das läßt Gott aus dem Himmelszelt\\
in alle Welt erschallen:\\
Daß er alleine König sei,\\
hoch über alle Götter,\\
groß, mächtig, freundlich, fromm \\
und treu der Frommen Schutz und Retter,\\
ein Wesen, drei Personen.

\flagverse{2.} Gott Vater, Sohn und heilger Geist\\
heißt sein hochheilger Name,\\
so kennt, so nennt, so rühmt und preist\\
ihn der gerechte Same,\\
Gott Abraham, Gott Isaak,\\
Gott Jacob, den er liebet,\\
herr Zebaoth, der Tag und Nacht\\
uns alle Gaben gibet\\
und Wunder tut alleine.

\flagverse{3.} Der Vater hat von Ewigkeit\\
den Sohn, sein Bild, erzeuget;\\
der Sohn hat in der Füll der Zeit\\
im Fleische sich gezeiget.\\
Der Geist geht ohne Zeit herfür\\
vom Vater und vom Sohne,\\
mit beiden gleicher Ehr und Zier,\\
gleich ewig, gleicher Krone\\
und ungeteilter Stärke.

\flagverse{4.} Sieh hier, mein Herz, das ist dein Gut,\\
dein Schatz, dem keiner gleichet!\\
Das ist dein Freund, der alles tut,\\
was dir zum Heil gereichet,\\
der dich gebaut nach seinem Bild,\\
für deine Schuld gebüßet,\\
der dich mit wahrem Glauben füllt\\
und all dein Kreuz durchsüßet\\
mit seinen heilgen Worten.

\flagverse{5.} Erhebe dich! Steig zu ihm zu\\
und lern ihn recht erkennen!\\
Denn solch Erkenntnis bringt dir Ruh\\
und macht die Seele brennen\\
in reiner Liebe, die uns nährt\\
zum ewgen Freudenleben,\\
da, was allhier kein Ohr gehört,\\
Gott wird zu schauen geben\\
den Augen seiner Kinder.

\flagverse{6.} Weh aber dem verstockten Heer,\\
das sich hie selbst verblendet,\\
Gott von sich stößt und seine Ehr\\
auf Kreaturen wendet!\\
Dem wird gewiß des Himmels Tür\\
einmal verschlossen bleiben;\\
denn wer Gott von sich treibt allhier,\\
den wird er dort auch treiben\\
von seinem heilgen Throne.

\flagverse{7.} Ei nun so gib, du großer Held,\\
Gott Himmels und der Erden,\\
daß alle Menschen in der Welt\\
zu dir bekehret werden.\\
Erleuchte, was verblendet geht,\\
bring wieder, was verirret,\\
reiß aus, was uns im Wege steht\\
und freventlich verwirret\\
die Schwachen in dem Glauben,

\flagverse{8.} Auf daß wir also allzugleich\\
zur Himmelspforte dringen\\
und dermaleinst in deinem Reich\\
ohn alles Ende singen,\\
daß du alleine König seist\\
hoch über alle Götter,\\
Gott Vater, Sohn und heilger Geist,\\
der Frommen Schutz und Retter\\
ein Wesen, drei Personen.

\end{verse}
\end{multicols}
%\attrib{\small{THZE}}

\index{Was aller Weisheit in der Welt}
\newpage

%\section*{\centerline{\LARGE PFINGSTEN}}
%\centerline{\scshape Gott Vater, sende deinen Geist}
%\centerline{\scshape O du allersüß'ste Freude}
%\centerline{\scshape Zeuch ein zu deinen Toren}

%\newpage

%
%\subsection*{Gott Vater, sende deinen Geist}
%\subsection*{O du allersüß'ste Freude}
%\subsection*{Zeuch ein zu deinen Toren}


%\section*{\centerline{Trinitatis}}
%\subsection*{ Was alle Weisheit in der Welt}

\section*{\centerline{\LARGE TAUFE - ABENDMAHL - BEICHTE}}
\addcontentsline{toc}{section}{TAUFE - ABENDMAHl - BEICHTE}
\rule{\textwidth}{0.2pt}\vspace*{-\baselineskip}\vspace{3.2pt}
\rule{\textwidth}{1.2pt}\\[\baselineskip]

\index{ Gedichte zu Taufe und Abendmahl}

\centerline{\scshape Du Volk, das du getaufet bist }
\vspace*{2\baselineskip}
\centerline{\scshape Herr Jesu, meine Liebe }
\vspace*{2\baselineskip}
\centerline{\scshape Herr, höre, was mein Mund }
\vspace*{2\baselineskip}
\centerline{\scshape Herr, ich will gar gerne bleiben }
\vspace*{2\baselineskip}
\centerline{\scshape Weg, mein Herz, mit den Gedanken }

\newpage

\subsection*{\centerline{Du Volk, das du getaufet bist}}
\addcontentsline{toc}{subsection}{Du Volk das du getaufet bist}
%StartInfo%%%%%%%%%%%%%%%%%%%%%%%%%%%%%%%%%%%%%%%%%%%%%%%%%%%%%%%%%%%%%%%%%%%%
%  Autor:
%  Titel:
%  File:
%  Ref:
%  Mod:
%EndInfo%%%%%%%%%%%%%%%%%%%%%%%%%%%%%%%%%%%%%%%%%%%%%%%%%%%%%%%%%%%%%%%%%%%%%%
%\poemtitle{Du Volk, das du getaufet bist}
\begin{multicols}{2}
\settowidth{\versewidth}{Dein Leib und Seel war mit der Sünd}
\begin{verse}[\versewidth]
%du Volk, das du getaufet bist – Nimms wohl in acht

\flagverse{1.} Du Volk, das du getaufet bist\\
und deinen Gott erkennest,\\
auch nach dem Namen Jesu Christ\\
dich und die Deinen nennest,\\
nimms wohl in Acht und denke dran,\\
wie viel dir Gutes sei getan\\
am Tage deiner Taufe.

\flagverse{2.} Du warst, noch eh du wurdst geborn\\
und eh du Milch gesogen,\\
verdammt, verstoßen und verlorn,\\
darum daß du gezogen\\
aus deiner Eltern Fleisch und Blut\\
ein Art, die sich vom höchsten Gut,\\
dem ewgen Gott, stets wendet.

\flagverse{3.} Dein Leib und Seel war mit der Sünd\\
als einem Gift durchkrochen,\\
und du warst nicht mehr Gottes Kind,\\
nachdem der Bund gebrochen,\\
den unser Schöpfer aufgericht,\\
da er uns seines Bildes Licht\\
und herrlichs Kleid erteilet.

\flagverse{4.} Der Zorn, der Fluch, der ewge Tod,\\
und was in diesen allen\\
enthalten ist für Angst und Not,\\
das war auf dich gefallen;\\
du warst des Satans Sklav und Knecht,\\
der hielt dich fest nach seinem Recht\\
in seinem Reich gefangen.

\flagverse{5.} Das alles hebt auf einmal auf\\
und schlägt und drückt es nieder\\
das Wasserbad der heilgen Tauf,\\
ersetzt dagegen wieder,\\
was Adam hat verderbt gemacht\\
und was wir selbsten durchgebracht\\
bei unserm bösen Wesen.

\flagverse{6.} Es macht dies Bad von Sünden los\\
und gibt die rechte Schöne.\\
Die Satans Kerker vor beschloß,\\
die werden frei und Söhne\\
des, der da trägt die höchste Kron;\\
der läßt sie, was sein einger Sohn\\
ererbt, auch mit ihm erben.

\flagverse{7.} Was von Natur vermaledeit\\
und mit dem Fluch umfangen,\\
das wird hier in der Tauf erneut,\\
den Segen zu erlangen.\\
Hier stirbt der Tod und würgt nicht mehr,\\
hier bricht die Höll, und all ihr Heer\\
muß uns zu Füßen liegen.

\flagverse{8.} Hier ziehn wir Jesum Christum an\\
und decken unsre Schanden\\
mit dem, was er für uns getan\\
und willig ausgestanden;\\
hier wäscht uns sein hochteures Blut\\
und macht uns heilig, fromm und gut\\
in seines Vaters Augen.

\flagverse{9.} O großes Werk! O heilges Bad,\\
o Wasser, dessengleichen\\
man in der ganzen Welt nicht hat,\\
kein Sinn kann dich erreichen!\\
Du hast recht eine Wunderkraft,\\
und die hat der, so alles schafft,\\
dir durch sein Wort geschenket.

\flagverse{10.} Du bist kein schlichtes Wasser nicht,\\
wies unsre Brunnen geben:\\
Was Gott mit seinem Munde spricht,\\
das hast du in dir leben.\\
Du bist ein Wasser, das den Geist\\
des Allerhöchsten in sich schleußt\\
und seinen großen Namen.

\flagverse{11.} Das halt, o Mensch, in allem wert\\
und danke für die Gaben,\\
die dein Gott dir darin beschert\\
und die uns alle laben,\\
wenn nichts mehr sonst uns laben will,\\
die laß, bis daß des Todes Ziel\\
dich trifft, nicht ungepreiset.

\flagverse{12.} Brauch alles wohl, und weil du bist\\
nun rein in Christo worden,\\
so leb und tu auch als ein Christ\\
und halte Christi Orden,\\
bis daß dort in der ewgen Freud\\
er dir das Ehr- und Freudenkleid\\
um deine Seele lege!

\end{verse}
\end{multicols}
%\attrib{\small{THZE}}

\index{Du Volk, das du getaufet bist}
\newpage
\subsection*{\centerline{Herr Jesu, meine Liebe}}
\addcontentsline{toc}{subsection}{Herr Jesu meine Liebe}
%StartInfo%%%%%%%%%%%%%%%%%%%%%%%%%%%%%%%%%%%%%%%%%%%%%%%%%%%%%%%%%%%%%%%%%%%%
%  Autor:
%  Titel:
%  File:
%  Ref:
%  Mod:
%EndInfo%%%%%%%%%%%%%%%%%%%%%%%%%%%%%%%%%%%%%%%%%%%%%%%%%%%%%%%%%%%%%%%%%%%%%%
%\poemtitle{Herr Jesu, meine Liebe}
\begin{multicols}{2}
\settowidth{\versewidth}{Nun weißt du meine Plagen}
\begin{verse}[\versewidth]


\flagverse{1.} Herr Jesu, meine Liebe,\\
ich hätte nimmer Ruh und Rast,\\
wo nicht fest in mir bliebe\\
was du für mich geleistet hast;\\
es müßt in meinen Sünden,\\
die sich sehr hoch erhöhn,\\
all meine Kraft verschwinden\\
und wie ein Rauch vergehn,\\
wenn sich mein Herz nicht hielte\\
zu dir und deinem Tod,\\
und ich nicht stets mich kühlte\\
an deines Leidens Not.

\flagverse{2.} Nun weißt du meine Plagen\\
und Satans, meines Feindes, List.\\
Wenn Meinen Geist zu nagen,\\
er emsig und bemühet ist,\\
da hat er tausend Künste,\\
von dir mich abzuziehn:\\
Bald treibt er mir die Dünste\\
des Zweifels in den Sinn,\\
bald nimmt er mir dein Meinen\\
und Wollen aus der Acht\\
und lehrt mich ganz verneinen,\\
was du doch fest gemacht.

\flagverse{3.} Solch Unheil abzuweisen,\\
hast du, Herr, deinen Tisch gesetzt,\\
da lässest du mich speisen,\\
so daß sich Mark und Bein ergötzt.\\
Du reichst mir zu genießen\\
dein teures Fleisch und Blut\\
und lässet Worte fließen,\\
da all mein Herz auf ruht.\\
Komm, sprichst du, komm und nahe\\
dich ungescheut zu mir,\\
was ich dir geb, empfahe\\
und nimms getrost zu dir.

\flagverse{4.} Hier ist beim Brot vorhanden\\
mein Leib, der dargegeben wird\\
zum Tod- und Kreuzesbanden\\
für dich, der sich von mir verirrt.\\
Beim Wein ist, was geflossen\\
zu Tilgung deiner Schuld,\\
mein Blut, das ich vergossen\\
in Sanftmut und Geduld.\\
Nimms beides mit dem Munde\\
und denk auch mit darbei,\\
wie fromm im Herzensgrunde\\
ich, dein Erlöser, sei.

\flagverse{5.} Herr, ich will dein gedenken,\\
so lang ich Luft und Leben hab,\\
und bis man mich wird senken\\
an meinem End ins finstre Grab.\\
Ich sehe dein Verlangen\\
nach einem ewgen Heil,\\
am Holz bist du gehangen\\
und hast so manchen Pfeil\\
des Trübsals lassen dringen\\
in dein unschuldigs Herz,\\
auf daß ich möcht entspringen\\
des Todes Pein und Schmerz.

\flagverse{6.} So hast du auch befohlen,\\
daß, was den Glauben stärken kann,\\
ich bei dir solle holen,\\
und soll doch ja nicht zweifeln dran,\\
du habst für alle Sünden,\\
die in der ganzen Welt\\
bei Menschen je zu finden,\\
ein völligs Lösegeld\\
und Opfer, das bestehet\\
vor dem, der alles trägt,\\
in dem auch alles gehet,\\
bezahlet und erlegt.

\flagverse{7.} Und daß ja mein Gedanke,\\
der voller Falschheit und Betrug,\\
nicht im geringsten wanke,\\
als wär es dir nicht Ernst genug:\\
So neigst du dein Gemüte\\
zusamt der rechten Hand\\
und gibst mit großer Güte\\
mir das hochwerte Pfand\\
zu essen und zu trinken.\\
Ist das nicht Trost und Licht\\
dem, der sich läßt bedünken,\\
du wollest seiner nicht?

\flagverse{8.} Ach Herr, du willst uns alle,\\
das sagt uns unser Herze zu,\\
die, so der Feind zu Falle\\
gebracht, rufst du zu deiner Ruh.\\
Ach hilf, Herr, hilf uns eilen\\
zu dir, der jederzeit\\
uns allesamt zu heilen\\
geneigt ist und bereit!\\
Gib Lust und heilges Dürsten\\
nach deinem Abendmahl,\\
und dort mach uns zu Fürsten\\
im güldnen Himmelssaal.

\end{verse}
\end{multicols}
%\attrib{\small{THZE}}

\index{Herr Jesu, meine Liebe}
\newpage
\subsection*{\centerline{Herr, höre, was mein Mund}}
\addcontentsline{toc}{subsection}{Herr höre was mein Mund}
%StartInfo%%%%%%%%%%%%%%%%%%%%%%%%%%%%%%%%%%%%%%%%%%%%%%%%%%%%%%%%%%%%%%%%%%%%
%  Autor:
%  Titel:
%  File:
%  Ref:
%  Mod:
%EndInfo%%%%%%%%%%%%%%%%%%%%%%%%%%%%%%%%%%%%%%%%%%%%%%%%%%%%%%%%%%%%%%%%%%%%%%
%\poemtitle{Herr, höre, was mein Mund spricht}
\begin{multicols}{2}
\settowidth{\versewidth}{Herr, höre, was mein Mund }
\begin{verse}[\versewidth]
%Der 143. Psalm


\flagverse{1.} Herr, höre, was mein Mund\\
aus innerm Herzensgrund\\
ohn alle Falschheit spricht,\\
wend, Herr, dein Angesicht,\\
vernimm meine Bitte!

\flagverse{2.} Ich bitte nicht um Gut,\\
das auf der Welt beruht,\\
auch endlich mit der Welt\\
bricht und zu Boden fällt\\
und mag gar nicht retten.

\flagverse{3.} Der Schatz, den ich begehr,\\
ist deine Gnad, o Herr,\\
die Gnade, die dein Sohn,\\
mein Heil und Gnadenthron,\\
mir sterbend erworben.

\flagverse{4.} Du bist rein und gerecht,\\
ich bin ein böser Knecht,\\
ich bin in Sünden tot,\\
du bist der fromme Gott,\\
der Sünde vergibet.

\flagverse{5.} Laß deine Frömmigkeit\\
sein meinen Trost und Freud,\\
laß über meine Schuld\\
dein edle Lieb und Huld\\
sich reichlich ergießen.

\flagverse{6.} Betrachte, wer ich bin,\\
im Hui fahr ich dahin,\\
zerbrechlich wie ein Glas,\\
vergänglich wie ein Gras,\\
ein Wind kann mich fällen.

\flagverse{7.} Willst du nichts sehen an\\
als was ein Mensch getan,\\
so wird kein Menschenkind\\
von wegen seiner Sünd\\
im Himmel bestehen.

\flagverse{8.} Sieh an, wie Jesus Christ\\
für mich gegeben ist,\\
der hat, was ich nicht kann,\\
erfüllt und gnug getan\\
im Leben und im Leiden.

\flagverse{9.} Du liebest Reu und Schmerz,\\
schau her, hier ist mein Herz,\\
das seine Sünd erkennt\\
und wie im Feuer brennt\\
vor Angst, Leid und Sorgen.

\flagverse{10.} Ich lechze wie ein Land,\\
dem deine milde Hand\\
den Regen lang entzeucht,\\
bis Saft und Kraft entweicht\\
und alles verdorret.

\flagverse{11.} Gleich wie auch auf der Heid\\
ein Hirsch begehrlich schreit\\
nach frischem Wasserquell,\\
so ruf ich laut und hell\\
nach dir, o mein Leben.

\flagverse{12.} Erquicke mein Gebein,\\
geuß Trost und Labsal ein\\
und sprich mir freundlich zu,\\
daß meine Seele ruh\\
im Schoß deiner Liebe.

\flagverse{13.} Gib mir getrosten Mut,\\
wenn meiner Sünden Flut\\
aufsteiget in die Höh,\\
ersäuf all Angst und Weh\\
im Meer deiner Gnaden.

\flagverse{14.} Treib weg den bösen Feind,\\
der mich zu stürzen meint,\\
du bist mein Hirt, und ich\\
will bleiben ewiglich\\
ein Schaf deiner Weide.

\flagverse{15.} So lang auf dieser Erd\\
ich Atem holen werd,\\
o Herr, so will ich dein\\
und deines Willens sein\\
gehorsamer Diener.

\flagverse{16.} Ich will dir dankbar sein,\\
doch ist mein Können klein,\\
allein in deiner Kraft,\\
die Tun und Wollen schafft,\\
steht all mein Vermögen.

\flagverse{17.} Drum sende deinen Geist,\\
der deinen Kindern weist\\
den Weg, der dir gefällt;\\
wer den bewahrt und hält,\\
wird nimmermehr fehlen.

\flagverse{18.} Ich richte mich nach dir,\\
du sollst mir gehen für.\\
Du sollst mir schließen auf\\
die Bahn im Tugendlauf,\\
ich will treulich folgen.

\end{verse}
\end{multicols}


\begin{center}
\settowidth{\versewidth}{Und wenn des Himmels Pfort}
\begin{verse}[\versewidth]

\flagverse{19.} Und wenn des Himmels Pfort\\
ich werd ergreifen dort,\\
so will im Engelheer\\
ich ewig deiner Ehr\\
in Freuden lobsingen.  
  
\end{verse}
\end{center}

%\attrib{\small{THZE}}

\index{Herr, höre, was mein Mund}
\newpage
\subsection*{\centerline{Herr, ich will gar gerne bleiben}}
\addcontentsline{toc}{subsection}{Herr ich will gar gerne bleiben}
%StartInfo%%%%%%%%%%%%%%%%%%%%%%%%%%%%%%%%%%%%%%%%%%%%%%%%%%%%%%%%%%%%%%%%%%%%
%  Autor:
%  Titel:
%  File:
%  Ref:
%  Mod: 05.11.2017 Spell check
%EndInfo%%%%%%%%%%%%%%%%%%%%%%%%%%%%%%%%%%%%%%%%%%%%%%%%%%%%%%%%%%%%%%%%%%%%%%
%\poemtitle{Herr, ich will gar gerne bleiben}
\begin{multicols}{2}
\settowidth{\versewidth}{Herr, ich will gar gerne bleiben,}
\begin{verse}[\versewidth]
%Selbsterniedrigung\\
%(Matth. 15, 21 u. Mark. 7, 28)\\
%nach den lateinischen Distichen des Nathan Chyträus »Sum canis indignus« 1568

\flagverse{1.} Herr, ich will gar gerne bleiben,\\
wie ich bin, dein armer Hund,\\
will auch anders nicht beschreiben\\
mich und meines Herzens Grund.\\
Denn ich fühle, was ich sei:\\
Alles Böse wohnt mir bei,\\
ich bin aller Schand ergeben,\\
unrein ist mein ganzes Leben.

\flagverse{2.} Hündisch ist mein Zorn und Eifer,\\
hündisch ist mein Neid und Haß,\\
hündisch ist mein Zank und Geifer,\\
hündisch ist mein Raub und Fraß;\\
ja, wenn ich mich recht genau,\\
als ich billig soll, beschau,\\
halt ich mich in vielen Sachen\\
ärger, als die Hund es machen.

\flagverse{3.} Ich will auch nicht mehr begehren,\\
als mir zukommt und gebührt,\\
wollst mich nur des Rechts gewähren,\\
das ein Hund im Hause führt!\\
Deine Heilgen, die sich dir\\
hier ergeben für und für,\\
mögen oben an der Spitzen\\
deiner Himmelstafel sitzen.

\flagverse{4.} Deine Kinder, die dich ehren\\
und in voller Tugend stehn,\\
mögen sich von Wollust nähren\\
und im Erbe sich erhöhn,\\
das du ihnen in dem Licht\\
deines Saals hast zugericht't,\\
ich will, wenn ich nur kann liegen\\
unterm Tisch, mir lassen gnügen.

\flagverse{5.} Ich will ins Verborgne kriechen,\\
da die Nacht den Tag verhüllt,\\
und hin nach der Erden riechen,\\
suchen, was den Hunger stillt;\\
ich will mit den Brosamlein,\\
die ich finde, friedlich sein\\
und mich freuen über allen,\\
was die Herren lassen fallen.

\flagverse{6.} Murren will ich auch und bellen,\\
aber gleichwohl weiter nicht,\\
als nur wenn in Sündenfällen\\
dir von mir ein Schimpf geschicht,\\
wenn mein Fleisch mich übereilt\\
und zur Buße, die uns heilt,\\
sich viel träger als zur Sünden\\
und zur Bosheit lässet finden.

\flagverse{7.} Dennoch will ohn alles Heucheln,\\
das so fest sonst in uns steckt,\\
ich dir auch hinwieder schmeicheln,\\
wenn ich deinen Zorn erweckt\\
und du meinen Übermut\\
strafest mit der scharfen Rut.\\
Ach Herr, schone, will ich sprechen,\\
laß mein Wort dein Herze brechen!

\flagverse{8.} Mache mich zum wackern Hüter,\\
dessen Augen offen sein,\\
wenn das schönste deiner Güter,\\
deine Kinder, schlafen ein.\\
Wenn das Haus zu Bette geht\\
und der Dieb mit Listen steht\\
nach des Nächsten Gut und Gelde,\\
ei, so gib, daß ich ihn melde!

\flagverse{9.} Mehre meinen kleinen Glauben\\
und wehr allem, das da will\\
dieses Schatzes mich berauben;\\
führe mich zum rechten Ziel!\\
Laß mich sein, o ewges Heil,\\
deines Hauses kleines Teil\\
auch den Kleinsten unter allen,\\
die nach deinem Reiche wallen.

\flagverse{10.} Hab ich dies, so ruht mein Wille,\\
denn ich habe selber dich,\\
dich, du unvermessne Fülle\\
dessen, was mich ewiglich\\
in dem Himmel laben soll.\\
Wohl mir, wohl und aber wohl!\\
Soll mich Gottes Fülle laben,\\
woran will ich Mangel haben?

\end{verse}
\end{multicols}
%\attrib{\small{THZE}}

\index{Herr, ich will gar gerne bleiben}
\newpage
\subsection*{\centerline{Weg, mein Herz, mit den Gedanken}}
\addcontentsline{toc}{subsection}{Weg mein Herz mit den Gedanken}
%StartInfo%%%%%%%%%%%%%%%%%%%%%%%%%%%%%%%%%%%%%%%%%%%%%%%%%%%%%%%%%%%%%%%%%%%%
%  Autor:
%  Titel:
%  File:
%  Ref:
%  Mod:
%EndInfo%%%%%%%%%%%%%%%%%%%%%%%%%%%%%%%%%%%%%%%%%%%%%%%%%%%%%%%%%%%%%%%%%%%%%%
%\poemtitle{pt}
\begin{multicols}{2}
\settowidth{\versewidth}{Weg, mein Herz, mit den Gedanken,}
\begin{verse}[\versewidth]
%Buß- und Trostlied\\
%weg, mein Herz, mit den Gedanken als ob du verstoßen wärst\\
%(Luc. 15)

\flagverse{1.} Weg, mein Herz, mit den Gedanken,\\
als ob du verstoßen wärst;\\
bleib in Gottes Wort und Schranken,\\
da du anders reden hörst.\\
Bist du bös und ungerecht,\\
ei, so ist Gott fromm und schlecht;\\
hast du Zorn und Tod verdienet,\\
sinke nicht! Gott ist versühnet.

\flagverse{2.} Du bist, wie die Menschen alle,\\
angesteckt mit Sündengift,\\
welches Adam mit dem Falle\\
samt der Schlangen hat gestift't;\\
aber so du kehrst zu Gott\\
und dich besserst, hats nicht Not!\\
Seht getrost! Gott wird dein Flehen\\
und Abbitten nicht verschmähen.

\flagverse{3.} Er ist ja kein Bär noch Leue,\\
der sich nur nach Blute sehnt,\\
sein Herz ist zu lauter Treue\\
und zur Sanftmut angewöhnt.\\
Gott hat einen Vatersinn,\\
unser Jammer jammert ihn,\\
unser Unglück ist sein Schmerze,\\
unser Sterben kränkt sein Herze.

\flagverse{4.} »So wahrhaftig als ich lebe,\\
will ich keines Menschen Tod,\\
sondern, daß er sich ergebe\\
an mir aus dem Sündenkot.«\\
Gottes Freud ist, wenn auf Erd\\
ein Verirrter wiederkehrt;\\
will nicht, daß aus seiner Herde\\
das Geringst entzogen werde.

\flagverse{5.} Kein Hirt kann so fleißig gehen\\
nach dem Schaf, das sich verläuft;\\
sollst du Gottes Herze sehen,\\
wie sich da der Kummer häuft,\\
wie es dürstet, jächt und brennt\\
nach dem, der sich abgewendt\\
von ihm und auch von den Seinen,\\
würdest du für Liebe weinen.

\flagverse{6.} Gott, der liebt nicht nur die Frommen,\\
die in seinem Hause seind,\\
sondern auch die ihm genommen\\
durch den grimmen Seelenfeind,\\
der dort in der Hölle sitzt\\
und der Menschen Herz erhitzt\\
wider den, der, wenn sich reget\\
sein Fuß, alle Welt beweget.

\flagverse{7.} Dennoch bleibt in Liebesflammen\\
sein Verlangen allzeit groß,\\
ruft und locket uns zusammen\\
in den weiten Himmelsschoß;\\
wer sich nun da stellet ein,\\
suchet frei und los zu sein\\
aus des Satans Reich und Rachen,\\
der macht Gott und Engel lachen.

\flagverse{8.} Gott und alles Heer hoch droben,\\
dem der Himmel schweigen muß,\\
wenn sie ihren Schöpfer loben,\\
jauchzen über unsre Buß.\\
Aber was gesündigt ist,\\
das verdeckt er, und vergißt,\\
wie wir ihn beleidigt haben;\\
alles, Alles ist vergraben.

\flagverse{9.} Kein See kann sich so ergießen,\\
kein Grund mag so grundlos sein,\\
kein Strom so gewaltig fließen,\\
gegen Gott ist alles klein,\\
gegen Gott und seine Huld,\\
die er über unsre Schuld\\
alle Tage lässet schweben\\
durch das ganze Sündenleben.

\flagverse{10.} Nun, so ruh und sei zufrieden,\\
seele, die du traurig bist,\\
was willst du dich viel ermüden,\\
da es nicht vonnöten ist.\\
Deiner Sünden großes Meer,\\
wie dirs scheinet, ist nicht mehr\\
gegen Gottes Herz zu sagen\\
als was wir mit Fingern tragen.

\flagverse{11.} Wären tausend Welt zu finden,\\
von dem Höchsten zugericht't,\\
und du hättest alle Sünden,\\
die darinnen sind, verricht't,\\
wär es viel; doch lange nicht\\
so viel, daß das volle Licht\\
seiner Gnaden hier auf Erden\\
dadurch könnt erlöschet werden.

\flagverse{12.} Mein Gott, öffne mir die Pforten\\
solcher Gnad und Gütigkeit,\\
laß mich allzeit aller Orten\\
schmecken deine Süßigkeit;\\
liebe mich und treib mich an,\\
daß ich dich, so gut ich kann,\\
wiederum umfang und liebe\\
und ja nun nicht mehr betrübe!


\end{verse}
\end{multicols}
%\attrib{\small{THZE}}

\index{Weg, mein Herz, mit den Gedanken}
\newpage


%\section*{\centerline{Abendmahl}}
%\subsection*{ Herr Jesu, meine Liebe}

%\section*{\centerline{Buße}}
%\subsection*{Herr, höre, was mein Mund}
%\subsection*{Herr, ich will gar gerne bleiben}
%\subsection*{Weg, mein Herz, mit den Gedanken}

\section*{\centerline{\LARGE LEID UND TROST}}
\addcontentsline{toc}{section}{LEID UND TROST}
\rule{\textwidth}{0.2pt}\vspace*{-\baselineskip}\vspace{3.2pt}
\rule{\textwidth}{1.2pt}\\[\baselineskip]

\index{ Gedichte von Leid und Trost}

\centerline{\scshape Ach Herr, wie lange willst du mein}
\vspace*{0.8\baselineskip}
\centerline{\scshape Ach treuer Gott, barmherzigs Herz }
\vspace*{0.8\baselineskip}
\centerline{\scshape Auf den Nebel folgt die Sonne}                    %witt: Sonn'
\vspace*{0.8\baselineskip}
\centerline{\scshape Barmherzger Vater, höchster Gott }
\vspace*{0.8\baselineskip}
\centerline{\scshape Befiehl du deine Wege }
\vspace*{0.8\baselineskip}
\centerline{\scshape Du bist ein Mensch, das weißt du wohl }
\vspace*{0.8\baselineskip}
\centerline{\scshape Du liebe Unschuld du, wie schlecht wirst du geacht't }
\vspace*{0.8\baselineskip}
\centerline{\scshape Geduld ist euch vonnöten }
\vspace*{0.8\baselineskip}
\centerline{\scshape Gib dich zufrieden und sei stille }
\vspace*{0.8\baselineskip}
\centerline{\scshape Gott ist mein Licht, der Herr mein Heil }
\vspace*{0.8\baselineskip}
\centerline{\scshape Herr, der du vormals hast dein Land }
\vspace*{0.8\baselineskip}
\centerline{\scshape Herr, was hast du im Sinn? }
\vspace*{0.8\baselineskip}
\centerline{\scshape Ich erhebe, Herr, zu dir }
\vspace*{0.8\baselineskip}
\centerline{\scshape Ich hab in Gottes Herz und Sinn }
\vspace*{0.8\baselineskip}
\centerline{\scshape Ich habs verdient, was will ich noch }
\vspace*{0.8\baselineskip}
\centerline{\scshape Ist Ephraim nicht meine Kron? }
\vspace*{0.8\baselineskip}
\centerline{\scshape Ist Gott für mich, so trete }
\vspace*{0.8\baselineskip}
\centerline{\scshape Kommt, ihr traurigen Gemüter }
\vspace*{0.8\baselineskip}
\centerline{\scshape Meine Seele ist in der Stille }
\vspace*{0.8\baselineskip}
\centerline{\scshape Nicht so traurig, nicht so sehr} 
\vspace*{0.8\baselineskip}
\centerline{\scshape Noch dennoch mußt du drum }
\vspace*{0.8\baselineskip}
\centerline{\scshape Nun, du lebest, unsre Krone}                   %witt:? --Trost
\vspace*{0.8\baselineskip}
\centerline{\scshape Schwing dich auf zu deinem Gott }
\vspace*{0.8\baselineskip}
\centerline{\scshape Sei wohlgemut, o Christenseel }
\vspace*{0.8\baselineskip}
\centerline{\scshape Warum sollt ich mich doch grämen? }
\vspace*{0.8\baselineskip}
\centerline{\scshape Was Gott gefällt, mein frommes Kind} 
\vspace*{0.8\baselineskip}
\centerline{\scshape Was soll ich doch, o Ephraim }
\vspace*{0.8\baselineskip}
\centerline{\scshape Was trotzest du, stolzer Tyrann? }
\vspace*{0.8\baselineskip}
\centerline{\scshape Wer unterm Schirm des Höchsten sitzt }
\vspace*{0.8\baselineskip}
\centerline{\scshape Wie ist so groß und schwer die Last }
\vspace*{0.8\baselineskip}
\centerline{\scshape Wie lang, o Herr, wie lange }

\newpage

\subsection*{\centerline{Ach Herr, wie lange willst du mein}}
\addcontentsline{toc}{subsection}{Ach Herr wie lange willst du mein}
%StartInfo%%%%%%%%%%%%%%%%%%%%%%%%%%%%%%%%%%%%%%%%%%%%%%%%%%%%%%%%%%%%%%%%%%%%
%  Autor: Dietmar Volkmann
%  Titel: Paul Gerhardt: Ach Herr, wie lange
%  File:  pg-Ach-Herr-wie.tex
%  Ref:   paulgerhardt
%  Mod:   
%EndInfo%%%%%%%%%%%%%%%%%%%%%%%%%%%%%%%%%%%%%%%%%%%%%%%%%%%%%%%%%%%%%%%%%%%%%%
%ANM:\poemtitle{Ach Herr, wie lange willst du mein so ganz und gar vergessen?}
\begin{multicols}{2}
\settowidth{\versewidth}{Ach Herr, wie lange willst du mein}
\begin{verse}[\versewidth]
%ANM:Der 13. Psalm\\
%ANM:Gedichtet zur Leichenfeier des 1660 verstorbenen Rittmeisters Christoph Ludwig von Thümen\\
\flagverse{1.} Ach Herr, wie lange willst du mein\\
so ganz und gar vergessen?\\
Wie lange soll der Sorgen Stein\\
mich und mein Herze pressen?\\
Wie lange soll dein Angesicht\\
sich von mir wenden? Willst du nicht\\
dich meiner mehr erbarmen?

\flagverse{2.} Wie lange soll ich armes Kind\\
der Seelen Ruh entbehren?\\
Wie lange soll der Sturm und Wind\\
der Herzensangst gewähren?\\
Wie lange soll mein stolzer Feind,\\
ders niemals gut, stets böse meint,\\
sich über mich erheben?

\flagverse{3.} Ach, schaue doch, mein Gott und Hort,\\
von deiner heilgen Hütte\\
und höre meiner Klage Wort\\
und hochbetrübte Bitte;\\
gib meinen Augen Kraft und Macht\\
und laß des Todes finstre Nacht\\
mich nicht so bald befallen!

\flagverse{4.} Sonst würde meiner Feinde Mund\\
des Ruhms kein Ende machen;\\
sie würden mein, als der zu Grund\\
und Boden gangen, lachen:\\
Da liegt der, würden sie mit Freud\\
herprahlen, der uns jederzeit\\
so viel zu schaffen machte!

\flagverse{5.} Ich kenne sie und weiß gar wohl,\\
was sie im Schilde führen,\\
ihr Herz ist aller Bosheit voll,\\
läßt sich nichts Guts regieren.\\
Du aber bist der fromme Mann,\\
Herr, mein Gott, der nicht lassen kann\\
die, so sich zu dir halten.

\flagverse{6.} Des tröst ich mich und hoffe drauf,\\  % KORR: Komma?
du wirst auch mir fromm bleiben\\
und aller bösen Tücke Lauf\\
gewaltig hintertreiben.\\
Mein Herze freut sich, wenns bedenkt,\\
wie gern du stets dein Heil geschenkt\\
dem, der sich dir vertrauet.

\end{verse}
\end{multicols}

\begin{center}
\settowidth{\versewidth}{Ach Herr, wie lange willst du mein}
\begin{verse}[\versewidth]
\flagverse{7.} Das tu ich, Herr; ich traue dir:\\
du bist mein einzge Freude,\\
bewehrest mich, tust wohl an mir\\
und führst mich aus dem Leide.\\
Dafür will ich mein Leben lang\\
dir manchen schönen Lobgesang\\
zum Dank und Opfer bringen.
\end{verse}
\end{center}

%\end{verse}
%\end{multicols}
%\attrib{\small{1660}}

\index{Ach Herr, wie lange willst du mein}
\newpage
\subsection*{\centerline{Ach treuer Gott, barmherzigs Herz}}
\addcontentsline{toc}{subsection}{Ach treuer Gott barmherzigs Herz}
%StartInfo%%%%%%%%%%%%%%%%%%%%%%%%%%%%%%%%%%%%%%%%%%%%%%%%%%%%%%%%%%%%%%%%%%%%
%  Autor:
%  Titel:
%  File:
%  Ref:
%  Mod:
%EndInfo%%%%%%%%%%%%%%%%%%%%%%%%%%%%%%%%%%%%%%%%%%%%%%%%%%%%%%%%%%%%%%%%%%%%%%
%\poemtitle{Ach treuer Gott, barmherzigs Herz}
\begin{multicols}{2}
\settowidth{\versewidth}{Denn das ist allzeit dein Gebrauch:}
\begin{verse}[\versewidth]
%ach treuer Gott, barmherzigs Herz\\
%nach Johann Arnds »Paradiesgärtlein« III, 27

\flagverse{1.} Ach treuer Gott, barmherzigs Herz,\\
des Güte sich nicht endet,\\
ich weiß, daß mir dies Kreuz und Schmerz\\
dein Vaterhand zusendet.\\
Ja, Herr, ich weiß, daß diese Last\\
du mir aus Lieb erteilet hast\\
und gar aus keinem Hasse.

\flagverse{2.} Denn das ist allzeit dein Gebrauch:\\
Wer Kind ist, muß was leiden;\\
und wen du liebst, den stäupst du auch,\\
schickst Trauern vor den Freuden,\\
führst uns zur Höllen, tust uns weh\\
und führst uns wieder in die Höh,\\
und so geht eins ums ander.

\flagverse{3.} Du führst ja wohl recht wunderlich\\
die, so dein Herz ergötzen:\\
Was Leben soll, muß erstlich sich\\
ins Todes Höhle setzen;\\
was steigen soll zur Ehr empor,\\
liegt auf der Erd und muß sich vor\\
im Kot und Staube wälzen.

\flagverse{4.} Das hat, Herr, dein geliebter Sohn\\
selbst wohl erfahrn auf Erden;\\
denn eh er kam zum Ehrenthron,\\
mußt er gekreuzigt werden.\\
Er ging durch Trübsal, Angst und Not,\\
ja durch den herben bittern Tod\\
drang er zur Himmelsfreude.

\flagverse{5.} Hat nun dein Sohn, der fromm und recht,\\
so willig sich ergeben,\\
was will ich armer Sündenknecht\\
dir viel zuwider streben?\\
Er ist der Spiegel der Geduld,\\
und wer sich sehnt nach seiner Huld,\\
der muß ihm endlich werden.

\flagverse{6.} Ach, liebster Vater, wie so schwer\\
ists der Vernunft, zu glauben,\\
daß du demselben, den du sehr\\
schlägst, solltest günstig bleiben!\\
Wie macht doch Kreuz so lange Zeit!\\
Wie schwerlich will sich Lieb und Leid\\
zusammen lassen reimen!

\flagverse{7.} Was ich nicht kann, das gib du mir,\\
o höchstes Gut der Frommen!\\
Gib, daß mir nicht des Glaubens Zier\\
durch Trübsal werd entnommen!\\
Erhalte mich, o starker Hort!\\
Befestge mich in deinem Wort,\\
behüte mich vor Murren!

\flagverse{8.} Bin ich ja schwach, laß deine Treu\\
mir an die Seite treten,\\
hilf, daß ich unverdrossen sei\\
zum Rufen, Seufzen, Beten!\\
So lang ein Herze hofft und gläubt\\
und im Gebet beständig bleibt,\\
so lang ists unbezwungen.

\flagverse{9.} Greif mich auch nicht zu heftig an,\\
damit ich nicht vergehe!\\
Du weißt wohl, was ich tragen kann,\\
wies um mein Leben stehe;\\
ich bin ja weder Stahl noch Stein:\\
Wie balde geht ein Wind herein,\\
so fall ich hin und sterbe.

\flagverse{10.} Ach Jesu, der du worden bist\\
mein Heil mit deinem Blute,\\
du weißt gar wohl, was Kreuze ist\\
und wie dem sei zu Mute,\\
den Kreuz und großes Unglück plagt;\\
drum wirst du, was mein Herze klagt,\\
gar gern zu Herzen fassen.

\flagverse{11.} Ich weiß, du wirst in deinem Sinn\\
mit mir Mitleiden haben\\
und mich, wie ichs jetzt dürftig bin,\\
mit Gnad und Hilfe laben.\\
Ach stärke meine schwache Hand,\\
ach heil und bring in bessern Stand\\
das Straucheln meiner Füße!

\flagverse{12.} Sprich meiner Seel ein Herze zu\\
und tröste mich aufs beste,\\
denn du bist ja der Müden Ruh,\\
der Schwachen Turm und Feste,\\
ein Schatten für der Sonnen Hitz,\\
ein Hütte, da ich sicher sitz\\
in Sturm und Ungewitter.

\flagverse{13.} Und weil ich ja nach deinem Rat\\
hie soll ein wenig leiden,\\
so laß mich auch in deiner Gnad\\
als wie ein Schäflein weiden,\\
daß ich im Glauben die Geduld\\
und durch Geduld die edle Huld\\
nach schwerer Prob erhalte.

\flagverse{14.} O heilger Geist, du Freudenöl,\\
das Gott vom Himmel schicket,\\
erfreue mich, gib meiner Seel\\
was Mark und Bein erquicket!\\
Du bist der Geist der Herrlichkeit,\\
weißt, was für Freud und Seligkeit\\
mein in dem Himmel warte.

\flagverse{15.} Ach laß mich schauen, wie so schön\\
und lieblich sei das Leben,\\
das denen, die durch Trübsal gehn,\\
du dermaleinst wirst geben.\\
Ein Leben, gegen welches hier\\
die ganze Welt mit ihrer Zier\\
durchaus nicht zu vergleichen.

\flagverse{16.} Daselbst wirst du in ewger Lust\\
aufs süß'ste mit mir handeln:\\
Mein Kreuz, das dir und mir bewußt,\\
in Freud und Ehre wandeln;\\
da wird mein Weinen lauter Wein,\\
mein Ächzen lauter Jauchzen sein!\\
Das glaub ich. Hilf mir! Amen.

\end{verse}
\end{multicols}
%\attrib{\small{THZE}}

\index{Ach treuer Gott, barmherzigs Herz}
\newpage
\subsection*{\centerline{Auf den Nebel folgt die Sonne}}            %witt: Sonn'
\addcontentsline{toc}{subsection}{Auf den Nebel folgt die Sonne}            %witt: Sonn'}
%StartInfo%%%%%%%%%%%%%%%%%%%%%%%%%%%%%%%%%%%%%%%%%%%%%%%%%%%%%%%%%%%%%%%%%%%%
%  Autor:
%  Titel:
%  File:
%  Ref:
%  Mod:
%EndInfo%%%%%%%%%%%%%%%%%%%%%%%%%%%%%%%%%%%%%%%%%%%%%%%%%%%%%%%%%%%%%%%%%%%%%%
%\poemtitle{Auf den Nebel folgt die Sonne}
\begin{multicols}{2}
\settowidth{\versewidth}{Der, vor dem die Welt erschrickt,}
\begin{verse}[\versewidth]

\flagverse{1.} Auf den Nebel folgt die Sonn,\\
auf das Trauern Freud und Wonn,\\
auf die schwere bittre Pein\\
stellt sich Trost und Labsal ein.\\
Meine Seele, die zuvor\\
sank bis zu dem Höllentor,\\
steigt nun bis zum Himmelschor.

\flagverse{2.} Der, vor dem die Welt erschrickt,\\
hat mir meinen Geist erquickt,\\
seine hohe starke Hand\\
reißt mich aus der Höllen Band;\\
alle seine Lieb und Güt\\
überschwemmt mir mein Gemüt\\
und erfrischt mir mein Geblüt.

\flagverse{3.} Hab ich vormals Angst gefühlt,\\
hat der Gram mein Herz zerwühlt,\\
hat der Kummer mich beschwert,\\
hat der Satan mich betört:\\
Ei, so bin ich nunmehr frei,\\
Heil und Rettung, Schutz und Treu\\
steht mir wieder treulich bei.

\flagverse{4.} Nun erfahr ich, schnöder Feind,\\
wie du's habst mit mir gemeint,\\
du hast wahrlich mich mit Macht\\
in dein Netz zu ziehn gedacht.\\
Hätt ich dir zuviel getraut,\\
hättst du, eh ich zugeschaut,\\
mir zu Fall ein Sieb gebaut.

\flagverse{5.} Ich erkenne deine List,\\
da du mit erfüllet bist;\\
du belügst mir meinen Gott\\
und machst seinen Ruhm zu Spott:\\
Wann er setzt, so wirfst du üm.\\
Wann er spricht, verkehrt dein Grimm\\
seine süße Vaterstimm.

\flagverse{6.} Hoff und wart ich alles Guts,\\
bin ich froh und gutes Muts,\\
rückst du mir aus meinem Sinn\\
alles gute Sinnen hin:\\
Gott ist, sprichst du, fern von dir,\\
alles Unglück bricht herfür,\\
steht und liegt vor deiner Tür.

\flagverse{7.} Heb dich weg, verlogner Mund!\\
Hie ist Gott und Gottes Grund,\\
hie ist Gottes Angesicht\\
und das schöne helle Licht\\
seines Segens, seiner Gnad;\\
all sein Wort und weiser Rat\\
steht vor mir in voller Tat.

\flagverse{8.} Gott läßt keinen traurig stehn,\\
noch mit Schimpf zurückegehn,\\
der sich ihm zu eigen schenkt\\
und ihn in sein Herze senkt;\\
wer auf Gott sein Hoffnung setzt,\\
findet endlich und zuletzt\\
was ihm Leib und Seel ergötzt.

\flagverse{9.} Kommts nicht heute, wie man will,\\
sei man nur ein wenig still:\\
Ist doch morgen auch ein Tag,\\
da die Wohlfahrt kommen mag.\\
Gottes Zeit hält ihren Schritt,\\
wann die kommt, kommt unsre Bitt\\
und die Freude reichlich mit.

\flagverse{10.} Ach, wie ofte dacht ich doch,\\
da mir noch des Trübsals Joch\\
auf dem Haupt und Halse saß\\
und das Leid mein Herze fraß:\\
Nun ist keine Hoffnung mehr,\\
auch kein Ruhen, bis ich kehr\\
in das schwarze Totenmeer.

\flagverse{11.} Aber mein Gott wandt es bald,\\
heilt und hielt mich dergestalt,\\
daß ich, was sein Arm getan,\\
nimmermehr gnug preisen kann;\\
da ich weder hie noch da\\
einen Weg zur Rettung sah,\\
hatt ich seine Hilfe nah.

\flagverse{12.} Als ich furchtsam und verzagt\\
mich selbst und mein Herze plagt,\\
als ich manche liebe Nacht\\
mich mit Wachen krank gemacht,\\
als mir aller Mut entfiel:\\
Tratst du, mein Gott, selbst ins Spiel,\\
gabst dem Unfall Maß und Ziel.

\flagverse{13.} Nun, so lang ich in der Welt\\
haben werde Haus und Zelt,\\
soll mir dieser Wunderschein\\
stets vor meinen Augen sein.\\
Ich will all mein Leben lang\\
meinem Gott mit Lobgesang\\
hiefür bringen Lob und Dank.

\flagverse{14.} Allen Jammer, allen Schmerz,\\
den des ewgen Vaters Herz\\
mir schon jetzo zugezählt\\
oder künftig auserwählt,\\
will ich hier in diesem Lauf\\
meines Lebens allzuhauf\\
frisch und freudig nehmen auf.

\end{verse}
\end{multicols}

\begin{center}
\settowidth{\versewidth}{Der, vor dem die Welt erschrickt,}
\begin{verse}[\versewidth]
  
\flagverse{15.} Ich will gehn in Angst und Not,\\
ich will gehn bis in den Tod,\\
ich will gehn ins Grab hinein\\
und doch allzeit fröhlich sein.\\
Wem der Stärkste bei will stehn,\\
wen der Höchste will erhöhn,\\
kann nicht ganz zugrunde gehn.

\end{verse}
\end{center}


%\attrib{\small{THZE}}

\index{Auf den Nebel folgt die Sonne}
\newpage
\subsection*{\centerline{Barmherzger Vater, höchster Gott}}
\addcontentsline{toc}{subsection}{Barmherzger Vater höchster Gott}
%StartInfo%%%%%%%%%%%%%%%%%%%%%%%%%%%%%%%%%%%%%%%%%%%%%%%%%%%%%%%%%%%%%%%%%%%%
%  Autor:
%  Titel:
%  File:
%  Ref:
%  Mod:
%EndInfo%%%%%%%%%%%%%%%%%%%%%%%%%%%%%%%%%%%%%%%%%%%%%%%%%%%%%%%%%%%%%%%%%%%%%%
%\poemtitle{Barmherzger Vater, höchster Gott, gedenk an deine Worte!}
\begin{multicols}{2}
\settowidth{\versewidth}{Ach, süßer Hort, wie tröstlich klingt,}
\begin{verse}[\versewidth]
%Barmherzger Vater, höchster Gott, gedenk an deine Worte!\\
%Nach Johann Arnds »ParadiesgäRtlein« III, 26\\
\flagverse{1.} Barmherzger Vater, höchster Gott,\\
gedenk an deine Worte!\\
Du sprichst: Ruf mich an in der Not\\
und klopf an meine Pforte,\\
so will ich dir\\
Errettung hier\\
nach deinem Wunsch erweisen,\\
daß du mit Mund\\
und Herzensgrund\\
in Freuden mich sollst preisen.

\flagverse{2.} Befiehl dem Herren früh und spat\\
all deine Weg und Sachen,\\
er weiß zu geben Rat und Tat,\\
kann alles richtig machen.\\
Wirf auf ihn hin,\\
was dir im Sinn\\
liegt und dein Herz betrübet,\\
er ist dein Hirt,\\
der wissen wird\\
zu schützen, was er liebet.

\flagverse{3.} Der fromme Vater wird sein Kind\\
in seine Arme fassen\\
und, die gerecht und gläubig sind,\\
nicht stets in Unruh lassen.\\
Drum, liebe Leut,\\
hofft allezeit\\
auf den, der völlig labet;\\
dem schüttet aus,\\
was ihr im Haus\\
und auf dem Herzen habet.

\flagverse{4.} Ach, süßer Hort, wie tröstlich klingt,\\
was du versprichst den Frommen:\\
Ich will, wann Trübsal einher dringt,\\
ihm selbst zu Hilfe kommen,\\
er liebet mich,\\
drum will auch ich\\
ihn lieben und beschützen,\\
er soll bei mir\\
im Schoße hier\\
frei aller Sorgen sitzen.

\flagverse{5.} Der Herr ist allen denen nah,\\
die sich zu ihme finden,\\
wann sie ihn rufen, steht er da,\\
hilft fröhlich überwinden\\
all Angst und Weh,\\
hebt in die Höh\\
die schon darnieder liegen;\\
er macht und schafft,\\
daß sie viel Kraft\\
und große Stärke kriegen.

\begin{verbatim}







\end{verbatim}

\flagverse{6.} Fürwahr, wer meinen Namen ehrt,\\
spricht Christus, und fest gläubet,\\
des Bitte wird von Gott erhört,\\
sein Herzenswunsch bekleibet.\\
So tret heran\\
ein jedermann!\\
Wer bittet, wird empfangen,\\
und wer da sucht,\\
der wird die Frucht\\
mit großem Nutz erlangen.

\flagverse{7.} Hört, was dort jener Richter sagt:\\
Ich muß die Witwe hören,\\
dieweil sie mich so treibt und plagt.\\
Sollt denn sich Gott nicht kehren\\
zu seiner Schar,\\
die hier und dar\\
bei Nacht und Tage schreien?\\
Ich sag und halt:\\
Er wird sie bald\\
aus aller Angst befreien.

\flagverse{8.} Wann der Gerecht in Nöten weint,\\
will Gott ihn fröhlich machen;\\
und die zerbrochnes Herzens seind,\\
die sollen wieder lachen.\\
Wer fromm will sein,\\
muß in der Pein\\
und Jammerstraße wallen;\\
doch steht ihm bei\\
des Höchsten Treu\\
und hift ihm aus dem allen.

\flagverse{9.} Ich habe dich ein'n Augenblick,\\
o liebes Kind, verlassen,\\
sieh aber, sieh, mit großem Glück\\
und Trost ohn alle Maßen\\
will ich dir schon\\
die Freudenkron\\
aufsetzen und verehren;\\
dein kurzes Leid\\
soll sich in Freud\\
und ewges Heil verkehren.

\flagverse{10.} Ach lieber Gott, ach Vaterherz,\\
mein Trost von so viel Jahren,\\
wie läßt du mich so manchen Schmerz\\
und große Angst erfahren!\\
Mein Herze schmacht,\\
mein Auge wacht\\
und weint sich krank und trübe;\\
mein Angesicht\\
verliert sein Licht\\
vom Seufzen, das ich übe.

\begin{verbatim}







\end{verbatim}

\flagverse{11.} Ach Herr, wie lange willst du mein\\
so ganz und gar vergessen?\\
Wie lange soll ich traurig sein\\
und mein Leid in mich fressen?\\
Wie lang ergrimmt\\
dein Herz und nimmt\\
dein Antlitz meiner Seelen?\\
Wie lange soll\\
ich sorgenvoll\\
mein Herz im Leibe quälen?

\flagverse{12.} Willst du verstoßen ewiglich\\
und kein Guts mehr erzeigen?\\
Soll dein Wort und Verheißung sich\\
nun ganz zu Grunde neigen?\\
Zürnst du so sehr,\\
daß du nicht mehr\\
dein Heil magst zu mir senden?\\
Doch Herr, ich will\\
dir halten still,\\
dein Hand kann alles wenden.

\flagverse{13.} Nach dir, o Herr, verlanget mich\\
im Jammer dieser Erden.\\
Mein Gott, ich harr und hoff auf dich,\\
laß nicht zuschanden werden,\\
Herr, deinen Freund,\\
daß nicht mein Feind\\
sich freu und jubiliere,\\
gib mir vielmehr,\\
daß ich zur Ehr\\
ersteig und triumphiere.

\flagverse{14.} Ach Herr, du bist und bleibst auch wohl\\
getreu in deinem Sinne,\\
darum, wann ich ja kämpfen soll,\\
so gib, daß ich gewinne.\\
Leg auf die Last,\\
die du mir hast\\
beschlossen aufzulegen,\\
leg auf, doch daß\\
auch nicht das Maß\\
sei über mein Vermögen!

\flagverse{15.} Du bist ja ungebundner Kraft\\
ein Held, der alles stürzet,\\
du hast ein Hand, die alles schafft,\\
die ist noch unverkürzet.\\
Herr Zebaoth\\
wirst du, mein Gott,\\
genannt zu deinen Ehren,\\
bist groß von Rat,\\
und deiner Tat\\
kann keine Stärke wehren.

\flagverse{16.} Du bist der Tröster Israel\\
und Retter aus Trübsalen,\\
wie kommts denn, daß du meine Seel\\
jetzt sinken läßt und fallen?\\
Du stellst und hast\\
dich als ein Gast,\\
der fremd ist in dem Lande,\\
und wie ein Held,\\
dems Herz entfällt\\
mit Schimpf und großer Schande.

\flagverse{17.} Nein Herr, ein solcher bist du nicht,\\
des ist mein Herz gegründet,\\
du stehest fest, der du dein Licht\\
hier bei uns angezündet.\\
Ja hier hältst du,\\
Herr, deine Ruh\\
bei uns, die nach dir heißen,\\
und bist bereit,\\
zu rechter Zeit\\
uns aus der Not zu reißen.

\flagverse{18.} Nun, Herr, nach aller dieser Zahl\\
der jetzt erzählten Worten\\
hilf mir, der ich so manchesmal\\
geklopft an deine Pforten!\\
Hilf, Helfer, mir,\\
so will ich hier\\
dir Freudenopfer bringen,\\
auch nachmals dort\\
dir fort und fort\\
im Himmel herrlich singen.

\end{verse}
\end{multicols}
%\attrib{\small{THZE}}

\index{Barmherziger Vater, höchster Gott}
\newpage
\newpage
\subsection*{\centerline{Befiehl du deine Wege}}
\addcontentsline{toc}{subsection}{Befiehl du deine Wege}
%StartInfo%%%%%%%%%%%%%%%%%%%%%%%%%%%%%%%%%%%%%%%%%%%%%%%%%%%%%%%%%%%%%%%%%%%%
%  Desc:  Befiehl du deine Wege - Paul Gerhard
%  Desc:  Include 
%  Desc:  Zweispaltiger Satz mit \verse
%  Tags:  GERHARD VERSE EG361 INCLUDE
%  File:  pg-befiehl-du-deine.tex
%  Autor: dv
%  Ref:   
%  Mod:   06.10.2016/dv/initial
%  Mod:   
%EndInfo%%%%%%%%%%%%%%%%%%%%%%%%%%%%%%%%%%%%%%%%%%%%%%%%%%%%%%%%%%%%%%%%%%%%%%



%\poemtitle{Befiehl du deine Wege}
\begin{multicols}{2}
\settowidth{\versewidth}{Befiehl du deine Wege und was}                                                  
\begin{verse}[\versewidth]                                                                                              
  \flagverse{1.} \emph{Befiehl} du deine Wege\\
  und was dein Herze kränkt\\
  der allertreusten Pflege des,\\
  der den Himmel lenkt.\\
  Der Wolken, Luft Und Winden\\
  gibt Wege Lauf und Bahn,\\
  der wird auch Wege finden,\\
  die dein Fuß gehen kann.
  
  \flagverse{2.} \emph{Dem Herren} mußt du trauen,\\
  wenn dir's soll wohlergehn;\\
  auf sein Werk mußt du schauen,\\
  wenn dein Werk soll bestehn.\\
  Mit Sorgen und mit Grämen\\
  und mit selbsteigner Pein\\
  läßt Gott sich gar nichts nehmen,\\
  es muß erbeten sein.

  
  \flagverse{3.} \emph{Dein} ewge Treu und Gnade,\\
  o Vater,  weiß und sieht,\\
  was gut sei oder schade\\
  dem sterblichen Geblüt;\\
  und was du dann erlesen,\\
  das treibst du, starker Held\\
  und bringst zum Stand und Wesen,\\
  was deinem Rat gefällt.
  
  \flagverse{4.} \emph{Weg} hast du allerwegen,\\
  an Mitteln fehlt dir's nicht\\
  dein Tun ist lauter Segen,\\
  dein Gang ist lauter Licht;\\
  dein Werk kann niemand hindern,\\
  dein Arbeit darf nicht ruhn,\\
  wenn du, was deinen Kindern\\
  ersprießlich ist, willst tun.

  \flagverse{5.} \emph{Und} ob gleich alle Teufel\\
  hier wollten widerstehn,\\
  so wird doch ohne Zweifel\\
  Gott nicht zurücke gehn;\\
  was er sich vorgenommen\\
  und was er haben will,\\
  das muß doch endlich kommen\\
  zu seinem Zweck und Ziel.

  \flagverse{6.} \emph{Hoff}, o du arme Seele,\\
  hoff und sei unverzagt!\\
  Gott Wird dich aus der Höhle,\\
  da dich der Kummern plagt,\\
  mit großen Gnaden rücken;\\
  erwarte nur die Zeit,\\
  so wirst du schon erblicken\\
  die Sonn der schönsten Freud.

  \flagverse{7.} \emph{Auf}, auf, gib deinem Schmerze\\
  und Sorgen gute Nacht,\\
  laß fahren, was das Herze\\
  betrübt und traurig macht;\\
  bist du doch nicht Regente,\\
  der alles führen soll,\\
  Gott sitzt im Regimente\\
  und führet alles wohl.

  \flagverse{8.} \emph{Ihn,} ihn laß tun und walten,\\
  er ist ein weiser Fürst\\
  und wird sich so verhalten,\\
  daß du dich wundern wirst,\\
  wenn er, wie ihm gebühret,\\
  mit wunderbaren Rat\\
  das Werk hinausgeführet,\\
  das dich bekümmert hat.

  \flagverse{9.} \emph{Er} wird zwar eine Weile\\
  mit seinem Trost verziehn\\
  und tun an seinem Teile,\\
  als hätt in seinem Sinn\\
  er deiner sich begeben\\
  und sollt'st du für und für\\
  in Angst und Nöten schweben,\\
  als frag er nichts nach dir.

  \flagverse{10.} \emph{Wird's} aber sich befinden,\\
  daß du ihm treu verbleibst,\\
  so wird er dich entbinden,\\
  da du's am mindsten glaubst;\\
  er wird dein Herze lösen\\
  von der so schweren Last,\\
  die du zu keinem Bösen\\
  bisher getragen hast.

  \flagverse{11.} \emph{Wohl} dir, du Kind der Treue,\\
  du hast und trägst davon\\
  mit Ruhm und Dankgeschreie\\
  den Sieg und Ehrenkron;\\
  Gott gibt dir selbst die Palmen\\
  in deine rechte Hand,\\
  und du singst Freudenpsalmen\\
  dem, der dein Leid gewandt.

  \flagverse{12.} \emph{Mach En}d, o Herr, mach Ende\\
  mit aller unsrer Not;\\
  stärk unsre Füß und Hände\\
  und laß bis in den Tod\\
  uns allzeit deiner Pflege\\
  und Treu empfohlen sein,\\
  so gehen unsre Wege\\
  gewiß zum Himmel ein.
  
  

\end{verse}
\end{multicols}
%\attrib{\small{Paul Gerhard 1653}}

  

\index{Befiel du deine Wege}
\newpage
\subsection*{\centerline{Du bist ein Mensch, das weißt du wohl}}
\addcontentsline{toc}{subsection}{Du bist ein Mensch das weißt du wohl}
%StartInfo%%%%%%%%%%%%%%%%%%%%%%%%%%%%%%%%%%%%%%%%%%%%%%%%%%%%%%%%%%%%%%%%%%%%
%  Autor:
%  Titel:
%  File:
%  Ref:
%  Mod:
%EndInfo%%%%%%%%%%%%%%%%%%%%%%%%%%%%%%%%%%%%%%%%%%%%%%%%%%%%%%%%%%%%%%%%%%%%%%
%\poemtitle{Du bist ein Mensch, das weißt du wohl}
\begin{multicols}{2}
\settowidth{\versewidth}{Du bist ein Mensch, das weißt du wohl,}
\begin{verse}[\versewidth]

\flagverse{1.} Du bist ein Mensch, das weißt du wohl,\\
was strebst du denn nach Dingen,\\
die Gott, der Höchst, alleine soll\\
und kann zu Werke bringen?\\
Du fährst mit deinem Witz und Sinn\\
durch so viel tausend Sorgen hin\\
und denkst: Wie wills auf Erden\\
doch endlich mit mir werden?

\flagverse{2.} Es ist umsonst. Du wirst fürwahr\\
mit allem deinem Dichten\\
auch nicht ein einzges kleinstes Haar\\
in aller Welt ausrichten,\\
und dient dein Gram sonst nirgend zu,\\
als daß du dich aus deiner Ruh\\
in Angst und Schmerzen stürzest\\
und selbst das Leben kürzest.

\flagverse{3.} Willst du was tun, was Gott gefällt\\
und dir zum Heil gedeihet,\\
so wirf dein Sorgen auf den Held,\\
den Erd und Himmel scheuet,\\
und gib dein Leben, Tun und Stand\\
nur fröhlich hin in Gottes Hand,\\
so wird er deinen Sachen\\
ein fröhlich Ende machen.

\flagverse{4.} Wer, hat gesorgt, da deine Seel\\
im Anfang deiner Tage\\
noch in der Mutterleibeshöhl\\
und finsterm Kerker lage?\\
Wer hat allda dein Heil bedacht?\\
Was tat da aller Menschen Macht,\\
da Geist und Sinn und Leben\\
dir ward ins Herz gegeben?

\flagverse{5.} Durch wessen Kunst steht dein Gebein\\
in ordentlicher Fülle?\\
Wer gab den Augen Licht und Schein,\\
dem Leibe Haut und Hülle?\\
Wer zog die Adern hie und dort\\
ein jed an ihre Stell und Ort?\\
Wer setzte hin und wieder\\
so viel und schöne Glieder?

\flagverse{6.} Wo war dein Herz, Will und Verstand,\\
da sich des Himmels Decken\\
erstreckten über See und Land\\
und aller Erden Ecken?\\
Wer brachte Sonn und Mond herfür?\\
Wer machte Kräuter, Bäum und Tier\\
und hieß sie deinen Willen\\
und Herzenslust erfüllen?

\flagverse{7.} Heb auf dein Haupt, schau überall\\
hier unten und dort oben,\\
wie Gottes Sorg auf allen Fall\\
vor dir sich hab erhoben:\\
Dein Brot, dein Wasser und dein Kleid\\
war eher noch als du bereit,\\
die Milch, die du erst nahmest,\\
war auch schon, da du kamest.

\flagverse{8.} Die Windeln, die dich allgemach\\
umfingen in der Wiegen,\\
dein Bettlein, Kammer, Stub und Dach\\
und wo du solltest liegen,\\
das war ja alles zugericht't,\\
eh als dein Aug und Angesicht\\
eröffnet ward und sahe,\\
was in der Welt geschahe.

\flagverse{9.} Noch dennoch soll dein Angesicht\\
dein ganzes Leben führen;\\
du traust und glaubest weiter nicht,\\
als was dein Augen spüren;\\
was du beginnst, da soll allein\\
dein Kopf dein Licht und Meister sein,\\
was der nicht auserkoren,\\
das hältst du als verloren.

\flagverse{10.} Nun siehe doch, wie viel und oft\\
ist schändlich umgeschlagen,\\
was du gewiß und fest gehofft\\
mit Händen zu erjagen.\\
Hingegen, wie so manchesmal\\
ist das geschehn, das überall\\
kein Mensch, kein Rat, kein Sinnen\\
ihm hat ersinnen können!

\flagverse{11.} Wie oft bist du in große Not\\
durch eignen Willen kommen,\\
da dein verblendter Sinn den Tod\\
fürs Leben angenommen;\\
und hätte Gott dein Werk und Tat\\
ergehen lassen nach dem Rat,\\
in dem du's angefangen,\\
du wärst zugrunde gangen.

\flagverse{12.} Der aber, der uns ewig liebt,\\
macht gut, was wir verwirren,\\
erfreut, wo wir uns selbst betrübt,\\
und führt uns, wo wir irren;\\
und dazu treibt ihn sein Gemüt\\
und die so reine Vatergüt,\\
in der uns arme Sünder\\
er trägt als seine Kinder.

\flagverse{13.} Ach, wie so oftmals schweigt er still\\
und tut doch, was uns nützet,\\
da unterdessen unser Will\\
und Herz in Ängsten sitzet,\\
sucht hier und da und findet nichts,\\
will sehn und mangelt doch des Lichts,\\
will aus der Angst sich winden\\
und kann den Weg nicht finden.

\flagverse{14.} Gott aber geht gerade fort\\
auf seinen weisen Wegen,\\
er geht und bringt uns an den Ort,\\
da Wind und Sturm sich legen.\\
Hernachmals, wann das Werk geschehn,\\
so kann alsdann der Mensche sehn,\\
was der, so ihn regieret,\\
in seinem Rat geführet.

\flagverse{15.} Drum, liebes Herz, sei wohlgemut\\
und laß von Sorg und Grämen!\\
Gott hat ein Herz, das nimmer ruht,\\
dein Bestes fürzunehmen.\\
Er kanns nicht lassen, glaube mir,\\
sein Eingeweid ist gegen dir\\
und uns hier allzusammen\\
voll allzu süßer Flammen.

\flagverse{16.} Er hitzt und brennt für Gnad und Treu,\\
und also kannst du denken,\\
wie seinem Mut zu Mute sei,\\
wenn wir uns oftmals kränken\\
mit so vergebner Sorgenbürd,\\
als ob er uns nun gänzlich würd\\
aus lauter Zorn und Hassen\\
ganz hilf- und trostlos lassen.

\flagverse{17.} Das schlag hinweg und laß dich nicht\\
so liederlich betören;\\
obgleich nicht allzeit das geschicht,\\
was Freude kann vermehren,\\
so wird doch wahrlich das geschehn,\\
was Gott dein Vater ausersehn;\\
was er dir zu will kehren,\\
das wird kein Mensche wehren.

\flagverse{18.} Tu als sein Kind und lege dich\\
in deines Vaters Arme,\\
bitt ihn und flehe, bis er sich\\
dein, wie er pflegt, erbarme:\\
So wird er dich durch seinen Geist\\
auf Wegen, die du jetzt nicht weißt,\\
nach wohlgehaltnem Ringen\\
aus allen Sorgen bringen.

\end{verse}
\end{multicols}
%\attrib{\small{THZE}}

\index{Du bist ein Mensch, das weißt du wohl}
\newpage
\subsection*{\centerline{Du liebe Unschuld du, wie schlecht wirst du geacht't}}
\addcontentsline{toc}{subsection}{Du liebe Unschuld du wie schlecht wirst du geacht't}
%StartInfo%%%%%%%%%%%%%%%%%%%%%%%%%%%%%%%%%%%%%%%%%%%%%%%%%%%%%%%%%%%%%%%%%%%%
%  Autor:
%  Titel:
%  File:
%  Ref:
%  Mod:
%EndInfo%%%%%%%%%%%%%%%%%%%%%%%%%%%%%%%%%%%%%%%%%%%%%%%%%%%%%%%%%%%%%%%%%%%%%%
%\poemtitle{Du liebe Unschuld du}
\begin{multicols}{2}
\settowidth{\versewidth}{Du strafst der Bösen Werk}
\begin{verse}[\versewidth]

  \flagverse{1.} Du liebe Unschuld du,\\
  wie schlecht wirst du geacht't!\\
  Wie oftmals wird dein Tun\\
  von aller Welt verlacht!\\
  Du dienest deinem Gott,\\
  hältst dich nach seinen Worten,\\
  darüber höhnt man dich\\
  und drückt dich aller Orten.

  \flagverse{2.} Du gehst geraden Weg,\\
  fleuchst von der krummen Bahn,\\
  ein andrer tut sich zu\\
  und wird ein reicher Mann,\\
  vermehrt sein kleines Gut,\\
  füllt Kästen, Böden, Scheunen;\\
  du bleibst ein armer Tropf\\
  und darbest mit den Deinen.

  \flagverse{3.} Du strafst der Bösen Werk\\
  und sagst, was Unrecht sei.\\
  Ein Andrer braucht die Kunst\\
  der süßen Heuchelei;\\
  die bringt ihm Lieb und Huld\\
  und hebt ihn auf die Höhen,\\
  du aber bleibst zurück\\
  und mußt da unten stehen.

  \flagverse{4.} Du sprichst, die Tugend sei\\
  der Christen schönste Kron;\\
  hingegen hält die Welt\\
  auf Reputation:\\
  Wer diese haben will, sagt sie,\\
  der muß gar eben\\
  sich schicken in die Zeit\\
  und gleich den andern leben.

  \flagverse{5.} Du rühmest viel von Gott\\
  und streichst gewaltig aus\\
  den Segen, den er schickt\\
  in seiner Kinder Haus.\\
  Ist diesem nun also, spricht man,\\
  so laß doch sehen,\\
  was dir denn ist für Guts,\\
  für Glück und Heil geschehen.

  \flagverse{6.} Halt fest, o frommes Herz,\\
  halt fest und bleib getreu\\
  in Widerwärtigkeit,\\
  denn Gott, der steht dir bei;\\
  laß diesen deine Sach handhaben,\\
  schützen, führen,\\
  so wirst du wohl bestehn\\
  und endlich triumphieren.

\vfill\null
\columnbreak

  \flagverse{7.} Gefällst du Menschen nicht,\\
  das ist ein schlechter Schad;\\
  all gnug ists, wenn du hast\\
  des ewgen Vaters Gnad.\\
  Ein Mensch kann doch nicht mehr\\
  als irren, fehlen, lügen;\\
  Gott aber ist gerecht,\\
  sein Urteil kann nicht trügen.

  \flagverse{8.} Spricht er nun: du bist mein,\\
  dein Tun gefällt mir wohl!\\
  Wohlan, so sei dein Herz\\
  getrost und freudenvoll.\\
  Schlag alles in den Wind,\\
  was böse Leute dichten,\\
  sei still und siehe zu:\\
  Gott wird sie balde richten.

  \flagverse{9.} Stolz, Übermut und Pracht\\
  währt in die Länge nicht;\\
  wanns Glas am hellsten scheint,\\
  fällts auf die Erd und bricht,\\
  und wann des Menschen Glück\\
  am höchsten ist gestiegen,\\
  so stürzt es unter sich\\
  und muß zu Boden liegen.

  \flagverse{10.} Das ungerechte Gut,\\
  wers recht und wohl besieht,\\
  ist lauter Zentnerlast,\\
  die Herz, Sinn und Gemüt\\
  ohn Unterlaß beschwert,\\
  Seel und Gewissen dringet\\
  und aus der sanften Ruh\\
  in schweres Leiden bringet.

  \flagverse{11.} Was hat doch mancher mehr\\
  als armer Leute Schweiß?\\
  Was ißt und trinket er?\\
  Worin besteht sein Preis\\
  als im geraubten Gut\\
  und armer Leute Tränen,\\
  die wie ein dürres Land\\
  sich nach Erquickung sehnen?

  \flagverse{12.} Heißt das nun selig sein?\\
  Ist das nun Herrlichkeit?\\
  O, Welch ein hartes Wort\\
  wird über solche Leut\\
  am Tage des Gerichts\\
  aus Gottes Thron erschallen!\\
  Wie schändlich wird ihr Ruhm\\
  und großes Prahlen fallen!

\vfill\null
\columnbreak

  \flagverse{13.} Du aber, der du Gott\\
  von ganzem Herzen ehrst\\
  und deine Füße nicht\\
  von seinem Wege kehrst,\\
  wirst in der schönen Schar,\\
  die Gott mit Manna weidet,\\
  hergehn, mit Lob und Ehr\\
  als einem Rock gekleidet.

  \flagverse{14.} Drum fasse deine Seel\\
  ein wenig mit Geduld,\\
  fahr immer fort, tu recht,\\
  leb außer Sündenschuld;\\
  halt, daß den höchsten Schatz\\
  dort in dem andern Leben\\
  des Höchsten milde Hand\\
  dir werd aus Gnaden geben.

\end{verse}
\end{multicols}


\begin{center}
\settowidth{\versewidth}{Was hier ist in der Welt,}
\begin{verse}[\versewidth]

  \flagverse{15.} Was hier ist in der Welt,\\
  da sei nur unbemüht,\\
  wird dirs ersprießlich sein,\\
  wies Gott am besten sieht,\\
  so glaube du gewiß,\\
  er wird dir deinen Willen\\
  schon geben und mit Freud\\
  all dein Begehren stillen.
  
\end{verse}
\end{center}


%\attrib{\small{THZE}}

\index{Du liebe Unschuld}
\newpage
\subsection*{\centerline{Geduld ist euch vonnöten}}
\addcontentsline{toc}{subsection}{Geduld ist euch vonnöten}
%StartInfo%%%%%%%%%%%%%%%%%%%%%%%%%%%%%%%%%%%%%%%%%%%%%%%%%%%%%%%%%%%%%%%%%%%%
%  Autor:
%  Titel:
%  File:
%  Ref:
%  Mod:
%EndInfo%%%%%%%%%%%%%%%%%%%%%%%%%%%%%%%%%%%%%%%%%%%%%%%%%%%%%%%%%%%%%%%%%%%%%%
%\poemtitle{Gedult ist euch vonnöten}
\begin{multicols}{2}
\settowidth{\versewidth}{Geduld ist Fleisch und Blute}
\begin{verse}[\versewidth]

%(Hebr. 10, 35-375)

\flagverse{1.} Geduld ist euch vonnöten,\\
wann Sorge, Gram und Leid\\
und was euch mehr will töten,\\
euch in das Herze schneidt,\\
o auserwählte Zahl!\\
Soll euch kein Tod nicht töten,\\
ist euch Geduld vonnöten:\\
Das sag ich noch einmal.

\flagverse{2.} Geduld ist Fleisch und Blute\\
ein herbes, bittres Kraut;\\
wenn unsers Kreuzes Rute\\
uns nur ein wenig draut,\\
erschrickt der zarte Sinn.\\
Im Glück ist er verwegen,\\
kommt aber Sturm und Regen,\\
fällt Herz und Mut dahin.

\flagverse{3.} Geduld ist schwer zu leiden,\\
dieweil wir irdisch seind\\
und nur in lautern Freuden\\
bei Gott zu sein vermeint.\\
Der doch sich klar erklärt:\\
Ich strafe, die ich liebe,\\
und die ich hoch betrübe,\\
die halt ich hoch und wert.

\flagverse{4.} Geduld ist Gottes Gabe\\
und seines Geistes Gut,\\
der zeucht und löst uns abe,\\
sobald er in uns ruht,\\
der edle werte Gast,\\
erlöst uns von dem Zagen\\
und hilft uns treulich tragen\\
die große Bürd und Last.

\flagverse{5.} Geduld kommt aus dem Glauben\\
und hängt an Gottes Wort;\\
das läßt sie ihr nicht rauben,\\
das ist ihr Heil und Hort,\\
das ist ihr hoher Wall,\\
da hält sie sich verborgen,\\
läßt Gott den Vater sorgen\\
und fürchtet keinen Fall.

\flagverse{6.} Geduld setzt ihr Vertrauen\\
auf Christi Tod und Schmerz,\\
macht Satan ihr ein Grauen,\\
so faßt sie hier ein Herz\\
und spricht: Zürn immerhin,\\
du wirst mich doch nicht fressen,\\
ich bin zu hoch gesessen,\\
weil ich in Christo bin!

\flagverse{7.} Geduld ist wohl zufrieden\\
mit Gottes weisem Rat,\\
läßt sich nicht leicht ermüden\\
durch Aufschub seiner Gnad,\\
hält frisch und fröhlich aus,\\
läßt sich getrost beschweren\\
und denkt: Wer wills ihm wehren?\\
Ist er doch Herr im Haus.

\flagverse{8.} Geduld kann lange warten,\\
vertreibt die lange Weil\\
in Gottes schönem Garten,\\
durchsucht zu ihrem Heil\\
das Paradies der Schrift\\
und schützt sich früh und späte\\
mit eifrigem Gebete\\
vor Satans List und Gift.

\flagverse{9.} Geduld tut Gottes Willen,\\
erfüllet sein Gebot\\
und weiß sich wohl zu stillen\\
in aller Feinde Spott.\\
Es lache, wems beliebt:\\
Wird sie doch nicht zuschanden,\\
es ist bei ihr vorhanden\\
ein Herz, das nichts drauf gibt.

\flagverse{10.} Geduld dient Gott zu Ehren\\
und läßt sich nimmermehr\\
von seiner Liebe kehren;\\
und schlüg er noch so sehr,\\
so ist sie doch bedacht,\\
sein heilge Hand zu loben,\\
spricht: Der im Himmel droben\\
hat alles wohl gemacht.

\flagverse{11.} Geduld erhält das Leben,\\
vermehrt der Jahre Zahl,\\
vertreibt und dämpft darneben\\
manch Angst und Herzensqual;\\
ist wie ein schönes Licht,\\
davon, wer an ihr hanget,\\
mit Gottes Hilf erlanget\\
ein fröhlichs Angesicht.

\flagverse{12.} Geduld macht große Freude,\\
bringt aus dem Himmelsthron\\
ein schönes Halsgeschmeide,\\
dem Haupt ein edle Kron\\
und königlichen Hut;\\
stillt die betrübten Tränen\\
und füllt das heiße Sehnen\\
mit rechtem guten Gut.

\flagverse{13.} Geduld ist mein Verlangen\\
und meines Herzens Lust,\\
nach der ich oft gegangen:\\
Das ist dir wohl bewußt,\\
Herr voller Gnad und Huld,\\
ach, gib mir und gewähre\\
mein Bitten! Ich begehre\\
nichts andres als Geduld.

\flagverse{14.} Geduld ist meine Bitte,\\
die ich sehr oft und viel\\
aus dieser Leibeshütte\\
zu dir, Herr, schicken will.\\
Kommt dann der letzte Zug,\\
so gib durch deine Hände\\
auch ein geduldigs Ende!\\
So hab ich alles gnug.

\end{verse}
\end{multicols}
%\attrib{\small{THZE}}

\index{Geduld ist euch vonnöten}
\newpage
\subsection*{\centerline{Gib dich zufrieden und sei stille}}
\addcontentsline{toc}{subsection}{Gib dich zufrieden und sei stille}
%StartInfo%%%%%%%%%%%%%%%%%%%%%%%%%%%%%%%%%%%%%%%%%%%%%%%%%%%%%%%%%%%%%%%%%%%%
%  Autor:
%  Titel:
%  File:
%  Ref:
%  Mod:
%EndInfo%%%%%%%%%%%%%%%%%%%%%%%%%%%%%%%%%%%%%%%%%%%%%%%%%%%%%%%%%%%%%%%%%%%%%%
%\poemtitle{Gib dich zufrieden und sei stille}
\begin{multicols}{2}
\settowidth{\versewidth}{Er ist voll Lichtes, Trosts und Gnaden,}
\begin{verse}[\versewidth]
%der 37. Psalm (Vers 7)

\flagverse{1.} Gib dich zufrieden und sei stille\\
in dem Gotte deines Lebens;\\
in ihm ruht aller Freuden Fülle,\\
ohn ihn mühst du dich vergebens.\\
Er ist dein Quell\\
und deine Sonne,\\
scheint täglich hell\\
zu deiner Wonne.\\
Gib dich zufrieden!

\flagverse{2.} Er ist voll Lichtes, Trosts und Gnaden,\\
ungefärbten treuen Herzens;\\
wo er steht, tut dir keinen Schaden\\
auch die Pein des größten Schmerzens;\\
Kreuz, Angst und Not\\
kann er bald wenden,\\
ja auch den Tod\\
hat er in Händen.\\
Gib dich zufrieden!

\flagverse{3.} Wie dirs und andern oft ergehe,\\
ist ihm wahrlich nicht verborgen,\\
er sieht und kennet aus der Höhe\\
der betrübten Herzen Sorgen.\\
Er zählt den Lauf\\
der heißen Tränen\\
und faßt zuhauf\\
all unser Sehnen.\\
Gib dich zufrieden!

\flagverse{4.} Wenn gar kein einzger mehr auf Erden,\\
dessen Treue darfst du trauen,\\
alsdann will er dein Treuster werden\\
und zu deinem Besten schauen.\\
Er weiß dein Leid\\
und heimlich Grämen,\\
auch weiß er Zeit,\\
dich zu benehmen.\\
Gib dich zufrieden!

\flagverse{5.} Er hört die Seufzer deiner Seelen\\
und des Herzens stilles Klagen,\\
und was du keinem darfst erzählen,\\
magst du Gott gar kühnlich sagen,\\
er ist nicht fern,\\
steht in der Mitten,\\
hört bald und gern\\
der Armen Bitten.\\
Gib dich zufrieden!

\flagverse{6.} Laß dich dein Elend nicht bezwingen,\\
halt an Gott, so wirst du siegen;\\
ob alle Fluten einher gingen,\\
dennoch mußt du oben liegen.\\
Denn wenn du wirst\\
zu hoch beschweret,\\
hat Gott, dein Fürst,\\
dich schon erhöret.\\
Gib dich zufrieden!

\begin{verbatim}



\end{verbatim}

\flagverse{7.} Was sorgst du für dein armes Leben,\\
wie du's halten wollst und nähren?\\
Der dir das Leben hat gegeben,\\
wird auch Unterhalt bescheren.\\
Er hat ein Hand\\
voll aller Gaben,\\
da See und Land\\
sich muß von laben.\\
Gib dich zufrieden!

\flagverse{8.} Der allen Vöglein in den Wäldern\\
ihr bescheidnes Körnlein weiset,\\
der Schaf und Rinder in den Feldern\\
alle Tage tränkt und speiset,\\
der wird ja auch\\
dich eingen füllen\\
und deinen Bauch\\
zur Notdurft stillen.\\
Gib dich zufrieden!

\flagverse{9.} Sprich nicht: Ich sehe keine Mittel;\\
wo ich such, ist nichts zum Besten;\\
denn das ist Gottes Ehrentitel:\\
Helfen, wann die Not am größten.\\
Wenn ich und du\\
ihn nicht mehr spüren,\\
da schickt er zu,\\
uns wohl zu führen.\\
Gib dich zufrieden!

\flagverse{10.} Bleibt gleich die Hilf in etwas lange,\\
wird sie dennoch endlich kommen,\\
macht dir das Harren angst und bange,\\
glaube mir, es ist dein Frommen.\\
Was langsam schleicht,\\
faßt man gewisser,\\
und was verzeucht,\\
ist desto süßer.\\
Gib dich zufrieden!

\flagverse{11.} Nimm nicht zu Herzen, was die Rotten\\
deiner Feinde von dir dichten,\\
laß sie nur immer weidlich spotten,\\
Gott wirds hören und recht richten.\\
Ist Gott dein Freund\\
und deiner Sachen,\\
was kann dein Feind,\\
der Mensch, groß machen!\\
Gib dich zufrieden!

\flagverse{12.} Hat er doch selbst auch wohl das Seine,\\
wenn ers sehen könnt und wollte.\\
Wo ist ein Glück so klar und reine,\\
dem nicht etwas fehlen sollte?\\
Wo ist ein Haus,\\
das könnte sagen:\\
Ich weiß durchaus\\
von keinen Plagen?\\
Gib dich zufrieden!

\begin{verbatim}




\end{verbatim}

\flagverse{13.} Es kann und mag nicht anders werden,\\
alle Menschen müssen leiden;\\
was webt und lebet auf der Erden,\\
kann das Unglück nicht vermeiden.\\
Des Kreuzes Stab\\
schlägt unsre Lenden\\
bis in das Grab:\\
Da wird sichs enden.\\
Gib dich zufrieden!

\flagverse{14.} Es ist ein Ruhetag vorhanden,\\
da uns unser Gott wird lösen,\\
er wird uns reißen aus den Banden\\
dieses Leibs und allem Bösen.\\
Es wird einmal\\
der Tod herspringen\\
und aus der Qual\\
uns sämtlich bringen.\\
Gib dich zufrieden!
\end{verse}
\end{multicols}

\begin{center}
\settowidth{\versewidth}{Der, vor dem die Welt erschrickt,}
\begin{verse}[\versewidth]

\flagverse{15.} Er wird uns bringen zu den Scharen\\
der Erwählten und Getreuen,\\
die hier mit Frieden abgefahren,\\
sich auch nun im Frieden freuen,\\
da sie den Grund,\\
der nicht kann brechen,\\
den ewgen Mund\\
selbst hören sprechen:\\
Gib dich zufrieden!

\end{verse}
\end{center}

%\attrib{\small{THZE}}

\index{Gib dich zufrieden und sei stille}
\newpage
\subsection*{\centerline{Gott ist mein Licht, der Herr mein Heil}}
\addcontentsline{toc}{subsection}{Gott ist mein Licht der Herr mein Heil}
%StartInfo%%%%%%%%%%%%%%%%%%%%%%%%%%%%%%%%%%%%%%%%%%%%%%%%%%%%%%%%%%%%%%%%%%%%
%  Autor:
%  Titel:
%  File:
%  Ref:
%  Mod:
%EndInfo%%%%%%%%%%%%%%%%%%%%%%%%%%%%%%%%%%%%%%%%%%%%%%%%%%%%%%%%%%%%%%%%%%%%%%
%\poemtitle{Gott ist mein Licht}
\begin{multicols}{2}
\settowidth{\versewidth}{Gott ist mein Licht, der Herr mein Heil,}
\begin{verse}[\versewidth]
%der 27. Psalm

\flagverse{1.} Gott ist mein Licht, der Herr mein Heil,\\
das ich erwählet habe;\\
er ist die Kraft, dahin ich eil\\
und meine Seele labe.\\
Was will ich mich doch fürchten nun?\\
Und wer kann mir doch Schaden tun\\
auf dieser ganzen Erden?

\flagverse{2.} Wann mich die böse Rott anfällt\\
und mein Fleisch will verschlingen,\\
so kann sie dieser starke Held\\
gar leicht zu Boden bringen.\\
Wann sich auch gleich ein ganzes Heer\\
legt um mich her, was ists denn mehr?\\
Mein Gott kann sie bald schlagen.

\flagverse{3.} Eins bitt ich nur, das hätt ich gern,\\
wenn mirs Gott wollte geben,\\
daß ich bei ihm, als meinem Herrn,\\
stets wohnen sollt und leben\\
und alle meine Tag und Jahr\\
in seinem Hause bei der Schar\\
der Heiligen vollbringen.

\flagverse{4.} Da wollt ich meine Herzensfreud\\
an seinen Diensten sehen\\
und rühmen, wie zur bösen Zeit\\
mir so viel Guts geschehen,\\
da er mich fleißig hat verdeckt\\
in seiner Hütten und versteckt\\
in einem starken Felsen.

\flagverse{5.} Und also wird er ferner noch\\
mich wissen zu regieren;\\
er wird mich schützen und sehr hoch\\
in sichre Örter führen;\\
mein Haupt wird über meine Feind\\
ob sie gleich hoch erhaben seind,\\
allzeit erhöhet bleiben.

\flagverse{6.} Dafür will ich denn wiederum\\
Gott auf das Best erhöhen;\\
sein Ruhm soll in dem Heiligtum\\
aus meinem Munde gehen;\\
ich will ihm opfern Dank und Preis,\\
ich will sein Lob, so gut ich weiß,\\
vor allem Volke singen.

\flagverse{7.} Herr, mein Gott, höre, wie ich schrei\\
und seufz in meinem Sinne;\\
gib, daß mein Bitten kräftig sei\\
und dein Herz eingewinne.\\
Mein Herz hält dir, o treuer Hort,\\
beständig vor dein eigen Wort:\\
Ihr sollt mein Antlitz suchen.

\flagverse{8.} Nun such ich jetzt, ach laß mich nicht\\
entgelten meiner Sünden!\\
Ich Suche, Herr, dein Angesicht,\\
das laß mich gnädig finden.\\
Verstoße ja nicht deinen Knecht,\\
denn du bists, der mir hilft zu recht\\
und bringst aus allen Nöten.

\flagverse{9.} Mein Vater, Mutter und was hier\\
sonst ist von guten Leuten,\\
das ist zu schwach und können mir\\
nicht treten an die Seiten.\\
Ich bin entsetzt von aller Welt,\\
Gott aber nimmt mich in sein Zelt,\\
da find ich alle Gnüge.

\flagverse{10.} Herr, mache mir gerade Bahn,\\
halt mich in deiner Gnade\\
und nimm dich meiner herzlich an,\\
daß mir kein Feind nicht schade;\\
Denn viel die reden wider mich\\
und zeugen, das sie ewiglich\\
nicht können überweisen.

\end{verse}
\end{multicols}

\begin{center}
\settowidth{\versewidth}{Der, vor dem die Welt erschrickt,}
\begin{verse}[\versewidth]

\flagverse{11.} Noch dennoch hab ich guten Mut\\
und glaube, daß ich werde\\
im Lebenslande Gottes Gut\\
dort sehn und auf der Erde.\\
Frisch auf, getrost und unverzagt!\\
Wers nur mit Gott im Glauben wagt,\\
der wird den Sieg erhalten.

\end{verse}
\end{center}


%\attrib{\small{THZE}}

\index{Gott ist mein Licht, der Herr mein Heil}
\newpage
\subsection*{\centerline{Herr, der du vormals hast dein Land}}
\addcontentsline{toc}{subsection}{Herr der du vormals hast dein Land}
%StartInfo%%%%%%%%%%%%%%%%%%%%%%%%%%%%%%%%%%%%%%%%%%%%%%%%%%%%%%%%%%%%%%%%%%%%
%  Autor:
%  Titel:
%  File:
%  Ref:
%  Mod:
%EndInfo%%%%%%%%%%%%%%%%%%%%%%%%%%%%%%%%%%%%%%%%%%%%%%%%%%%%%%%%%%%%%%%%%%%%%%
%\poemtitle{Herr, der du vormals hast dein Land}
\begin{multicols}{2}
\settowidth{\versewidth}{Herr, der du vormals hast dein Land}
\begin{verse}[\versewidth]
%der 85. Psalm

\flagverse{1.} Herr, der du vormals hast dein Land\\
mit Gnaden angeblicket,\\
und des gefangnen Jakobs Band\\
gelöst und ihn erquicket,\\
der du die Sünd und Missetat,\\
die dein Volk vor begangen hat,\\
hast väterlich verziehen.

\flagverse{2.} Herr, der du deines Eifers Glut\\
zuvor oft abgewendet\\
und nach dem Zorn das süße Gut\\
der Lieb und Huld gesendet,\\
ach, frommes Herz, ach unser Heil,\\
nimm weg und heb auf in der Eil,\\
was uns betrübt und kränket!

\flagverse{3.} Lösch aus, Herr, deinen großen Grimm\\
im Brunnen deiner Gnaden,\\
erfreu und tröst uns wiederüm\\
nach ausgestandnem Schaden!\\
Willst du denn zürnen ewiglich,\\
und sollen deine Fluten sich\\
ohn alles End ergießen?

\flagverse{4.} Willst du, o Vater, uns denn nicht\\
nun einmal wieder laben?\\
Und sollen wir an deinem Licht\\
nicht wieder Freude haben?\\
Ach geuß aus deines Himmels Haus,\\
Herr, deine Güt und Segen aus\\
auf uns und unsre Häuser!

\flagverse{5.} Ach, daß ich hören sollt das Wort\\
erschallen bald auf Erden,\\
daß Friede sollt an allem Ort,\\
wo Christen wohnen, werden!\\
Ach, daß uns doch Gott sagte zu\\
des Krieges Schluß, der Waffen Ruh\\
und alles Unglücks Ende.

\flagverse{6.} Ach, daß doch diese böse Zeit\\
sich stellt in guten Tagen,\\
damit wir in dem großen Leid\\
nicht mögen ganz verzagen;\\
doch ist ja Gottes Hilfe nah\\
und seine Gnade stehet da\\
all denen, die ihn fürchten.

\flagverse{7.} Wenn wir nur fromm sind, wird sich Gott\\
schon wieder zu uns wenden,\\
den Krieg und alle andre Not\\
nach Wunsch und also enden,\\
daß seine Ehr in unserm Land\\
und über alle werd erkannt,\\
ja stetig bei uns wohne.

\flagverse{8.} Die Güt und Treue werden schön\\
einander grüßen müssen;\\
Gerechtigkeit wird einher gehn,\\
und Friede wird sie küssen;\\
die Treue wird mit Lust und Freud\\
auf Erden blühn, Gerechtigkeit\\
wird von dem Himmel schauen.

\end{verse}
\end{multicols}

\begin{center}
\settowidth{\versewidth}{Der Herr wird uns viel Gutes tun,}
\begin{verse}[\versewidth]

\flagverse{9.} Der Herr wird uns viel Gutes tun,\\
das Land wird Früchte geben,\\
und die in seinem Schoße ruhn,\\
die werden davon leben.\\
Gerechtigkeit Wird dennoch stehn\\
und stets in vollem Schwange gehn\\
zur Ehre seines Namens.

\end{verse}
\end{center}
%\attrib{\small{THZE}}

\index{Herr, der du vormals hast dein Land}
\newpage
\subsection*{\centerline{Herr, was hast du im Sinn?}}
\addcontentsline{toc}{subsection}{Herr was hast du im Sinn?}
%StartInfo%%%%%%%%%%%%%%%%%%%%%%%%%%%%%%%%%%%%%%%%%%%%%%%%%%%%%%%%%%%%%%%%%%%%
%  Autor:
%  Titel:
%  File:
%  Ref:
%  Mod:
%EndInfo%%%%%%%%%%%%%%%%%%%%%%%%%%%%%%%%%%%%%%%%%%%%%%%%%%%%%%%%%%%%%%%%%%%%%%
%\poemtitle{Herr, was hast du im Sinn?}
\begin{multicols}{2}
\settowidth{\versewidth}{Kein Mensche hört fast mehr,}
\begin{verse}[\versewidth]
%Herr, was hast du im Sinn?\\
%Gedichtet Auf die Erscheinung des Kometen von 1664 (nicht 1652, vgl. P. Anm. 378)

\flagverse{1.} Herr, was hast du im Sinn?\\
Wo denkt dein Eifer hin?\\
Von was für neuen Plagen\\
soll uns der Himmel sagen?\\
Was soll uns armen Leuten\\
der neue Stern bedeuten?

\flagverse{2.} Die Zeichen in der Höh\\
erwecken Ach und Weh,\\
es hats in nächsten Jahren\\
die ganze Welt erfahren:\\
Die brennenden Kometen\\
sind traurige Propheten.

\flagverse{3.} Sie brennen in der Luft,\\
und unsers Herzens Kluft\\
ist blind und kalt zum Guten,\\
erkennet nicht die Ruten,\\
die uns zu unsern Wunden\\
des Höchsten Hand gebunden.

\flagverse{4.} Kein Mensche hört fast mehr,\\
was Gottes Geist uns lehr\\
in seinen heilgen Worten;\\
drum muß an so viel Orten\\
von großem Zorn und Dräuen\\
das Sternenland selbst schreien.

\flagverse{5.} Die Welt hält keine Zucht,\\
der Glaub ist in der Flucht,\\
die Treu ist hart gebunden,\\
die Wahrheit ist verschwunden\\
barmherzig sein und lieben,\\
das sieht man selten üben.

\flagverse{6.} Daher wächst Gottes Grimm\\
und dringt mit Ungestüm\\
aus seines Eifers Kammer\\
und will mit großem Jammer,\\
wo wir uns nicht bekehren,\\
uns allesamt verheeren.

\flagverse{7.} Und das will der Prophet,\\
der in der Luft da steht,\\
uns, die wir sicher leben,\\
klar zu verstehen geben\\
mit seinem hellen Lichte\\
und klarem Angesichte.

\flagverse{8.} Sein Lauf ist gar geschwind.\\
Ach, Gott, Laß unsre Sünd\\
uns nicht geschwind hinrücken\\
und eilends unterdrücken;\\
laß uns der Strafen Haufen\\
nicht plötzlich überlaufen!

\flagverse{9.} Sein Strahl ist breit und lang,\\
macht uns fast angst und bang,\\
ach, Jesu, hilf uns allen,\\
auf das nicht auf uns fallen\\
die hochbetrübten Zahlen\\
der letzten Zornesschalen.

\flagverse{10.} Erhalt uns unsern Herrn,\\
den schönen edlen Stern,\\
laß uns sein Licht beleuchten,\\
laß seinen Tau uns feuchten,\\
daß wir uns seiner freuen\\
und unter ihm gedeihen.

\flagverse{11.} Laß auch noch immerfort\\
dein liebes wertes Wort\\
in unserm Land und Grenzen\\
schön rein und helle glänzen;\\
wenn dein Wort uns nur blicket,\\
so sind wir gnug erquicket.

\flagverse{12.} Gedenk an deine Güt\\
und laß doch dein Gemüt\\
erweichen von uns Armen!\\
Regier uns mit Erbarmen,\\
damit die bösen Zeichen\\
ein gutes End erreichen!

\end{verse}
\end{multicols}
%\attrib{\small{THZE}}

\index{Herr, was hast du im Sinn}
\newpage
\subsection*{\centerline{Ich erhebe, Herr, zu dir}}
\addcontentsline{toc}{subsection}{Ich erhebe Herr zu dir}
%StartInfo%%%%%%%%%%%%%%%%%%%%%%%%%%%%%%%%%%%%%%%%%%%%%%%%%%%%%%%%%%%%%%%%%%%%
%  Autor:
%  Titel:
%  File:
%  Ref:
%  Mod:
%EndInfo%%%%%%%%%%%%%%%%%%%%%%%%%%%%%%%%%%%%%%%%%%%%%%%%%%%%%%%%%%%%%%%%%%%%%%
%\poemtitle{pt}
\begin{multicols}{2}
\settowidth{\versewidth}{wenn dich Sturm und Wetter schreckt.}
\begin{verse}[\versewidth]
%der 121. Psalm\\

\flagverse{1.} Ich erhebe, Herr, zu dir\\
meiner beiden Augen Licht,\\
mein Gesicht ist für und für\\
zu den Bergen aufgericht,\\
zu den Bergen, da herab\\
ich mein Heil und Hilfe hab.

\flagverse{2.} Meine Hilfe kommt allein\\
von des Höchsten Händen her,\\
der so künstlich, hübsch und fein\\
Himmel, Erde, Luft und Meer,\\
und was in den allen ist,\\
uns zum Besten ausgerüst.

\flagverse{3.} Er nimmt deiner Füße Tritt,\\
o mein Herze, wohl in Acht,\\
wenn du gehest, geht er mit\\
und bewahrt dich Tag und Nacht.\\
Sei getrost! Das Höllenheer\\
wird dir schaden nimmermehr.

\flagverse{4.} Siehe, wie sein Auge wacht,\\
wenn du liegest in der Ruh,\\
wenn du schläfest, kommt mit Macht\\
auf dein Bett geflogen zu\\
seiner Engel güldne Schar,\\
daß sie deiner nehmen wahr.

\flagverse{5.} Alles was du bist und hast,\\
ist umringt mit seiner Hut,\\
deiner Sorgen schwere Last\\
nimmt er weg, macht alles gut;\\
Leib und Seel hält er verdeckt,\\
wenn dich Sturm und Wetter schreckt.

\flagverse{6.} Wenn der Sonnen Hitze brennt\\
und des Leibes Kräfte bricht,\\
wenn dich Stern und Monde blendt\\
mit dem klaren Angesicht,\\
hat er seine starke Hand\\
dir zum Schatten vorgewandt.

\flagverse{7.} Nun, er fahre immer fort,\\
der getreue fromme Hirt,\\
bleibe stets dein Schild und Hort,\\
wenn dein Herz geängstet wird;\\
wenn die Not wird viel und groß,\\
schließt er dich in seinen Schoß.

\flagverse{8.} Wenn du sitzest, wenn du stehst,\\
wenn du redest, wenn du hörst,\\
wenn du aus dem Hause gehst\\
und zurücke wieder kehrst,\\
wenn du trittst aus oder ein,\\
woll er dein Gefährte sein.

\end{verse}
\end{multicols}
%\attrib{\small{THZE}}

\index{Ich erhebe, Herr, zu dir}
\newpage
\subsection*{\centerline{Ich hab in Gottes Herz und Sinn}}
\addcontentsline{toc}{subsection}{Ich hab in Gottes Herz und Sinn}
%StartInfo%%%%%%%%%%%%%%%%%%%%%%%%%%%%%%%%%%%%%%%%%%%%%%%%%%%%%%%%%%%%%%%%%%%%
%  Autor:
%  Titel:
%  File:
%  Ref:
%  Mod:
%EndInfo%%%%%%%%%%%%%%%%%%%%%%%%%%%%%%%%%%%%%%%%%%%%%%%%%%%%%%%%%%%%%%%%%%%%%%
%\poemtitle{pt}
\begin{multicols}{2}
\settowidth{\versewidth}{Wenn er mich auch gleich wirft ins Meer,}
\begin{verse}[\versewidth]
%ich hab in Gottes Herz und Sinn mein Herz und Sinn ergeben

\flagverse{1.} Ich hab in Gottes Herz und Sinn\\
mein Herz und Sinn ergeben:\\
Was böse scheint, ist mir Gewinn,\\
der Tod selbst ist mein Leben.\\
Ich bin ein Sohn des, der den Thron\\
des Himmels aufgezogen;\\
ob er gleich schlägt und Kreuz auflegt,\\
bleibt doch sein Herz gewogen.

\flagverse{2.} Das kann mir fehlen nimmermehr,\\
mein Vater muß mich lieben!\\
Wenn er mich auch gleich wirft ins Meer,\\
so will er mich nur üben\\
und mein Gemüt in seiner Güt\\
gewöhnen fest zu stehen;\\
halt ich den Stand, weiß seine Hand\\
mich wieder zu erhöhen.

\flagverse{3.} Ich bin ja von mir selber nicht\\
entsprungen noch formieret,\\
mein Gott ists, der mich zugericht't,\\
an Leib und Seel gezieret,\\
der Seelen Sitz mit Sinn und Witz,\\
den Leib mit Fleisch und Beinen:\\
Wer so viel tut, des Herz und Mut\\
kanns nimmer böse meinen.

\flagverse{4.} Woher wollt ich mein Aufenthalt\\
auf dieser Erd erlangen?\\
Ich wäre längsten tot und kalt,\\
wo mich nicht Gott umfangen\\
mit seinem Arm, der alles warm,\\
gesund und fröhlich machet;\\
was er nicht hält, das bricht und fällt,\\
was er erfreut, das lachet.

\flagverse{5.} Zudem ist Weisheit und Verstand\\
bei ihm ohn alle Maßen,\\
zeit, Ort und Stund ist ihm bekannt,\\
zu tun und auch zu lassen.\\
Er weiß, wenn Freud, er weiß, wenn Leid\\
uns, seinen Kindern, diene;\\
und was er tut, ist alles gut,\\
obs noch so traurig schiene.

\flagverse{6.} Du denkest zwar, wenn du nicht hast,\\
was Fleisch und Blut begehret,\\
als sei mit einer großen Last\\
dein Glück und Heil beschweret,\\
hast spät und früh viel Sorg und Müh,\\
an deinen Wunsch zu kommen,\\
und denkest nicht, daß, was geschicht,\\
gescheh zu deinem Frommen.

\flagverse{7.} Fürwahr, der dich geschaffen hat\\
und sich zur Ehr erbauet,\\
der hat schon längst in seinem Rat\\
ersehen und beschauet\\
aus wahrer Treu, was dienlich sei\\
dir und den Deinen alle;\\
laß ihm doch zu, daß er nur tu\\
das, was ihm wohlgefalle.

\flagverse{8.} Wanns Gott gefällt, so kanns nicht sein,\\
er wird dich letzt erfreuen:\\
Was du jetzt nennest Kreuz und Pein,\\
wird dir zum Trost gedeihen.\\
Wart in Geduld: Die Gnad und Huld\\
wird sich doch endlich finden;\\
all Angst und Qual wird auf einmal\\
gleichwie ein Dampf verschwinden.

\flagverse{9.} Das Feld kann ohne Ungestüm\\
gar keine Früchte tragen:\\
So fällt auch Menschenwohlfahrt üm\\
bei lauter guten Tagen.\\
Die Aloe bringt bittres Weh,\\
macht gleichwohl rote Wangen:\\
So muß ein Herz durch Angst und Schmerz\\
zu seinem Heil gelangen.

\flagverse{10.} Ei nun, mein Gott, so fall ich dir\\
getrost in deine Hände;\\
nimm mich und mach es du mit mir\\
bis an mein letztes Ende\\
wie du wohl weißt, daß meinem Geist\\
dadurch sein Nutz entstehe\\
und deine Ehr je mehr und mehr\\
sich in ihr selbst erhöhe.

\flagverse{11.} Willst du mir geben Sonnenschein,\\
so nehm ichs an mit Freuden,\\
solls aber Kreuz und Unglück sein,\\
will ichs geduldig leiden.\\
Soll mir allhier des Lebens Tür\\
noch ferner offen stehen:\\
Wie du mich führst und führen wirst,\\
so will ich gern mitgehen.

\flagverse{12.} Soll ich denn auch des Todes Weg\\
und finstre Straßen reisen:\\
Wohlan, so tret ich Bahn und Steg,\\
den mir dein Augen weisen.\\
Du bist mein Hirt, der alles wird\\
zu solchem Ende kehren,\\
daß ich einmal in deinem Saal\\
dich ewig möge ehren.

\end{verse}
\end{multicols}
%\attrib{\small{THZE}}

\index{Ich hab in Gottes Herz und Sinn}
\newpage
\subsection*{\centerline{Ich habs verdient, was will ich doch}}
\addcontentsline{toc}{subsection}{Ich habs verdient was will ich doch}
%StartInfo%%%%%%%%%%%%%%%%%%%%%%%%%%%%%%%%%%%%%%%%%%%%%%%%%%%%%%%%%%%%%%%%%%%%
%  Autor:
%  Titel:
%  File:
%  Ref:
%  Mod:
%EndInfo%%%%%%%%%%%%%%%%%%%%%%%%%%%%%%%%%%%%%%%%%%%%%%%%%%%%%%%%%%%%%%%%%%%%%%
%\poemtitle{pt}
\begin{multicols}{2}
\settowidth{\versewidth}{Ich habs verdient, was will ich doch}
\begin{verse}[\versewidth]
%ich habs verdient, was will ich doch mich wider Gott viel sperren?\\
%(Micha 7)

\flagverse{1.} Ich habs verdient, was will ich doch\\
mich wider Gott viel sperren?\\
Komm immer her, du Kreuzesjoch\\
und bittrer Kelch des Herren!\\
Ohn Angst und Pein\\
mag der nicht sein,\\
der wider Gott gehandelt,\\
wie ich getan,\\
da ich die Bahn\\
der schnöden Welt gewandelt.

\flagverse{2.} Ich will des Herren Straf und Zorn\\
mit willgem Herzen tragen,\\
in Sünden bin ich ja geborn,\\
hab auch im Sündenwagen\\
mit eitler Freud\\
oft meine Zeit\\
ganz liederlich verzehret,\\
Gott, meinen Hort,\\
in seinem Wort\\
nicht, wie ich soll, gehöret.

\flagverse{3.} Ich habe den gebahnten Steg\\
verlassen und geliebet\\
den gottvergessnen Irreweg;\\
drum wird auch nun betrübet\\
mein Herz und Mut\\
durch Gottes Rut;\\
er hält ein recht Gerichte\\
vor seinem Thron,\\
gibt Sold und Lohn\\
mit völligem Gewichte.

\flagverse{4.} Gott ist gerecht, doch auch dabei\\
sehr fromm und voller Güte,\\
die Vaterlieb und Muttertreu,\\
die wohnt ihm im Gemüte.\\
Gott zürnet nicht,\\
wie wohl geschicht\\
bei uns hier auf der Erden,\\
da mancher Mann\\
nicht wieder kann\\
zur Sühn erweichet werden.

\flagverse{5.} Nein, traun! Das ist nicht Gottes Sinn,\\
sein Zorn der hat ein Ende,\\
wann wir uns bessern, fällt er hin\\
und macht die strengen Hände\\
sanft und gelind,\\
hört auf, die Sünd\\
hier bei uns heimzusuchen;\\
Gott kehrt den Grimm\\
mit Gnaden üm\\
und segnet nach dem Fluchen.

\flagverse{6.} Das wird fürwahr auch mir geschehn!\\
Es solls ein jeder spüren.\\
Gott wird einmal zum Rechten sehn\\
und meine Sach ausführen.\\
Sein Angesicht\\
wird mich ans Licht\\
aus meiner Höhle bringen,\\
daß seine Treu\\
ich frisch und frei\\
erzählen mög und singen.

\flagverse{7.} Drum freut euch nicht, ihr meine Feind,\\
ob ich darniederliege,\\
denn mein Gott wird, eh ihr vermeint,\\
mir helfen, daß ich siege.\\
Sein heilge Hand\\
wird meinen Stand\\
schon wieder feste gründen;\\
es wird sich Freud\\
und gute Zeit\\
nach trübem Wetter finden.

\flagverse{8.} Ich bin in Not und weiß doch nicht\\
von rechter Not zu sagen,\\
denn Gott ist meines Herzens Licht;\\
wo das ist, muß es tagen\\
auch in der Nacht,\\
da sich die Macht\\
der Finsternis vermehret.\\
Wann dieses Licht\\
mir scheint, so bricht\\
und fällt, was mich beschweret.

\end{verse}
\end{multicols}



\begin{center}
\settowidth{\versewidth}{Der, vor dem die Welt erschrickt,}
\begin{verse}[\versewidth]


\flagverse{9.} Es kommt die Zeit und ist nicht weit,\\
da will ich jubilieren;\\
der aber, der mich jetzt verspeit\\
und pfleget zu vexieren\\
in meiner Not:\\
Wo ist dein Gott?\\
Der wird mit Schanden stehen;\\
er wird mit Hohn,\\
ich mit der Kron\\
der Ehren davon gehen.

\end{verse}
\end{center}

%\attrib{\small{THZE}}

\index{Ich habs verdient, was will ich noch}
\newpage
\subsection*{\centerline{Ist Ephraim nicht meine Kron?}}
\addcontentsline{toc}{subsection}{Ist Ephraim nicht meine Kron?}
%StartInfo%%%%%%%%%%%%%%%%%%%%%%%%%%%%%%%%%%%%%%%%%%%%%%%%%%%%%%%%%%%%%%%%%%%%
%  Autor:
%  Titel:
%  File:
%  Ref:
%  Mod:
%EndInfo%%%%%%%%%%%%%%%%%%%%%%%%%%%%%%%%%%%%%%%%%%%%%%%%%%%%%%%%%%%%%%%%%%%%%%
%\poemtitle{pt}
\begin{multicols}{2}
\settowidth{\versewidth}{Ich höre seines Seufzers Stimm}
\begin{verse}[\versewidth]
%ist Ephraim nicht meine Kron?\\
%(Jer. 31, 20)

\flagverse{1.} Ist Ephraim nicht meine Kron\\
und meines Herzens Wonne,\\
mein trautes Kind, mein teurer Sohn,\\
mein Stern und meine Sonne,\\
mein Augenlust, mein edle Blum,\\
mein auserwähltes Eigentum\\
und meiner Seelen Freude?

\flagverse{2.} Ich höre seines Seufzens Stimm\\
und hochbetrübtes Klagen:\\
Mein Gott hat mich, spricht Ephraim,\\
gestraft und wohl geschlagen.\\
Er sucht mich heim mit harter Zucht,\\
das ist mein Lohn, das ist die Frucht\\
und Nutzen meiner Sünden.

\flagverse{3.} Hör alle Welt! Ich bin getreu\\
und halte mein Versprechen;\\
was ich geredt, da bleibt es bei,\\
mein Wort werd ich nicht brechen.\\
Das soll mein Ephraim gar bald\\
erfahren und mich dergestalt\\
recht aus dem Grund erkennen.

\flagverse{4.} Ich denk noch wohl an meinen Eid,\\
den ich geschworen habe,\\
da ich, aus lauter Gütigkeit,\\
mich ihm zu eigen gabe;\\
ich sprach: du hast mein Herz erfüllt\\
mit deiner Lieb, ich bin dein Schild\\
und wills auch ewig bleiben.

\flagverse{5.} Ich will mit meiner starken Hand\\
dich als ein Vater führen,\\
ich selbst will dich und auch dein Land\\
aufs schönst und beste zieren.\\
Und wirst du mir gehorsam sein,\\
so soll dich meines Segens Schein\\
ohn alles End erfreuen.

\flagverse{6.} Wo du dich aber bösen Rat\\
wirst von mir wenden lassen,\\
so will ich Deine Missetat\\
heimsuchen, doch mit Maßen;\\
und wenn du wieder kehrst zu mir,\\
so will ich wieder auch zu dir\\
mich mit Erbarmen kehren.

\flagverse{7.} Nun kehrt zu mir mein Ephraim,\\
sucht Gnad in meinen Armen,\\
drum bricht mein Herze gegen ihm\\
und muß mich sein erbarmen.\\
Der Unmut fällt mir mit Gewalt,\\
mein Eingeweide hitzt und wallt\\
in treuer Lieb und Gnade.

\flagverse{8.} Kommt, alle Sünder, kommt zu mir,\\
bereuet eure Sünden\\
und suchet Gnad an meiner Tür,\\
ihr sollt sie reichlich finden!\\
Wer sich mit Ephraim bekehrt,\\
wird auch mit Ephraim erhört\\
und hier und dort getröstet.
   
\end{verse}
\end{multicols}
%\attrib{\small{THZE}}

\index{Ist Ephraim nicht meine Kron}
\newpage
\subsection*{\centerline{Ist Gott für mich, so trete}}
\addcontentsline{toc}{subsection}{Ist Gott für mich so trete}
%StartInfo%%%%%%%%%%%%%%%%%%%%%%%%%%%%%%%%%%%%%%%%%%%%%%%%%%%%%%%%%%%%%%%%%%%%
%  Autor:
%  Titel:
%  File:
%  Ref:
%  Mod:
%EndInfo%%%%%%%%%%%%%%%%%%%%%%%%%%%%%%%%%%%%%%%%%%%%%%%%%%%%%%%%%%%%%%%%%%%%%%
%\poemtitle{pt}
\begin{multicols}{2}
\settowidth{\versewidth}{Der Grund, da ich mich gründe}
\begin{verse}[\versewidth]
%ist Gott für mich, so trete gleich alles wider mich\\
%(Römer 8)

\flagverse{1.} Ist Gott für mich, so trete\\
gleich alles wider mich,\\
so oft ich ruf und bete,\\
weicht alles hinter sich.\\
Hab ich das Haupt zum Freunde\\
und bin geliebt bei Gott,\\
was kann mir tun der Feinde\\
und Widersacher Rott?

\flagverse{2.} Nun weiß und glaub ich feste,\\
ich rühms auch ohne Scheu,\\
daß Gott, der Höchst und Beste,\\
mir gänzlich günstig sei,\\
und daß in allen Fällen\\
er mir zur Rechten steh\\
und dämpfe Sturm und Wellen\\
und was mir bringet Weh.

\flagverse{3.} Der Grund, da ich mich gründe,\\
ist Christus und sein Blut,\\
das machet, daß ich finde\\
das ewge wahre Gut.\\
An mir und meinem Leben\\
ist nichts auf dieser Erd,\\
was Christus mir gegeben,\\
das ist der Liebe wert.

\flagverse{4.} Mein Jesus ist mein Ehre,\\
mein Glanz und schönes Licht,\\
wenn der nicht in mir wäre,\\
so dürft und könnt ich nicht\\
vor Gottes Augen stehen\\
und vor dem Sternensitz,\\
ich müßte stracks vergehen\\
wie Wachs in Feuers Hitz.

\flagverse{5.} Der, der hat ausgelöschet\\
was mit sich führt den Tod,\\
der ists, der mich rein wäschet,\\
macht schneeweiß, was ist rot.\\
In ihm kann ich mich freuen,\\
hab einen Heldenmut,\\
darf kein Gerichte scheuen,\\
wie sonst ein Sünder tut.

\flagverse{6.} Nichts, nichts kann mich verdammen,\\
nichts nimmet mir mein Herz,\\
die Höll und ihre Flammen,\\
die sind mir nur ein Scherz.\\
Kein Urteil mich erschrecket,\\
kein Unheil mich betrübt,\\
weil mich mit Flügeln decket\\
mein Heiland, der mich liebt.

\flagverse{7.} Sein Geist wohnt mir im Herzen,\\
regiert mir meinen Sinn,\\
vertreibet Sorg und Schmerzen,\\
nimmt allen Kummer hin,\\
gibt Segen und Gedeihen\\
dem, was er in mir schafft,\\
hilft mir das Abba schreien\\
aus aller meiner Kraft.

  % weiter in der anderen Spalte:
\vfill\null
\columnbreak

\flagverse{8.} Und wenn an meinem Orte\\
sich Furcht und Schrecken findt,\\
so seufzt und spricht er Worte,\\
die unaussprechlich sind\\
mir zwar und meinem Munde,\\
gott aber wohlbewußt,\\
der an des Herzens Grunde\\
ersiehet seine Lust.

\flagverse{9.} Sein Geist spricht meinem Geiste\\
manch süßes Trostwort zu:\\
Wie Gott dem Hilfe leiste,\\
der bei ihm suchet Ruh,\\
und wie er hab erbauet\\
ein neue edle Stadt,\\
da Aug und Herze schauet\\
was es geglaubet hat.

\flagverse{10.} Da ist mein Teil und Erbe\\
mir prächtig zugericht't;\\
wenn ich gleich fall und sterbe,\\
fällt doch mein Himmel nicht.\\
Muß ich auch gleich hier feuchten\\
mit Tränen meine Zeit,\\
mein Jesus und sein Leuchten\\
durchsüßet alles Leid.

\flagverse{11.} Wer sich mit dem verbindet,\\
den Satan fleucht und haßt,\\
der wird verfolgt und findet\\
ein hohe schwere Last\\
zu leiden und zu tragen,\\
gerät in Hohn und Spott;\\
das Kreuz und alle Plagen,\\
die sind sein täglich Brot.

\flagverse{12.} Das ist mir nicht verborgen,\\
doch bin ich unverzagt,\\
gott will ihn lassen sorgen,\\
dem ich mich zugesagt.\\
Es kostet Leib und Leben\\
und alles, was ich hab:\\
An dir will ich fest kleben\\
und nimmer lassen ab.

\flagverse{13.} Die Welt, die mag zerbrechen,\\
du stehst mir ewiglich,\\
kein Brennen, Hauen, Stechen\\
soll trennen mich und dich.\\
Kein Hunger und kein Dürsten,\\
kein Armut, keine Pein,\\
kein Zorn der großen Fürsten\\
soll mir ein Hindrung sein.

\flagverse{14.} Kein Engel, keine Freuden,\\
kein Thron, kein Herrlichkeit,\\
kein Lieben und kein Leiden,\\
kein Angst und Fährlichkeit,\\
was man nur kann erdenken,\\
es sei klein oder groß,\\
der keines soll mich lenken\\
aus deinem Arm und Schoß.

\begin{verbatim}

\end{verbatim}

\end{verse}
\end{multicols}

\begin{center}
\settowidth{\versewidth}{Der, vor dem die Welt erschrickt,}
\begin{verse}[\versewidth]



\flagverse{15.} Mein Herze geht in Springen\\
und kann nicht traurig sein,\\
ist voller Freud und Singen,\\
sieht lauter Sonnenschein.\\
Die Sonne, die mir lachet,\\
ist mein Herr Jesus Christ,\\
das, was mich singen machet,\\
ist, was im Himmel ist.

\end{verse}
\end{center}
%\attrib{\small{THZE}}

\index{Ist Gott für mich, so trete}
\newpage
\subsection*{\centerline{Kommt, ihr traurigen Gemüter}}
\addcontentsline{toc}{subsection}{Kommt ihr traurigen Gemüter}
%StartInfo%%%%%%%%%%%%%%%%%%%%%%%%%%%%%%%%%%%%%%%%%%%%%%%%%%%%%%%%%%%%%%%%%%%%
%  Autor:
%  Titel:
%  File:
%  Ref:
%  Mod:
%EndInfo%%%%%%%%%%%%%%%%%%%%%%%%%%%%%%%%%%%%%%%%%%%%%%%%%%%%%%%%%%%%%%%%%%%%%%
%\poemtitle{pt}
\begin{multicols}{2}
\settowidth{\versewidth}{Zwar er hat uns ja zerrissen}
\begin{verse}[\versewidth]
%kommt, ihr traurigen Gemüter (Hosea 6)

\flagverse{1.} Kommt, ihr traurigen Gemüter,\\
kommt, wir wollen wiederkehrn\\
zu dem Herren, dessen Güter\\
kein Verderben kann verzehrn;\\
dessen Macht kein Unglück fällt,\\
dessen Gnade wieder stellt,\\
was sein Eifer umgestürzet:\\
Seine Gnad bleibt unverkürzet.

\flagverse{2.} Zwar er hat uns ja zerrissen\\
mit ergrimmtem Angesicht\\
und uns, da er uns geschmissen,\\
sehr erbärmlich zugericht't.\\
Doch deswegen unverzagt!\\
Eben der uns schlägt und plagt,\\
wird die Wunden unsrer Sünden\\
wieder heilen und verbinden.

\flagverse{3.} Alle Not, die uns umfangen,\\
springt vor seinem Arm entzwei;\\
wenn zwei Tage sind vergangen,\\
macht er uns vom Tode frei,\\
daß wir, wenn des dritten Licht\\
durch des Himmels Fenster bricht,\\
fröhlich auf erneurter Erden\\
vor ihm stehn und leben werden.

\flagverse{4.} Alsdann wird man acht drauf haben\\
und mit großem Fleiße sehn,\\
was für Wundergnad und Gaben\\
uns von obenher geschehn.\\
Da wird dieses nur allein\\
unsers Herzens Sorge sein,\\
daß wir Gott, des wir uns nennen,\\
mögen recht und wohl erkennen.

\flagverse{5.} Denn er wird sich zu uns machen\\
wie die schöne Morgenröt,\\
über welche Lust und Lachen\\
bei der ganzen Welt entsteht.\\
Er wird kommen uns zur Freud\\
eben zu der rechten Zeit,\\
voller süßen Kraft und Segen,\\
wie die früh und spaten Regen.

\flagverse{6.} Ach, wie will ich dich ergötzen,\\
o mein hochgeliebtes Volk!\\
Meine Gnade soll dich netzen\\
wie ein ausgespannte Wolk,\\
eine Wolke, die das Feld,\\
wann der Morgen weckt die Welt\\
und die Sonne noch nicht leuchtet,\\
mit dem frischen Tau befeuchtet.
   
\end{verse}
\end{multicols}
%\attrib{\small{THZE}}

\index{Kommt, Ihr traurigen Gemüter}
\newpage
\subsection*{\centerline{Meine Seele ist in der Stille}}
\addcontentsline{toc}{subsection}{Meine Seele ist in der Stille}
%StartInfo%%%%%%%%%%%%%%%%%%%%%%%%%%%%%%%%%%%%%%%%%%%%%%%%%%%%%%%%%%%%%%%%%%%%
%  Autor:
%  Titel:
%  File:
%  Ref:
%  Mod:
%EndInfo%%%%%%%%%%%%%%%%%%%%%%%%%%%%%%%%%%%%%%%%%%%%%%%%%%%%%%%%%%%%%%%%%%%%%%
%\poemtitle{pt}
\begin{multicols}{2}
\settowidth{\versewidth}{Meine Hasser, hört! Wie lange}
\begin{verse}[\versewidth]
%der 62. Psalm\\
%meine Seele ist in der Stille

\flagverse{1.} Meine Seel ist in der Stille,\\
tröstet sich des Höchsten Kraft,\\
dessen Rat und heilger Wille\\
mir bald Rat und Hilfe schafft.\\
Der kann mehr als alle Götter,\\
ist mein Hort, mein Heil, mein Retter,\\
daß kein Fall mich stürzen kann,\\
trät er noch so heftig an.

\flagverse{2.} Meine Hasser, hört! Wie lange\\
stellt ihr alle einem nach?\\
Ihr macht meinem Herzen bange,\\
mir zur Ehr und euch zur Schmach,\\
hanget wie zerrißne Mauern\\
und wie Wände, die nicht dauern,\\
über mir und seid bedacht,\\
wie ich werde totgemacht.

\flagverse{3.} Ja fürwahr, daß einge denken,\\
die, so mir zuwider seind,\\
wie sie mir mein Leben senken\\
dahin, wo kein Licht mehr scheint:\\
Darum geht ihr Mund aufs Lügen\\
und das Herz auf lauter Trügen;\\
gute Wort und falsche Tück\\
ist ihr bestes Meisterstück.

\flagverse{4.} Dennoch bleib ich ungeschrecket,\\
und mein Geist ist unverzagt\\
in dem Gotte, der mich decket,\\
wann die arge Welt mich plagt.\\
Auf den harret meine Seele;\\
da ist Trost, den ich erwähle,\\
da ist Schutz, der mir gefällt,\\
und Errettung, die mich hält.

\flagverse{5.} Nimmer, nimmer werd ich fallen,\\
nimmer werd ich untergehn,\\
denn hier ist, der mich vor allen,\\
die mich drücken, kann erhöhn;\\
bei dem ist mein Heil und Ehre,\\
meine Stärke, meine Wehre;\\
meine Freud und Zuversicht\\
ist nur stets auf Gott gericht.

\flagverse{6.} Hoffet allzeit, lieben Leute,\\
hoffet allzeit stark auf ihn.\\
Kommt die Hilfe nicht bald heute,\\
falle doch der Mut nicht hin.\\
Sondern schüttet aus dem Herzen\\
eures Herzens Sorg und Schmerzen,\\
legt sie vor sein Angesicht,\\
traut ihm fest und zweifelt nicht.

\flagverse{7.} Gott kann alles Unglück enden,\\
wirds auch herzlich gerne tun\\
denen, die sich zu ihm wenden\\
und auf seiner Güte ruhn.\\
Aber Menschenhilf ist nichtig,\\
ihr Vermögen ist nicht tüchtig,\\
wär es gleich noch eins so groß,\\
uns zu machen frei und los.

\flagverse{8.} Große Leute, große Toren!\\
Prangen sehr und sind doch Kot,\\
füllen Sinnen, Aug und Ohren:\\
Kommts zur Tat, so sind sie tot;\\
will man ihres Tuns und Sachen\\
eine Prob und Rechnung machen,\\
nach dem Ausschlag des Gewichts\\
sind sie weniger denn nichts.

\flagverse{9.} Laßt sie fahren, liebe Kinder,\\
da ist schlechter Vorteil bei!\\
Habt vor allem, was die Sünder\\
frechlich treiben, Furcht und Scheu!\\
Laßt euch Eitelkeit nicht fangen,\\
nach, was nichts ist, nicht verlangen;\\
käm auch Gut und Reichtum an,\\
ei, so hängt das Herz nicht dran!

\flagverse{10.} Wo das Herz am besten stehe,\\
lehrt am besten Gottes Wort\\
aus der güldnen Himmelshöhe;\\
denn da hör ich fort und fort,\\
daß er groß und reich von Kräften,\\
rein und heilig in Geschäften,\\
gütig dem, der Gutes tut.\\
Nun, der sei mein schönstes Gut.
   
\end{verse}
\end{multicols}
%\attrib{\small{THZE}}

\index{Meine Seele ist in der Stille}
\newpage
\subsection*{\centerline{Nicht so traurig, nicht so sehr}}
\addcontentsline{toc}{subsection}{Nicht so traurig nicht so sehr}
%StartInfo%%%%%%%%%%%%%%%%%%%%%%%%%%%%%%%%%%%%%%%%%%%%%%%%%%%%%%%%%%%%%%%%%%%%
%  Autor:
%  Titel:
%  File:
%  Ref:
%  Mod:
%EndInfo%%%%%%%%%%%%%%%%%%%%%%%%%%%%%%%%%%%%%%%%%%%%%%%%%%%%%%%%%%%%%%%%%%%%%%
%\poemtitle{pt}
\begin{multicols}{2}
\settowidth{\versewidth}{Nicht so traurig, nicht so sehr,}
\begin{verse}[\versewidth]
%nicht so traurig, nicht so sehr, meine Seele, sei betrübt\\
%(1. Timoth. 6, 6 ff.)

\flagverse{1.} Nicht so traurig, nicht so sehr,\\
meine Seele, sei betrübt,\\
daß dir Gott Glück, Gut und Ehr\\
nicht so viel wie andern gibt!\\
Nimm vorlieb mit deinem Gott!\\
Hast du Gott, so hats nicht not.

\flagverse{2.} Du noch einzig Menschenkind\\
habt ein Recht in dieser Welt;\\
alle, die geschaffen sind,\\
sind nur Gäst im fremden Zelt;\\
Gott ist Herr in seinem Haus,\\
wie er will, so teilt er aus.

\flagverse{3.} Bist du doch darum nicht hier,\\
daß du Erden haben sollt,\\
schau den Himmel über dir,\\
da, da ist dein edles Gold,\\
da ist Ehre, da ist Freud,\\
Freud ohn End, Ehr ohne Neid.

\flagverse{4.} Der ist albern, der sich kränkt\\
um ein Hand voll Eitelkeit,\\
wenn ihm Gott dagegen schenkt\\
schätze der Beständigkeit;\\
bleibt der Zentner dein Gewinn,\\
fahr der Heller immer hin!

\flagverse{5.} Schaue alle Güter an,\\
die dein Herz für Güter hält,\\
keines mit dir gehen kann,\\
wann du gehest aus der Welt;\\
alles bleibet hinter dir,\\
wann du trittst ins Grabes Tür.

\flagverse{6.} Aber was die Seele nährt,\\
Gottes Huld und Christi Blut,\\
wird von keiner Zeit verzehrt,\\
ist und bleibet allzeit gut;\\
Erdengut zerfällt und bricht,\\
Seelengut das schwindet nicht.

\flagverse{7.} Ach, wie bist du doch so blind\\
und im Denken unbedacht!\\
Augen hast du, Menschenkind,\\
und hast doch noch nie betracht\\
deiner Augen helles Glas:\\
Siehe, welch ein Schatz ist das!

\flagverse{8.} Zähle deine Finger her\\
und der andern Glieder Zahl;\\
keins ist, das dir unwert wär,\\
ehrst und liebst sie allzumal;\\
keines gäbst du weg um Gold,\\
wenn man dirs abnehmen wollt.

\flagverse{9.} Nun, so gehe in den Grund\\
deines Herzens, das dich lehrt,\\
wie viel Gutes alle Stund\\
dir von oben wird beschert:\\
Du hast mehr als Sand am Meer,\\
und willst doch noch immer mehr.

\flagverse{10.} Wüßte, der im Himmel lebt,\\
daß dir wäre nütz und gut,\\
wonach so begierig strebt\\
dein verblendet Fleisch und Blut,\\
würde seine Frömmigkeit\\
dich nicht lassen unerfreut.

\flagverse{11.} Gott ist deiner Liebe voll\\
und von ganzem Herzen treu;\\
wenn du wünschest, prüft er wohl,\\
wie dein Wunsch beschaffen sei;\\
ist dirs gut, so geht ers ein,\\
ists dein Schade, spricht er: Nein.

\flagverse{12.} Unterdessen trägt sein Geist\\
dir in deines Herzens Haus\\
Manna, das die Engel speist,\\
ziert und schmückt es herrlich aus,\\
ja erwählet, dir zum Heil,\\
dich zu seinem Gut und Teil.

\flagverse{13.} Ei, so richte dich empor,\\
du betrübtes Angesicht!\\
Laß das Seufzen, nimm hervor\\
deines Glaubens Freudenlicht;\\
das behalt, wenn dich die Nacht\\
deines Kummers traurig macht.

\flagverse{14.} Setze als ein Himmelssohn\\
deinem Willen Maß und Ziel,\\
rühre stets vor Gottes Thron\\
deines Dankens Saitenspiel,\\
weil dir schon gegeben ist\\
mehres als du würdig bist.

\end{verse}
\end{multicols}
%\attrib{\small{THZE}}

\begin{center}
\settowidth{\versewidth}{Der, vor dem die Welt erschrickt,}
\begin{verse}[\versewidth]

\flagverse{15.} Führe deines Lebens Lauf\\
allzeit Gottes eingedenk.\\
Wie es kommt, nimm alles auf\\
als ein wohlbedacht Geschenk.\\
Geht dirs widrig, laß es gehn!\\
Gott und Himmel bleibt dir stehn.
  
\end{verse}
\end{center}




\index{Nicht so traurig, nicht so sehr}
\newpage
\subsection*{\centerline{Noch dennoch mußt du drum nicht ganz}}
\addcontentsline{toc}{subsection}{Noch dennoch mußt du drum nicht ganz}
%StartInfo%%%%%%%%%%%%%%%%%%%%%%%%%%%%%%%%%%%%%%%%%%%%%%%%%%%%%%%%%%%%%%%%%%%%
%  Autor:
%  Titel:
%  File:
%  Ref:
%  Mod:
%EndInfo%%%%%%%%%%%%%%%%%%%%%%%%%%%%%%%%%%%%%%%%%%%%%%%%%%%%%%%%%%%%%%%%%%%%%%
%\poemtitle{pt}
\begin{multicols}{2}
\settowidth{\versewidth}{Noch dennoch mußt du drum nicht ganz}
\begin{verse}[\versewidth]
%noch dennoch mußt du drum nicht ganz in Traurigkeit versinken

\flagverse{1.} Noch dennoch mußt du drum nicht ganz\\
in Traurigkeit versinken,\\
gott wird des süßen Trostes Glanz\\
schon wieder lassen blinken.\\
Steh in Geduld, wart in der Still\\
und laß Gott machen, wie er will,\\
er kanns nicht böse machen.

\flagverse{2.} Ist denn dies unser erstes Mal,\\
daß wir betrübet werden?\\
Was haben wir als Angst und Qual\\
bisher gehabt auf Erden?\\
Wir sind wohl mehr so hoch gekränkt,\\
und hat doch Gott uns drauf geschenkt\\
ein Stündlein voller Freuden.

\flagverse{3.} So ist auch Gottes Meinung nicht,\\
wenn er uns Unglück sendet,\\
als sollt darum sein Angesicht\\
ganz von uns sein gewendet;\\
nein, sondern dieses ist sein Rat,\\
daß der, so ihn verlassen hat,\\
durchs Unglück wiederkehre.

\flagverse{4.} Denn das ist unser Fleisches Mut,\\
wenn wir in Freuden leben,\\
daß wir dann unserm höchsten Gut\\
am ersten Urlaub geben,\\
wir sind von Erd und halten wert\\
viel mehr, was hier ist auf der Erd\\
als was im Himmel wohnet.

\flagverse{5.} Drum fährt uns Gott durch unsern Sinn\\
und läßt uns Weh geschehen;\\
er nimmt oft, was uns lieb, dahin,\\
damit wir aufwärts sehen\\
und uns zu seiner Güt und Macht,\\
die wir bisher nicht groß geacht,\\
als Kinder wiederfinden.

\flagverse{6.} Tun wir nun das, ist er bereit,\\
uns wieder anzunehmen,\\
macht aus dem Leide lauter Freud\\
und Lachen aus dem Grämen,\\
und ist ihm das gar schlichte Kunst;\\
wen er umfängt mit Lieb und Gunst,\\
dem ist geschwind geholfen.

\flagverse{7.} Drum falle, du betrübtes Heer,\\
in Demut vor ihm nieder;\\
sprich: Herr, wir geben dir die Ehr,\\
ach, nimm uns Sünder wieder\\
in deine Gnade! Reiß die Last,\\
die du uns aufgeleget hast,\\
hinweg, heil unsern Schaden!

\flagverse{8.} Denn Gnade gehet doch vor Recht,\\
zorn muß der Liebe weichen,\\
wenn wir erliegen, muß uns schlecht\\
gott sein Erbarmen reichen;\\
dies ist die Hand, die uns erhält,\\
wo wir die lassen, bricht und fällt\\
all unser Tun in Haufen.

\flagverse{9.} Auf Gottes Liebe mußt du stehn\\
und dich nicht lassen fällen,\\
wenn auch der Himmel ein wollt gehn\\
und alle Welt zerschellen;\\
gott hat uns Gnade zugesagt,\\
sein Wort ist klar, wer sich drauf wagt,\\
dem kann es nimmer fehlen.

\flagverse{10.} So darfst du auch an seiner Kraft\\
gar keinen Zweifel haben.\\
Wer ists, der alle Dinge schafft?\\
Wer teilt aus alle Gaben?\\
Gott tuts! Und das ist auch der Mann,\\
der Rat und Tat erfinden kann,\\
wann jedermann verzaget.

\flagverse{11.} Deucht dir die Hilf unmöglich sein,\\
so sollst du gleichwohl wissen:\\
Gott räumt uns dieses nimmer ein,\\
daß er sich laß einschließen\\
in unsers Sinnes engen Stall;\\
sein Arm ist frei, tut überall\\
viel mehr als wir verstehen.

\flagverse{12.} Was ist sein ganzes wertes Reich\\
als lauter Wundersachen?\\
Er hilft und baut, wann wir uns gleich\\
des gar kein Hoffnung machen,\\
und das ist seines Namens Ruhm,\\
den du, wann du sein Heiligtum\\
willst sehen, ihm mußt geben.

\end{verse}
\end{multicols}
%\attrib{\small{THZE}}

\index{Noch dennoch mußt du drum}
\newpage
\subsection*{\centerline{Nun, du lebest, unsre Krone}}              %witt:? --Trost
\addcontentsline{toc}{subsection}{Nun du lebest unsre Krone}              %witt:? --Trost}
%StartInfo%%%%%%%%%%%%%%%%%%%%%%%%%%%%%%%%%%%%%%%%%%%%%%%%%%%%%%%%%%%%%%%%%%%%
%  Autor:
%  Titel:
%  File:
%  Ref:
%  Mod:
%EndInfo%%%%%%%%%%%%%%%%%%%%%%%%%%%%%%%%%%%%%%%%%%%%%%%%%%%%%%%%%%%%%%%%%%%%%%
%\poemtitle{pt}
\begin{multicols}{2}
\settowidth{\versewidth}{Freunden soll man Freuden gönnen}
\begin{verse}[\versewidth]
%nun, du lebest, unsre Krone\\
%auf den Tod des Hofkammergerichtsrats Petrus Fritzen (1650)

\flagverse{1.} Nun, du lebest, unsre Krone,\\
in der süßen sanften Ruh,\\
bringst die Zeit bei Gottes Throne\\
ohne Zeit und Ende zu!\\
Du hast ewge Freud und Zier,\\
und wir sollten für und für\\
uns mit unsern Tränen kränken?\\
Auf! Und laßt uns recht bedenken!

\flagverse{2.} Freunden soll man Freuden gönnen\\
lachen, wenn sie fröhlich sein!\\
Tränen laß zu der Zeit rinnen,\\
wenn sie liegen in der Pein;\\
aber wenn der Sieg erlangt\\
und der Held im Kranze prangt,\\
wenn das Herzleid weggeschlagen,\\
legt sich billig Schmerz und Klagen.

\flagverse{3.} Edles Herz, du hast bezwungen\\
alles, was dir widrig war:\\
Alle Schmerzen, die sich drungen\\
in dein Herz mit großer Schar;\\
allen Jammer, alle Müh,\\
alle Sorgen, die dich früh,\\
auch oft bei den späten Nachten\\
voller Angst und Wehmut machten.

\flagverse{4.} Gott weiß wohl, was wir vermögen\\
und wie stark die Schulter sei,\\
da er will sein Kreuz hinlegen;\\
dessen Huld und Vatertreu\\
hat auch dir die schwere Last,\\
die du ausgestanden hast,\\
über dein Haupt lassen gehen.\\
Wer viel kann, muß viel ausstehen.

\flagverse{5.} Wärst du einer aus dem Orden,\\
denen Herz und Mund entfällt,\\
wenn sie nur berühret worden\\
von des rauhen Unglücks Kält,\\
ei, so würde nimmermehr\\
ein so großes Jammerheer\\
Gott, der Geber aller Gaben,\\
über dich verhänget haben.

\flagverse{6.} Freue dich, Du hast gewonnen\\
durch des Höchsten Stärk und Kraft;\\
jetzo gehst du, gleich der Sonnen,\\
mitten in der Bürgerschaft\\
der sehr schönen neuen Stadt,\\
die uns Gott gebauet hat,\\
springst und singst und holest wieder\\
mit den Engeln süße Lieder.

\flagverse{7.} Christus wischet selbst die Tränen\\
dir von deinem Angesicht;\\
dein Herz hört auf, sich zu sehnen,\\
weiß von keinem Mangel nicht,\\
ohne daß du, die du hier\\
hast gelassen hinter dir,\\
auch in solchem Freudenleben\\
balde möchtest sehen schweben.

\flagverse{8.} Nun, wir werden balde kommen\\
aus dem Leide zu der Freud\\
und dich mit viel tausend Frommen\\
schauen in der Seligkeit!\\
O wie herrlich! O wie schön\\
wirst du und wir mit dir gehn,\\
wenn uns wird, anstatt der Erden,\\
Gottes Reich zu Teile werden.

\end{verse}
\end{multicols}
%\attrib{\small{THZE}}

\index{Nun, du lebest, unsre Krone}
\newpage
\subsection*{\centerline{Schwing dich auf zu deinem Gott}}
\addcontentsline{toc}{subsection}{Schwing dich auf zu deinem Gott}
%StartInfo%%%%%%%%%%%%%%%%%%%%%%%%%%%%%%%%%%%%%%%%%%%%%%%%%%%%%%%%%%%%%%%%%%%%
%  Autor:
%  Titel:
%  File:
%  Ref:
%  Mod:
%EndInfo%%%%%%%%%%%%%%%%%%%%%%%%%%%%%%%%%%%%%%%%%%%%%%%%%%%%%%%%%%%%%%%%%%%%%%
%\poemtitle{pt}
\begin{multicols}{2}
\settowidth{\versewidth}{Schwing dich auf zu deinem Gott,}
\begin{verse}[\versewidth]
%schwing dich auf zu deinem Gott

\flagverse{1.} Schwing dich auf zu deinem Gott,\\
du betrübte Seele!\\
Warum liegst du, Gott zum Spott,\\
in der Schwermutshöhle?\\
Merkst du nicht des Satans List?\\
Er will durch sein Kämpfen\\
deinen Trost, den Jesus Christ\\
dir erworben, dämpfen.

\flagverse{2.} Schüttle deinen Kopf und sprich:\\
Fleuch, du alte Schlange!\\
Was erneust du deinen Stich,\\
machst mir angst und bange?\\
Ist dir doch der Kopf zerknickt,\\
und ich bin durchs Leiden\\
meines Heilands dir entzückt\\
in den Saal der Freuden.

\flagverse{3.} Wirfst du mir mein Sünd'gen für?\\
Wo hat Gott befohlen,\\
daß mein Urteil über mir\\
ich bei dir soll holen?\\
Wer hat dir die Macht geschenkt,\\
andre zu verdammen,\\
der du selbst doch liegst versenkt\\
in der Höllen Flammen?

\flagverse{4.} Hab ich was nicht recht getan,\\
ist mirs leid von Herzen;\\
dahingegen nehm ich an\\
Christi Blut und Schmerzen.\\
Denn das ist die Ranzion\\
meiner Missetaten.\\
Bring ich dies vor Gottes Thron,\\
ist mir wohl geraten.

\flagverse{5.} Christi Unschuld ist mein Ruhm,\\
sein Recht meine Krone,\\
sein Verdienst mein Eigentum,\\
da ich frei in wohne\\
als in einem festen Schloß,\\
das kein Feind kann fällen,\\
brächt er gleich davor Geschoß\\
und Gewalt der Höllen.

\flagverse{6.} Stürme, Teufel und du Tod,\\
was könnt ihr mir schaden?\\
Deckt mich doch in meiner Not\\
Gott mit seiner Gnaden.\\
Der Gott, der mir seinen Sohn\\
selbst verehrt aus Liebe,\\
daß der ewge Spott und Hohn\\
mich nicht dort betrübe.

\flagverse{7.} Schreie, tolle Welt, es sei\\
mir Gott nicht gewogen,\\
es ist lauter Täuscherei\\
und im Grund erlogen.\\
Wäre Gott mir gram und feind,\\
würd er seine Gaben,\\
die mein eigen worden seind,\\
wohl behalten haben.

\flagverse{8.} Denn was ist im Himmelszelt,\\
was im tiefen Meere,\\
was ist Gutes in der Welt,\\
das nicht mir gut wäre?\\
Weme brennt das Sternenlicht?\\
Wozu ist gegeben\\
luft und Wasser? Dient es nicht\\
mir und meinem Leben?

\flagverse{9.} Weme wird das Erdreich naß\\
von dem Tau und Regen?\\
Weme grünet Laub und Gras?\\
Weme füllt der Segen\\
berg und Tale, Feld und Wald?\\
Wahrlich, mir zur Freude,\\
daß ich meinen Aufenthalt\\
hab und Leibesweide.

\flagverse{10.} Meine Seele lebt in mir\\
durch die süßen Lehren,\\
so die Christen mit Begier\\
alle Tage hören.\\
Gott eröffnet früh und spat\\
meinen Geist und Sinnen,\\
daß sie seines Geistes Gnad\\
in sich ziehen können.

\flagverse{11.} Was sind der Propheten Wort\\
und Apostel Schreiben\\
als ein Licht am dunklen Ort,\\
fackeln, die vertreiben\\
meines Herzens Finsternis\\
und in Glaubenssachen\\
das Gewissen fein gewiß\\
und recht grundfest machen?

\flagverse{12.} Nun, auf diesen heilgen Grund\\
bau ich mein Gemüte,\\
sehe, wie der Höllenhund\\
zwar dawider wüte;\\
gleichwohl muß er lassen stehn,\\
was Gott aufgerichtet,\\
aber schändlich muß vergehn,\\
was er selber dichtet.

\flagverse{13.} Ich bin Gottes, Gott ist mein:\\
Wer ist, der uns scheide?\\
Dringt das liebe Kreuz herein\\
mit dem bittern Leide,\\
laß es dringen, kommt es doch\\
von geliebten Händen,\\
bricht und kriegt geschwind ein Loch,\\
wenn es Gott will wenden.

\flagverse{14.} Kinder, die der Vater soll\\
ziehn zu allem Guten,\\
die gedeihen selten wohl\\
ohne Zucht und Ruten.\\
Bin ich denn nun Gottes Kind,\\
warum will ich fliehen,\\
wenn er mich von meiner Sünd\\
auf was Guts will ziehen?

\flagverse{15.} Es ist herzlich gut gemeint\\
mit der Christen Plagen:\\
Wer hier zeitlich wohl geweint,\\
darf nicht ewig klagen,\\
sondern hat vollkommne Lust\\
dort in Christi Garten\\
(dem er einig recht bewußt)\\
endlich zu gewarten.

\flagverse{16.} Gottes Kinder säen zwar\\
traurig und mit Tränen,\\
aber endlich bringt das Jahr,\\
wonach sie sich sehnen;\\
denn es kommt die Erntezeit,\\
da sie Garben machen,\\
da wird all ihr Gram und Leid\\
lauter Freud und Lachen.

\end{verse}
\end{multicols}

\begin{center}
\settowidth{\versewidth}{Der, vor dem die Welt erschrickt,}
\begin{verse}[\versewidth]



\flagverse{17.} Ei, so faß, o Christenherz,\\
alle deine Schmerzen,\\
wirf sie fröhlich hinterwärts,\\
laß des Trostes Kerzen\\
dich entzünden mehr und mehr,\\
gib dem großen Namen\\
deines Gottes Preis und Ehr,\\
er wird helfen. Amen.

  
\end{verse}
\end{center}



%\attrib{\small{THZE}}

\index{Schwing die auf zu deinem Gott}
\newpage
\subsection*{\centerline{Sei wohlgemut, o Christenseel}}            %60
\addcontentsline{toc}{subsection}{Sei wohlgemut o Christenseel}            %60}
%StartInfo%%%%%%%%%%%%%%%%%%%%%%%%%%%%%%%%%%%%%%%%%%%%%%%%%%%%%%%%%%%%%%%%%%%%
%  Autor:
%  Titel:
%  File:
%  Ref:
%  Mod:
%EndInfo%%%%%%%%%%%%%%%%%%%%%%%%%%%%%%%%%%%%%%%%%%%%%%%%%%%%%%%%%%%%%%%%%%%%%%
%\poemtitle{pt}
\begin{multicols}{2}
\settowidth{\versewidth}{ein frommer Mensch, der nicht geschwebt}
\begin{verse}[\versewidth]
%der 73. Psalm\\
%sei wohlgemut, o Christenseel

\flagverse{1.} Sei wohlgemut, o Christenseel,\\
im Hochmut deiner Feinde;\\
es hat das rechte Israel\\
noch dennoch Gott zum Freunde,\\
wer glaubt und hofft, der wird geliebt\\
von dem, der unsern Herzen gibt\\
Trost, Friede, Freud und Leben.

\flagverse{2.} Zwar tut es weh und ärgert sehr,\\
wenn man vor Augen siehet,\\
wie dieser Welt gottloses Heer\\
so schön und herrlich blühet;\\
sie sind in keiner Todesfahr,\\
erleben hier so manches Jahr\\
und stehen wie Paläste.

\\flagverse{3.} Sie haben Glück und wissen nicht,\\
wie Armen sei zu Mute;\\
Gold ist ihr Gott, Geld ist ihr Licht.\\
Sind stolz bei großem Gute;\\
sie reden hoch, und das gilt schlecht:\\
Was andre sagen, ist nicht recht,\\
es ist ihn'n viel zu wenig.

\flagverse{4.} Des Pöbelvolks unweiser Hauf\\
ist auch auf Ihrer Seite;\\
sie sperren Maul und Nasen auf\\
und sprechen: Das sind Leute!\\
Das sind ohn allen Zweifel die,\\
die Gott vor allen andern hie\\
zu Kindern auserkoren.

\flagverse{5.} Was sollte doch der große Gott\\
nach jenen andern fragen,\\
die sich mit Armut, Kreuz und Not\\
bis in die Grube tragen?\\
Wem hier des Glückes Gunst und Schein\\
nicht leuchtet, kann kein Christe sein,\\
er ist gewiß verstoßen.

\flagverse{6.} Solls denn, mein Gott, vergebens sein\\
daß dich mein Herze liebet?\\
Ich liebe dich und leide Pein,\\
bin dein und doch betrübet.\\
Ich hätte bald auch so gedacht\\
wie jene Rotte, die nichts acht't\\
als was vor Augen pranget.

\flagverse{7.} Sieh aber, sieh, in solchem Sinn\\
wär ich zu weit gekommen,\\
ich hätte bloß verdammt dahin\\
die ganze Schar der Frommen;\\
denn hat auch je einmal gelebt\\
ein frommer Mensch, der nicht geschwebt\\
in großem Kreuz und Leiden?

\flagverse{8.} Ich dachte hin, ich dachte her,\\
ob ich es möcht ergründen,\\
es war mir aber viel zu schwer,\\
den rechten Schluß zu finden,\\
bis daß ich ging ins Heiligtum\\
und merkte, wie du, unser Ruhm,\\
die Bösen führst zu Ende.

\flagverse{9.} Ihr Gang ist schlüpfrig, glatt ihr Pfad,\\
ihr Tritt ist ungewisse;\\
du suchst sie heim nach ihrer Tat\\
und stürzest ihre Füße.\\
Im Hui ist alles umgewendt,\\
da nehmen sie ein plötzlich End\\
und fahren hin mit Schrecken.

\flagverse{10.} Heut grünen sie gleich wie ein Baum,\\
ihr Herz ist froh und lachet,\\
und morgen sind sie wie ein Traum,\\
von dem der Mensch aufwachet,\\
ein bloßer Schatt, ein totes Bild,\\
das weder Hand noch Augen füllt,\\
verschwindt im Augenblicke.

\flagverse{11.} Es mag drum sein; es wäre gleich\\
mein Kreuz so lang ich lebe,\\
ich habe gnug am Himmelreich,\\
dahin ich täglich strebe.\\
Hält mich die Welt gleich als ein Tier,\\
ei, lebst du, Gott, doch über mir,\\
du bist mein Ehr und Krone.

\flagverse{12.} Du heilest meines Herzens Stich\\
mit deiner süßen Liebe\\
und wehrst dem Unglück, daß es mich\\
nicht allzu hoch betrübe;\\
du leitest mich mit deiner Hand\\
und wirst mich endlich in den Stand\\
der rechten Ehren setzen.

\flagverse{13.} Wenn ich nur dich, o starker Held,\\
behalt in meinem Leide,\\
so acht ichs nicht, wenn gleich zerfällt\\
das große Weltgebäude.\\
Du bist mein Himmel, und dein Schoß\\
bleibt allezeit mein Burg und Schloß,\\
wann diese Erd entweichet.

\flagverse{14.} Wann mir gleich Leib und Seel verschmacht,\\
so kann ich doch nicht sterben,\\
denn du bist meines Lebens Macht\\
und läßt mich nicht verderben.\\
Was frag ich nach dem Erb und Teil\\
auf dieser Welt? Du, du, mein Heil,\\
du bist mein Teil und Erbe.

\flagverse{15.} Das kann die gottvergessne Rott\\
mit Wahrheit nimmer sagen;\\
sie weicht von dir und wird zum Spott,\\
verdirbt in großen Plagen.\\
Mir aber ists, wie dir bewußt,\\
die größte Freud und höchste Lust,\\
daß ich mich zu dir halte.

\flagverse{16.} So will ich nun die Zuversicht\\
auf dich beständig setzen,\\
es werde mich dein Angesicht\\
zu rechter Zeit ergötzen.\\
Indessen will ich stille ruhn\\
und deiner weisen Hände Tun\\
mit meinem Munde preisen.
   
\end{verse}
\end{multicols}
%\attrib{\small{THZE}}

\index{Sei wohlgemut, o Christenheit}
\newpage
\subsection*{\centerline{Warum sollt ich mich doch grämen?}}        %witt:denn
\addcontentsline{toc}{subsection}{Warum sollt ich mich doch grämen?}        %witt:denn}
%StartInfo%%%%%%%%%%%%%%%%%%%%%%%%%%%%%%%%%%%%%%%%%%%%%%%%%%%%%%%%%%%%%%%%%%%%
%  Autor:
%  Titel:
%  File:
%  Ref:
%  Mod:
%EndInfo%%%%%%%%%%%%%%%%%%%%%%%%%%%%%%%%%%%%%%%%%%%%%%%%%%%%%%%%%%%%%%%%%%%%%%
%\poemtitle{pt}
\begin{multicols}{2}
\settowidth{\versewidth}{Gut und Blut, Leib, Seel und Leben}
\begin{verse}[\versewidth]

\flagverse{1.} Warum sollt ich mich doch grämen?\\
Hab ich doch\\
Christum noch,\\
wer will mir den nehmen?\\
Wer will mir den Himmel rauben,\\
den mir schon\\
Gottes Sohn\\
beigelegt im Glauben?

\flagverse{2.} Nackend lag ich auf dem Boden,\\
da ich kam,\\
da ich nahm\\
meinen ersten Odem;\\
nackend werd ich auch hinziehen,\\
wann ich werd\\
von der Erd\\
als ein Schatten fliehen.

\flagverse{3.} Gut und Blut, Leib, Seel und Leben\\
ist nicht mein;\\
Gott allein\\
ist es, ders gegeben.\\
Will ers wieder zu sich kehren,\\
nehm ers hin!\\
Ich will ihn\\
dennoch fröhlich ehren.

\flagverse{4.} Schickt er mir ein Kreuz zu tragen,\\
dringt herein\\
Angst und Pein,\\
sollt ich drum verzagen?\\
Der es schickt, der wird es wenden!\\
Er weiß wohl,\\
wie er soll\\
all mein Unglück enden.

\flagverse{5.} Gott hat mich bei guten Tagen\\
oft ergötzt:\\
Sollt ich jetzt\\
auch nicht etwas tragen?\\
Fromm ist Gott und schärft mit Maßen\\
sein Gericht;\\
kann mich nicht\\
ganz und gar verlassen.

\flagverse{6.} Satan, Welt und ihre Rotten\\
können mir\\
nichts mehr hier\\
tun, als meiner spotten.\\
Laß sie spotten, laß sie lachen!\\
Gott, mein Heil,\\
wird in Eil\\
sie zu Schanden machen.

\flagverse{7.} Unverzagt und ohne Grauen\\
soll ein Christ,\\
wo er ist,\\
stets sich lassen schauen.\\
Wollt ihn auch der Tod aufreiben,\\
soll der Mut\\
dennoch gut\\
und fein stille bleiben.

\flagverse{8.} Kann uns doch kein Tod nicht töten,\\
sondern reißt\\
unsern Geist\\
aus viel tausend Nöten;\\
schleußt das Tor des bittern Leiden\\
und macht Bahn,\\
da man kann\\
gehn zur Himmelsfreuden.

\flagverse{9.} Allda will in süßen Schätzen\\
ich mein Herz\\
auf den Schmerz\\
ewiglich ergötzen.\\
Hier ist kein recht Gut zu finden:\\
Was die Welt\\
in sich hält,\\
muß im Hui verschwinden.

\flagverse{10.} Was sind dieses Lebens Güter?\\
Eine Hand\\
voller Sand,\\
Kummer der Gemüter.\\
Dort, dort sind die edlen Gaben,\\
da mein Hirt,\\
Christus, wird\\
mich ohn Ende laben.

\flagverse{11.} Herr, mein Hirt, Brunn aller Freuden,\\
du bist mein,\\
ich bin dein,\\
niemand kann uns scheiden:\\
Ich bin dein, weil du dein Leben\\
und dein Blut\\
mir zugut\\
in den Tod gegeben.

\flagverse{12.} Du bist mein, weil ich dich fasse\\
und dich nicht,\\
o mein Licht,\\
aus dem Herzen lasse.\\
Laß mich, laß mich hingelangen,\\
da du mich\\
und ich dich\\
lieblich werd umfangen.

\end{verse}
\end{multicols}
%\attrib{\small{THZE}}

\index{Warum soll ich mich doch grämen}
\newpage
\subsection*{\centerline{Was Gott gefällt, mein frommes Kind}}
\addcontentsline{toc}{subsection}{Was Gott gefällt mein frommes Kind}
%StartInfo%%%%%%%%%%%%%%%%%%%%%%%%%%%%%%%%%%%%%%%%%%%%%%%%%%%%%%%%%%%%%%%%%%%%
%  Autor:
%  Titel:
%  File:
%  Ref:
%  Mod:
%EndInfo%%%%%%%%%%%%%%%%%%%%%%%%%%%%%%%%%%%%%%%%%%%%%%%%%%%%%%%%%%%%%%%%%%%%%%
%\poemtitle{pt}
\begin{multicols}{2}
\settowidth{\versewidth}{Was Gott gefällt, mein frommes Kind,}
\begin{verse}[\versewidth]
%was Gott gefällt, mein frommes Kind, nimm fröhlich an!

\flagverse{1.} Was Gott gefällt, mein frommes Kind,\\
nimm fröhlich an! Stürmt gleich der Wind\\
und braust, daß alles kracht und bricht,\\
so sei getrost, denn dir geschicht,\\
was Gott gefällt.

\flagverse{2.} Der beste Will ist Gottes Will,\\
auf diesem ruht man sanft und still,\\
da gib dich allzeit frisch hinein,\\
begehre nichts, als nur allein,\\
was Gott gefällt.

\flagverse{3.} Der klügste Sinn ist Gottes Sinn,\\
was Menschen sinnen, fället hin,\\
wird plötzlich kraftlos, müd und laß,\\
tut oft, was bös, und selten das,\\
was Gott gefällt.

\flagverse{4.} Der frömmste Mut ist Gottes Mut,\\
der niemand Arges gönnt und tut,\\
er segnet, wenn uns schilt und flucht\\
die böse Welt, die nimmer sucht,\\
was Gott gefällt.

\flagverse{5.} Das treuste Herz ist Gottes Herz,\\
treibt alles Unglück hinterwärts,\\
beschirmt und schützet Tag und Nacht\\
den, der stets hoch und herrlich acht,\\
was Gott gefällt.

\flagverse{6.} Ach könnt ich singen, wie ich wohl\\
im Herzen wünsch und billig soll,\\
so wollt ich öffnen meinen Mund\\
und singen jetzo diese Stund,\\
was Gott gefällt.

\flagverse{7.} Ich wollt erzählen seinen Rat\\
und übergroße Wundertat,\\
das süße Heil, die ewge Kraft,\\
die allenthalben wirkt und schafft,\\
was Gott gefällt.

\flagverse{8.} Er ist der Herrscher in der Höh,\\
auf ihm steht unser Wohl und Weh,\\
er trägt die Welt in seiner Hand,\\
hinwieder trägt uns See und Land,\\
was Gott gefällt.

\flagverse{9.} Er hält der Elemente Lauf,\\
und damit hält er uns auch auf,\\
gibt Sommer, Winter, Tag und Nacht\\
und alles, davon lebt und lacht,\\
was Gott gefällt.

\flagverse{10.} Sein Heer, die Sterne, Sonn und Mond\\
gehn ab und zu, wie sie gewohnt,\\
die Erd ist fruchtbar, bringt herfür\\
Korn, Öl und Most, Brot, Wein und Bier,\\
was Gott gefällt.

\flagverse{11.} Sein ist die Weisheit und Verstand,\\
ihm ist bewußt und wohlbekannt\\
sowohl wer Böses tut und übt\\
als auch wer Gutes tut und liebt,\\
was Gott gefällt.

\flagverse{12.} Sein Häuflein ist ihm lieb und wert;\\
sobald es sich zu Sünden kehrt,\\
so winkt er mit der Vaterrut\\
und locket, bis man wieder tut,\\
was Gott gefällt.

\flagverse{13.} Was unserm Herzen dienlich sei,\\
das weiß sein Herz, ist fromm dabei,\\
der keinem jemals Guts versagt,\\
der Guts gesucht, dem nachgejagt,\\
was Gott gefällt.

\flagverse{14.} Ist dem also, so mag die Welt\\
behalten, was ihr wohlgefällt;\\
du aber, mein Herz, halt genehm\\
und nimm fürlieb mit Gott und dem,\\
was Gott gefällt.

\flagverse{15.} Laß andre sich mit stolzem Mut\\
erfreuen über großes Gut,\\
du aber nimm des Kreuzes Last\\
und sei geduldig, wenn du hast,\\
was Gott gefällt.

\flagverse{16.} Lebst du in Sorg und großem Leid,\\
hast lauter Gram und Herzeleid,\\
ei, sei zufrieden; trägst du doch\\
in diesem sauren Lebensjoch,\\
was Gott gefällt.

\flagverse{17.} Mußt du viel leiden hie und dort,\\
so bleibe fest an deinem Hort,\\
denn alle Welt und Kreatur\\
ist unter Gott, kann nichts als nur,\\
was Gott gefällt.

\flagverse{18.} Wirst du veracht't von jedermann,\\
höhnt dich dein Feind und speit dich an:\\
Sei wohlgemut, denn Jesus Christ\\
erhöhet dich, weil in dir ist,\\
was Gott gefällt.

\flagverse{19.} Glaub, Hoffnung, Sanftmut und Geduld\\
erhalten Gottes Gnad und Huld;\\
die schleuß in deines Herzens Schrein,\\
so wird dein ewges Erbe sein,\\
was Gott gefällt.

\flagverse{20.} Dein Erb ist in dem Himmelsthron,\\
hier ist dein Zepter, Reich und Kron,\\
hier wirst du schmecken, hören, sehn,\\
hier wird ohn Ende dir geschehn,\\
was Gott gefällt.

\end{verse}
\end{multicols}
%\attrib{\small{THZE}}

\index{Was Gott gefällt, mein frommes Kind}
\newpage
\subsection*{\centerline{Was soll ich doch, o Ephraim}}
\addcontentsline{toc}{subsection}{Was soll ich doch o Ephraim}
%StartInfo%%%%%%%%%%%%%%%%%%%%%%%%%%%%%%%%%%%%%%%%%%%%%%%%%%%%%%%%%%%%%%%%%%%%
%  Autor:
%  Titel:
%  File:
%  Ref:
%  Mod:
%EndInfo%%%%%%%%%%%%%%%%%%%%%%%%%%%%%%%%%%%%%%%%%%%%%%%%%%%%%%%%%%%%%%%%%%%%%%
%\poemtitle{pt}
\begin{multicols}{2}
\settowidth{\versewidth}{Sollt ich nicht billig deiner Tat}
\begin{verse}[\versewidth]
%was soll ich doch, o Ephraim, was soll ich aus dir machen?\\
%(Hosea 11, 8/9)

\flagverse{1.} Was soll ich doch, o Ephraim,\\
was soll ich aus dir machen?\\
Der du so oftmals meinen Grimm\\
hast pflegen zu verlachen?\\
Soll ich dich schützen, Israel?\\
Soll ich dir deine freche Seel\\
hinfürder noch bewahren?\\
Aus welcher doch von Jugend auf\\
ein solcher großer Sündenhauf\\
ohn alle Scheu gefahren.

\flagverse{2.} Sollt ich nicht billig deiner Tat\\
und Leben gleich mich stellen?\\
Und dich wie Sodom ohne Gnad\\
und wie Adama fällen?\\
Sollt ich nicht billig meine Glut\\
auf dein verfluchtes Gut und Blut\\
wie auf Zeboim schütten?\\
Dieweil du ja mein Wort und Bahn\\
fast ärger noch, als sie getan,\\
bis hieher überschritten.

\flagverse{3.} Ja, billig sollt ich dich dahin\\
in alles Herzleid senken,\\
allein es will mir nicht zu Sinn,\\
ich hab ein andres Denken;\\
mein Herze will durchaus nicht dran,\\
daß es dir tu, wie du getan,\\
es brennt für Gnad und Liebe;\\
mich jammert dein von Herzen sehr\\
und kann nicht sehen, daß das Heer\\
der Höllen dich betrübe.

\flagverse{4.} Ich kann und mag nicht, wie du wohl\\
verdienet, dich verderben;\\
ich bin und bleib Erbarmens voll\\
und halte nichts vom Sterben;\\
denn ich bin Gott, der treue Gott,\\
mitnichten einer aus der Rott\\
der bösen Adamskinder,\\
die ohne Treu und Glauben seind\\
und werden ihren Feinden feind\\
und täglich größre Sünder.

\flagverse{5.} So bin ich nicht, das glaube mir,\\
und nimms recht zu Gemüte,\\
ich bin der Heilge unter dir,\\
der ich aus lauter Güte\\
für meine Feinde in den Tod\\
und in des bittern Kreuzes Not\\
mich als ein Lamm begeben;\\
ich, ich will tragen deine Last,\\
die du dir, Mensch, gehäufet hast,\\
auf daß du mögest leben.

\flagverse{6.} O heilger Herr, o ewges Heil,\\
versöhner meiner Sünden,\\
ach, heilge mich und laß mich teil\\
in, bei und an dir finden!\\
Erwecke mich zur wahren Reu\\
und gib, daß ich dein edle Treu\\
im festen Glauben fasse;\\
auch töte mich durch deinen Tod,\\
damit ich allen Sündenkot\\
hinfort von Herzen hasse.

\end{verse}
\end{multicols}
%\attrib{\small{THZE}}

\index{Was soll ich doch, o Ephraim}
\newpage
\subsection*{\centerline{Was trotzest du, stolzer Tyrann?}}
\addcontentsline{toc}{subsection}{Was trotzest du stolzer Tyrann?}
%StartInfo%%%%%%%%%%%%%%%%%%%%%%%%%%%%%%%%%%%%%%%%%%%%%%%%%%%%%%%%%%%%%%%%%%%%
%  Autor:
%  Titel:
%  File:
%  Ref:
%  Mod:
%EndInfo%%%%%%%%%%%%%%%%%%%%%%%%%%%%%%%%%%%%%%%%%%%%%%%%%%%%%%%%%%%%%%%%%%%%%%
%\poemtitle{pt}
\begin{multicols}{2}
\settowidth{\versewidth}{Dein Dichten, dein Trachten, dein Tun}
\begin{verse}[\versewidth]
%der 52. Psalm\\
%was trotzest du, stolzer Tyrann?

\flagverse{1.} Was trotzest du, stolzer Tyrann,\\
daß deine verkehrte Gewalt\\
den Armen viel Schaden tun kann?\\
Verkreuch dich und schweige nur bald!\\
Denn Gottes, des Ewigen Güte\\
bleibt immer in voller Blüte\\
und währet noch täglich und stehet,\\
ob alles gleich sonsten vergehet.

\flagverse{2.} Die Zunge, dein schändliches Glied,\\
du falscher verlogener Mund,\\
tut manchen gefährlichen Schnitt,\\
schlägt alles zu Schanden und wund;\\
was unrecht, das sprichst du mit Freuden,\\
was recht ist, das kannst du nicht leiden,\\
die Wahrheit verdrückst du, die Lügen\\
muß Oberhand haben und siegen.

\flagverse{3.} Dein Dichten, dein Trachten, dein Tun\\
ist einzig auf Schaden bedacht;\\
da ist dir unmöglich zu ruhn,\\
du habest denn Böses verbracht;\\
dein Rachen sucht lauter Verderben,\\
und wenn nur viel Fromme ersterben\\
von deiner vergällten Zungen,\\
so meinst du, es sei dir gelungen.

\flagverse{4.} Drum wird dich auch Gottes Gericht\\
zerstören, verheeren im Grimm;\\
die Rechte, die alles zerbricht\\
mit Donner und blitzender Stimm,\\
die wird dich zugrunde zuschlagen\\
und wird dich mit schrecklichen Plagen\\
aus deinem bisherigen Bleiben\\
samt allen den Deinen vertreiben.

\flagverse{5.} Das werden mit Freuden und Lust\\
die Frommen, Gerechten ersehn,\\
die anders bisher nicht gewußt,\\
als ob es nun gänzlich geschehn;\\
die werden mit Schrecken da stehen,\\
wenn jene zugrunde vergehen,\\
und endlich mit heiligem Lachen\\
sich wiederum lustig bei machen.

\flagverse{6.} Ei siehe! wirds heißen, da liegt\\
der prächtige, mächtige Mann,\\
der stetig mit Erden vergnügt,\\
der Himmel beiseite getan;\\
vom Reichtum war immer sein Prangen,\\
und wann er die Unschuld gefangen,\\
so hielt ers für treffliche Taten;\\
ei siehe, wie ists ihm geraten!

\flagverse{7.} Ich hoffe mit freudigem Geist\\
ein anders und besseres Glück,\\
denn was mir mein Vater verheißt,\\
das bleibet doch nimmer zurück.\\
Ich werde des Friedens genießen,\\
auch wird sich der Segen ergießen\\
und mich mit erwünschtem Gedeihen\\
samt allen den meinen erfreuen.

\flagverse{8.} Ich werde nach Weise des Baums,\\
der Öle trägt, grünen und blühn,\\
mich freuen des seligen Raums,\\
den ohne mein eignes Bemühn\\
mein Herrscher, mein Helfer, mein Leben\\
mir selber zu eigen gegeben\\
im Hause, da täglich mit Loben\\
sein Name wird herrlich erhoben.

\end{verse}
\end{multicols}
%\attrib{\small{THZE}}

\begin{center}
\settowidth{\versewidth}{Der, vor dem die Welt erschrickt,}
\begin{verse}[\versewidth]


\flagverse{9.} Trotz sei dir, du trotzender Kot!\\
Ich habe den Höchsten bei mir;\\
wo der ist, da hat es nicht Not,\\
und fürcht ich mich gar nicht vor dir.\\
Du, mein Gott, kannst alles wohl machen,\\
dich setz ich zum Richter der Sachen,\\
und weißt es: es wird sich mein Leiden\\
bald enden in Jauchzen und Freuden.

  
\end{verse}
\end{center}


\index{Was trotzest du, stolzer Tyrann}
\newpage
\subsection*{\centerline{Wer unterm Schirm des Höchsten sitzt}}
\addcontentsline{toc}{subsection}{Wer unterm Schirm des Höchsten sitzt}
%StartInfo%%%%%%%%%%%%%%%%%%%%%%%%%%%%%%%%%%%%%%%%%%%%%%%%%%%%%%%%%%%%%%%%%%%%
%  Autor:
%  Titel:
%  File:
%  Ref:
%  Mod:
%EndInfo%%%%%%%%%%%%%%%%%%%%%%%%%%%%%%%%%%%%%%%%%%%%%%%%%%%%%%%%%%%%%%%%%%%%%%
%\poemtitle{pt}
\begin{multicols}{2}
\settowidth{\versewidth}{Frisch auf, mein Herz! Gott stärket dich}
\begin{verse}[\versewidth]
%der 91. Psalm\\
%wer unterm Schirm des Höchsten sitzt, der ist sehr wohl bedecket

\flagverse{1.} Wer unterm Schirm des Höchsten sitzt,\\
der ist sehr wohl bedecket,\\
wenn alles donnert, kracht und blitzt,\\
bleibt sein Herz ungeschrecket;\\
er spricht zum Herrn: Du bist mein Licht,\\
mein Hoffnung, meine Zuversicht,\\
mein Turm und starke Feste,\\
du rettest mich vons Jägers Strick\\
und treibst des Todes Netz zurück\\
und schützest mich aufs beste.

\flagverse{2.} Frisch auf, mein Herz! Gott stärket dich\\
mit Kraft auf allen Seiten;\\
schau her, wie seine Flügel sich\\
ganz über dich ausbreiten!\\
Sein Schirm umfängt und deckt dich gar,\\
sein Schild fängt auf, was hie und dar\\
von Pfeilen fleugt und tobet:\\
Der Schild ist Gottes wahres Wort,\\
der Schirm ist, was der starke Hort\\
versprochen und gelobet.

\flagverse{3.} Wenn dich die schwarze Nacht umgibt,\\
kannst du fein sicher schlafen,\\
des Tages bleibst du unbetrübt\\
von deines Feindes Waffen.\\
Die Peste, die im Finstern schleicht,\\
und des Mittages umherkreucht,\\
wird von dir abgeführet;\\
und wenn gleich tausend fallen hier,\\
und zehentausend hart bei dir,\\
bleibst du doch unberühret.

\flagverse{4.} Hingegen wirst du Lust und Freud\\
an deinen Feinden sehen,\\
wenn ihnen alles Herzeleid\\
vom Höchsten wird geschehen;\\
wer Gott verläßt, wird wiederüm\\
verlassen und mit großem Grimm\\
zu seiner Zeit geschlagen;\\
du aber, der du bleibst bei Gott,\\
findst Gnad und darfst in keiner Not\\
ohn Hilf und Trost verzagen.

\flagverse{5.} Kein Übels wird zu deiner Hütt\\
eingehn und dir begegnen,\\
gott wird all deine Tritt und Schritt\\
auf deinen Wegen segnen;\\
denn er hat seiner Engelschar\\
befohlen, daß sie vor Gefahr\\
dich ganz genau bewahren,\\
daß dein Fuß möge sicher sein\\
und nicht vielleicht an einen Stein\\
zu deinem Schaden fahren.

\flagverse{6.} Du wirst auf wilden Leuen stehn\\
und treten auf die Drachen;\\
du wirst ihr Gift und scharfe Zähn\\
in deinem Sinn verlachen.\\
Das machts, daß Gott will bei dir sein,\\
der spricht: Mein Knecht begehret mein,\\
so will ich ihm beispringen;\\
er kennet meines Namens Zier,\\
drum will ich ihm auch nach Begier\\
mein Hilf und Rettung bringen.

\end{verse}
\end{multicols}

\begin{center}
\settowidth{\versewidth}{Der, vor dem die Welt erschrickt,}
\begin{verse}[\versewidth]



\flagverse{7.} Er ruft mich an, so will ich ihn\\
ganz gnädiglich erhören;\\
wenn sein Feind auf ihn aus will ziehn,\\
so will ich stehn und wehren.\\
Ich will ihn reißen aus dem Tod\\
und nach erlittner Angst und Not\\
mit großer Ehr ergötzen;\\
ich will ihn machen lebenssatt\\
und, wenn er gnug gelebet hat,\\
ins ewge Heil versetzen.

  
\end{verse}
\end{center}

%\attrib{\small{THZE}}

\index{Wer unterm Schirm des Höchsten sitzt}
\newpage
\subsection*{\centerline{Wie ist so groß und schwer die Last}}
\addcontentsline{toc}{subsection}{Wie ist so groß und schwer die Last}
%StartInfo%%%%%%%%%%%%%%%%%%%%%%%%%%%%%%%%%%%%%%%%%%%%%%%%%%%%%%%%%%%%%%%%%%%%
%  Autor:
%  Titel:
%  File:
%  Ref:
%  Mod:
%EndInfo%%%%%%%%%%%%%%%%%%%%%%%%%%%%%%%%%%%%%%%%%%%%%%%%%%%%%%%%%%%%%%%%%%%%%%
%\poemtitle{pt}
\begin{multicols}{2}
\settowidth{\versewidth}{Wir unsers Teils sind dir verpflicht't}
\begin{verse}[\versewidth]
%dankgebet in Kriegszeiten\\
%wie ist so groß und schwer die Last

\flagverse{1.} Wie ist so groß und schwer die Last,\\
die du uns aufgeleget hast,\\
o aller Götter Gott!\\
Gott, der du streng und eifrig bist\\
dem, der nicht fromm und heilig ist.

\flagverse{2.} Die Last, die ist die Kriegesflut,\\
so jetzt die Welt mit rotem Blut\\
und heißen Tränen füllt;\\
es ist das Feur, das hitzt und brennt,\\
so weit fast Sonn und Mond sich wendt.

\flagverse{3.} Groß ist die Last, doch ist dabei\\
dein starker Schutz und Vatertreu\\
uns gar nicht unbekannt;\\
du strafst, und mitten in dem Leid\\
erzeigst du Lieb und Freundlichkeit.

\flagverse{4.} Wir unsers Teils sind dir verpflicht't\\
dafür, daß du dein Heil und Licht\\
uns niemals ganz versagt;\\
viel andre hast du abgelohnt,\\
uns hast du ja noch oft verschont.

\flagverse{5.} Wie manchmal hat sich hier und dar\\
ein großes Wetter der Gefahr\\
um uns gezogen auf;\\
dein Hand, die Erd und Himmel trägt,\\
hat Sturm und Wetter beigelegt.

\flagverse{6.} Wie oftmals hat bei Tag und Nacht\\
der Feinde List und große Macht\\
uns, deine Herd, umringt;\\
du aber, o du treuer Hirt\\
hast unsern Wolf zurückgeführt.

\flagverse{7.} Viel unsrer Brüder sind geplagt,\\
von Haus und Hof dazu verjagt;\\
wir aber haben noch\\
beim Weinstock und beim Feigenbaum\\
ein jeder seinen Sitz und Raum.

\flagverse{8.} Sieh an, mein Herz, wie Stadt und Land\\
an vielen Orten ist gewandt\\
zum tiefen Untergang;\\
der Menschen Hütten sind verstört,\\
die Gotteshäuser umgekehrt.

\flagverse{9.} Bei uns ist ja noch Polizei,\\
auch leisten wir noch ohne Scheu\\
dem Herren seinen Dienst;\\
man lehrt und hört ja fort und fort\\
alltäglich bei uns Gottes Wort.

\flagverse{10.} Wer dieses nun will nicht verstehn,\\
läßts in die Luft und Winde gehn\\
und bei so hellem Licht\\
nicht Gottes Gnad und Güt erkennt,\\
der ist fürwahr durchaus verblendt.

\flagverse{11.} O frommer Gott, nimm von uns hin\\
solch Unvernunft, richt unsern Sinn,\\
daß wir zur Dankbarkeit\\
mit Lobgesang und süßem Ton\\
uns finden stets vor deinem Thron.

\flagverse{12.} Nicht unserm Werk, nicht unserm Tun,\\
allein dir, dir, o Gnadenbrunn,\\
gebührt all Ehr und Ruhm.\\
Wir haben Zorn und Tod verschuldt,\\
du zahlest uns mit Lieb und Huld.

\flagverse{13.} Laß diese Lieb, als eine Glut,\\
in uns entzünden Herz und Mut,\\
gib engelische Brunst,\\
daß alle unsre Äderlein\\
zu singen dir bereitet sein.

\flagverse{14.} Laß auch einmal nach so viel Leid\\
uns wieder scheinen unsre Freud,\\
des Friedens Angesicht,\\
das mancher Mensch noch nie einmal\\
geschaut in diesem Jammertal.

\flagverse{15.} Sind wirs nicht wert, so sieh doch an\\
die, so kein Unrecht je getan,\\
die kleinen Kinderlein;\\
solln sie denn in der Wiegen noch\\
mittragen solches schweres Joch?

\flagverse{16.} Erbarm dich, o barmherzigs Herz,\\
so vieler Seufzer, die der Schmerz\\
uns aus dem Herzen zwingt.\\
Du bist ja Gott und nicht ein Stein,\\
wie kannst du denn so harte sein?

\flagverse{17.} Wir sind an bösen Wunden krank,\\
voll Eiter, Striemen, Kot und Stank,\\
du Herr bist unser Arzt!\\
Geuß ein, geuß ein dein Gnadenöl,\\
so wird geheilet Leib und Seel.

\flagverse{18.} Nun, du wirsts tun, das glauben wir,\\
obgleich noch wenig scheinen für\\
die Mittel in der Welt.\\
Wenn alle Mittel stille stehn,\\
dann pflegt dein Helfen anzugehn.

\end{verse}
\end{multicols}
%\attrib{\small{THZE}}

\index{Wie ist so groß und schwer die Last}
\newpage
\subsection*{\centerline{Wie lang, o Herr, wie lange}}
\addcontentsline{toc}{subsection}{Wie lang o Herr wie lange}
%StartInfo%%%%%%%%%%%%%%%%%%%%%%%%%%%%%%%%%%%%%%%%%%%%%%%%%%%%%%%%%%%%%%%%%%%%
%  Autor:
%  Titel:
%  File:
%  Ref:
%  Mod:
%EndInfo%%%%%%%%%%%%%%%%%%%%%%%%%%%%%%%%%%%%%%%%%%%%%%%%%%%%%%%%%%%%%%%%%%%%%%
%\poemtitle{pt}
\begin{multicols}{2}
\settowidth{\versewidth}{Wie lang, o Herr, wie lange soll}
\begin{verse}[\versewidth]
%der 13. Psalm\\
%wie lang, o Herr, wie lange soll dein Herze mein vergessen?

\flagverse{1.} Wie lang, o Herr, wie lange soll\\
dein Herze mein vergessen?\\
Wie lange soll ich Jammers voll\\
mein Brot mit Tränen essen?\\
Wie lange willst du nicht\\
mir dein Angesicht\\
zu schauen reichen dar?\\
Willst du denn ganz und gar\\
dich nun von mir verbergen?

\flagverse{2.} Wie lange soll die Trauerhöhl\\
in Sorgen ich besitzen?\\
Wie lange soll mein arme Seel\\
in diesem Bade schwitzen?\\
Soll ich denn alle Tag\\
immer lauter Plag,\\
die Welt im Gegenteil\\
nur immer lauter Heil\\
nach ihrem Wunsche haben?

\flagverse{3.} Ach, schaue doch von deinem Saal\\
und siehe, wie ich leide!\\
Mein Herzensweh und große Qual\\
ist meiner Feinde Freude.\\
Herr, mein getreuer Hort,\\
hör an meine Wort,\\
die ich, durch Trübsal hier\\
gepresset, schütt herfür;\\
laß dein Gemüt erweichen!

\flagverse{4.} Erleuchte meiner Augen Licht,\\
mit deinem Gnadenwinke,\\
damit ich in dem Tode nicht\\
enschlafe noch versinke!\\
Gib, daß die böse Rott\\
nicht treib ihren Spott\\
aus mir und meinem Fall,\\
als hätt ich überall\\
verspielet und verloren.

\flagverse{5.} Ich steh und hoffe steif und fest\\
darauf, daß du die Deinen\\
nicht endlich untergehen läßt.\\
Kannsts auch nicht böse meinen;\\
obs gleich bisweilen scheint,\\
als wärst du uns feind\\
und gänzlich abgewendt,\\
so find sich doch behend\\
dein Vaterherze wieder.

\flagverse{6.} Mein Herze lacht vor großer Freud,\\
wann ich bei mir bedenke,\\
wie herzlich gern in böser Zeit\\
dein Herz sich zu uns lenke.\\
Der Herr ist frommes Muts,\\
tut uns nichts als Guts.\\
Das ist mein Lobgesang,\\
den ihm zum Ehrendank\\
ich hier und dort will singen.

\end{verse}
\end{multicols}
%\attrib{\small{THZE}}

\index{Wie lang o Herr, wie lange}
\newpage

\section*{\centerline{\LARGE BITTEN UND BETEN}}
\addcontentsline{toc}{section}{BITTEN UND BETEN}
\rule{\textwidth}{0.2pt}\vspace*{-\baselineskip}\vspace{3.2pt}
\rule{\textwidth}{1.2pt}\\[\baselineskip]

\index{ Gedichte vom Bitten und Beten}

\centerline{\scshape Herr, aller Weisheit Quell und Grund }
\vspace*{2\baselineskip}
\centerline{\scshape Hörst du hier die Ewigkeit?}                  %witt:? --> Bitten und Beten 
\vspace*{2\baselineskip}
\centerline{\scshape Ich danke dir demütiglich }
\vspace*{2\baselineskip}
\centerline{\scshape Ich danke dir mit Freuden }
\vspace*{2\baselineskip}
\centerline{\scshape Jesu, allerliebster Bruder }
\vspace*{2\baselineskip}
\centerline{\scshape Nach dir, o Herr, verlanget mich }
\vspace*{2\baselineskip}
\centerline{\scshape O Gott, mein Schöpfer, edler Fürst} 
\vspace*{2\baselineskip}
\centerline{\scshape O Herrscher in dem Himmelszelt }
\vspace*{2\baselineskip}
\centerline{\scshape O Jesu Christ, mein schönstes Licht }
\vspace*{2\baselineskip}
\centerline{\scshape Wie der Hirsch im großen Dürsten }

\newpage

\subsection*{\centerline{Herr, aller Weisheit Quell und Grund}}
\addcontentsline{toc}{subsection}{Herr aller Weisheit Quell und Grund}
%StartInfo%%%%%%%%%%%%%%%%%%%%%%%%%%%%%%%%%%%%%%%%%%%%%%%%%%%%%%%%%%%%%%%%%%%%
%  Autor:
%  Titel:
%  File:
%  Ref:
%  Mod:
%EndInfo%%%%%%%%%%%%%%%%%%%%%%%%%%%%%%%%%%%%%%%%%%%%%%%%%%%%%%%%%%%%%%%%%%%%%%
%\poemtitle{Herr, aller Weisheit Quell und Grund}
\begin{multicols}{2}
\settowidth{\versewidth}{Ja, Herr, ich bin gar viel zu schlecht,}
\begin{verse}[\versewidth]
%Herr, aller Weisheit Quell und Grund\\
%(in Anlehnung an Sprüche Sal. 7-9)\\
%nach Johann Arnds »Paradiesgärtlein«, Goslar 1621, I, 14

\flagverse{1.} Herr, aller Weisheit Quell und Grund,\\
dir ist all mein Vermögen kund,\\
wo du nicht hilfst und deine Gunst,\\
ist all mein Tun und Werk umsunst.

\flagverse{2.} Ich leider als ein Sündenkind\\
bin von Natur zum Guten blind,\\
mein Herze, wann dirs dienen soll,\\
ist ungeschickt und Torheit voll.

\flagverse{3.} Ja, Herr, ich bin gar viel zu schlecht,\\
zu handeln dein Gesetz und Recht,\\
was meinem Nächsten nütz im Land,\\
ist mir verdeckt und unbekannt.

\flagverse{4.} Mein Leben ist sehr kurz und schwach,\\
ein Lüftlein, das bald lässet nach;\\
was in der Welt zu prangen pflegt,\\
das ist mir wenig beigelegt.

\flagverse{5.} Wann ich auch gleich vollkommen wär,\\
hätt aller Gaben Ruhm und Ehr\\
und sollt entraten deines Lichts,\\
so wär ich doch ein lauter Nichts.

\flagverse{6.} Was hilfts, wann einer gleich viel weiß,\\
und hat zuvorderst nicht mit Fleiß\\
gelernet deine Furcht und Dienst,\\
der hat mehr Schaden als Gewinst.

\flagverse{7.} Das Wissen, das ein Mensche führt,\\
wird leichthin in ihm selbst verirrt;\\
wann unsre Kunst am meisten kann,\\
so stößt sie aller Enden an.

\flagverse{8.} Wie mancher stürzet seine Seel\\
durch Klugheit, wie Ahitophel,\\
und nimmt, weil er dich nicht recht kennt,\\
durch seinen Witz ein schrecklich End!

\flagverse{9.} O Gott, mein Vater, kehre dich\\
zu meiner Bitt und höre mich:\\
Nimm solche Torheit von mir hin\\
und gib mir einen bessern Sinn!

\flagverse{10.} Gib mir die Weisheit, die du liebst\\
und denen, die dich lieben, gibst,\\
die Weisheit, die vor deinem Thron\\
allstets erscheint in ihrer Kron.

\flagverse{11.} Ich lieb ihr liebes Angesicht,\\
sie ist meins Herzens Freud und Licht,\\
sie ist die Schönste, die mich hält\\
und meinen Augen wohlgefällt.

\flagverse{12.} Sie ist hochedel, auserkorn,\\
von dir, o Höchster, selbst geborn,\\
sie ist der hellen Sonnen gleich,\\
an Tugend und an Gaben reich.

\flagverse{13.} Ihr Mund ist süß und tröstet schön,\\
wenn uns die Augen übergehn;\\
wenn uns der Kummer niederdrückt,\\
so ist sies, die das Herz erquickt.

\flagverse{14.} Sie ist voll Ehr und Herrlichkeit,\\
bewehrt vorm Tod und großem Leid;\\
wer fleißig um sie kämpft und wirbt,\\
der bleibet lebend, wenn er stirbt.

\flagverse{15.} Sie ist des Schöpfers nächster Rat,\\
von Worten mächtig und von Tat;\\
durch sie erfährt die blinde Welt,\\
was Gott gedenkt in seinem Zelt.

\flagverse{16.} Denn welcher Mensch weiß Gottes Rat?\\
Wer ists, der je erfunden hat\\
den Schluß, den er im Himmel schleußt,\\
den Weg, den er uns laufen heißt?

\flagverse{17.} Die Seele wohnet in der Erd\\
und wird durch ihre Last beschwert;\\
die Sinnen, hin und her zerstreut,\\
sind ja von Irrtum nicht befreit.

\flagverse{18.} Wer will erforschen, was Gott setzt,\\
und sagen, was sein Herz ergötzt?\\
Es sei denn, der du ewig lebst,\\
daß du uns deine Weisheit gebst.

\flagverse{19.} Drum sende sie von deinem Thron\\
und gib sie deinem Kind und Sohn!\\
Ach, schütt und geuß sie reichlich aus\\
in meines Herzens armes Haus!

\flagverse{20.} Befiehl ihr, daß sie mit mir sei\\
und, wo ich gehe, stehe bei;\\
bin ich in Arbeit, helfe sie\\
mir tragen meine schwere Müh!

\flagverse{21.} Gib mir durch ihre weise Hand\\
die recht Erkenntnis und Verstand,\\
daß ich an dir alleine kleb\\
und nur nach deinem Willen leb!

\flagverse{22.} Gib mir durch sie Geschicklichkeit,\\
zur Wahrheit laß mich sein bereit,\\
daß ich nicht mach aus sauer süß,\\
noch aus dem Lichte Finsternis!

\flagverse{23.} Gib Lieb und Lust zu deinem Wort,\\
hilf, daß ich bleib an meinem Ort\\
und mich zur frommen Schar gesell,\\
in ihrem Rat mein Wesen stell!

\flagverse{24.} Gib auch, daß ich gern jedermann\\
mit Rat und Tat, so gut ich kann,\\
aus rechter unverfälschter Treu\\
zu helfen allzeit willig sei.

\end{verse}
\end{multicols}
%\attrib{\small{THZE}}

\begin{center}
\settowidth{\versewidth}{Auf daß in allem, was ich tu,}
\begin{verse}[\versewidth]

\flagverse{25.} Auf daß in allem, was ich tu,\\
in deiner Lieb ich nehme zu;\\
denn wer sich nicht der Weisheit gibt,\\
der bleibt von dir auch ungeliebt.

  
\end{verse}
\end{center}


%\attrib{\small{THZE}}

\index{Herr, aller Weisheit Quell und Grund}
\newpage
\subsection*{\centerline{Hörst du hier die Ewigkeit?}}
\addcontentsline{toc}{subsection}{Hörst du hier die Ewigkeit?}
%StartInfo%%%%%%%%%%%%%%%%%%%%%%%%%%%%%%%%%%%%%%%%%%%%%%%%%%%%%%%%%%%%%%%%%%%%
%  Autor:
%  Titel:
%  File:
%  Ref:
%  Mod:
%EndInfo%%%%%%%%%%%%%%%%%%%%%%%%%%%%%%%%%%%%%%%%%%%%%%%%%%%%%%%%%%%%%%%%%%%%%%
%\poemtitle{pt}
\begin{multicols}{2}
\settowidth{\versewidth}{Zwar heißts ja Tod und Sterbensnot,}
\begin{verse}[\versewidth]
%nun sei getrost und unbetrübt\\
%auf den Tod der Regina Leyser, geb. Calow, in Wittenberg (1664)

\flagverse{1.} Nun sei getrost und unbetrübt,\\
du mein Geist und Gemüte!\\
Dein Jesus lebt, der dich geliebt\\
eh, als dir dein Geblüte\\
und Fleisch und Haut ward zugericht;\\
der wird dich auch gewißlich nicht\\
an deinem Ende hassen.

\flagverse{2.} Erschrecke nicht vor deinem End,\\
es ist nichts Böses drinnen;\\
dein lieber Herr streckt seine Händ\\
und fordert dich von hinnen\\
aus soviel tausend Angst und Qual,\\
die du in diesem Jammertal\\
bisher hast ausgestanden.

\flagverse{3.} Zwar heißts ja Tod und Sterbensnot,\\
doch ist da gar kein Sterben;\\
denn Jesus ist des Todes Tod\\
und nimmt ihm das Verderben,\\
daß alle seine Stärk und Kraft\\
mir, wenn ich jetzt werd hingerafft,\\
nicht auf ein Härlein schade.

\flagverse{4.} Des Todes Kraft steht in der Sünd\\
und schnöden Missetaten,\\
darin ich armes Adamskind\\
so oft und viel geraten;\\
nun ist die Sünd in Jesu Blut\\
ersäuft, erstickt, getilgt und tut\\
fort gar nichts mehr zur Sachen.

\flagverse{5.} Die Sünd ist hin und ich bin rein;\\
trotz dem, der mir das nehme!\\
Hinfüro ist das Leben mein,\\
darf nicht, daß ich mich gräme\\
um einger Sünden Lohn und Sold;\\
wer ausgesöhnt, dem ist man hold\\
und tut ihm nichts zuwider.

\flagverse{6.} Ei nun, so nehm ich Gottes Gnad\\
und alle seine Freude\\
mit mir auf meinen letzten Pfad\\
und weiß von keinem Leide.\\
Der wilde Feind muß nur ein Schaf,\\
sein Ungestüm ein süßer Schlaf\\
und sanfte Ruhe werden.

\flagverse{7.} Du Jesu, allerliebster Freund,\\
bist selbst mein Licht und Leben:\\
Du hältst mich fest, und kann kein Feind\\
dich, wo du stehest, heben.\\
In dir steh ich, und du in mir;\\
und wie wir stehn, so bleiben wir\\
hier und dort ungeschieden.

\flagverse{8.} Mein Leib, der legt sich hin zur Ruh,\\
als der fast müde worden;\\
die Seele fährt dem Himmel zu\\
und mischt sich in den Orden\\
der auserwählten Gottesschar\\
und hält das ewge Jubeljahr\\
mit allen heilgen Engeln.

\flagverse{9.} Kommt dann der Tag, o höchster Fürst\\
der Kleinen und der Großen,\\
da du zum allerletzten wirst\\
in die Posaunen stoßen,\\
so soll denn Seel und Leib zugleich\\
mit dir in deines Vaters Reich\\
zu deiner Freud eingehen.

\flagverse{10.} Ists nun dein Will, so stell dich ein,\\
mich selig zu versetzen.\\
Ach, ewig bei und mit dir sein,\\
wie hoch muß das ergötzen!\\
Eröffne dich, du Todespfort,\\
auf daß an solchen schönen Ort\\
ich durch dich möge fahren!

\end{verse}
\end{multicols}
%\attrib{\small{THZE}}

\index{Hörst du hier die Ewigkeit}
\newpage
\subsection*{\centerline{Ich danke dir demütiglich}}
\addcontentsline{toc}{subsection}{Ich danke dir demütiglich}
%StartInfo%%%%%%%%%%%%%%%%%%%%%%%%%%%%%%%%%%%%%%%%%%%%%%%%%%%%%%%%%%%%%%%%%%%%
%  Autor:
%  Titel:
%  File:
%  Ref:
%  Mod:
%EndInfo%%%%%%%%%%%%%%%%%%%%%%%%%%%%%%%%%%%%%%%%%%%%%%%%%%%%%%%%%%%%%%%%%%%%%%
%\poemtitle{pt}
\begin{multicols}{2}
\settowidth{\versewidth}{Ich bitte, was ich bitten kann,}
\begin{verse}[\versewidth]
%ich danke dir demütiglich\\
  %nach Johann Arnds »Paradiesgärtlein«, Goslar 1621, III, 17:
  %»Gebet um zeitliche und ewige Wohlfahrt«

\flagverse{1.} Ich danke dir demütiglich,\\
o Gott, mein Vater, daß du dich\\
von deinem Zorn gewendet\\
und deinen Sohn\\
zur Freud und Kron\\
uns in die Welt gesendet.

\flagverse{2.} Er ist gekommen, hat sein Blut\\
vergossen und in solcher Flut\\
all unser Sünd ersticket.\\
Wer ihn nur faßt,\\
wird aller Last\\
benommen und erquicket.

\flagverse{3.} Ich bitte, was ich bitten kann,\\
herzlieber Vater, nimm mich an\\
in diesen edlen Orden,\\
der durch dies Blut\\
gerecht und gut\\
und ewig selig worden.

\flagverse{4.} Laß meines Glaubens Aug und Hand\\
ergreifen dieses werte Pfand\\
und nimmermehr verlieren;\\
laß dieses Licht\\
mein Angesicht\\
zum ewgen Lichte führen!

\flagverse{5.} Bereite meiner Seelen Haus,\\
wirf allen Kot und Unflat aus,\\
bau in mir deine Hütte,\\
daß deine Güt\\
in mein Gemüt\\
all ihre Lieb ausschütte!

\flagverse{6.} Wann ich dich hab, ist alles mein;\\
du kannst nicht ohne Gaben sein,\\
hast tausend Weg und Weisen,\\
dein arme Herd\\
auf dieser Erd\\
zu nähren und zu speisen.

\flagverse{7.} Gib mir, daß ich an meinem Ort\\
allstets dich fürcht in deinem Wort\\
und meinen Stand so führe,\\
daß Glaub und Treu\\
stets bei mir sei\\
und all mein Leben ziere!

\flagverse{8.} Gib nur ein gnügsam Herz und Sinn!\\
Denn das ist ja ein großer Gwinn,\\
in steter Andacht liegen\\
und, wenn Gott gibt\\
was ihm beliebt,\\
ihm lassen gern genügen.

\flagverse{9.} Das Wen'ge, das durch Gottes Hand\\
ein Frommer und Gerechter hat,\\
ist vielmal mehr geehret\\
als alles Geld,\\
davon die Welt\\
mit frechem Herzen zehret.

\flagverse{10.} Die Frommen sind dir, Herr, bewußt;\\
du bist ihr und sie deine Lust\\
und werden nicht zuschanden,\\
kommt teure Zeit,\\
findt sich bereit\\
ihr Brot in allen Landen.

\flagverse{11.} Gott hat den, der ihn fürchtet, lieb,\\
sieht zu, daß ihn kein Unfall trüb,\\
hat Lust zu seinen Wegen;\\
und wenn er fällt,\\
steht Gott und hält\\
ihn fest in seinem Segen.

\flagverse{12.} Des Höchsten Auge sieht auf die,\\
so auf ihn hoffen spat und früh,\\
daß er sie schütz und rette\\
aus aller Not,\\
wann sie der Tod\\
auch selbst verschlungen hätte.

\flagverse{13.} Herr, du kannst nichts als Güte sein,\\
du wollest deiner Güte Schein\\
uns und all denen gönnen,\\
die sich mit Mund\\
und Herzensgrund\\
allein zu dir bekennen!

\flagverse{14.} Insonderheit nimm wohl in Acht\\
den Fürsten, den du uns gemacht\\
zu unsers Landes Krone,\\
laß immerzu\\
sein Fried und Ruh\\
auf seinem Stuhl und Throne.

\flagverse{15.} Halt unser liebes Vaterland\\
in deinem Schoß und starker Hand!\\
Behüt uns allzusammen\\
vor falscher Lehr\\
und Feindes Heer,\\
vor Pest und Feuersflammen.

\flagverse{16.} Nimm all der Meinen eben wahr,\\
treib, Herr, die böse Höllenschar\\
von Jungen und von Alten,\\
daß deine Herd\\
hie zeitlich werd\\
und ewig dort erhalten.

\end{verse}
\end{multicols}
%\attrib{\small{THZE}}

\index{Ich danke dir demütiglich}
\newpage
\subsection*{\centerline{Ich danke dir mit Freuden}}
\addcontentsline{toc}{subsection}{Ich danke dir mit Freuden}
%StartInfo%%%%%%%%%%%%%%%%%%%%%%%%%%%%%%%%%%%%%%%%%%%%%%%%%%%%%%%%%%%%%%%%%%%%
%  Autor:
%  Titel:
%  File:
%  Ref:
%  Mod:
%EndInfo%%%%%%%%%%%%%%%%%%%%%%%%%%%%%%%%%%%%%%%%%%%%%%%%%%%%%%%%%%%%%%%%%%%%%%
%\poemtitle{pt}
\begin{multicols}{2}
\settowidth{\versewidth}{Ich danke dir mit Freuden,}
\begin{verse}[\versewidth]
%ich danke dir mit Freuden\\
%(Sir. 51)

\flagverse{1.} Ich danke dir mit Freuden,\\
mein König und mein Heil,\\
daß du manch schweres Leiden,\\
so mir zu meinem Teil\\
oft häufig zugedrungen,\\
durch deine Wunderhand\\
gewaltig hast bezwungen\\
und von mir abgewandt.

\flagverse{2.} Du hast in harten Zeiten\\
mir diese Gnad erteilt,\\
daß meiner Feinde Streiten\\
mein Leben nicht ereilt,\\
wenn sie an hohen Orten\\
mich, der ichs nicht gedacht,\\
mit bösen falschen Worten\\
sehr übel angebracht.

\flagverse{3.} Wenn sie wie wilde Leuen\\
die Zungen ausgestreckt\\
und mich mit ihrem Schreien\\
bis auf den Tod erschreckt,\\
so hat denn dein Erbarmen,\\
das alles lindern kann,\\
gewaltet und mir Armen\\
den treusten Dienst getan.

\flagverse{4.} Sie haben oft zusammen\\
sich wider mich gelegt\\
und wie die Feuerflammen\\
Gefahr und Brand erregt:\\
Da hab ich denn gesessen\\
und Blut vor Angst geschwitzt,\\
als ob du mein vergessen,\\
und hast mich doch geschützt.

\flagverse{5.} Du hast mich aus dem Brande\\
und aus dem Feur gerückt,\\
und wenn der Höllen Bande\\
mich um und um bestrickt,\\
so hast du auf mein Bitten\\
dich, Herr, zu mir gesellt\\
und aus des Unglücks Mitten\\
mich frei ins Feld gestellt.

\flagverse{6.} Den Kläffer, der mit Lügen\\
gleich als mit Waffen kämpft\\
und nichts kann als betrügen,\\
den hast du oft gedämpft;\\
wenn er, gleich einem Drachen,\\
das Maul hoch aufgezerrt,\\
so hast du ihm den Rachen\\
durch deine Kraft gesperrt.

\flagverse{7.} Ich war nah am Verderben,\\
du nahmst mich in den Schoß;\\
es kam mit mir zum Sterben,\\
du aber sprachst mich los\\
und hieltest mich beim Leben\\
und gabst mir Rat und Tat,\\
die sonst kein Mensch zu geben\\
in seinen Mächten hat.

\flagverse{8.} Es war in allen Landen,\\
so weit die Wolken gehn,\\
kein einzger Freund vorhanden,\\
der bei mir wollte stehn;\\
da dacht ich an die Güte,\\
die du, Herr, täglich tust,\\
und hub Herz und Gemüte\\
zur Höhe, da du ruhst.

\end{verse}
\end{multicols}


\begin{center}
\settowidth{\versewidth}{Der, vor dem die Welt erschrickt,}
\begin{verse}[\versewidth]




\flagverse{9.} Ich rief mit vollem Munde,\\
du nahmest alles an\\
und halfst recht aus dem Grunde\\
so, daß ichs nimmer kann\\
nach Würden gnugsam loben:\\
Doch will ich Tag und Nacht\\
dich in dem Himmel droben\\
zu preisen sein bedacht.
  
\end{verse}
\end{center}


%\attrib{\small{THZE}}

\index{Ich danke dir mit Freuden}
\newpage
\subsection*{\centerline{Jesu, allerliebster Bruder}}
\addcontentsline{toc}{subsection}{Jesu allerliebster Bruder}
%StartInfo%%%%%%%%%%%%%%%%%%%%%%%%%%%%%%%%%%%%%%%%%%%%%%%%%%%%%%%%%%%%%%%%%%%%
%  Autor:
%  Titel:
%  File:
%  Ref:
%  Mod:
%EndInfo%%%%%%%%%%%%%%%%%%%%%%%%%%%%%%%%%%%%%%%%%%%%%%%%%%%%%%%%%%%%%%%%%%%%%%
%\poemtitle{pt}
\begin{multicols}{2}
\settowidth{\versewidth}{Wenn die Zung und Mund nur liebet,}
\begin{verse}[\versewidth]
%nach Johann Arnds »Paradiesgärtlein«, Goslar 1621, I, 34

\flagverse{1.} Jesu, allerliebster Bruder,\\
ders am besten mit mir meint,\\
du mein Anker, Mast und Ruder\\
und mein treuester Herzensfreund;\\
der du, ehe was geboren,\\
dir das Menschenvolk erkoren,\\
auch mich armen Erdengast\\
dir zur Lieb ersehen hast.

\flagverse{2.} Du bist ohne Falsch und Tücke,\\
dein Herz weiß von keiner List,\\
aber wenn ich nur erblicke\\
was hier auf Erden ist,\\
find ich alles voller Lügen:\\
Wer am besten kann betrügen,\\
wer am schönsten heucheln kann,\\
ist der allerbeste Mann.

\flagverse{3.} Ach, wie untreu und verlogen\\
ist die Liebe dieser Welt;\\
ist sie jemand wohl gewogen,\\
währts nicht länger als sein Geld.\\
Wenn das Glück uns fügt und grünet,\\
sind wir schön und hübsch bedienet,\\
kommt ein wenig Ungestüm,\\
kehrt sich alle Freundschaft üm.

\flagverse{4.} Treib, Herr, von mir und verhüte\\
solchen unbeständgen Sinn;\\
hätt ich aber mein Gemüte,\\
weil ich auch ein Mensche bin,\\
schon mit diesem Kot besprenget\\
und der Falschheit nachgehänget,\\
so erkenn ich meine Schuld,\\
bitt um Gnad und um Geduld.

\flagverse{5.} Laß mir ja nicht widerfahren,\\
was du Herr zur Straf und Last\\
denen, die mit falschen Waren\\
handeln, angedräuet hast,\\
da du sprichst, du wollest scheuen\\
und als Unflat von dir speien\\
aller Heuchler falschen Mut,\\
der Guts fürgibt und nicht tut.

\flagverse{6.} Gib mir ein beständges Herze\\
gegen alle meine Freund;\\
auch dann, wann mit Kreuz und Schmerze\\
sie von dir beleget seind,\\
daß ich mich nicht ihrer schäme,\\
sondern mich nach dir bequeme,\\
der du, da wir arm und bloß,\\
uns gesetzt in deinen Schoß.

\flagverse{7.} Gib mir auch nach deinem Willen\\
einen Freund, in dessen Treu\\
ich mein Herze möge stillen,\\
da mein Mund sich ohne Scheu\\
öffnen und erklären möge,\\
da ich alles abelege\\
(nach dem Maße, das mir gnügt),\\
was mir auf dem Herzen liegt.

\flagverse{8.} Laß mich Davids Glück erleben:\\
Gib mir einen Jonathan,\\
der mir sein Herz möge geben,\\
der auch, wenn nun jedermann\\
mir nichts Gutes mehr will gönnen,\\
sich nicht lasse von mir trennen,\\
sondern fest in Wohl und Weh\\
als ein Felsen bei mir steh.

\flagverse{9.} Herr, ich bitte dich, erwähle\\
mir aus aller Menschen Meng\\
eine fromme heilge Seele,\\
die an dir fein kleb und häng,\\
auch nach deinem Sinn und Geiste\\
mir stets Trost und Hilfe leiste:\\
Trost, der in der Not besteht,\\
Hilfe, die von Herzen geht.

\flagverse{10.} Wenn die Zung und Mund nur liebet,\\
ist die Liebe schlecht bestellt.\\
Wer mir gute Worte gibet\\
und den Haß im Herzen hält,\\
wer nur seinen Kuchen schmieret\\
und, wanns Bienlein nicht mehr führet,\\
alsdann geht er nach der Tür -\\
ei, der bleibe fern von mir.

\flagverse{11.} Hab ich Schwachheit und Gebrechen,\\
Herr, so lenke meinen Freund,\\
mich in Güte zu besprechen\\
und nicht als ein Leu und Feind.\\
Wer mich freundlich weiß zu schlagen,\\
ist, als der in Freudentagen\\
reichlich auf mein Haupt mir geußt\\
Balsam, der am Jordan fleußt.

\flagverse{12.} O, wie groß ist meine Habe,\\
o, wie köstlich ist mein Gut,\\
Jesu, wenn mit dieser Gabe\\
dein Hand meinen Willen tut,\\
daß mich meines Freundes Treue\\
und beständigs Herz erfreue!\\
Wer dich fürchtet, liebt und ehrt,\\
dem ist solch ein Schatz beschert.

\flagverse{13.} Gute Freunde sind wie Stäbe,\\
da der Menschen Gang sich hält,\\
daß der schwache Fuß sich hebe,\\
wann der Leib zu Boden fällt.\\
Wehe dem, der nicht zum Frommen\\
solches Stabes weiß zu kommen!\\
Der hat einen schweren Lauf;\\
wann er fällt, wer hilft ihn auf?

\flagverse{14.} Nun, Herr, laß dirs wohl gefallen,\\
bleib mein Freund bis in mein Grab!\\
Bleib mein Freund und unter allen\\
mein getreuster, stärkster Stab!\\
Wenn du dich mir wirst verbinden,\\
wird sich schon ein Herze finden,\\
das, durch deinen Geist gerührt,\\
mir was Gutes gönnen wird.

\end{verse}
\end{multicols}
%\attrib{\small{THZE}}

\index{Jesu, allerliebster Bruder}
\newpage
\subsection*{\centerline{Nach dir, o Herr, verlanget mich}}
\addcontentsline{toc}{subsection}{Nach dir o Herr verlanget mich}
%StartInfo%%%%%%%%%%%%%%%%%%%%%%%%%%%%%%%%%%%%%%%%%%%%%%%%%%%%%%%%%%%%%%%%%%%%
%  Autor:
%  Titel:
%  File:
%  Ref:
%  Mod:
%EndInfo%%%%%%%%%%%%%%%%%%%%%%%%%%%%%%%%%%%%%%%%%%%%%%%%%%%%%%%%%%%%%%%%%%%%%%
%\poemtitle{pt}
\begin{multicols}{2}
\settowidth{\versewidth}{Der wird zu Schanden, der dich schändt}
\begin{verse}[\versewidth]
%der 25. Psalm\\
%nach dir, o Herr, verlanget mich

\flagverse{1.} Nach dir, o Herr, verlanget mich,\\
du bist mein Gott, ich hoff auf dich,\\
ich hoff und bin der Zuversicht,\\
du werdest mich beschämen nicht.

\flagverse{2.} Der wird zu Schanden, der dich schändt\\
und sein Gemüte von dir wendt,\\
der aber, der sich dir ergibt\\
und dich recht liebt, bleibt unbetrübt.

\flagverse{3.} Herr, nimm dich meiner Seelen an\\
und führe sie die rechte Bahn,\\
laß deine Wahrheit leuchten mir\\
im Steige, der uns bringt zu dir.

\flagverse{4.} Denn du bist ja mein einigs Licht,\\
sonst weiß ich keinen Helfer nicht,\\
ich harre dein bei Tag und Nacht:\\
Was ists, das dich so säumend macht?

\flagverse{5.} Ach wende, Herr, dein Augen ab\\
von dem, wo ich geirret hab.\\
Was denkst du an den Sündenlauf,\\
den ich geführt von Jugend auf?

\flagverse{6.} Gedenk an deine Gütigkeit\\
und an die große Süßigkeit,\\
damit dein Herz zu trösten pflegt\\
das, was sich dir zu Füßen legt.

\flagverse{7.} Der Herr ist fromm und herzlich gut\\
dem, der sich prüft und Buße tut,\\
wer seinen Bund und Zeugnis hält,\\
der wird erhalten, wenn er fällt.

\flagverse{8.} Ein Herz, das Gott von Herzen scheut,\\
das wird in seinem Leid erfreut,\\
und wenn die Not am tiefsten steht,\\
so wird sein Kreuz zur Wonn erhöht.

\flagverse{9.} Nun, Herr, ich bin dir wohlbekannt,\\
mein Geist, der schwebt in deiner Hand,\\
du siehst, wie meine Seele tränt\\
und sich nach deiner Hilfe sehnt.

\flagverse{10.} Die Angst, die mir mein Herze dringt\\
und daraus soviel Seufzer zwingt,\\
ist groß; du aber bist der Mann,\\
dem nichts zu groß entstehen kann.

\flagverse{11.} Drum steht mein Auge stets nach dir\\
und trägt dir mein Begehren für.\\
Ach laß doch, wie du pflegst zu tun,\\
dein Aug auf meinen Augen ruhn.

\flagverse{12.} Wenn ich dein darf, so wende nicht\\
von mir dein Aug und Angesicht,\\
laß deiner Antwort Gegenschein\\
mit meinem Beten stimmen ein.

\flagverse{13.} Die Welt ist falsch, du bist mein Freund,\\
ders treulich und von Herzen meint,\\
der Menschen Gunst steht nur im Mund,\\
du aber liebst von Herzensgrund.

\flagverse{14.} Zerreiß die Netz, heb auf die Strick\\
und brich des Feindes List und Tück,\\
und wenn mein Unglück ist vorbei,\\
so gib, daß ich auch dankbar sei.

\flagverse{15.} Laß mich in deiner Furcht bestehn,\\
fein schlecht und recht stets einhergehn;\\
gib mir die Einfalt, die dich ehrt\\
und lieber duldet als beschwert.

\flagverse{16.} Regier und führe mich zu dir,\\
auch andre Christen neben mir,\\
nimm, was dir mißfällt, von uns hin,\\
gib neue Herzen, neuen Sinn.

\end{verse}
\end{multicols}
%\attrib{\small{THZE}}

\begin{center}
\settowidth{\versewidth}{Der, vor dem die Welt erschrickt,}
\begin{verse}[\versewidth]
  
\flagverse{17.} Wasch ab all unsern Sündenkot,\\
erlös aus aller Angst und Not,\\
und führ uns bald mit Gnaden ein\\
zum ewgen Fried und Freudenschein.
   
\end{verse}
\end{center}



\index{Nach dir, o Herr, verlanget mich}
\newpage
\subsection*{\centerline{O Gott, mein Schöpfer, edler Fürst}}
\addcontentsline{toc}{subsection}{O Gott mein Schöpfer edler Fürst}
%StartInfo%%%%%%%%%%%%%%%%%%%%%%%%%%%%%%%%%%%%%%%%%%%%%%%%%%%%%%%%%%%%%%%%%%%%
%  Autor:
%  Titel:
%  File:
%  Ref:
%  Mod:
%EndInfo%%%%%%%%%%%%%%%%%%%%%%%%%%%%%%%%%%%%%%%%%%%%%%%%%%%%%%%%%%%%%%%%%%%%%%
%\poemtitle{pt}
\begin{multicols}{2}
\settowidth{\versewidth}{der nichts mehr schmeckt, nichts sieht und hört,}
\begin{verse}[\versewidth]
%sirachs Gebet um fleckenlosen Wandel (Sirach 23, 1-6)\\
%o Gott, mein Schöpfer, edler Fürst

\flagverse{1.} O Gott, mein Schöpfer, edler Fürst\\
und Vater meines Lebens,\\
wo du mein Leben nicht regierst,\\
so leb ich hier vergebens.\\
Ja lebendig bin ich auch tot,\\
der Sünden ganz ergeben,\\
wer sich wälzt\\
in dem Sündenkot,\\
der hat das rechte Leben\\
noch niemals recht gesehen.

\flagverse{2.} Darnum so wende deine Gnad\\
zu deinem armen Kinde\\
und gib mir allzeit guten Rat,\\
zu meiden Schand und Sünde;\\
behüte meines Mundes Tür,\\
daß mir ja nicht entfahre\\
ein solches Wort,\\
dadurch ich dir\\
und deiner frommen Schare\\
verdrießlich sei und schade.

\flagverse{3.} Bewahr, o Vater, mein Gehör\\
auf dieser schnöden Erde\\
vor allem, dadurch deine Ehr\\
und Reich beschimpfet werde;\\
laß mich der Lästrer Gall und Gift\\
ja nimmermehr berühren,\\
denn wen ein solcher\\
Unflat trifft,\\
den pflegt er zu verführen,\\
auch wohl gar umzukehren.

\flagverse{4.} Regiere meiner Augen Licht,\\
daß sie nichts Arges treiben,\\
ein unverschämtes Angesicht\\
laß ferne von mir bleiben;\\
was ehrbar ist, was Zucht erhält,\\
wonach die Englein trachten,\\
was dir beliebt\\
und wohlgefällt,\\
das laß auch mich hochachten,\\
all Üppigkeit verlachen.

\flagverse{5.} Gib, daß ich mich nicht lasse ein\\
zum Schlemmen und zum Prassen,\\
laß deine Lust mein eigen sein,\\
die andre fliehn und hassen.\\
Die Lust, die unser Fleisch ergötzt,\\
die zeucht uns nach der Höllen,\\
und was die Welt\\
für Freude schätzt,\\
pflegt Seel und Geist zu fällen\\
und ewiglich zu quälen.

\flagverse{6.} O selig ist, der stets sich nährt\\
mit Himmels Speis und Tränken,\\
der nichts mehr schmeckt,\\
nichts sieht und hört,\\
auch nichts begehrt zu denken,\\
als nur was zu dem Leben bringt,\\
da man bei Gotte lebet\\
und bei der Schar, die fröhlich singt\\
und in der Wollust schwebet,\\
die keine Zeit aufhebet.

\end{verse}
\end{multicols}
%\attrib{\small{THZE}}

\index{O Gott, mein Schöpfer, edler Fürst}
\newpage
\subsection*{\centerline{O Herrscher in dem Himmelszelt}}
\addcontentsline{toc}{subsection}{O Herrscher in dem Himmelszelt}
%StartInfo%%%%%%%%%%%%%%%%%%%%%%%%%%%%%%%%%%%%%%%%%%%%%%%%%%%%%%%%%%%%%%%%%%%%
%  Autor:
%  Titel:
%  File:
%  Ref:
%  Mod:
%EndInfo%%%%%%%%%%%%%%%%%%%%%%%%%%%%%%%%%%%%%%%%%%%%%%%%%%%%%%%%%%%%%%%%%%%%%%
%\poemtitle{pt}
\begin{multicols}{2}
\settowidth{\versewidth}{Nichts anders, traun, als daß die Schar}
\begin{verse}[\versewidth]
%o Herrscher in dem Himmelszelt\\
%»Buß- und Betgesang bei unzeitiger Nässe und betrübtem Gewitter«

\flagverse{1.} O Herrscher in dem Himmelszelt,\\
was ist es doch, das unser Feld\\
und was es uns hervorgebracht,\\
so ungestalt und traurig macht?

\flagverse{2.} Nichts anders, traun, als daß die Schar\\
der Menschen sich so ganz und gar\\
bis in den tiefsten Grund verkehrt\\
und täglich ihre Schuld vermehrt.

\flagverse{3.} Die, so, als Gottes Eigentum,\\
stets preisen sollten Gottes Ruhm\\
und lieben seines Wortes Kraft,\\
sind gleich der blinden Heidenschaft.

\flagverse{4.} Drum wird uns auch der Himmel blind,\\
des Firmamentes Glanz verschwind't,\\
wir warten, wann der Tag anbricht,\\
aufs Tageslicht und kommt doch nicht.

\flagverse{5.} Man zankt noch immer fort und fort,\\
es bleibet Krieg an allem Ort,\\
in allen Winkeln Haß und Neid,\\
in allen Ständen Streitigkeit.

\flagverse{6.} Drum strecken auch all Element\\
hier wider uns aus ihre Händ,\\
Angst kommt uns aus der Tief und See,\\
Angst kommt uns aus der Luft und Höh.

\flagverse{7.} Es ist ein hochbetrübte Zeit;\\
man plagt und jagt die armen Leut,\\
eh als es Zeit, zur Grube zu\\
und gönnet ihnen keine Ruh.

\flagverse{8.} Drum trauert auch der Freudenquell,\\
die Sonn, und scheint uns nicht so hell;\\
die Wolken gießen allzumal\\
die Tränen ohne Maß und Zahl.

\flagverse{9.} Ach, wein auch du, o Menschenkind,\\
und traure über deine Sünd;\\
halt doch von deinen Lastern ein\\
und mache dich durch Buße rein.

\flagverse{10.} Fall auf die Knie, fall in die Arm\\
des Herrn, daß sich sein Herz erbarm\\
und der so wohl verdienten Rach\\
in Gnaden bald ein Ende mach!

\flagverse{11.} Er ist ja fromm und bleibet fromm,\\
begehrt nichts mehr, als daß man komm\\
und mit geneigter Furcht und Scheu\\
ihn bitt um Gnad und Vatertreu.

\flagverse{12.} Ach Vater, Vater, höre doch\\
und lös uns aus dem Sündenjoch\\
und zeuch uns aus der Welt herfür\\
und kehr uns selbsten du zu dir!

\flagverse{13.} Erweiche unsern harten Mut\\
und mach uns Böse fromm und gut;\\
wen du bekehrst, der wird bekehrt,\\
und wer dich hört, der wird erhört.

\flagverse{14.} Laß deine Augen freundlich sein\\
und nimm mit gnädgen Ohren ein\\
das Angstgeschrei, das von der Erd\\
aus unserm Herzen zu dir fährt.

\flagverse{15.} Reiß weg das schwarze Zorngewand,\\
erquicke uns und unser Land\\
und der so schönen Früchte Kranz\\
mit süßem, warmem Sonnenglanz.

\flagverse{16.} Verleih uns bis in unsern Tod\\
alltäglich unser liebes Brot\\
und dermaleinst nach dieser Zeit\\
das süße Brot der Ewigkeit!

\end{verse}
\end{multicols}
%\attrib{\small{THZE}}

\index{O Herscher in dem Himmelszelt}
\newpage
\subsection*{\centerline{O Jesu Christ, mein schönstes Licht}}
\addcontentsline{toc}{subsection}{O Jesu Christ mein schönstes Licht}
%StartInfo%%%%%%%%%%%%%%%%%%%%%%%%%%%%%%%%%%%%%%%%%%%%%%%%%%%%%%%%%%%%%%%%%%%%
%  Autor:
%  Titel:
%  File:
%  Ref:
%  Mod:
%EndInfo%%%%%%%%%%%%%%%%%%%%%%%%%%%%%%%%%%%%%%%%%%%%%%%%%%%%%%%%%%%%%%%%%%%%%%
%\poemtitle{pt}
\begin{multicols}{2}
\settowidth{\versewidth}{Mein Trost, mein Schatz, mein Licht und Heil,}
\begin{verse}[\versewidth]
%o Jesu Christ, mein schönstes Licht\\
%nach Johann Arnds »Paradiesgärtlein«, Goslar 1621, II, 5: »Gebet um die Liebe Christi«

\flagverse{1.} O Jesu Christ, mein schönstes Licht,\\
der du in deiner Seelen\\
so hoch mich liebst, daß ich es nicht\\
aussprechen kann noch zählen:\\
Gib, daß mein Herz dich wiederum\\
mit Lieben und Verlangen\\
mög umfangen\\
und als dein Eigentum\\
nur einzig an dir hangen!

\flagverse{2.} Gib, daß sonst nichts in meiner Seel\\
als deine Liebe wohne,\\
gib, daß ich deine Lieb erwähl\\
als meinen Schatz und Krone;\\
stoß alles aus, nimm alles hin,\\
was mich und dich will trennen\\
und nicht gönnen,\\
daß all mein Mut und Sinn\\
in deiner Liebe brennen!

\flagverse{3.} Wie freundlich, selig, süß und schön\\
ist, Jesu, deine Liebe!\\
Wann diese steht, kann nichts entstehn,\\
das meinen Geist betrübe.\\
Drum laß nichts anders denken mich,\\
nichts sehen, fühlen, hören,\\
lieben, ehren\\
als deine Lieb und dich,\\
der du sie kannst vermehren.

\flagverse{4.} O, daß ich dieses hohe Gut\\
möcht ewiglich besitzen!\\
O, daß in mir dies' edle Glut\\
ohn Ende möchte hitzen!\\
Ach, hilf mir wachen Tag und Nacht\\
und diesen Schatz bewahren\\
vor den Scharen,\\
die wider uns mit Macht\\
aus Satans Reiche fahren!

\flagverse{5.} Mein Heiland, du bist mir zulieb\\
in Not und Tod gegangen\\
und hast am Kreuz als wie ein Dieb\\
und Mörder da gehangen,\\
verhöhnt, verspeit und sehr verwundt;\\
ach, laß mich deine Wunden\\
alle Stunden\\
mit Lieb im Herzensgrund\\
auch ritzen und verwunden.

\flagverse{6.} Dein Blut, daß dir vergossen ward,\\
ist köstlich, gut und reine,\\
mein Herz hingegen böser Art\\
und hart gleich einem Steine.\\
O laß doch deines Blutes Kraft\\
mein hartes Herze zwingen,\\
wohl durchdringen\\
und diesen Lebenssaft\\
mir deine Liebe bringen!

\vfill\null
\columnbreak

\flagverse{7.} O daß mein Herze offen stünd\\
und fleißig möcht auffangen\\
die Tröpflein Bluts, die meine Sünd\\
im Garten dir abdrangen!\\
Ach daß sich meiner Augen Brunn\\
auftät und mit Stöhnen\\
heiße Tränen\\
vergösse, wie die tun,\\
die sich in Liebe sehnen.

\flagverse{8.} O daß ich wie ein kleines Kind\\
mit Weinen dir nachginge\\
so lange, bis dein Herz entzündt\\
mit Armen mich umfinge\\
und deine Seel in mein Gemüt\\
in voller süßer Liebe\\
sich erhübe\\
und also deiner Güt\\
ich stets vereinigt bliebe!

\flagverse{9.} Ach zeuch, mein Liebster, mich nach dir,\\
so lauf ich mit den Füßen;\\
ich lauf und will dich mit Begier\\
in meinem Herzen küssen.\\
Ich will aus deines Mundes Zier\\
den süßen Trost empfinden,\\
der die Sünden\\
und alles Unglück hier\\
kann leichtlich überwinden.

\flagverse{10.} Mein Trost, mein Schatz, mein Licht und Heil,\\
mein höchstes Gut und Leben,\\
ach nimm mich auf zu deinem Teil,\\
dir hab ich mich ergeben.\\
Denn außer dir ist lauter Pein,\\
ich find hier überalle\\
nichts denn Galle;\\
nichts kann mir tröstlich sein,\\
nichts ist, das mir gefalle.

\flagverse{11.} Du aber bist die rechte Ruh,\\
in dir ist Fried und Freude,\\
gib, Jesu, gib, daß immerzu\\
mein Herz in dir sich weide!\\
Sei meine Flamm und brenn in mir,\\
mein Balsam, wollest eilen,\\
lindern, heilen\\
den Schmerzen, der allhier\\
mich seufzen macht und heulen.

\flagverse{12.} Was ists, o Schönster, das ich nicht\\
in deiner Liebe habe?\\
Sie ist mein Stern, mein Sonnenlicht,\\
mein Quell, da ich mich labe,\\
mein süßer Wein, mein Himmelsbrot,\\
mein Kleid vor Gottes Throne,\\
meine Krone,\\
mein Schutz in aller Not,\\
mein Haus, darin ich wohne.

\vfill\null
\columnbreak

\flagverse{13.} Ach, liebstes Lieb, wann du entweichst,\\
was hilft mir sein geboren?\\
Wann du mir deine Lieb entzeuchst,\\
ist all mein Gut verloren.\\
So gib, daß ich dich, meinen Gast,\\
wohl such und bester Maßen\\
möge fassen\\
und, wenn ich dich gefaßt,\\
in Ewigkeit nicht lassen!

\flagverse{14.} Du hast mich je und je geliebt\\
und auch nach dir gezogen;\\
eh ich noch etwas Guts geübt,\\
warst du mir schon gewogen.\\
Ach, laß doch ferner, edler Hort,\\
mich diese Liebe leiten\\
und begleiten,\\
daß sie mir immerfort\\
beisteh auf allen Seiten!

\flagverse{15.} Laß meinen Stand, darin ich steh,\\
herr, deine Liebe zieren\\
und, wo ich etwa irre geh,\\
alsbald zurechte führen;\\
laß sie mir allzeit guten Rat\\
und gute Werke lehren,\\
steuern, wehren\\
der Sünd, und nach der Tat\\
bald wieder mich bekehren!

\flagverse{16.} Laß sie sein meine Freud im Leid,\\
in Schwachheit mein Vermögen,\\
und wann ich nach vollbrachter Zeit\\
mich soll zur Ruhe legen,\\
alsdann laß deine Liebestreu,\\
Herr Jesu, bei mir stehen,\\
Luft zuwehen,\\
daß ich getrost und frei\\
mög in dein Reich eingehen!

\end{verse}
\end{multicols}
%\attrib{\small{THZE}}

\index{O Jesu Christ, mein schönstes Licht}
\newpage
\subsection*{\centerline{Wie der Hirsch im großen Dürsten}}
\addcontentsline{toc}{subsection}{Wie der Hirsch im großen Dürsten}
%StartInfo%%%%%%%%%%%%%%%%%%%%%%%%%%%%%%%%%%%%%%%%%%%%%%%%%%%%%%%%%%%%%%%%%%%%
%  Autor:
%  Titel:
%  File:
%  Ref:
%  Mod:
%EndInfo%%%%%%%%%%%%%%%%%%%%%%%%%%%%%%%%%%%%%%%%%%%%%%%%%%%%%%%%%%%%%%%%%%%%%%
%\poemtitle{pt}
\begin{multicols}{2}
\settowidth{\versewidth}{Wie der Hirsch im großen Dürsten}
\begin{verse}[\versewidth]
%der 42. Psalm\\
%wie der Hirsch im großen Dürsten schreiet und frisch Wasser sucht

\flagverse{1.} Wie der Hirsch im großen Dürsten\\
schreiet und frisch Wasser sucht,\\
also sucht dich Lebensfürsten\\
meine Seel in ihrer Flucht;\\
meine Seele brennt in mir\\
lechzet, dürstet, trägt Begier\\
nach dir, o du süßes Leben,\\
der mir Leib und Seel gegeben.

\flagverse{2.} Ach, wann werd ich dahin kommen,\\
daß ich Gottes Angesicht,\\
das gewünschte Licht der Frommen,\\
schau mit meiner Augen Licht!\\
Meine Tränen sind mein Brot\\
Tag und Nacht in meiner Not,\\
wann mich schmähen meine Spötter:\\
Wo ist nun dein Gott und Retter?

\flagverse{3.} Wenn ich dann des inne werde,\\
schütt ich mein Herz bei mir aus,\\
wollte gerne mit der Herde\\
deiner Kinder in dein Haus;\\
ja, in dein Haus wollt ich gern\\
gehen und dir, meinem Herrn,\\
in der Schar, die Opfer bringen,\\
mit erhobner Stimme singen.

\flagverse{4.} Was bist du so hoch betrübet\\
und voll Unruh, meine Seel?\\
Harr auf Gott, der herzlich liebet\\
und wohl siehet, was dich quäl.\\
Ei, ich werd ihm dennoch hier\\
fröhlich danken, daß er mir,\\
wann mein Herz ich zu ihm richte,\\
hilft mit seinem Angesichte.

\flagverse{5.} Mein Gott, ich bin voller Schande,\\
meine Seele voller Leid,\\
darum denk ich dein im Lande\\
bei dem Jordan an der Seit,\\
da Hermonim hoch herfür\\
und hingegen meine Zier,\\
zion, ein klein wenig steiget\\
und dir Kron und Zepter neiget.

\flagverse{6.} Deines Zornes Fluten sausen\\
mit Gewalt auf mich daher;\\
dein Gericht und Eifer brausen\\
wie das tiefe weite Meer;\\
deine Wellen heben sich\\
hoch empor und haben mich\\
mit ergrimmten Wasserwogen\\
fast zu Grund hinabgezogen.

\flagverse{7.} Gott der Herr hat mir versprochen,\\
wann es Tag ist, seine Güt,\\
und wann sich die Sonn verkrochen,\\
heb ich zu ihm mein Gemüt,\\
spreche: Du mein Fels und Stein,\\
gegen welchen alles klein,\\
dem ich in dem Schoß gesessen,\\
warum hast du mein vergessen?

\flagverse{8.} Warum muß ich gehn und weinen\\
über meiner Feinde Wort?\\
Es ist mir in meinen Beinen\\
durch und durch als wie ein Mord,\\
wann sie sagen: Wo ist nun\\
dein Gott und sein großes Tun?\\
Davon, wann du sicher lagest,\\
du so viel zu rühmen pflagest.
\end{verse}
\end{multicols}

\begin{center}
\settowidth{\versewidth}{Der, vor dem die Welt erschrickt,}
\begin{verse}[\versewidth]



\flagverse{9.} Was bist du so hoch betrübet\\
und voll Unruh, meine Seel?\\
Harr auf Gott, der herzlich liebet\\
und wohl siehet, was dich quäl!\\
Ei, ich werd ihm dennoch hier\\
fröhlich danken für und für,\\
daß er meinem Angesichte\\
sich selbst gibt zum Heil und Lichte.
  
\end{verse}
\end{center}


%\attrib{\small{THZE}}

\index{Wie der Hirsch im großen Dürsten}
\newpage

%\newpage

\section*{\centerline{\LARGE LOB UND DANK}}
\addcontentsline{toc}{section}{LOB UND DANK}
\rule{\textwidth}{0.2pt}\vspace*{-\baselineskip}\vspace{3.2pt}
\rule{\textwidth}{1.2pt}\\[\baselineskip]

\index{ Gedichte vom Loben und Danken}

\centerline{\scshape Also hat Gott die Welt geliebt}
\vspace*{0.8\baselineskip}
\centerline{\scshape Das ist mir lieb, daß Gott, mein Hort}
\vspace*{0.8\baselineskip}
\centerline{\scshape Du meine Seele singe}
\vspace*{0.8\baselineskip}
\centerline{\scshape Geh aus, mein Herz, und suche Freud}
\vspace*{0.8\baselineskip}
\centerline{\scshape Gott Lob! Nun ist erschollen}                %witt:Gottlob
\vspace*{0.8\baselineskip}
\centerline{\scshape Herr, dir trau ich all mein Tage}
\vspace*{0.8\baselineskip}
\centerline{\scshape Herr, du erforschest meinen Sinn}
\vspace*{0.8\baselineskip}
\centerline{\scshape Ich danke dir mit Freuden}
\vspace*{0.8\baselineskip}
\centerline{\scshape Ich, der ich oft in tiefes Leid}
\vspace*{0.8\baselineskip}
\centerline{\scshape Ich preise dich und singe}
\vspace*{0.8\baselineskip}
\centerline{\scshape Ich singe dir mit Herz und Mund}
\vspace*{0.8\baselineskip}
\centerline{\scshape Ich will erhöhen immerfort}
\vspace*{0.8\baselineskip}
\centerline{\scshape Ich will mit Danken kommen}
\vspace*{0.8\baselineskip}
\centerline{\scshape Merkt auf, merkt, Himmel, Erde}
\vspace*{0.8\baselineskip}
\centerline{\scshape Nun danket all und bringet Ehr}
\vspace*{0.8\baselineskip}
\centerline{\scshape Nun geht frisch drauf, es geht nach Haus}
\vspace*{0.8\baselineskip}
\centerline{\scshape Nun ist der Regen hin}
\vspace*{0.8\baselineskip}
\centerline{\scshape Sollt ich meinen Gott nicht singen?}
\vspace*{0.8\baselineskip}
\centerline{\scshape Unter allen, die da leben}                %witt:?  --> Lob und Dank
\vspace*{0.8\baselineskip}
\centerline{\scshape Wer wohlauf ist und gesund}
\vspace*{0.8\baselineskip}
\centerline{\scshape Wie ist es möglich, höchstes Licht?}

\newpage

\subsection*{\centerline{Also hat Gott die Welt geliebt}}
\addcontentsline{toc}{subsection}{Also hat Gott die Welt geliebt}
%StartInfo%%%%%%%%%%%%%%%%%%%%%%%%%%%%%%%%%%%%%%%%%%%%%%%%%%%%%%%%%%%%%%%%%%%%
%  Autor:
%  Titel:
%  File:
%  Ref:
%  Mod:
%EndInfo%%%%%%%%%%%%%%%%%%%%%%%%%%%%%%%%%%%%%%%%%%%%%%%%%%%%%%%%%%%%%%%%%%%%%%
%ANM:\poemtitle{Also hat Gott die Welt geliebt}
\begin{multicols}{2}
\settowidth{\versewidth}{sein Kreuz und Leiden ist mein Schmuck,}
\begin{verse}[\versewidth]
\flagverse{1.} Also hat Gott die Welt geliebt\\
das merke, wer es höret\\
die Welt, die Gott so hoch betrübt,\\
hat Gott so hoch geehret,\\
daß er den eingebornen Sohn,\\
den eingen Schatz, die einge Kron,\\
das einge Herz und Leben\\
mit Willen hingegeben.

\flagverse{2.} Ach wie muß doch ein einges Kind\\
bei uns hier auf der Erden,\\
da man doch nichts als Bosheit findt,\\
so hoch geschonet werden;\\
wie hitzt, wie brennt der Vatersinn,\\
wie gibt und schenkt er alles hin,\\
eh als er an das Schenken\\
des Eingen nur will denken!

\flagverse{3.} Gott aber schenkt, aus freiem Mut\\
und mildem treuem Herzen,\\
sein einges Kind, sein schönstes Gut\\
in mehr als tausend Schmerzen;\\
er gibt ihn in den Tod hinein,\\
ja in die Höll und ewge Pein,\\
zu unerhörtem Leide\\
stößt Gott sein einge Freude!

\flagverse{4.} Warum doch das? Daß du, o Welt,\\
frei wieder möchtest stehen\\
und durch ein teures Lösegeld\\
aus deinem Kerker gehen;\\
denn du weißt wohl, du schnöde Braut,\\
wie, da dich Gott ihm anvertraut,\\
du, wider deinen Orden,\\
ihm allzu untreu worden.

\flagverse{5.} Darüber hat dich Sünd und Tod\\
und Satanas Gesellen\\
zu bittrer Angst und harter Not\\
beschlossen in der Höllen.\\
Und ist hier gar kein andrer Rat\\
als der, den Gott gegeben hat;\\
wer den hat, wird dem Haufen\\
der höllschen Feind entlaufen.

\flagverse{6.} Gott hat uns seinen Sohn verehrt,\\
daß aller Menschen Wesen,\\
so mit dem ewgen Fluch beschwert,\\
durch diesen soll genesen.\\
Wen die Verdammnis hat umschränkt,\\
der soll durch den, den Gott geschenkt,\\
Erlösung, Trost und Gaben\\
des ewgen Lebens haben.

\flagverse{7.} Ach mein Gott, mein Lebens Grund,\\
wo soll ich Worte finden?\\
Mit was für Lobe soll mein Mund\\
dein treues Herz ergründen?\\
Wie ist dir immermehr geschehn?\\
Was hast du an der Welt ersehn,\\
daß, die so hoch dich höhnet,\\
du so gar hoch gekrönet?

\flagverse{8.} Warum behieltst du nicht dein Recht\\
und ließest ewig pressen\\
diejenge, die dein Recht geschwächt\\
und freventlich vergessen?\\
Was hattest du an der für Lust,\\
von welcher dir doch war bewußt,\\
daß sie für dein Verschonen\\
dir schändlich würde lohnen?

\flagverse{9.} Das Herz im Leibe weinet mir\\
vor großem Leid und Grämen,\\
wenn ich bedenke, wie wir dir\\
so gar schlecht uns bequemen.\\
Die Meisten wollen deiner nicht,\\
und was du ihnen zugericht\\
durch deines Sohnes Büßen,\\
das treten sie mit Füßen.

\flagverse{10.} Du, frommer Vater, meinst es gut\\
mit allen Menschenkindern,\\
du ordnest deines Sohnes Blut\\
und reichst es allen Sündern,\\
willst, daß sie mit der Glaubenshand\\
das, was du ihnen zugewandt,\\
sich völlig zu erquicken,\\
fest in ihr Herze drücken.

\flagverse{11.} Sieh aber, ist nicht immerfort\\
dir alle Welt zuwider?\\
Du bauest hier, du bauest dort,\\
die Welt schlägt alles nieder.\\
Darum erlangt sie auch kein Heil,\\
sie bleibt im Tod und hat kein Teil\\
am Reiche, da die Frommen,\\
die Gott gefolgt, hinkommen.

\flagverse{12.} An dir, o Gott, ist keine Schuld,\\
du, du hast nichts verschlafen:\\
Der Feind und Hasser deiner Huld\\
ist Ursach deiner Strafen,\\
weil er den Sohn, der ihm so klar\\
und nah ans Herz gestellet war,\\
auch einzig helfen sollte,\\
durchaus nicht haben wollte.

\flagverse{13.} So fahre hin, du tolle Schar!\\
Ich bleibe bei dem Sohne.\\
Dem geb ich mich, des bin ich gar,\\
und er ist meine Krone.\\
Hab ich den Sohn, so hab ich gnug,\\
sein Kreuz und Leiden ist mein Schmuck,\\
sein Angst ist meine Freude,\\
sein Sterben meine Weide.

\flagverse{14.} Ich freue mich, so oft und viel\\
ich dieses Sohns gedenke.\\
Dies ist mein Lied und Saitenspiel,\\
wann ich mich heimlich kränke,\\
wann meine Sünd und Missetat\\
will größer sein als Gottes Gnad,\\
und wann mir meinen Glauben\\
mein eigen Herz will rauben.

\flagverse{15.} Ei, sprech ich, war mein Gott geneigt,\\
da wir noch Feinde waren,\\
so wird er ja, der kein Recht beugt,\\
nicht feindlich mit mir fahren\\
anitzo, da ich ihm versühnt,\\
da, was ich Böses je verdient,\\
sein Sohn, der nichts verschuldet,\\
so wohl für mich erduldet.

\flagverse{16.} Fehlts hier und dar? Ei unverzagt!\\
Laß Sorg und Kummer schwinden!\\
Der mir das Größte nicht versagt,\\
wird Rat zum Kleinern finden.\\
Hat Gott mir seinen Sohn geschenkt\\
und für mich in den Tod gesenkt:\\
Wie sollt er, laßt uns denken,\\
nicht alles mit ihm schenken!


\end{verse}
\end{multicols}

\begin{center}
\settowidth{\versewidth}{Ich bins gewiß und sterbe drauf:}
\begin{verse}[\versewidth]

\flagverse{17.} Ich bins gewiß und sterbe drauf:\\
Nach meines Gottes Willen\\
mein Kreuz und ganzer Lebenslauf\\
wird sich noch fröhlich stillen.\\
Hier hab ich Gott und Gottes Sohn,\\
und dort bei Gottes Stuhl und Thron:\\
Da wird fürwahr mein Leben.
  
\end{verse}
\end{center}

%\attrib{\small{THZE}}

\index{Also hat Gott die Welt geliebt}
\newpage
\subsection*{\centerline{Das ist mir lieb, daß Gott, mein Hort}}
\addcontentsline{toc}{subsection}{Das ist mir lieb daß Gott mein Hort}
%StartInfo%%%%%%%%%%%%%%%%%%%%%%%%%%%%%%%%%%%%%%%%%%%%%%%%%%%%%%%%%%%%%%%%%%%%
%  Autor:
%  Titel:
%  File:
%  Ref:
%  Mod:
%EndInfo%%%%%%%%%%%%%%%%%%%%%%%%%%%%%%%%%%%%%%%%%%%%%%%%%%%%%%%%%%%%%%%%%%%%%%
%\poemtitle{Das ist mir lieb, daß Gott, mein Hort, so treulich bei mir stehet}
\begin{multicols}{2}
\settowidth{\versewidth}{Das ist mir lieb, daß Gott, mein Hort,}
\begin{verse}[\versewidth]
%der 116. Psalm\\
%Das ist mir lieb, daß Gott, mein Hort, so treulich bei mir stehet

\flagverse{1.} Das ist mir lieb, daß Gott, mein Hort\\
so treulich bei mir stehet;\\
wann ich ihn bitte, wird kein Wort\\
in meiner Bitt verschmähet.\\
Des schwarzen Todes Hand\\
samt der Höllen Band\\
umfingen überall\\
mein Herz mit Angst und Qual;\\
doch hat mir Gott geholfen.

\flagverse{2.} Ich kam in Jammer und in Not\\
und sank fast gar zugrunde,\\
und da ich sank, rief ich zu Gott\\
mit Herzen und mit Munde:\\
O Herr, ich weiß, du wirst\\
als des Lebens Fürst\\
schon führen meine Sach!\\
And wie ich bat und sprach,\\
so ist's auch nun geschehen.

\flagverse{3.} Sei wieder froh und gutes Muts,\\
mein Herze, sei zufrieden,\\
der Herr, der tut dir alles Guts,\\
durch ihn ist nun geschieden\\
und ferne weggebracht,\\
was mich traurig macht;\\
er hat mich aus dem Loch\\
und schwarzen Todesjoch\\
mit seiner Hand gerissen.

\flagverse{4.} Mein Aug ist nun von Tränen frei,\\
mein Fuß von seinem Gleiten;\\
das will ich sagen ohne Scheu\\
und rühmen bei den Leuten.\\
Was gar kein Mensch nicht kann,\\
das hat Gott getan.\\
Der Mensch ist Lügen voll,\\
Gott aber weiß gar wohl,\\
wie er sein Wort soll halten.

\flagverse{5.} Ich glaube fest in meinem Sinn,\\
und was mein Herze glaubet,\\
das redt mein Mund in Einfalt hin:\\
Wer Gott vertraut, der bleibet.\\
Die Welt und böse Rott\\
lacht des, mir zum Spott,\\
ja plagt mich noch dazu;\\
ich aber steh und ruh\\
auf dir, mein Gott und Helfer.

\flagverse{6.} Du stürzest meiner Feinde Rat\\
und segnest, wenn sie schelten,\\
wie soll ich doch die große Gnad\\
dir immer mehr vergelten?\\
Ich will, Herr, meines Teils\\
den Kelch deines Heils,\\
der voller Bitterkeit,\\
doch mir zu Nutz gedeiht,\\
gehorsamlich annehmen.

\flagverse{7.} Was du mir zugemessen hast,\\
das will ich gerne leiden;\\
wer fröhlich trägt des Kreuzes Last,\\
dem hilfst du aus mit Freuden.\\
Du weißt der Deinen Not\\
und hältst ihren Tod\\
sehr hoch, sehr lieb und wert,\\
auch läßt du auf der Erd\\
ihr Blut nicht ungerochen.

\flagverse{8.} So zürne nun gleich alle Welt\\
mit mir, Herr, deinem Knechte:\\
Du, du deckst mich in deinem Zelt\\
und reichst mir deine Rechte.\\
Darüber will ich dich\\
allstets inniglich,\\
so gut ich immer kann,\\
mit Dank vor jedermann\\
in deinem Hause preisen.

\end{verse}
\end{multicols}
%\attrib{\small{THZE}}

\index{Das ist mir lieb}
\newpage
\subsection*{\centerline{Du meine Seele singe}}
\addcontentsline{toc}{subsection}{Du meine Seele singe}
%StartInfo%%%%%%%%%%%%%%%%%%%%%%%%%%%%%%%%%%%%%%%%%%%%%%%%%%%%%%%%%%%%%%%%%%%%
%  Autor:
%  Titel:
%  File:
%  Ref:
%  Mod:
%EndInfo%%%%%%%%%%%%%%%%%%%%%%%%%%%%%%%%%%%%%%%%%%%%%%%%%%%%%%%%%%%%%%%%%%%%%%
%\poemtitle{Du meine Seele, singe}
\begin{multicols}{2}
\settowidth{\versewidth}{Ihr Menschen laßt euch lehren,}
\begin{verse}[\versewidth]
%der 146. Psalm


\flagverse{1.} Du meine Seele, singe,
wohlauf, und singe schön\\
dem, welchem alle Dinge\\
zu Dienst und Willen stehn.\\
Ich will den Herren droben\\
hier preisen auf der Erd,\\
ich will ihn herzlich loben,\\
so lang ich leben werd.

\flagverse{2.} Ihr Menschen laßt euch lehren,\\
es wird sehr nützlich sein:\\
Laßt euch doch nicht betören\\
die Welt mit ihrem Schein.\\
Verlasse sich ja keiner\\
auf Fürstenmacht und –gunst,\\
weil sie wie unser einer\\
nichts sind, als nur ein Dunst.

\flagverse{3.} Was Mensch ist, muß erblassen\\
und sinken in den Tod;\\
er muß den Geist auslassen,\\
selbst werden Erd und Kot.\\
Allda ist's dann geschehen\\
mit seinem klugen Rat\\
und ist frei klar zu sehen,\\
wie schwach sei Menschentat.

\flagverse{4.} Wohl dem, der einzig schauet\\
nach Jakobs Gott und Heil;\\
wer dem sich anvertrauet,\\
der hat das beste Teil,\\
das höchste Gut erlesen,\\
den schönsten Schatz geliebt,\\
sein Herz und ganzes Wesen\\
bleibt ewig unbetrübt.

\flagverse{5.} Hier sind die starken Kräfte,\\
die unerschöpfte Macht,\\
das weisen die Geschäfte,\\
die seine Hand gemacht:\\
Der Himmel und die Erde\\
mit ihrem ganzen Heer,\\
der Fisch unzählig Herde\\
im großen wilden Meer.

\flagverse{6.} Hier sind die treuen Sinnen,\\
die niemand Unrecht tun,\\
all denen Gutes gönnen,\\
die in der Treu beruhn.\\
Gott hält sein Wort mit Freuden,\\
und was er spricht, geschicht,\\
und wer Gewalt muß leiden,\\
den schützt er im Gericht.

\flagverse{7.} Er weiß viel tausend Weisen,\\
zu retten aus dem Tod,\\
ernährt und gibet Speisen\\
zur Zeit der Hungersnot,\\
macht schöne rote Wangen\\
oft bei geringem Mahl,\\
und die da sind gefangen,\\
die reißt er aus der Qual.

\flagverse{8.} Er ist das Licht der Blinden,\\
erleuchtet ihr Gesicht,\\
und die sich schwach befinden,\\
die stellt er aufgericht:\\
Er liebet alle Frommen,\\
und die ihm günstig seind,\\
die finden, wenn sie kommen,\\
an ihm den besten Freund.

\flagverse{9.} Er ist der Fremden Hütte,\\
die Waisen nimmt er an,\\
erfüllt der Witwen Bitte,\\
wird selbst ihr Trost und Mann;\\
die aber, die ihn hassen,\\
bezahlet er mit Grimm,\\
ihr Haus und wo sie saßen,\\
das wirft er üm und üm.

\flagverse{10.} Ach, ich bin viel zu wenig,\\
zu rühmen seinen Ruhm!\\
Der Herr allein ist König,\\
ich eine welke Blum.\\
Jedoch weil ich gehöre\\
gen Zion in sein Zelt,\\
ist's billig, daß ich mehre\\
sein Lob vor aller Welt.

\end{verse}
\end{multicols}
%\attrib{\small{THZE}}

\index{Du meine Seele singe}
\newpage
\subsection*{\centerline{Geh aus, mein Herz, und suche Freud}}
\addcontentsline{toc}{subsection}{Geh aus mein Herz und suche Freud}
%StartInfo%%%%%%%%%%%%%%%%%%%%%%%%%%%%%%%%%%%%%%%%%%%%%%%%%%%%%%%%%%%%%%%%%%%%
%  Autor:
%  Titel:
%  File:
%  Ref:
%  Mod:
%EndInfo%%%%%%%%%%%%%%%%%%%%%%%%%%%%%%%%%%%%%%%%%%%%%%%%%%%%%%%%%%%%%%%%%%%%%%
%\poemtitle{Geh aus, mein Herz, und suche Freud}
\begin{multicols}{2}
\settowidth{\versewidth}{Die Lerche schwingt sich in die Luft,}
\begin{verse}[\versewidth]

\flagverse{1.} Geh aus, mein Herz, und suche Freud\\
in dieser lieben Sommerzeit\\
an deines Gottes Gaben;\\
schau an der schönen Gärten Zier\\
und siehe, wie sie mir und dir\\
sich ausgeschmücket haben.

\flagverse{2.} Die Bäume stehen voller Laub,\\
das Erdreich decket seinen Staub\\
mit einem grünen Kleide;\\
Narzissus und die Tulipan,\\
die ziehen sich viel schöner an\\
als Salomonis Seide.

\flagverse{3.} Die Lerche schwingt sich in die Luft,\\
das Täublein fleugt aus seiner Kluft\\
und macht sich in die Wälder;\\
die hochbegabte Nachtigall\\
ergötzt und füllt mit ihrem Schall\\
Berg, Hügel, Tal und Felder.

\flagverse{4.} Die Glucke führt ihr Völklein aus,\\
der Storch baut und bewohnt sein Haus,\\
das Schwälblein speist die Jungen;\\
der schnelle Hirsch, das leichte Reh\\
ist froh und kommt aus seiner Höh\\
ins tiefe Gras gesprungen.

\flagverse{5.} Die Bächlein rauschen in dem Sand\\
und malen sich in ihrem Rand\\
mit schattenreichen Myrten;\\
die Wiesen liegen hart dabei\\
und klingen ganz von Lustgeschrei\\
der Schaf und ihrer Hirten.

\flagverse{6.} Die unverdroßne Bienenschar\\
fleucht hin und her, sucht hie und dar\\
ihr edle Honigspeise.\\
Des süßen Weinstocks starker Saft\\
bringt täglich neue Stärk und Kraft\\
in seinem schwachen Reise.

\flagverse{7.} Der Weizen wächset mit Gewalt,\\
darüber jauchzet Jung und Alt\\
und rühmt die große Güte\\
des, der so überflüssig labt\\
und mit so manchem Gut begabt\\
das menschliche Gemüte.

\flagverse{8.} Ich selbsten kann und mag nicht ruhn;\\
des großen Gottes großes Tun\\
erweckt mir alle Sinnen;\\
ich singe mit, wenn alles singt,\\
und lasse, was dem Höchsten klingt,\\
aus meinem Herzen rinnen.

\flagverse{9.} Ach, denk ich, bist du hier so schön\\
und läßt du uns so lieblich gehn\\
auf dieser armen Erden,\\
was will doch wohl nach dieser Welt\\
dort in dem festen Himmelszelt\\
und güldnen Schlosse werden!

\flagverse{10.} Welch hohe Lust, welch heller Schein\\
wird wohl in Christi Garten sein!\\
Wie Muß es da wohl klingen,\\
da so viel tausend Seraphim\\
mit eingestimmtem Mund und Stimm\\
ihr Halleluja singen!

\flagverse{11.} O wär ich da, o stünd ich schon,\\
ach, süßer Gott, vor deinem Thron\\
und trüge meine Palmen,\\
so wollt ich nach der Engel Weis\\
erhöhen deines Namens Preis\\
mit tausend schönen Psalmen!

\flagverse{12.} Doch gleichwohl will ich, weil ich noch\\
hier trage dieses Leibes Joch,\\
auch nicht gar stille schweigen;\\
mein Herze soll sich fort und fort\\
an diesem und an allem Ort\\
zu deinem Lobe neigen.

\flagverse{13.} Hilf mir und segne meinen Geist\\
mit Segen, der vom Himmel fleußt,\\
daß ich dir stetig blühe!\\
Gib, daß der Sommer deiner Gnad\\
in meiner Seelen früh und spat\\
viel Glaubensfrücht erziehe!

\flagverse{14.} Mach in mir deinem Geiste Raum,\\
daß ich dir werd ein guter Baum,\\
und laß mich wohl bekleiben;\\
verleihe, daß zu deinem Ruhm\\
ich deines Gartens schöne Blum\\
und Pflanze möge bleiben!

\end{verse}
\end{multicols}

\begin{center}
\settowidth{\versewidth}{Der, vor dem die Welt erschrickt,}
\begin{verse}[\versewidth]

\flagverse{15.} Erwähle mich zum Paradeis\\
und laß mich bis zur letzten Reis\\
an Leib und Seele grünen;\\
so will ich dir und deiner Ehr\\
allein und sonsten keinem mehr\\
hier und dort ewig dienen.
  
\end{verse}
\end{center}


%\attrib{\small{THZE}}

\index{Geh aus, mein Herz und suche Freud}
\newpage
\subsection*{\centerline{Gott Lob! Nun ist erschollen}}         %witt:Gottlob
\addcontentsline{toc}{subsection}{Gott Lob! Nun ist erschollen}         %witt:Gottlob}
%StartInfo%%%%%%%%%%%%%%%%%%%%%%%%%%%%%%%%%%%%%%%%%%%%%%%%%%%%%%%%%%%%%%%%%%%%
%  Autor:
%  Titel:
%  File:
%  Ref:
%  Mod:
%EndInfo%%%%%%%%%%%%%%%%%%%%%%%%%%%%%%%%%%%%%%%%%%%%%%%%%%%%%%%%%%%%%%%%%%%%%%
%\poemtitle{Gott Lob! Nun ist erschollen das edle Fried- und Freudenwort}
\begin{multicols}{2}
\settowidth{\versewidth}{die Spieß und Schwerter und ihr Mord.}
\begin{verse}[\versewidth]
%Danklied für die Verkündigung des Friedens


\flagverse{1.} Gott Lob! Nun ist erschollen\\
das edle Fried- und Freudenwort,\\
daß nunmehr ruhen sollen\\
die Spieß und Schwerter und ihr Mord.\\
Wohlauf und nimm nun wieder\\
dein Saitenspiel hervor,\\
o Deutschland, und sing Lieder\\
im hohen vollen Chor.\\
Erhebe Dein Gemüte\\
zu deinem Gott und sprich:\\
Herr, deine Gnad und Güte\\
bleibt dennoch ewiglich!

\flagverse{2.} Wir haben nichts verdienet\\
als schwere Straf und großen Zorn,\\
weil stets noch bei uns grünet\\
der freche schnöde Sündendorn.\\
Wir sind fürwahr geschlagen\\
mit harter, scharfer Rut,\\
und dennoch muß man fragen:\\
Wer ist, der Buße tut?\\
Wir sind und bleiben böse,\\
Gott ist und bleibet treu,\\
hilft, daß sich bei uns löse\\
der Krieg und sein Geschrei.

\flagverse{3.} Sei tausendmal willkommen,\\
du teure werte Friedensgab!\\
Jetzt sehn wir, was für Frommen\\
dein Bei-uns-wohnen in sich hab;\\
in dir hat Gott versenket\\
all unser Glück und Heil.\\
Wer dich betrübt und kränket,\\
der drückt sich selbst den Pfeil\\
des Herzleids in das Herze\\
und löscht aus Unverstand\\
die güldne Freudenkerze\\
mit seiner eignen Hand.

\flagverse{4.} Das drückt uns niemand besser\\
in unser Herz und Seel hinein\\
als ihr zerstörten Schlösser\\
und Städte voller Schutt und Stein;\\
ihr vormals schönen Felder\\
mit frischer Saat bestreut,\\
jetzt aber lauter Wälder\\
und dürre wüste Heid;\\
ihr Gräber voller Leichen\\
und blutgen Heldenschweiß\\
der Helden, derengleichen\\
auf Erden man nicht weiß.

\flagverse{5.} Hier trübe deine Sinnen,\\
o Mensch, und laß die Tränenbach\\
aus beiden Augen rinnen,\\
geh in dein Herz und denke nach:\\
Was Gott bisher gesendet,\\
das hast du ausgelacht,\\
nun hat er sich gewendet\\
und väterlich bedacht,\\
vom Grimm und scharfen Dringen\\
zu deinem Heil zu ruhn,\\
ob er dich möchte zwingen\\
mit Lieb und Gutestun.

\flagverse{6.} Ach, laß dich doch erwecken,\\
wach auf, wach auf, du harte Welt,\\
eh als das harte Schrecken\\
dich schnell und plötzlich überfällt!\\
Wer aber Christum liebet,\\
sei unerschrocknes Muts,\\
der Friede, den er gibet,\\
bedeutet alles Guts.\\
Er will die Lehre geben:\\
Das Ende naht herzu,\\
da sollt ihr bei Gott leben\\
in ewgem Fried und Ruh.

\end{verse}
\end{multicols}
%\attrib{\small{THZE}}

\index{Gott Lob, nun ist erschollen}
\newpage
\subsection*{\centerline{Herr, dir trau ich all mein Tage}}
\addcontentsline{toc}{subsection}{Herr dir trau ich all mein Tage}
%StartInfo%%%%%%%%%%%%%%%%%%%%%%%%%%%%%%%%%%%%%%%%%%%%%%%%%%%%%%%%%%%%%%%%%%%%
%  Autor:
%  Titel:
%  File:
%  Ref:
%  Mod:
%EndInfo%%%%%%%%%%%%%%%%%%%%%%%%%%%%%%%%%%%%%%%%%%%%%%%%%%%%%%%%%%%%%%%%%%%%%%
%\poemtitle{Herr, dir trau ich all mein Tage}
\begin{multicols}{2}
\settowidth{\versewidth}{Herr, dir trau ich all mein Tage}
\begin{verse}[\versewidth]
%Der 71. Psalm\\
%Auf den Tod des Amtsschreibers Joachim Schröder zu Mittenwalde

\flagverse{1.} Herr, dir trau ich all mein Tage\\
laß mich nicht mit Schimpf bestehn.\\
Wie ich von dir glaub und sage,\\
also laß mirs auch ergehn.\\
Rette mich, laß deine Güte\\
mir erfrischen mein Gemüte,\\
neige deiner Ohren Treu\\
und vernimm mein Angstgeschrei!

\flagverse{2.} Sei mein Aufhalt, laß mich sitzen\\
bei dir, o mein starker Hort!\\
Laß mich deinen Schutz beschützen\\
und erfülle mir dein Wort,\\
da du selbsten meinem Leben\\
dich zum Fels und Burg gegeben.\\
Hilf mir aus des Heuchlers Band\\
und des Ungerechten Hand!

\flagverse{3.} Denn dich hab ich auserlesen\\
von der zarten Jugend an;\\
dein Arm ist mein Trost gewesen,\\
Herr, so lang ich denken kann;\\
auf dich hab ich mich erwogen,\\
alsbald du mich der entzogen,\\
der ich, ehe Nacht und Tag\\
mich erblickt, im Leibe lag.

\flagverse{4.} Von dir ist mein Ruhm, mein Sagen,\\
dein erwähn ich immerzu;\\
viel, die spotten meiner Plagen,\\
höhnen, was ich red und tu.\\
Aber du bist meine Stärke:\\
Wenn ich Angst und Trübsal merke,\\
lauf ich dich an. Gönne mir,\\
fröhlich stets zu sein in dir!

\flagverse{5.} Stoß mich nicht von deiner Seiten,\\
wenn mein hohes Alter kömmt,\\
da die schwachen Tritte gleiten\\
und man Trost vom Stecken nimmt;\\
da greif du mir an die Arme,\\
fall ich nieder, so erbarme\\
du dich, hilf mir in die Höh\\
und halt, bis ich wieder steh.

\flagverse{6.} Mach es nicht wie mirs die gönnen,\\
die mein abgesagte Feind,\\
auch mir, wo sie immer können,\\
mit Gewalt zuwider seind;\\
sprechen: Auf, laßt uns ihn fassen,\\
sein Gott hat ihn ganz verlassen,\\
jagt und schlagt ihn immerhin,\\
niemand schützt und rettet ihn!

\flagverse{7.} Ach, mein Helfer, sei nicht ferne,\\
komm und eile doch zu mir,\\
hilf mir, mein Gott, bald und gerne,\\
zeuch mich aus der Not herfür,\\
daß sich meine Feinde schämen\\
und vor Hohn und Schande grämen,\\
ich hingegen lustig sei\\
über mir erwiesne Treu.

\flagverse{8.} Mein Herz soll dir allzeit bringen\\
deines Lobs gebührlich Teil,\\
auch soll meine Zunge singen\\
täglich dein unzählig Heil.\\
Ich bin stark, hereinzugehen,\\
unerschrocken dazustehen\\
durch des großen Herrschers Kraft,\\
der die Erd und alles schafft.

\flagverse{9.} Herr, ich preise deine Tugend,\\
Wahrheit und Gerechtigkeit,\\
die mich noch in meiner Jugend\\
hoch ergötzet und erfreut;\\
hast mich als ein Kind ernähret,\\
deine Furcht dabei gelehret,\\
oftmals wunderlich bedeckt,\\
daß mein Feind mich nicht erschreckt.

\flagverse{10.} Fahre fort, o mein Erhalter,\\
fahre fort und laß mich nicht\\
in dem hohen grauen Alter,\\
wenn mir Lebenskraft gebricht;\\
laß mein Leben in dir leben,\\
bis ich Unterricht kann geben\\
Kindeskindern, daß dein Hand\\
ihnen gleichfalls sei bekannt.

\flagverse{11.} Gott, du bist sehr hoch zu loben,\\
dir ist nirgend etwas gleich,\\
weder hier bei uns noch droben\\
in dem Stern- und Engelreich.\\
Dein Tun ist nicht auszusprechen,\\
deinen Rat kann niemand brechen,\\
alles liegt dir in dem Schoß,\\
und dein Werk ist alles groß.

\flagverse{12.} Du ergibst mich großen Nöten,\\
gibst auch wieder große Freud,\\
heute läßt du mich ertöten,\\
morgen ist die Lebens Zeit,\\
da ermunterst du mich wieder\\
und erneuerst meine Glieder,\\
holst sie aus der Erdenkluft,\\
gibst dem Herzen wieder Luft.

\flagverse{13.} Such ich Trost und finde keinen?\\
Balde Werd ich wieder groß.\\
Dein Trost Trocknet mir mein Weinen,\\
das mir aus den Augen floß.\\
Ich Selbst werde wie ganz neue,\\
sing und klinge deine Treue,\\
meines Lebens einzges Ziel,\\
auf der Harf und Psalterspiel.

\flagverse{14.} Ich bin durch und durch entzündet,\\
fröhlich ist, was in mir ist,\\
alle mein Geblüt empfindet\\
dein Heil, das du selber bist.\\
Ich steh im gewünschten Stande,\\
mein Feind ist voll Scham und Schande;\\
der mein Unglück hat gesucht,\\
leidet, was er mir geflucht.

\end{verse}
\end{multicols}
%\attrib{\small{THZE}}

\index{Herr, die trau ich alle Tage}
\newpage
\subsection*{\centerline{Herr, du erforschest meinen Sinn}}
\addcontentsline{toc}{subsection}{Herr du erforschest meinen Sinn}
%StartInfo%%%%%%%%%%%%%%%%%%%%%%%%%%%%%%%%%%%%%%%%%%%%%%%%%%%%%%%%%%%%%%%%%%%%
%  Autor:
%  Titel:
%  File:
%  Ref:
%  Mod:
%EndInfo%%%%%%%%%%%%%%%%%%%%%%%%%%%%%%%%%%%%%%%%%%%%%%%%%%%%%%%%%%%%%%%%%%%%%%
%\poemtitle{Herr, du erforschest meinen Sinn}
\begin{multicols}{2}
\settowidth{\versewidth}{Das ist mir kund. Und bleibet doch}
\begin{verse}[\versewidth]
%der 139. Psalm


\flagverse{1.} Herr, du erforschest meinen Sinn\\
und kennest, was ich hab und bin,\\
ja, was mir selbst verborgen ist,\\
das weißt du, der du alles bist.

\flagverse{2.} Ich sitz hier oder stehe auf,\\
ich lieg, ich geh auch oder lauf:\\
So bist du um und neben mir,\\
und ich bin allzeit hart bei dir.

\flagverse{3.} All die Gedanken meiner Seel,\\
und was sich in der Herzenshöhl\\
hier reget, hast du schon betracht,\\
eh ich einmal daran gedacht.

\flagverse{4.} Auf meiner Zunge ist kein Wort,\\
das du nicht hörest allsofort,\\
du schaffests, was ich red und tu,\\
und siehst all meinem Leben zu.

\flagverse{5.} Das ist mir kund. Und bleibet doch\\
mir solch Erkenntnis viel zu hoch,\\
es ist die Weisheit, die kein Mann\\
recht aus dem Grunde wissen kann.

\flagverse{6.} Wo soll ich, der du alles weißt,\\
mich wenden hin vor deinem Geist?\\
Wo soll ich deinem Angesicht\\
entgehen, daß michs sehe nicht?

\flagverse{7.} Führ ich gleich an des Himmels Dach\\
so bist du da, hältst Hut und Wach,\\
stieg ich zur Höll und wollte mir\\
da betten, find ich dich auch hier.

\flagverse{8.} Wollt ich der Morgenröten gleich\\
geflügelt ziehn, so weit das Reich\\
der wilden Fluten netzt das Land,\\
käm ich doch nie aus deiner Hand.

\flagverse{9.} Rief ich zu Hilf die finstre Nacht,\\
hätt ich doch damit nichts verbracht;\\
denn laß die Nacht sein wie sie mag,\\
so ist sie bei dir heller Tag.

\flagverse{10.} Dich blendt der dunkle Schatten nicht,\\
die Finsternis ist dir ein Licht,\\
dein Augenglanz ist klar und rein,\\
darf weder Sonn noch Mondenschein.

\flagverse{11.} Mein Eingeweid ist dir bekannt,\\
es liegt frei da in deiner Hand,\\
der du von Mutterleibe an\\
mir lauter Lieb und Guts getan.

\flagverse{12.} Du bists, der Fleisch, Gebein und Haut\\
so künstlich in mir aufgebaut;\\
all deine Werk sind Wunder voll,\\
und das weiß meine Seele wohl.

\flagverse{13.} Du sahest mich, da ich noch gar\\
fast nichts und unbereitet war,\\
warst selbst mein Meister über mir\\
und zogst mich aus der Tief herfür.

\flagverse{14.} Auch meiner Tag und Jahre Zahl,\\
Minuten, Stunden allzumal\\
hast du, als meiner Zeiten Lauf,\\
vor meiner Zeit geschrieben auf.

\flagverse{15.} Wie köstlich, herrlich, süß und schön\\
seh ich, mein Gott, da vor mir stehn\\
dein weises Denken, was du denkst,\\
wenn du uns deine Güter schenkst!

\flagverse{16.} Wie ist doch des so trefflich viel!\\
Wenn ich bisweilen zählen will,\\
so find ich da bei weitem mehr\\
als Staub im Feld und Sand am Meer.

\flagverse{17.} Was macht denn nun die wüste Rott,\\
die dich, o großer Wundergott,\\
so schändlich lästert und mit Schmach\\
dir so viel Übels redet nach?

\flagverse{18.} Ach, stopfe ihren schnöden Mund!\\
Steh auf und stürze sie zu Grund!\\
Denn weil sie deine Feinde seind,\\
bin ich auch ihnen herzlich feind.

\flagverse{19.} Ob sie gleich nun hinwieder sehr\\
mich hassen, tu ich doch nicht mehr,\\
als daß ich wider ihren Trutz\\
mich leg in deinen Schoß und Schutz.

\flagverse{20.} Erforsch, Herr, all mein Herz und Mut,\\
sieh, ob mein Weg sei recht und gut,\\
und führe mich bald himmelan\\
den ewgen Weg, die Freudenbahn.

\end{verse}
\end{multicols}
%\attrib{\small{THZE}}

\index{Herr, du erforschest meinen Sinn}
\newpage
\subsection*{\centerline{Ich danke dir mit Freuden}}
\addcontentsline{toc}{subsection}{Ich danke dir mit Freuden}
%StartInfo%%%%%%%%%%%%%%%%%%%%%%%%%%%%%%%%%%%%%%%%%%%%%%%%%%%%%%%%%%%%%%%%%%%%
%  Autor:
%  Titel:
%  File:
%  Ref:
%  Mod:
%EndInfo%%%%%%%%%%%%%%%%%%%%%%%%%%%%%%%%%%%%%%%%%%%%%%%%%%%%%%%%%%%%%%%%%%%%%%
%\poemtitle{pt}
\begin{multicols}{2}
\settowidth{\versewidth}{Ich bitte, was ich bitten kann,}
\begin{verse}[\versewidth]
%ich danke dir demütiglich\\
  %nach Johann Arnds »Paradiesgärtlein«, Goslar 1621, III, 17:
  %»Gebet um zeitliche und ewige Wohlfahrt«

\flagverse{1.} Ich danke dir demütiglich,\\
o Gott, mein Vater, daß du dich\\
von deinem Zorn gewendet\\
und deinen Sohn\\
zur Freud und Kron\\
uns in die Welt gesendet.

\flagverse{2.} Er ist gekommen, hat sein Blut\\
vergossen und in solcher Flut\\
all unser Sünd ersticket.\\
Wer ihn nur faßt,\\
wird aller Last\\
benommen und erquicket.

\flagverse{3.} Ich bitte, was ich bitten kann,\\
herzlieber Vater, nimm mich an\\
in diesen edlen Orden,\\
der durch dies Blut\\
gerecht und gut\\
und ewig selig worden.

\flagverse{4.} Laß meines Glaubens Aug und Hand\\
ergreifen dieses werte Pfand\\
und nimmermehr verlieren;\\
laß dieses Licht\\
mein Angesicht\\
zum ewgen Lichte führen!

\flagverse{5.} Bereite meiner Seelen Haus,\\
wirf allen Kot und Unflat aus,\\
bau in mir deine Hütte,\\
daß deine Güt\\
in mein Gemüt\\
all ihre Lieb ausschütte!

\flagverse{6.} Wann ich dich hab, ist alles mein;\\
du kannst nicht ohne Gaben sein,\\
hast tausend Weg und Weisen,\\
dein arme Herd\\
auf dieser Erd\\
zu nähren und zu speisen.

\flagverse{7.} Gib mir, daß ich an meinem Ort\\
allstets dich fürcht in deinem Wort\\
und meinen Stand so führe,\\
daß Glaub und Treu\\
stets bei mir sei\\
und all mein Leben ziere!

\flagverse{8.} Gib nur ein gnügsam Herz und Sinn!\\
Denn das ist ja ein großer Gwinn,\\
in steter Andacht liegen\\
und, wenn Gott gibt\\
was ihm beliebt,\\
ihm lassen gern genügen.

\flagverse{9.} Das Wen'ge, das durch Gottes Hand\\
ein Frommer und Gerechter hat,\\
ist vielmal mehr geehret\\
als alles Geld,\\
davon die Welt\\
mit frechem Herzen zehret.

\flagverse{10.} Die Frommen sind dir, Herr, bewußt;\\
du bist ihr und sie deine Lust\\
und werden nicht zuschanden,\\
kommt teure Zeit,\\
findt sich bereit\\
ihr Brot in allen Landen.

\flagverse{11.} Gott hat den, der ihn fürchtet, lieb,\\
sieht zu, daß ihn kein Unfall trüb,\\
hat Lust zu seinen Wegen;\\
und wenn er fällt,\\
steht Gott und hält\\
ihn fest in seinem Segen.

\flagverse{12.} Des Höchsten Auge sieht auf die,\\
so auf ihn hoffen spat und früh,\\
daß er sie schütz und rette\\
aus aller Not,\\
wann sie der Tod\\
auch selbst verschlungen hätte.

\flagverse{13.} Herr, du kannst nichts als Güte sein,\\
du wollest deiner Güte Schein\\
uns und all denen gönnen,\\
die sich mit Mund\\
und Herzensgrund\\
allein zu dir bekennen!

\flagverse{14.} Insonderheit nimm wohl in Acht\\
den Fürsten, den du uns gemacht\\
zu unsers Landes Krone,\\
laß immerzu\\
sein Fried und Ruh\\
auf seinem Stuhl und Throne.

\flagverse{15.} Halt unser liebes Vaterland\\
in deinem Schoß und starker Hand!\\
Behüt uns allzusammen\\
vor falscher Lehr\\
und Feindes Heer,\\
vor Pest und Feuersflammen.

\flagverse{16.} Nimm all der Meinen eben wahr,\\
treib, Herr, die böse Höllenschar\\
von Jungen und von Alten,\\
daß deine Herd\\
hie zeitlich werd\\
und ewig dort erhalten.

\end{verse}
\end{multicols}
%\attrib{\small{THZE}}

\index{Ich danke dir mit Freuden}
\newpage
\subsection*{\centerline{Ich, der ich oft in tiefes Leid}}
\addcontentsline{toc}{subsection}{Ich der ich oft in tiefes Leid}
%StartInfo%%%%%%%%%%%%%%%%%%%%%%%%%%%%%%%%%%%%%%%%%%%%%%%%%%%%%%%%%%%%%%%%%%%%
%  Autor:
%  Titel:
%  File:
%  Ref:
%  Mod:
%EndInfo%%%%%%%%%%%%%%%%%%%%%%%%%%%%%%%%%%%%%%%%%%%%%%%%%%%%%%%%%%%%%%%%%%%%%%
%\poemtitle{pt}
\begin{multicols}{2}
\settowidth{\versewidth}{Die Welt, die deucht uns schön und groß}
\begin{verse}[\versewidth]
%der 145. Psalm\\
%ich, der ich oft in tiefes Leid und große Not muß gehen

\flagverse{1.} Ich, der ich oft in tiefes Leid\\
und große Not muß gehen,\\
will dennoch Gott mit großer Freud\\
und Herzenslust erhöhen.\\
Mein Gott, du König, höre mich,\\
ich will ohn alles Ende dich\\
und deinen Namen loben.

\flagverse{2.} Ich will dir mit der Morgenröt\\
ein täglich Opfer bringen,\\
so oft die liebe Sonn aufgeht,\\
so ofte will ich singen\\
dem großen Namen deiner Macht,\\
das soll auch in der späten Nacht\\
mein Werk sein und Geschäfte.

\flagverse{3.} Die Welt, die deucht uns schön und groß\\
und was für Gut und Gaben\\
sie trägt in ihrem Arm und Schoß,\\
das will ein jeder haben:\\
Und ist doch alles lauter Nichts;\\
eh als mans recht genießt, zerbrichts\\
und geht im Hui zugrunde.

\flagverse{4.} Gott ist alleine groß und schön,\\
unmöglich auszuloben\\
auch denen, die doch allzeit stehn\\
vor seinem Throne droben.\\
Laß sprechen, wer nur sprechen kann,\\
doch wird kein Engel noch kein Mann\\
des Höchsten Größ aussprechen.

\flagverse{5.} Die Alten, die nun nicht mehr sind,\\
die haben ihn gepreiset;\\
so hat ein jeder auch sein Kind\\
zu solchem Dienst geweiset;\\
die Kinder werden auch nicht ruhn\\
und werden doch, o Gott, dein Tun\\
und Werk nicht ganz auspreisen.

\flagverse{6.} Wie mancher hat vor mir dein Heil\\
und Lob mit Fleiß getrieben;\\
und siehe, mir ist doch mein Teil\\
zu loben übrig blieben.\\
Ich will von deiner Wundermacht\\
und der so herrlich schönen Pracht\\
bis an mein Ende reden.

\flagverse{7.} Und was ich rede, wird von mir\\
manch frommes Herze lernen,\\
man wird dich heben für und für\\
hoch über allen Sternen;\\
dein Herrlichkeit und starke Hand\\
wird in der ganzen Welt bekannt\\
und hoch berufen werden.

\flagverse{8.} Wer ist so gnädig als wie du?\\
Wer kann so viel erdulden?\\
Wer sieht mit solcher Langmut zu\\
so vielen schweren Schulden,\\
die aus der ganzen weiten Welt\\
ohn Unterlaß bis an das Zelt\\
des hohen Himmels steigen?

\flagverse{9.} Es muß ein treues Herze sein,\\
das uns so hoch kann lieben,\\
da wir doch in den Tag hinein,\\
was gar nicht gut ist, üben.\\
Gott muß nichts anders sein als gut,\\
daher fließt seiner Güte Flut\\
auf alle seine Werke.

\flagverse{10.} Drum, Herr, so sollen dir auch nun\\
all deine Werke danken,\\
voraus die Heilgen, deren Tun\\
sich hält in deinen Schranken,\\
die sollen deines Reichs Gewalt\\
und schöne Regimentsgestalt\\
mit vollem Munde rühmen.

\flagverse{11.} Sie sollen rühmen, daß der Ruhm\\
durch alle Welt erklinge,\\
daß jedermann zum Heiligtum\\
dir Dienst und Opfer bringe;\\
dein Reich, das ist ein ewges Reich,\\
dein Herrschaft ist dir selber gleich,\\
der du kein End erreichest.

\flagverse{12.} Der Herr ist bis in unsern Tod\\
beständig bei uns allen,\\
erleichtert unsers Kreuzes Not\\
und hält uns, wenn wir fallen;\\
er steuert manches Unglücks Lauf\\
und hilft uns wieder freundlich auf,\\
wenn wir ganz hingeschlagen.

\flagverse{13.} Herr, aller Augen sind nach dir\\
und deinem Stuhl gekehret;\\
denn du bists auch, der alles hier\\
so väterlich ernähret;\\
du tust auf deine milde Hand,\\
machst froh und satt, was auf dem Land,\\
im Meer und Lüften lebet.

\flagverse{14.} Du meinst es gut und tust uns Guts,\\
auch da wirs oft nicht denken,\\
wie mancher ist betrübtes Muts\\
und frißt sein Herz mit Kränken,\\
besorgt und fürcht sich Tag und Nacht,\\
Gott hab ihn gänzlich aus der Acht\\
gelassen und vergessen.

\flagverse{15.} Nein, Gott vergißt der Seinen nicht,\\
er ist uns viel zu treue,\\
sein Herz ist stets dahin gericht,\\
daß er uns letzt erfreue.\\
Gehts gleich bisweilen etwas schlecht,\\
ist er doch heilig und gerecht\\
in allen seinen Wegen.

\flagverse{16.} Der Herr ist nah und stets bereit\\
eim jeden, der ihn ehret,\\
und wer nur ernstlich zu ihm schreit,\\
der wird gewiß erhöret.\\
Gott weiß wohl, wer ihm günstig sei,\\
und deme steht er dann auch bei,\\
wann ihn die Angst nun treibet.

\flagverse{17.} Den Frommen wird nichts abgesagt;\\
Gott tut, was sie begehren,\\
er mißt das Unglück, das sie plagt,\\
und zählt all ihre Zähren\\
und reißt sie endlich aus der Last;\\
den aber, der sie kränkt und haßt,\\
den stürzt er ganz zu Boden.

\flagverse{18.} Dies alles und was sonsten mehr\\
man kann für Lob erzwingen,\\
das soll mein Mund zu Ruhm und Ehr\\
des Höchsten täglich singen:\\
Und also tut auch immerfort\\
was lebt und webt an jedem Ort:\\
Das wird Gott wohlgefallen.

\end{verse}
\end{multicols}
%\attrib{\small{THZE}}

\index{Ich, der ich oft in tiefes Leid}
\newpage
\subsection*{\centerline{Ich preise dich und singe}}
\addcontentsline{toc}{subsection}{Ich preise dich und singe}
%StartInfo%%%%%%%%%%%%%%%%%%%%%%%%%%%%%%%%%%%%%%%%%%%%%%%%%%%%%%%%%%%%%%%%%%%%
%  Autor:
%  Titel:
%  File:
%  Ref:
%  Mod:
%EndInfo%%%%%%%%%%%%%%%%%%%%%%%%%%%%%%%%%%%%%%%%%%%%%%%%%%%%%%%%%%%%%%%%%%%%%%
%\poemtitle{pt}
\begin{multicols}{2}
\settowidth{\versewidth}{Herr, mein Gott, da ich Kranker}
\begin{verse}[\versewidth]
%der 30. Psalm\\
%ich preise dich und singe, Herr

\flagverse{1.} Ich preise dich und singe,\\
Herr, deine Wundergnad,\\
die mir so große Dinge\\
bisher erwiesen hat;\\
denn das ist meine Pflicht,\\
in meinem ganzen Leben\\
dir Lob und Dank zu geben,\\
mehr hab und kann ich nicht.

\flagverse{2.} Du hast mein Herz erhöhet\\
aus mancher tiefen Not,\\
den aber, der da gehet\\
und suchet meinen Tod\\
und tut mir Herzleid an,\\
den hast du weggeschlagen,\\
daß er sich meiner Plagen\\
mit nicht erfreuen kann.

\flagverse{3.} Herr, mein Gott, da ich Kranker\\
vom Bette zu dir schrei,\\
da ward mein Heil mein Anker\\
und stund mir treulich bei;\\
da andre fuhren hin\\
zur finstern Todeshöhle,\\
da hieltst du meine Seele\\
und mich noch, wo ich bin.

\flagverse{4.} Ihr Heiligen, lobsinget\\
und danket eurem Herrn,\\
der, wenn die Not herdringet,\\
bald hört und herzlich gern\\
uns Gnad und Hilfe gibt;\\
rühmt den, des Hand uns träget\\
und, wenn er uns ja schläget,\\
nicht allzusehr betrübt.

\flagverse{5.} Gott hat ja Vaterhände\\
und strafet mit Geduld,\\
sein Zorn nimmt bald ein Ende,\\
sein Herz ist voller Huld\\
und gönnt uns lauter Guts.\\
Den Abend währt das Weinen,\\
des Morgens macht das Scheinen\\
der Sonn uns gutes Muts.

\flagverse{6.} Ich sprach zur guten Stunde,\\
da mirs noch wohl erging:\\
Ich steh auf festem Grunde,\\
acht alles Kreuz gering;\\
ich werde nimmermehr,\\
das weiß ich, niederliegen;\\
denn Gott der kann nicht trügen,\\
der liebt mich gar zu sehr.

\flagverse{7.} Als aber dein Gesichte,\\
ach Gott, sich von mir wandt,\\
da war mein Trost zunichte,\\
da lag mein Heldenstand;\\
es war mir angst und bang,\\
ich führte schwere Klagen\\
mit Zittern und mit Zagen:\\
Herr, mein Gott, wie so lang?

\flagverse{8.} Hast du dir vorgenommen,\\
mein ewger Feind zu sein?\\
Was werden dir denn frommen\\
die ausgedorrten Bein\\
und der elende Staub,\\
zu welchem in der Erden\\
wir werden, wenn wir werden\\
des blassen Todes Raub?

\flagverse{9.} So lang ichs Leben habe,\\
lobsing ich deiner Ehr,\\
dort aber in dem Grabe,\\
gedenk ich dein nicht mehr;\\
drum eil und hilf mir auf\\
und gib mir Kraft und Leben;\\
dafür will ich dir geben\\
meins ganzen Lebens Lauf.

\flagverse{10.} Nun wohl, ich bin erhöret,\\
mein Seufzen ist erfüllt,\\
mein Kreuz ist umgekehret,\\
mein Herzleid ist gestillt,\\
mein Grämen hat ein End;\\
es ist von meinem Herzen\\
der bittern Sorgen Schmerzen\\
durch dich, Herr, abgewendt.

\flagverse{11.} Du hast mit mir gehandelt\\
noch besser, als ich will;\\
mein Klagen ist verwandelt\\
in eines Reigens Spiel,\\
und für das Trauerkleid,\\
in dem ich vor gestöhnet,\\
da hast du mich gekrönet\\
mit süßer Lust und Freud.

\flagverse{12.} Auf daß zu deiner Ehre\\
mein Ehre sich erhüb\\
und nimmer stille wäre,\\
bis daß ich deine Lieb\\
und ungezählte Zahl\\
der großen Wunderdinge\\
mit ewgen Freuden singe\\
im güldnen Himmelssaal.
   
\end{verse}
\end{multicols}
%\attrib{\small{THZE}}

\index{Ich preise dich und singe}
\newpage
\subsection*{\centerline{Ich singe dir mit Herz und Mund}}
\addcontentsline{toc}{subsection}{Ich singe dir mit Herz und Mund}
%StartInfo%%%%%%%%%%%%%%%%%%%%%%%%%%%%%%%%%%%%%%%%%%%%%%%%%%%%%%%%%%%%%%%%%%%%
%  Autor:
%  Titel:
%  File:
%  Ref:
%  Mod:
%EndInfo%%%%%%%%%%%%%%%%%%%%%%%%%%%%%%%%%%%%%%%%%%%%%%%%%%%%%%%%%%%%%%%%%%%%%%
%\poemtitle{pt}
\begin{multicols}{2}
\settowidth{\versewidth}{Ich weiß, daß du der Brunn der Gnad}
\begin{verse}[\versewidth]
%ich singe dir mit Herz und Mund

\flagverse{1.} Ich singe dir mit Herz und Mund,\\
Herr, meines Herzens Lust,\\
ich sing und mach auf Erden kund,\\
was mir von dir bewußt.

\flagverse{2.} Ich weiß, daß du der Brunn der Gnad\\
und ewge Quelle seist,\\
daraus uns allen früh und spat\\
viel Heil und Gutes fleußt.

\flagverse{3.} Was sind wir doch? was haben wir\\
auf dieser ganzen Erd,\\
das uns, o Vater, nicht von dir\\
allein gegeben werd?

\flagverse{4.} Wer hat das schöne Himmelszelt\\
hoch über uns gesetzt?\\
Wer ist es, der uns unser Feld\\
mit Tau und Regen netzt?

\flagverse{5.} Wer wärmet uns in Kält und Frost?\\
Wer schützt uns vor dem Wind?\\
Wer macht es, daß man Öl und Most\\
zu seinen Zeiten findt?

\flagverse{6.} Wer gibt uns Leben und Geblüt?\\
Wer hält mit seiner Hand\\
den güldnen, werten, edlen Fried\\
in unserm Vaterland?

\flagverse{7.} Ach Herr, mein Gott, das kommt von dir!\\
Du, du mußt alles tun,\\
du hältst die Wacht an unsrer Tür\\
und läßt uns sicher ruhn.

\flagverse{8.} Du nährest uns von Jahr zu Jahr,\\
bleibst immer fromm und treu\\
und stehst uns, wann wir in Gefahr\\
geraten, treulich bei.

\flagverse{9.} Du strafst uns Sünder mit Geduld\\
und schlägst nicht allzu sehr,\\
ja endlich nimmst du unsre Schuld\\
und wirfst sie in das Meer.

\flagverse{10.} Wann unser Herze seufzt und schreit,\\
wirst du gar leicht erweicht,\\
und gibst uns, was uns hoch erfreut\\
und dir zu Ehren reicht.

\flagverse{11.} Du zählst, wie oft ein Christe wein\\
und was sein Kummer sei,\\
kein Zähr- und Tränlein ist so klein,\\
du hebst und legst es bei.

\flagverse{12.} Du füllst des Lebens Mangel aus\\
mit dem, was ewig steht,\\
und führst uns in das Himmelshaus,\\
wann uns die Erd entgeht.

\flagverse{13.} Wohlauf, mein Herze, sing und spring\\
und habe guten Mut,\\
dein Gott, der Ursprung aller Ding,\\
ist selbst und bleibt dein Gut.

\flagverse{14.} Er ist dein Schatz, dein Erb und Teil,\\
dein Glanz und Freudenlicht,\\
dein Schirm und Schild, dein Hilf und Heil,\\
schafft Rat und läßt dich nicht.

\flagverse{15.} Was kränkst du dich in deinem Sinn\\
und grämst dich Tag und Nacht?\\
Nimm deine Sorg und wirf sie hin\\
auf den, der dich gemacht!

\flagverse{16.} Hat er dich nicht von Jugend auf\\
versorget und ernährt?\\
Wie manches schweren Unglücks Lauf\\
hat er zurückgekehrt!

\flagverse{17.} Er hat noch niemals was versehn\\
in seinem Regiment,\\
nein, was er tut und läßt geschehn,\\
das nimmt ein gutes End.

\flagverse{18.} Ei nun, so laß ihn ferner tun\\
und red ihm nicht darein,\\
so wirst du hier im Frieden ruhn\\
und ewig fröhlich sein.
    
\end{verse}
\end{multicols}
%\attrib{\small{THZE}}

\index{Ich singe dir mit Herz und Mund}
\newpage
\subsection*{\centerline{Ich will erhöhen immerfort}}
\addcontentsline{toc}{subsection}{Ich will erhöhen immerfort}
%StartInfo%%%%%%%%%%%%%%%%%%%%%%%%%%%%%%%%%%%%%%%%%%%%%%%%%%%%%%%%%%%%%%%%%%%%
%  Autor:
%  Titel:
%  File:
%  Ref:
%  Mod:
%EndInfo%%%%%%%%%%%%%%%%%%%%%%%%%%%%%%%%%%%%%%%%%%%%%%%%%%%%%%%%%%%%%%%%%%%%%%
%\poemtitle{pt}
\begin{multicols}{2}
\settowidth{\versewidth}{Gott ist ein Gott, der reichlich tröst't,}
\begin{verse}[\versewidth]
%der 34. Psalm\\

\flagverse{1.} Ich will erhöhen immerfort\\
und preisen meiner Seelen Hort,\\
ich will ihn herzlich ehren.\\
Wer Gott liebt, stimme mit mir ein,\\
laß alle, die betrübet sein,\\
ein Freudenliedlein hören.

\flagverse{2.} Gott ist ein Gott, der reichlich tröst't,\\
wer ihn nur sucht, der wird erlöst,\\
ich hab es selbst erfahren:\\
Sobald ein Ach im Himmel klingt,\\
kommt Heil und was uns Freude bringt\\
vom Himmel ab gefahren.

\flagverse{3.} Der starken Engel Kompanie\\
zieht fröhlich an, macht dort und hie\\
sich selbst zum Wall und Mauern,\\
da weicht und fleucht die böse Rott,\\
der Satan wird zu Hohn und Spott,\\
kein Unglück kann da dauern.

\flagverse{4.} Ach, was ist das für Süßigkeit!\\
Ach, schmecket alle, die ihr seid\\
mit Sinnen wohl begabet!\\
Kein Honig ist mehr auf der Erd\\
hinfort des süßen Namens wert;\\
gott ists, der uns recht labet.

\flagverse{5.} O seligs Herz, o seligs Haus,\\
das alle Lust stößt von sich aus\\
und diese Lust beliebet!\\
All andre Schönheit wird verrückt,\\
der aber bleibet stets geschmückt,\\
wer sich nur Gott ergibet.

\flagverse{6.} Der Kön'ge Gut, der Fürsten Geld\\
ist Kot und bleibet in der Welt,\\
wann die Besitzer sterben.\\
Wie oft verarmt ein reicher Mann!\\
Wer Gott vertraut, bleibt reich und kann\\
die ewgen Schätz ererben.

\flagverse{7.} Kommt her, ihr Kinder, hört mir zu!\\
Ich will euch zeigen, wie ihr Ruh\\
und Wohlfahrt könnt erjagen:\\
Ergebet euch und euren Sinn\\
zu Gottes Wohlgefallen hin\\
in allen euren Tagen!

\flagverse{8.} Bewahrt die Zung! Habt solchen Mut,\\
der Zank, und was zum Zanken tut,\\
nicht reget, sondern stillet:\\
So werden eure Tage sein\\
mit stillem Fried und süßem Schein\\
des Segens überfüllet.

\flagverse{9.} Laß ab vom Bösen, fleuch die Sünd,\\
o Mensch, und halt dich als ein Kind\\
des Vaters in der Höhe!\\
Du wirsts erfahren in der Tat,\\
wies dem, der ihm gefolget hat,\\
so herzlich wohl ergehe.

\flagverse{10.} Den Frommen ist Gott wieder fromm\\
und machet, daß geflossen komm\\
auf uns all sein Gedeihen;\\
sein Aug ist unser Sonnenlicht,\\
sein Ohr ist Tag und Nacht gericht,\\
zu hören unser Schreien.

\flagverse{11.} Zwar, wer Gott dient, muß leiden viel,\\
doch hat sein Leiden Maß und Ziel,\\
gott hilft ihm aus dem allen;\\
er sorgt für alle seine Bein,\\
er hebt sie auf und legt sie ein,\\
kein einzges muß verfallen.

\flagverse{12.} Gott sieht ins Herz und weiß gar wohl,\\
was uns macht Angst und Sorgen voll,\\
kein Tränlein fällt vergebens.\\
Er zählt sie all und legt darvor\\
uns treulich bei im Himmelschor\\
all Ehr des ewgen Lebens.

\end{verse}
\end{multicols}
%\attrib{\small{THZE}}

\index{Ich will erhöhen immerfort}
\newpage
\subsection*{\centerline{Ich will mit Danken kommen}}
\addcontentsline{toc}{subsection}{Ich will mit Danken kommen}
%StartInfo%%%%%%%%%%%%%%%%%%%%%%%%%%%%%%%%%%%%%%%%%%%%%%%%%%%%%%%%%%%%%%%%%%%%
%  Autor:
%  Titel:
%  File:
%  Ref:
%  Mod:
%EndInfo%%%%%%%%%%%%%%%%%%%%%%%%%%%%%%%%%%%%%%%%%%%%%%%%%%%%%%%%%%%%%%%%%%%%%%
%\poemtitle{pt}
\begin{multicols}{2}
\settowidth{\versewidth}{Groß ist der Herr und mächtig}
\begin{verse}[\versewidth]
%der 111. Psalm\\
%ich will mit Danken kommen

\flagverse{1.} Ich will mit Danken kommen\\
in den gemeinen Rat\\
der rechten wahren Frommen,\\
die Gottes Rat und Tat\\
mit süßem Lohn erhöhn;\\
zu denen will ich treten,\\
und soll mein Dank und Beten\\
von ganzem Herzen gehn.

\flagverse{2.} Groß ist der Herr und mächtig,\\
groß ist auch, was er macht.\\
Wer aufmerkt und andächtig\\
nimmt seine Werk in Acht,\\
hat eitel Lust daran.\\
Was seine Weisheit setzet\\
und ordnet, das ergötzet\\
und ist sehr wohl getan.

\flagverse{3.} Sein Heil und große Güte\\
steht fest und unbewegt,\\
damit auch dem Gemüte,\\
das uns im Herzen schlägt,\\
dieselbe nicht entweich,\\
hat er zum Glaubenszunder\\
ein Denkmal seiner Wunder\\
gestift't in seinem Reich.

\flagverse{4.} Gott ist voll Gnad und Gaben,\\
gibt Speis aus milder Hand,\\
die Seinen wohl zu laben,\\
die ihm allein bekannt;\\
denkt stets an seinen Bund,\\
gibt denen, die er weiden\\
will mit dem Erb der Heiden,\\
all seine Taten kund.

\flagverse{5.} Das Wirken seiner Hände\\
und was er uns gebeut,\\
das hat ein gutes Ende,\\
bringt reichen Trost und Freud\\
und Wahrheit, die nicht treugt.\\
Gott leitet seine Knechte\\
in dem rechtschaffnen Rechte,\\
das sich zum Leben neigt.

\flagverse{6.} Sein Herz läßt ihm nicht reuen,\\
was uns sein Mund verspricht,\\
gibt redlich und mit Treuen,\\
was unser Unglück bricht;\\
ist freudig, unverzagt,\\
uns alle zu erlösen\\
vom Kreuz und allem Bösen,\\
das seine Kinder plagt.

\flagverse{7.} Sein Wort ist wohl gegründet,\\
sein Mund ist rein und klar,\\
wozu er sich verbindet,\\
das macht er fest und wahr\\
und wird ihm gar nicht schwer.\\
Seine Name, den er führet,\\
ist heilig und gezieret\\
mit großer Pracht und Ehr.

\flagverse{8.} Die Furcht des Herren gibet\\
den ersten besten Grund\\
zur Weisheit, die Gott liebet\\
und rühmt mit seinem Mund.\\
O, wie klug ist der Sinn,\\
der diesen Weg verstehet\\
und fleißig darauf gehet!\\
Des Lob fällt nimmer hin.

\end{verse}
\end{multicols}
%\attrib{\small{THZE}}

\index{Ich will mit Danken kommmen}
\newpage
\subsection*{\centerline{Merkt auf, merkt, Himmel, Erde}}
\addcontentsline{toc}{subsection}{Merkt auf merkt Himmel Erde}
%StartInfo%%%%%%%%%%%%%%%%%%%%%%%%%%%%%%%%%%%%%%%%%%%%%%%%%%%%%%%%%%%%%%%%%%%%
%  Autor:
%  Titel:
%  File:
%  Ref:
%  Mod:
%EndInfo%%%%%%%%%%%%%%%%%%%%%%%%%%%%%%%%%%%%%%%%%%%%%%%%%%%%%%%%%%%%%%%%%%%%%%
%\poemtitle{pt}
\begin{multicols}{2}
\settowidth{\versewidth}{Merkt auf, merkt, Himmel, Erde,}
\begin{verse}[\versewidth]
%merkt auf, merkt, Himmel, Erde\\
%danklied Moses vor seinem Tode

\flagverse{1.} Merkt auf, merkt, Himmel, Erde,\\
und du, o Meeresgrund,\\
was ich jetzt singen werde\\
aus Gottes heilgem Mund!\\
Es fließe meine Lehre,\\
wie Tau und Regen fleußt;\\
wer Ohren hat, der höre\\
des Höchsten Wort und Geist.

\flagverse{2.} Es läßt der Herr euch weisen\\
sein Ehr und Namenszier;\\
die soll und will ich preisen,\\
das tut auch ihr mit mir.\\
Er ist ein Gott der Götter,\\
ein Tröster in der Not,\\
ein Fels, ein einzger Retter\\
und selbst des Todes Tod.

\flagverse{3.} Sein Tun ist lauter Güte,\\
sein Werk ist rein und klar,\\
treu ist er am Gemüte,\\
im Wort und Reden wahr;\\
viel heilger als die Engel,\\
die doch nur recht getan,\\
frei aller Fehl und Mängel,\\
fern von der Unrechtsbahn.

\flagverse{4.} Er ist gerecht. Wir alle\\
sind schändlich angesteckt\\
mit Adams Sünd und Falle,\\
der täglich in uns heckt\\
viel böse schwere Taten,\\
die unserm großen Gott,\\
des kein Mensch kann entraten,\\
geraten nur zum Spott.

\flagverse{5.} Die ungeratnen Kinder,\\
die fallen von ihm ab\\
und werden freche Sünder,\\
vergessen aller Gab\\
und so viel tausend Güter\\
und so viel süßer Gnad,\\
die ihnen Gott, ihr Hüter,\\
so oft erwiesen hat.

\flagverse{6.} Dankst du denn solchermaßen,\\
du toll und töricht Volk,\\
dem, der dir regnen lassen\\
dein Manna aus der Wolk\\
und aus des Himmels Kammer\\
dir Speisen zugeschickt,\\
damit in deinem Jammer\\
dein Herze würd erquickt?

\flagverse{7.} Woher hast du dein Leben\\
und deines Leibes Bild?\\
Wer hat das Blut gegeben,\\
das dir die Adern füllt?\\
Ists nicht dein Herr, dein Schöpfer,\\
dein Vater und dein Licht,\\
der dich, gleich als ein Töpfer,\\
von Erde zugericht?

\flagverse{8.} Gedenk und geh zurücke\\
in die vergangnen Jahr;\\
erwäge, was für Glücke\\
Gott deiner Väter Schar\\
erzeigt in schweren Zeiten!\\
Das ist den Alten kund,\\
die werden dir andeuten\\
den rechten wahren Grund.

\flagverse{9.} Er stieß die wilden Heiden\\
mit seiner starken Hand\\
aus ihrer fetten Weiden\\
und gab das schöne Land\\
des Israels Geschlechte\\
zu seines Namens Ruhm\\
und Jakob, seinem Knechte,\\
zum Erb und Eigentum.

\flagverse{10.} Er fand ihn, wo es heulet,\\
in dürrer Wüstenei,\\
er fand ihn und erteilet\\
ihm alle Vatertreu;\\
er lehret ihn, was tauge\\
und er selbst Tugend heiß,\\
er hielt ihn wie ein Auge\\
und sparte keinen Fleiß.

\flagverse{11.} Gleichwie ein Adler sitzet\\
auf seiner zarten Brut\\
und gar genau beschützet,\\
was ihm am Herzen ruht;\\
er dehnt die starken Flügel,\\
wenn er sich hoch erschwingt\\
und über Tal und Hügel\\
sein edle Jungen bringt:

\flagverse{12.} So hat sich auch gebreitet\\
des Höchsten Lieb und Gnad\\
auf Jakob, den er leitet,\\
auf daß ihm ja kein Schad\\
hier oder da anstieße;\\
er hub, er trug mit Fleiß,\\
bewahrt ihm Gang und Füße\\
auf seiner ganzen Reis.

\flagverse{13.} Er, sein Gott, tats alleine\\
und sonst kein andrer Gott;\\
es gaben Feld und Steine\\
Öl, Honig, Wasser, Brot\\
ohn alle seine Mühe;\\
er hatte guten Mut\\
beim Fett der Schaf und Kühe\\
und trank gut Traubenblut.

\flagverse{14.} Da er nun wohl gegessen,\\
vergaß er Gottes Heil,\\
und da er des vergessen,\\
da ward er frech und geil;\\
da seine Not gestillet,\\
beschimpft er Gottes Ehr,\\
und da der Leib gefüllet,\\
da ward das Herze leer.

\flagverse{15.} Leer ward es an dem Guten,\\
des Bösen ward es voll,\\
ließ Götzenopfer bluten\\
und dient, als wär er toll,\\
den schändlichen Feldteufeln;\\
und den, an dessen Macht\\
die Teufel selbst nicht zweifeln,\\
den ließ er aus der Acht.

\flagverse{16.} Er ließ den ewgen Retter\\
und gab sich in den Schirm\\
der neuerdachten Götter,\\
hielt Bestien und Gewürm\\
und Bilder von Metallen,\\
von Holz, von Stein und Ton,\\
den Heiden zu gefallen,\\
für seiner Seelen Kron.

\flagverse{17.} Als das nun der erkannte,\\
der Herz und Nieren kennt,\\
da wuchs sein Zorn und brannte,\\
gleichwie ein Feuer brennt;\\
und die er vor so schöne\\
geliebt an seinem Teil\\
als Töchter und als Söhne,\\
die wurden ihm ein Greul.

\flagverse{18.} Ich will mich, sprach er, wenden\\
von dieser schnöden Art,\\
die so abscheulich schänden\\
mich, der ich nichts gespart\\
an meiner Treu und Güte;\\
ich habe recht geliebt,\\
dafür wird mein Gemüte\\
gekränket und betrübt.

\flagverse{19.} Sie reizen mich mit Sünden:\\
Was gilts, es soll einmal\\
sich wieder etwas finden\\
zu ihrem Zorn und Qual!\\
Es werden Völker kommen,\\
die blind sind als ein Stein;\\
die sollen meine frommen\\
und liebsten Kinder sein.

\flagverse{20.} Mein Feuer ist entstanden\\
und brennet lichterloh\\
in meines Volkes Landen,\\
die sind ihm wie das Stroh.\\
Es wird weit um sich greifen\\
bis zu der Höllen Grund\\
und alle Frucht abstreifen,\\
die auf der Erden stund.

\flagverse{21.} Ich will mit meinen Pfeilen\\
sie treiben in den Tod;\\
es soll sie übereilen\\
Schwert, Pest und Hungersnot.\\
Ich will viel Tiere schicken\\
und strenges Schlangengift,\\
das soll zermartern, drücken\\
und fressen, wen es trifft.

\flagverse{22.} Ich will sie recht belohnen,\\
mein Zorn soll gleich ergehn,\\
auch derer nicht verschonen,\\
die jung, gerad und schön;\\
ich will sie all zerstäuben\\
und fragen hier und dort:\\
Wo ist dann nun ihr Bleiben?\\
Welch ist ihr Sitz und Ort?

\flagverse{23.} Doch muß ich gleichwohl scheuen\\
den ungereimten Wahn\\
der Feinde, die sich freuen,\\
als hätten sies getan.\\
Sie bleiben wie die Narren\\
bei ihrem Gaukelspiel\\
und ziehn am Torheitskarren,\\
ich tu auch, was ich will.

\flagverse{24.} O, daß mein Volk verstünde\\
das edle schöne Gut,\\
das, wenns nun seine Sünde\\
bereut und Buße tut,\\
ihm nachmals wird begegnen!\\
Denn was ich jetzt geflucht,\\
das will ich wieder segnen,\\
sobald es Gnade sucht.

\flagverse{25.} Mein Volk kommt aus dem Weinen,\\
sein Feind kommt aus der Ruh,\\
ihr tausend flieht vor einem,\\
wie geht das immer zu?\\
Ihr Herr, ihr Fels und Leben,\\
ist weg aus ihrem Zelt,\\
er hat sie übergeben\\
zur Flucht ins freie Feld.

\flagverse{26.} Seid froh, ihr treuen Knechte\\
des Gottes Israel,\\
des Arm und starke Rechte\\
euch schützt an Leib und Seel,\\
habt fröhliches Vertrauen\\
und Glauben, der da siegt:\\
So wird Gott wieder bauen,\\
was jetzt darnieder liegt.

\end{verse}
\end{multicols}
%\attrib{\small{THZE}}

\begin{center}
\settowidth{\versewidth}{Der, vor dem die Welt erschrickt,}
\begin{verse}[\versewidth]

\flagverse{27.} Er wird am Feinde rächen,\\
was uns zuviel geschehn,\\
uns wird er Trost zusprechen,\\
uns wieder lassen sehn\\
die Sonne seiner Gnaden:\\
Die wird in kurzer Zeit\\
des Landes Klag und Schaden\\
verkehrn in Glück und Freud.

\end{verse}
\end{center}


\index{Merkt auf, Himmel, Erde}
\newpage
\subsection*{\centerline{Nun danket all und bringet Ehr}}
\addcontentsline{toc}{subsection}{Nun danket all und bringet Ehr}
%StartInfo%%%%%%%%%%%%%%%%%%%%%%%%%%%%%%%%%%%%%%%%%%%%%%%%%%%%%%%%%%%%%%%%%%%%
%  Autor:
%  Titel:
%  File:
%  Ref:
%  Mod:
%EndInfo%%%%%%%%%%%%%%%%%%%%%%%%%%%%%%%%%%%%%%%%%%%%%%%%%%%%%%%%%%%%%%%%%%%%%%
%\poemtitle{pt}
\begin{multicols}{2}
\settowidth{\versewidth}{Ermuntert euch und singt mit Schall}
\begin{verse}[\versewidth]
%nun danket all und bringet Ehr\\
%(Sirach 50, 24)

\flagverse{1.} Nun danket all und bringet Ehr,\\
ihr Menschen in der Welt,\\
dem, dessen Lob der Engel Heer\\
im Himmel stets vermeldt.

\flagverse{2.} Ermuntert euch und singt mit Schall\\
gott, unserm höchsten Gut,\\
der seine Wunder überall\\
und große Dinge tut.

\flagverse{3.} Der uns von Mutterleibe an\\
frisch und gesund erhält\\
und, wo kein Mensch nicht helfen kann,\\
sich selbst zum Helfer stellt.

\flagverse{4.} Der, ob wir ihn gleich hoch betrübt,\\
doch bleibet gutes Muts,\\
die Straf erläßt, die Schuld vergibt\\
und tut uns alles Guts.

\flagverse{5.} Er gebe uns ein fröhlich Herz,\\
erfrische Geist und Sinn\\
und werf all Angst, Furcht, Sorg und Schmerz\\
ins Meeres Tiefe hin!

\flagverse{6.} Er lasse seinen Frieden ruhn\\
in Israelis Land,\\
er gebe Glück zu unserm Tun\\
und Heil zu allem Stand.

\flagverse{7.} Er lasse seine Lieb und Güt\\
um, bei und mit uns gehn,\\
was aber ängstet und bemüht,\\
gar ferne von uns stehn.

\flagverse{8.} So lange dieses Leben währt,\\
sei er stets unser Heil,\\
und wenn wir scheiden von der Erd,\\
verbleib er unser Teil.

\end{verse}
\end{multicols}
%\attrib{\small{THZE}}

\begin{center}
\settowidth{\versewidth}{Der, vor dem die Welt erschrickt,}
\begin{verse}[\versewidth]

\flagverse{9.} Er drücke, wenn das Herze bricht,\\
uns unsre Augen zu\\
und zeig uns drauf sein Angesicht\\
dort in der ewgen Ruh.

\end{verse}
\end{center}



\index{Nun danked all}
\newpage
\subsection*{\centerline{Nun geht frisch drauf, es geht nach Haus}}
\addcontentsline{toc}{subsection}{Nun geht frisch drauf es geht nach Haus}
%StartInfo%%%%%%%%%%%%%%%%%%%%%%%%%%%%%%%%%%%%%%%%%%%%%%%%%%%%%%%%%%%%%%%%%%%%
%  Autor:
%  Titel:
%  File:
%  Ref:
%  Mod:
%EndInfo%%%%%%%%%%%%%%%%%%%%%%%%%%%%%%%%%%%%%%%%%%%%%%%%%%%%%%%%%%%%%%%%%%%%%%
%\poemtitle{pt}
\begin{multicols}{2}
\settowidth{\versewidth}{Nun geht frisch drauf, es geht nach Haus,}
\begin{verse}[\versewidth]

\flagverse{1.} Nun geht frisch drauf, es geht nach Haus,\\
ihr Rößlein, regt die Bein;\\
ich will dem, der uns ein und aus\\
begleitet, dankbar sein.

\flagverse{2.} Ich will ihm singen Lob und Preis,\\
so viel ich singen kann,\\
ich will sein Werk, so gut ichs weiß,\\
mit Freuden zeigen an.

\flagverse{3.} Es ist fürwahr nicht Menschenkunst,\\
auf sichern Wegen gehn,\\
führt uns nicht Gott und Gottes Gunst,\\
würds oftmals seltsam stehn.

\flagverse{4.} Wie manches Leid, wie manche Not,\\
wie manches Jammerheer\\
brächt uns in Angst, tät uns den Tod,\\
wo Gott nicht bei uns wär.

\flagverse{5.} Wie mancher Feind, wie mancher Dieb,\\
wo ihn nicht Gott gerührt,\\
hätt uns das Unsre, das uns lieb,\\
genommen und entführt.

\flagverse{6.} Wie mancher böser schwarzer Geist\\
hätt unser Leib und Seel,\\
wo uns der Herr nicht Gnad erweist,\\
erschreckt aus seiner Höhl.

\flagverse{7.} Es ist der alte große Drach\\
doch allzeit ohne Ruh,\\
wohin wir gehn, da geht er nach\\
und setzt uns heftig zu.

\flagverse{8.} Er sucht zu Haus, er sucht zu Feld,\\
er sucht zur See und Land,\\
er sucht uns in der ganzen Welt\\
mit unverdroßner Hand.

\flagverse{9.} Noch dennoch trifft er uns nicht an,\\
sein Anschlag geht zurück,\\
denn Gottes Schutz hegt unsre Bahn\\
für unsres Feindes Tück.

\flagverse{10.} Es zeucht der heilgen Engel Schar,\\
mit Waffen ausgerüst,\\
und wehren fleißig hie und dar\\
des Tausendkünstlers List.

\flagverse{11.} Es müssen ja noch immerfort\\
die Mahanaim gehn\\
und Gottes Volk auf Gottes Wort\\
zu Dienst und Willen stehn.

\flagverse{12.} Wenn Gott mir meiner Augen Licht\\
mit Licht erfüllen wollt,\\
als wie dem Jakob, der sich nicht\\
für Esau fürchten sollt:

\flagverse{13.} Ach, was für Wunder würd ich hier\\
auf meinen Reisen sehn,\\
wie schön, wie lieblich würde mir\\
in solchem Sehn geschehn.

\flagverse{14.} Nun, was den Augen nicht vergunnt,\\
das sieht mein Herz und Geist,\\
dem Gott der heilgen Weisheit Grund\\
in seinem Geiste weist.

\flagverse{15.} Es ist sein Wort, er hats gesagt:\\
Sein Heervolk sei bereit,\\
uns zu umlagern, wenn uns plagt\\
des Satans Neid und Streit.

\flagverse{16.} Was Gott geredt, das ist vollbracht,\\
mein Herz, sei wohlgemut\\
und laß ja nimmer aus der Acht,\\
was dein Gott an dir tut.

\flagverse{17.} Du siehst und greifst, wie gut er sei\\
dem, der ihn ehrt und liebt,\\
er ziert mit Lieb, er führt mit Treu\\
ein Herz, das ihm sich gibt.

\flagverse{18.} Er trägt uns, wie (wenn einher schlägt\\
Blitz, Hagel, Sturm und Wind)\\
ein treuer frommer Vater trägt\\
sein kleines zartes Kind.

\flagverse{19.} Er deckt uns zu mit seiner Hand,\\
wie eine Mutter tut,\\
in derer Schoß das süßte Pfand\\
der keuschen Liebe ruht.

\flagverse{20.} Er räumt aus unsern Wegen weg\\
des Unglücks scharfen Stein\\
und schafft, daß unsre Bahn und Steg\\
fein schlicht und eben sein.

\flagverse{21.} Er führt uns über Berg und Tal,\\
und wenns nun rechte Zeit,\\
so führt er uns in seinen Saal\\
zur ewgen Himmelsfreud.

\flagverse{22.} Alsdann werd ich die letzte Reis\\
und schönste Heimfahrt tun\\
und nach dem sauren Erdenschweiß\\
in süßer Stille ruhn.

\end{verse}
\end{multicols}
\attrib{\small{THZE}}

\index{Nun geht frisch drauf}
\newpage
\subsection*{\centerline{Nun ist der Regen hin}}
\addcontentsline{toc}{subsection}{Nun ist der Regen hin}
%StartInfo%%%%%%%%%%%%%%%%%%%%%%%%%%%%%%%%%%%%%%%%%%%%%%%%%%%%%%%%%%%%%%%%%%%%
%  Autor:
%  Titel:
%  File:
%  Ref:
%  Mod:
%EndInfo%%%%%%%%%%%%%%%%%%%%%%%%%%%%%%%%%%%%%%%%%%%%%%%%%%%%%%%%%%%%%%%%%%%%%%
%\poemtitle{pt}
\begin{multicols}{2}
\settowidth{\versewidth}{Sein Zorn war sehr entbrannt}
\begin{verse}[\versewidth]
%nun ist der Regen hin\\
%danklied vor einem gnädigen Sonnenschein

\flagverse{1.} Nun ist der Regen hin;\\
wohlauf, mein Herz und Sinn,\\
sing nach betrübtem Leiden\\
Gott, deinem Herrn, mit Freuden!\\
Gott hat sein Herz gekehret\\
und unser Bitt erhöret.

\flagverse{2.} Sein Zorn war sehr entbrannt\\
auf uns und unser Land;\\
er sprach: Ihr Menschenkinder,\\
geht, seid und bleibet Sünder,\\
wollt von der Bosheit Straßen\\
euch gar nicht wenden lassen.

\flagverse{3.} Drum soll mein Himmelslicht\\
sein klares Angesicht\\
in schwarze trübe Decken\\
und dunkle Wolken stecken\\
und für das helle Scheinen\\
nur immer zu euch weinen.

\flagverse{4.} Bald aber fiel sein Grimm\\
durch unsers Seufzens Stimm;\\
das ewige Gemüte\\
dacht an sein ewge Güte\\
und ließ auf unser Schreien\\
ihm seinen Zorn gereuen.

\flagverse{5.} Die Wolken flohen weg,\\
der feuchten Winde Steg,\\
daher die Wasser flossen,\\
nahm ab und ward verschlossen;\\
des hohen Himmels Tiefen,\\
die hörten auf zu triefen.

\flagverse{6.} Steh auf, du mattes Feld,\\
aus deinem Trauerzelt,\\
steh auf und laß nun wieder\\
die süßen Sommerlieder\\
zu deines Schöpfers Ehren\\
mit Lust und Freuden hören.

\flagverse{7.} Sie hie, der Sonnen Zier\\
geht wieder schön herfür,\\
bringt nach dem Schlag und Regen\\
den lieben warmen Segen\\
und wirkt auf Berg und Talen\\
mit wunderlichen Strahlen.

\flagverse{8.} Die Erde wird erquickt,\\
und was durch Näß erstickt,\\
das wird nun wieder leben\\
und reife Früchte geben:\\
Die Äcker gut Getreide,\\
die Wiesen Gras und Weide.

\flagverse{9.} Die Bäume werden schön\\
in ihrer Fülle stehn,\\
die Berge werden fließen\\
und Wein und Öle gießen,\\
das Bienlein wird wohl tragen\\
bei guten warmen Tagen.

\flagverse{10.} Davon wird unser Teil\\
das ewge Gut und Heil\\
uns allensamt zumessen,\\
wir werdens sehn und essen\\
und mit dem Gut der Erden\\
zur Gnüg ersättigt werden.

\flagverse{11.} Nun, Gott ist fromm und treu,\\
sein Huld ist immer neu\\
und läßt sich leicht versühnen,\\
gibt, was wir nicht verdienen,\\
läßt gnädiglich sich finden\\
und nicht nach unsern Sünden.

\flagverse{12.} Darum so richte nun,\\
o Mensch, auch du dein Tun\\
zu Gottes Lob und Liebe,\\
daß dein Herz nicht betrübe\\
mit mehrem Zorn und Schmerze\\
das allerfrömmste Herze.

\end{verse}
\end{multicols}
%\attrib{\small{THZE}}

\index{Nun ist der Regen hin}
\newpage
\subsection*{\centerline{Sollt ich meinen Gott nicht singen?}}
\addcontentsline{toc}{subsection}{Sollt ich meinen Gott nicht singen?}
%StartInfo%%%%%%%%%%%%%%%%%%%%%%%%%%%%%%%%%%%%%%%%%%%%%%%%%%%%%%%%%%%%%%%%%%%%
%  Autor:
%  Titel:
%  File:
%  Ref:
%  Mod:
%EndInfo%%%%%%%%%%%%%%%%%%%%%%%%%%%%%%%%%%%%%%%%%%%%%%%%%%%%%%%%%%%%%%%%%%%%%%
%\poemtitle{pt}
\begin{multicols}{2}
\settowidth{\versewidth}{Sollt ich meinem Gott nicht singen}
\begin{verse}[\versewidth]

\flagverse{1.} Sollt ich meinem Gott nicht singen\\
sollt ich ihm nicht dankbar sein?\\
Denn ich seh in allen Dingen,\\
wie so gut ers mit mir mein.\\
Ist doch nichts als lauter Lieben,\\
das sein treues Herze regt,\\
das ohn Ende hebt und trägt,\\
die in seinem Dienst sich üben.\\
Alles Ding währt seine Zeit,\\
Gottes Lieb in Ewigkeit.

\flagverse{2.} Wie ein Adler sein Gefieder\\
über seine Jungen streckt,\\
also hat auch hin und wieder\\
mich des Höchsten Arm bedeckt,\\
alsobald im Mutterleibe,\\
da er mir mein Wesen gab\\
und das Leben, das ich hab\\
und noch diese Stunde treibe.\\
Alles Ding währt seine Zeit,\\
Gottes Lieb in Ewigkeit.

\flagverse{3.} Sein Sohn ist ihm nicht zu teuer,\\
nein, er gibt ihn für mich hin,\\
daß er mich vom ewgen Feuer\\
durch sein teures Blut gewinn.\\
O du ungegründter Brunnen,\\
wie will doch mein schwacher Geist,\\
ob er sich gleich hoch befleißt,\\
deine Tief ergründen können?\\
Alles Ding währt seine Zeit,\\
Gottes Lieb in Ewigkeit.

\flagverse{4.} Seinen Geist, den edlen Führer,\\
gibt er mir in seinem Wort,\\
daß er werde mein Regierer\\
durch die Welt zur Himmelspfort,\\
daß er mir mein Herz erfülle\\
mit dem hellen Glaubenslicht,\\
das des Todes Macht zerbricht\\
und die Hölle selbst macht stille.\\
Alles Ding währt seine Zeit,\\
Gottes Lieb in Ewigkeit.

\flagverse{5.} Meiner Seele Wohlergehen\\
hat er ja recht wohl bedacht;\\
will dem Leibe Not zustehen,\\
nimmt ers gleichfalls wohl in Acht.\\
Wann mein Können, mein Vermögen\\
nichts vermag, nichts helfen kann,\\
kommt mein Gott und hebt mir an,\\
sein Vermögen beizulegen.\\
Alles Ding währt seine Zeit,\\
Gottes Lieb in Ewigkeit.

\vfill\null
\columnbreak

\flagverse{6.} Himmel, Erd und ihre Heere\\
hat er mir zum Dienst bestellt;\\
wo ich nur mein Aug hinkehre,\\
find ich, was mich nährt und hält:\\
Tier und Kräuter und Getreide\\
in den Gründen, in der Höh,\\
in den Büschen, in der See,\\
überall ist meine Weide.\\
Alles Ding währt seine Zeit,\\
Gottes Lieb in Ewigkeit.

\flagverse{7.} Wenn ich schlafe, wacht sein Sorgen\\
und ermuntert mein Gemüt,\\
daß ich alle lieben Morgen\\
schaue neue Lieb und Güt.\\
Wäre mein Gott nicht gewesen,\\
hätte mich sein Angesicht\\
nicht geleitet, wär ich nicht\\
aus so mancher Angst genesen.\\
Alles Ding währt seine Zeit,\\
Gottes Lieb in Ewigkeit.

\flagverse{8.} Wie so manche schwere Plage\\
wird vom Satan umgeführt,\\
die mich doch mein Lebetage\\
niemals noch bisher gerührt.\\
Gottes Engel, den er sendet,\\
hat das Böse, was der Feind\\
anzurichten war gemeint,\\
in die Ferne weggewendet.\\
Alles Ding währt seine Zeit,\\
Gottes Lieb in Ewigkeit.

\flagverse{9.} Wie ein Vater seinem Kinde\\
sein Herz niemals ganz entzeucht,\\
ob es gleich bisweilen Sünde\\
tut und aus der Bahne weicht:\\
Also hält auch mein Verbrechen\\
mir mein frommer Gott zu gut,\\
will mein Fehlen mit der Rut\\
und nicht mit dem Schwerte rächen.\\
Alles Ding währt seine Zeit,\\
Gottes Lieb in Ewigkeit.

\flagverse{10.} Seine Strafen, seine Schläge,\\
ob sie mir gleich bitter seind,\\
dennoch, wenn ichs recht erwäge,\\
sind es Zeichen, daß mein Freund,\\
der mich liebet, mein gedenke\\
und mich von der schnöden Welt,\\
die uns hart gefangen hält,\\
durch das Kreuze zu ihm lenke.\\
Alles Ding währt seine Zeit,\\
Gottes Lieb in Ewigkeit.

\vfill\null
\columnbreak

\flagverse{11.} Das weiß ich fürwahr und lasse\\
mirs nicht aus dem Sinne gehn:\\
Christenkreuz hat seine Maße\\
und muß endlich stille stehn;\\
wenn der Winter ausgeschneiet,\\
tritt der schöne Sommer ein:\\
Also wird auch nach der Pein,\\
wers erwarten kann, erfreuet.\\
Alles Ding währt seine Zeit,\\
Gottes Lieb in Ewigkeit.

\flagverse{12.} Weil dann weder Ziel noch Ende\\
sich in Gottes Liebe findt,\\
ei, so heb ich meine Hände\\
zu dir, Vater, als dein Kind;\\
bitte, wollst mir Gnade geben,\\
dich aus aller meiner Macht\\
zu umfangen Tag und Nacht\\
hier in meinem ganzen Leben,\\
bis ich dich nach dieser Zeit\\
lob und lieb in Ewigkeit.
   
\end{verse}
\end{multicols}
%\attrib{\small{THZE}}

\index{Sollt ich meinem Gott nicht singen}
\newpage
\subsection*{\centerline{Unter allen, die da leben}}                    %witt:?  --> Lob und Dank
\addcontentsline{toc}{subsection}{Unter allen die da leben}
%StartInfo%%%%%%%%%%%%%%%%%%%%%%%%%%%%%%%%%%%%%%%%%%%%%%%%%%%%%%%%%%%%%%%%%%%%
%  Autor:
%  Titel:
%  File:
%  Ref:
%  Mod:
%EndInfo%%%%%%%%%%%%%%%%%%%%%%%%%%%%%%%%%%%%%%%%%%%%%%%%%%%%%%%%%%%%%%%%%%%%%%
%\poemtitle{pt}
%\begin{multicols}{2}
\begin{center}
\settowidth{\versewidth}{Unter allen, die da singen}
\begin{verse}[\versewidth]
%an Joachim Pauli\\
%unter allen, die da leben, hat ein jeder seinen Fleiß\\
%(Am Schluß der »Vier geistlichen Lieder« von Joachim Pauli, Berlin o.J. – wahrscheinlich 1665)

\flagverse{1.} Unter allen, die da leben,\\
hat ein jeder seinen Fleiß\\
und weiß dessen Frucht zu geben;\\
doch hat der den größten Preis,\\
der dem Höchsten Ehre bringt\\
und von Gottes Namen singt.

\flagverse{2.} Unter allen, die da singen\\
und mit wohlgefaßter Kunst\\
ihrem Schöpfer Opfer bringen,\\
hat ein jeder seine Gunst;\\
doch ist der am besten dran,\\
der mit Andacht singen kann.

\end{verse}
\end{center}
%\end{multicols}
%\attrib{\small{THZE}}

\index{Unter allen, die da leben}
\newpage
\subsection*{\centerline{Wer wohlauf ist und gesund}}
\addcontentsline{toc}{subsection}{Wer wohlauf ist und gesund}
%StartInfo%%%%%%%%%%%%%%%%%%%%%%%%%%%%%%%%%%%%%%%%%%%%%%%%%%%%%%%%%%%%%%%%%%%%
%  Autor:
%  Titel:
%  File:
%  Ref:
%  Mod:
%EndInfo%%%%%%%%%%%%%%%%%%%%%%%%%%%%%%%%%%%%%%%%%%%%%%%%%%%%%%%%%%%%%%%%%%%%%%
%\poemtitle{pt}
\begin{multicols}{2}
\settowidth{\versewidth}{Wär ich gleich wie Krösus reich,}
\begin{verse}[\versewidth]
%wer wohlauf ist und gesund\\
%danklied für Gesundheit des Leibes

\flagverse{1.} Wer wohlauf ist und gesund,\\
hebe sein Gemüte\\
und erhöhe seinen Mund\\
zu des Höchsten Güte.\\
Laßt uns danken Tag und Nacht\\
mit gesunden Liedern\\
unserm Gott, der uns bedacht\\
mit gesunden Gliedern.

\flagverse{2.} Ein gesundes frisches Blut\\
hat ein fröhlich Leben;\\
gibt uns Gott dies einzge Gut,\\
ist uns gnug gegeben\\
hier in dieser armen Welt,\\
da die schönsten Gaben\\
und des güldnen Himmels Zelt\\
wir noch künftig haben.

\flagverse{3.} Wär ich gleich wie Krösus reich,\\
hätte Barschaft liegen,\\
wär ich Alexandern gleich\\
an Triumph und Siegen;\\
müßte gleichwohl siech und schwach\\
Pfühl und Betten drücken:\\
Würd auch mich im Ungemach\\
all mein Gut erquicken?

\flagverse{4.} Stünde gleich mein ganzer Tisch\\
voller Lust und Freude,\\
hätt ich Wildbret, Wein und Fisch\\
und die ganze Weide,\\
die den Hals und Schmack ergötzt:\\
Wozu würd es nützen,\\
wenn ich dennoch ausgesetzt\\
müßt in Schmerzen sitzen?

\flagverse{5.} Hätt ich aller Ehren Pracht,\\
säß im höchsten Stande,\\
wär ich mächtig aller Macht\\
und ein Herr im Lande;\\
mein Leib aber hätte doch\\
auf- und angenommen\\
der betrübten Krankheit Joch:\\
Was hätt ich für Frommen?

\flagverse{6.} Ich erwähl ein Stücklein Brot,\\
das mir wohl gedeihet,\\
vor des roten Goldes Kot,\\
da man Ach bei schreiet;\\
schmeckt mir Speis und Mahlzeit wohl\\
und darf mein nicht schonen,\\
halt ich ein Gerichtlein Kohl\\
höher als Melonen.

\vfill\null
\columnbreak

\flagverse{7.} Samt und Purpur hilft mir nicht\\
mein Elende tragen,\\
wenn mich Hauptweh, Stein und Gicht\\
und die Schwindsucht plagen.\\
Lieber will ich fröhlich gehn\\
im geringen Kleide,\\
als mit Leid und Ängsten stehn\\
in der schönsten Seide.

\flagverse{8.} Sollt ich stumm und sprachlos sein\\
oder lahm an Füßen,\\
sollt ich nicht des Tages Schein\\
sehen und genießen;\\
sollt ich gehen spat und früh\\
mit verschlossnen Ohren:\\
Würd ich wünschen, daß ich nie\\
wär ein Mensch geboren.

\flagverse{9.} Lebt ich ohne Rat und Witz,\\
wär im Haupt verirret,\\
hätte meiner Seelen Sitz,\\
mein Herz, sich verwirret;\\
wäre mir mein Mut und Sinn\\
niemals guter Dinge:\\
Wär es besser, daß ich hin,\\
wo ich her bin, ginge.

\flagverse{10.} Aber nun gebricht mir nichts\\
an erzählten Stücken,\\
ich erfreue mich des Lichts\\
und der Sonnen Blicken,\\
mein Gesichte sieht sich üm,\\
mein Gehöre höret,\\
wie der Vöglein süße Stimm\\
ihren Schöpfer ehret.

\flagverse{11.} Händ und Füße, Herz und Geist\\
sind bei guten Kräften,\\
alle mein Vermögen fleußt\\
und geht in Geschäften,\\
die mein Herrscher hat gestellt\\
hier in meinem Bleiben,\\
alsolang es ihm gefällt,\\
in der Welt zu treiben.

\flagverse{12.} Ist es Tag, so mach und tu\\
ich, was mir gebühret,\\
kommt die Nacht und süße Ruh,\\
die zum Schlafen führet,\\
schlaf und ruh ich unbewegt,\\
bis die Sonne wieder\\
mit den hellen Strahlen regt\\
meine Augenlider.

\vfill\null
\columnbreak

\flagverse{13.} Habe Dank, du milde Hand,\\
die du aus dem Throne\\
deines Himmels mir gesandt\\
diese schöne Krone\\
deiner Gnad und großen Huld,\\
die ich all mein Tage\\
niemals hab um dich verschuldt\\
und doch an mir trage.

\flagverse{14.} Gib, so lang ich bei mir hab\\
ein lebendges Hauchen,\\
daß ich solche teure Gab\\
auch wohl möge brauchen;\\
hilf, daß mein gesunder Mund\\
und erfreute Sinnen\\
dir zu aller Zeit und Stund\\
alles Liebs beginnen!

\end{verse}
\end{multicols}

\begin{verbatim}


\end{verbatim}

\begin{center}
\settowidth{\versewidth}{Der, vor dem die Welt erschrickt,}
\begin{verse}[\versewidth]



\flagverse{15.} Halte mich bei Stärk und Kraft,\\
wenn ich nun alt werde,\\
bis mein Stündlein hin mich rafft\\
in das Grab und Erde;\\
gib mir meine Lebenszeit\\
ohne sonderm Leide,\\
und dort in der Ewigkeit\\
die vollkommne Freude!

  
\end{verse}
\end{center}


%\attrib{\small{THZE}}

\index{Wer wohlauf ist und gesund}
\newpage
\subsection*{\centerline{Wie ist es möglich, höchstes Licht?}}
\addcontentsline{toc}{subsection}{Wie ist es möglich höchstes Licht?}
%StartInfo%%%%%%%%%%%%%%%%%%%%%%%%%%%%%%%%%%%%%%%%%%%%%%%%%%%%%%%%%%%%%%%%%%%%
%  Autor:
%  Titel:
%  File:
%  Ref:
%  Mod:
%EndInfo%%%%%%%%%%%%%%%%%%%%%%%%%%%%%%%%%%%%%%%%%%%%%%%%%%%%%%%%%%%%%%%%%%%%%%
%\poemtitle{pt}
\begin{multicols}{2}
\settowidth{\versewidth}{Herr, ich bin nichts! Du aber bist}
\begin{verse}[\versewidth]
%wie ist es möglich, höchstes Licht?

\flagverse{1.} Wie ist es möglich, höchstes Licht,\\
daß, weil vor deinem Angesicht\\
doch alles muß erblassen,\\
ich und mein armes Fleisch und Blut\\
dir zu entgegen eingen Mut\\
und Herze sollten fassen?

\flagverse{2.} Was bin ich mehr als Erd und Staub?\\
Was ist mein Leib als Gras und Laub?\\
Was taugt mein ganzes Leben?\\
Was kann ich, wenn ich alles kann?\\
Was hab und trag ich um und an,\\
als was du mir gegeben?

\flagverse{3.} Ich bin ein arme Mad und Wurm,\\
ein Strohhalm, den ein kleiner Sturm\\
gar leichtlich hin kann treiben;\\
wenn deine Hand, die alles trägt,\\
mich nur ein wenig trifft und schlägt,\\
so weiß ich nicht zu bleiben.

\flagverse{4.} Herr, ich bin nichts! Du aber bist\\
der Mann, der alles hat und ist,\\
in dir steht all mein Wesen;\\
wo du mit deiner Hand mich schreckst,\\
und nicht mit Huld und Gnaden deckst,\\
so mag ich nicht genesen.

\flagverse{5.} Du bist getreu, ich ungerecht,\\
du fromm, ich gar ein böser Knecht\\
und muß mich wahrlich schämen,\\
daß ich bei solchem schnöden Stand\\
aus deiner milden Vaterhand\\
ein einzges Gut sollt nehmen.

\flagverse{6.} Ich habe dir von Jugend an\\
nichts andres als Verdruß getan,\\
bin Sünden voll geboren;\\
und wo du nicht durch deine Treu\\
mich wieder machest los und frei,\\
so wär ich gar verloren.

\flagverse{7.} Drum sei das Rühmen fern von mir,\\
was dir gebührt, das geb ich dir,\\
du bist allein zu ehren.\\
Ach laß, Herr Jesu, meinen Geist\\
und was aus meinem Geiste fleußt,\\
zu dir sich allzeit kehren!

\flagverse{8.} Auch wenn ich gleich was wohl gemacht,\\
so hab ichs doch nicht selbst verbracht,\\
aus dir ist es entsprungen;\\
dir sei auch dafür Ehr und Dank,\\
mein Heiland, all mein Leben lang\\
und Lob und Preis gesungen.

\end{verse}
\end{multicols}
%\attrib{\small{THZE}}

\index{Wie ist es möglich, höchstes Licht}
\newpage

\newpage

\section*{\centerline{\LARGE LEBEN UND ARBEIT}}
\addcontentsline{toc}{section}{LEBEN UND ARBEIT}
\rule{\textwidth}{0.2pt}\vspace*{-\baselineskip}\vspace{3.2pt}
\rule{\textwidth}{1.2pt}\\[\baselineskip]

\index{ Gedichte über Leben und Arbeit}

\centerline{\scshape Der aller Herz und Willen lenkt}
\vspace*{2\baselineskip}
\centerline{\scshape Der Herr, der aller Enden}
\vspace*{2\baselineskip}
\centerline{\scshape Ein Weib, das Gott den Herren liebt}
\vspace*{2\baselineskip}
\centerline{\scshape Hört an, ihr Völker, hört doch an}
\vspace*{2\baselineskip}  
\centerline{\scshape Ich hab oft bei mir selbst gedacht}
\vspace*{2\baselineskip}  
\centerline{\scshape Ich weiß, mein Gott, daß all mein Tun}
\vspace*{2\baselineskip}
\centerline{\scshape Tapfre Leute soll man loben}
\vspace*{2\baselineskip}  
\centerline{\scshape Voller Wunder, voller Kunst}
\vspace*{2\baselineskip}
\centerline{\scshape Weltskribenten und Poeten}                          %witt:? --> Leben
\vspace*{2\baselineskip}
\centerline{\scshape Wie schön ists doch, Herr Jesu Christ}
\vspace*{2\baselineskip}
\centerline{\scshape Wohl dem, der den Herren scheuet}
\vspace*{2\baselineskip}
\centerline{\scshape Wohl dem Menschen, der nicht wandelt}
\vspace*{2\baselineskip}
\centerline{\scshape Zweierlei bitt ich von dir}

\newpage

\subsection*{\centerline{Der aller Herz und Willen lenkt}}            %witt:? -- Hochzeit --> Leben
\addcontentsline{toc}{subsection}{Der aller Herz und Willen lenkt}            %witt:? -- Hochzeit --> Leben}
%StartInfo%%%%%%%%%%%%%%%%%%%%%%%%%%%%%%%%%%%%%%%%%%%%%%%%%%%%%%%%%%%%%%%%%%%%
%  Autor:
%  Titel:
%  File:
%  Ref:
%  Mod:
%EndInfo%%%%%%%%%%%%%%%%%%%%%%%%%%%%%%%%%%%%%%%%%%%%%%%%%%%%%%%%%%%%%%%%%%%%%%
%\poemtitle{Der aller Herz und Willen lenkt}
\begin{multicols}{2}
\settowidth{\versewidth}{Wie Gott will, brennen auf der Erd}
\begin{verse}[\versewidth]
% Der aller Herz und Willen lenkt
% Hochzeitsgedicht für Joachim Fromm und Sabina Barthold 1643

\flagverse{1.} Der aller Herz und Willen lenkt\\
und wie er will regieret,\\
der ist's, der euch, Herr Bräutgam, schenkt\\
die man euch hier zuführet.\\
Glück zu, Glück zu, ruft jedermann,\\
Gott gebe, daß es sei getan\\
zu beider Wohlergehen!

\flagverse{2.} Wie sollte nicht sein wohlgetan,\\
was Gott denkt zu vollbringen?\\
Sein Will und Rat nicht fehlen kann,\\
es wird ihm nichts mißlingen.\\
Er regt den Mund und spricht ein Wort,\\
so geht das Werk und dringet fort,\\
muß alles wohl geraten.

\flagverse{3.} Wie Gott will, brennen auf der Erd\\
die ehelichen Flammen.\\
Wie eins dem andern ist beschert,\\
so kommen sie zusammen.\\
Im Himmel wird der Schluß gemacht,\\
auf Erden wird das Werk vollbracht:\\
Das gibt ein schönes Leben.

\flagverse{4.} Ein Leben, daß sehr hoch beliebt\\
dem, der es hat erfunden,\\
da er auch seinen Segen gibt\\
und mehret alle Stunden.\\
Das ist und bleibet sein Gebrauch:\\
Was er gestift't, das hält er auch\\
und lässet es nicht fallen.

\flagverse{5.} Die Bäumlein, die man fortgesetzt\\
in wohlbestallten Garten,\\
die pfleget man zur Erst und Letzt\\
vor allem wohl zu warten.\\
Ihr Bäumlein Gottes, freuet euch!\\
Der Gärtner ist von Liebe reich,\\
der ihm euch hat erwählet.

\flagverse{6.} Was er gepflanzt mit seiner Hand,\\
hält er in großen Ehren;\\
sein Sinn und Aug ist stets gewandt,\\
dasselbe zu vermehren,\\
kommt oft und sieht aus reiner Treu,\\
was seines Gartens Zustand sei,\\
was seine Reislein machen.

\flagverse{7.} Und wenn denn unterweilen will\\
ein rauhes Lüftlein wehen,\\
ist er bald da, setzt Maß und Ziel,\\
läßts eilend übergehen.\\
Wenn er betrübt, ists gut gemeint,\\
er stellt sich hart und ist doch Freund\\
voll süßer Gnad und Hulde.

\flagverse{8.} O selig der, wenns Gott gefällt,\\
ein Wölklein einzuführen,\\
ein treues, fröhlich Herz behält,\\
läßt keinen Unmut spüren!\\
Ein Wölklein geht ja bald vorbei,\\
es währt ein Stündlein oder zwei,\\
so kommt die Sonne wieder.

\flagverse{9.} Ein Schifflein, das im Meere läuft,\\
muß manchen Sturm erfahren,\\
und bleibet dennoch überhäuft\\
mit edlem Gut und Waren;\\
es streicht dahin, und Gottes Hand,\\
die führt und bringt es an das Land\\
bei gutem Wind und Wetter.

\flagverse{10.} Ein Röslein, wenns im Lenzen lacht\\
und in den Farben pranget,\\
wird oft vom Regen matt gemacht,\\
daß es sein Köpflein hanget,\\
doch wenn die Sonne leucht herfür,\\
siehts wieder auf und bleibt die Zier\\
und Fürstin aller Blumen.

\flagverse{11.} Wohlan, laß Regen, Reif und Wind\\
bald oder lang ansetzen,\\
wer Gott liebt, bleibet Gottes Kind,\\
kein Fall wird ihn verletzen.\\
Er sitzet in des Vaters Arm,\\
er gibt ihm Schutz, der hält ihn warm,\\
und spricht: Sei unerschrocken!

\flagverse{12.} Wer fromm ist, hat schon großen Teil\\
der Wohlfahrt in den Händen,\\
Gott gönnt ihm Guts und kann sein Heil\\
von ihme nicht abwenden.\\
Herr Fromm ist fromm, das weiß man wohl,\\
drum er nichts anders haben soll\\
als lauter Glück und Freude.

\flagverse{13.} Die auch, die ihm zur Seiten geht\\
und die Gott selbst gezieret,\\
was Menschenseelen wohl ansteht\\
und Himmelsgunst gebieret;\\
was Tugend bringt, was Tugend heißt,\\
was Tugend auch selbst lobt und preist,\\
das findt sich hier beisammen:

\flagverse{14.} Ein züchtig Herz, ein reiner Mut,\\
von denen angeboren,\\
die ihnen Gottesfurcht zum Gut\\
und Schätzen auserkoren.\\
Was ist doch gut ohn diesem Gut?\\
Wenn dies Gut nicht im Herzen ruht,\\
ist alles Gut verworfen.

\flagverse{15.} Die Augen Gottes sehen bald,\\
die ihm sein Herz erfreuen,\\
wen er nun findet recht gestalt,\\
dem gibt er sein Gedeihen,\\
ja schütts mit vollen Händen aus,\\
da wird denn ein gesegntes Haus,\\
dems nicht kann übel gehen.

\flagverse{16.} Und dieses wird o edles Paar,\\
euch beiden auch geschehen!\\
Was Gott Verspricht, ist ja und wahr,\\
man wirds mit Augen sehen.\\
Es fehlt ihm nicht an Gütigkeit,\\
auch fehlts ihm nicht an Möglichkeit,\\
wie sollt er Guts versagen?

\flagverse{17.} So gehet nun mit Freuden ein\\
zu eurem Stand und Orden!\\
Der Weg Wird ohne Schaden sein,\\
der euch gezeuget worden:\\
Es geht ein Englein vornen an,\\
und wo es geht, bestreuts die Bahn\\
mit Rosen und Violen.

\flagverse{18.} Ein einzig Wunsch vermag den Saal\\
des Himmels durch zu dringen,\\
hier gehn die Wünsch in voller Zahl,\\
sie werden Gutes bringen:\\
Der Frommen Lohn, der euch bereit,\\
euch, die ihr tragt die Frömmigkeit\\
im Herzen und im Namen.

\end{verse}
\end{multicols}
%\attrib{\small{THZE}}

\index{Der aller Herz und Willen lenkt}
\newpage
\subsection*{\centerline{Der Herr, der aller Enden}}
\addcontentsline{toc}{subsection}{Der Herr der aller Enden}
%StartInfo%%%%%%%%%%%%%%%%%%%%%%%%%%%%%%%%%%%%%%%%%%%%%%%%%%%%%%%%%%%%%%%%%%%%
%  Autor:
%  Titel:
%  File:
%  Ref:
%  Mod:
%EndInfo%%%%%%%%%%%%%%%%%%%%%%%%%%%%%%%%%%%%%%%%%%%%%%%%%%%%%%%%%%%%%%%%%%%%%%
%\poemtitle{Der Herr, der aller Enden regiert, der ist mein Hirt und Hüter}
\begin{multicols}{2}
\settowidth{\versewidth}{Der Herr, der aller Enden}
\begin{verse}[\versewidth]
%Der 23. Psalm\\
%Der Herr, der aller Enden regiert, der ist mein Hirt und Hüter

\flagverse{1.} Der Herr, der aller Enden\\
regiert mit seinen Händen,\\
der Brunn der ewgen Güter,\\
der ist mein Hirt und Hüter.

\flagverse{2.} So lang ich diesen habe,\\
fehlt mirs an keiner Gabe,\\
der Reichtum seiner Fülle\\
gibt mir die Füll und Hülle.

\flagverse{3.} Er lässet mich mit Freuden\\
auf grüner Aue weiden,\\
führt mich zu frischen Quellen,\\
schafft Rat in schweren Fällen.

\flagverse{4.} Wann meine Seele zaget\\
und sich mit Sorgen plaget,\\
weiß er sie zu erquicken,\\
aus aller Not zu rücken.

\flagverse{5.} Er lehrt mich tun und lassen,\\
führt mich auf rechter Straßen,\\
läßt Furcht und Angst sich stillen\\
um seines Namens willen.

\flagverse{6.} Und ob ich gleich vor andern\\
im finstern Tal muß wandern,\\
fürcht ich doch keine Tücke,\\
bin frei vom Ungelücke.

\flagverse{7.} Denn du stehst mir zur Seiten,\\
schützst mich vor bösen Leuten,\\
dein Stab, Herr, und dein Stecken\\
benimmt mir all mein Schrecken.

\flagverse{8.} Du setzest mich zu Tische,\\
machst, daß ich mich erfrische,\\
wann mir mein Feind viel Schmerzen\\
erweckt in meinem Herzen.

\flagverse{9.} Du salbst mein Haupt mit Öle\\
und füllest meine Seele,\\
die leer und durstig saße,\\
mit vollgeschenktem Maße.

\flagverse{10.} Barmherzigkeit und Gutes\\
wird mein Herz gutes Mutes,\\
voll Lust, voll Freud, voll Lachen,\\
so lang ich lebe, machen.

\flagverse{11.} Ich will dein Diener bleiben\\
und dein Lob herrlich treiben\\
im Hause, da du wohnest\\
und Frommsein wohl belohnest.

\flagverse{12.} Ich will dich hier auf Erden\\
und dort, da wir dich werden\\
selbst schaun, im Himmel droben\\
hoch rühmen, singn und loben.

\end{verse}
\end{multicols}
%\attrib{\small{THZE}}

\index{Der Herr, der aller Enden}
\newpage
\subsection*{\centerline{Ein Weib, das Gott den Herren liebt}}
\addcontentsline{toc}{subsection}{Ein Weib das Gott den Herren liebt}
%StartInfo%%%%%%%%%%%%%%%%%%%%%%%%%%%%%%%%%%%%%%%%%%%%%%%%%%%%%%%%%%%%%%%%%%%%
%  Autor:
%  Titel:
%  File:
%  Ref:
%  Mod:
%EndInfo%%%%%%%%%%%%%%%%%%%%%%%%%%%%%%%%%%%%%%%%%%%%%%%%%%%%%%%%%%%%%%%%%%%%%%
%\poemtitle{Ein Weib, das Gott den Herren liebt}
\begin{multicols}{2}
\settowidth{\versewidth}{Sie ist ein Schifflein auf dem Meer,}
\begin{verse}[\versewidth]
%frauenlob\\
%ein Weib, das Gott den Herren liebt\\
%(Spr. Sal. 31, 10-30)

\flagverse{1.} Ein Weib, das Gott den Herren liebt\\
und sich stets in der Tugend übt,\\
ist viel mehr Lobs und Liebens wert\\
als alle Perlen auf der Erd.

\flagverse{2.} Ihr Mann darf mit dem Herzen frei\\
verlassen sich auf ihre Treu,\\
sein Haus ist voller Freud und Licht,\\
an Nahrung wirds ihm mangeln nicht.

\flagverse{3.} Sie tut ihm Liebes und kein Leid,\\
durchsüßet seine Lebenszeit,\\
sie nimmt sich seines Kummers an\\
mit Trost und Rat, so gut sie kann.

\flagverse{4.} Die Woll und Flachs sind ihre Lust,\\
was hierzu dien, ist ihr bewußt,\\
ihr Händlein greifet selber zu,\\
hat oftmals Müh und selten Ruh.

\flagverse{5.} Sie ist ein Schifflein auf dem Meer,\\
wann dieses kommt, so kommts nicht leer:\\
So schafft auch sie aus allem Ort\\
und setzet ihre Nahrung fort.

\flagverse{6.} Sie schläft mit Sorg, ist früh heraus,\\
gibt Futter, wo sie soll, im Haus\\
und speist die Dirnen, derer Hand\\
zu ihren Diensten ist gewandt.

\flagverse{7.} Sie gürtet ihre Lenden fest\\
und stärket ihre Arm aufs best,\\
ist froh, wenns wohl von statten geht,\\
worauf ihr Sinn und Herze steht.

\flagverse{8.} Wenn andre löschen Feur und Licht,\\
verlöscht doch ihre Leuchte nicht,\\
ihr Herze wachet Tag und Nacht\\
zu dem, der Tag und Nacht gemacht.

\flagverse{9.} Sie nimmt den Wocken, setzt sich hin\\
und schämt sich nicht, daß sie ihn spinn,\\
ihr Finger faßt die Spindel wohl\\
und macht sie schnell mit Garne voll.

\flagverse{10.} Sie hört gar leicht der Armen Bitt,\\
ist gütig, teilet gerne mit,\\
ihr Haus und alles Hausgesind\\
ist wohl verwahrt vor Schnee und Wind.

\flagverse{11.} Sie näht, sie sitzt, sie wirkt mit Fleiß,\\
macht Decken nach der Künstler Weis,\\
hält sich selbst sauber; weiße Seid\\
und Purpur ist ihr schönes Kleid.

\flagverse{12.} Ihr Mann ist in der Stadt berühmt,\\
bestellt sein Amt, wie sichs geziemt,\\
er geht, steht und sitzt obenan,\\
und was er tut, ist wohlgetan.

\flagverse{13.} Ihr Schmuck ist, daß sie reinlich ist,\\
ihr Ehr ist, daß sie ausgerüst\\
mit Fleiße, der gewiß zuletzt\\
den, der ihn liebet, hoch ergötzt.

\flagverse{14.} Sie öffnet ihren weisen Mund,\\
tut Kindern und Gesinde kund\\
des Höchsten Wort und lehrt sie fein\\
fromm, ehrbar und gehorsam sein.

\flagverse{15.} Sie schauet, wies im Hause steht\\
und wie es hier und dort ergeht,\\
sie ißt ihr Brot und sagt dabei,\\
wie so groß Unrecht Faulsein sei.

\flagverse{16.} Die Söhne, die ihr Gott beschert,\\
die halten sie hoch, lieb und wert,\\
ihr Mann, der lobt sie spat und früh\\
und preiset selig sich und sie.

\flagverse{17.} Viel Töchter bringen Geld und Gut,\\
sind zart am Leib und stolz am Mut,\\
du aber, meine Kron und Zier,\\
gehst wahrlich ihnen allen für.

\flagverse{18.} Was hilft der äußerliche Schein?\\
Was ists doch, schön und lieblich sein?\\
Ein Weib, Das Gott liebt, ehrt und scheut,\\
das soll man loben weit und breit.

\end{verse}
\end{multicols}

\begin{center}
\settowidth{\versewidth}{Der, vor dem die Welt erschrickt,}
\begin{verse}[\versewidth]

\flagverse{19.} Die Werke, die sie hier verricht,\\
sind wie ein schönes helles Licht,\\
sie dringen bis zur Himmelspfort\\
und werden leuchten hier und dort.
  
\end{verse}
\end{center}



%\attrib{\small{THZE}}

\index{Ein Weib, das Gott den Herren liebt}
\newpage
\subsection*{\centerline{Hört an, ihr Völker, hört doch an}}
\addcontentsline{toc}{subsection}{Hört an ihr Völker hört doch an}
%StartInfo%%%%%%%%%%%%%%%%%%%%%%%%%%%%%%%%%%%%%%%%%%%%%%%%%%%%%%%%%%%%%%%%%%%%
%  Autor:
%  Titel:
%  File:
%  Ref:
%  Mod:
%EndInfo%%%%%%%%%%%%%%%%%%%%%%%%%%%%%%%%%%%%%%%%%%%%%%%%%%%%%%%%%%%%%%%%%%%%%%
%\poemtitle{pt}
\begin{multicols}{2}
\settowidth{\versewidth}{Was hilft ihm all sein Hab und Gut,}
\begin{verse}[\versewidth]
%der 49. Psalm\\
%hört an, ihr Völker, hört doch an

\flagverse{1.} Hört an, ihr Völker, hört doch an,\\
hört alle, die ihr lebet,\\
arm, reich, Herr, Diener, Frau und Mann\\
und was auf Erden schwebet:\\
Mein Mund soll reden von Verstand\\
und rechte Weisheit lehren;\\
wir wollen, was mein Herz erfand,\\
ein fein Gedichte hören\\
und spielen auf der Harfen.

\flagverse{2.} Was sollt ich fürchten meinen Feind\\
in meinen bösen Tagen,\\
da mich, ders böse mit mir meint,\\
umgibt mit vielen Plagen,\\
wann mich mein Untertreter drückt\\
mit seinen Missetaten\\
und sich, weil ihm sein Tun geglückt\\
und alles wohl geraten,\\
erhebet, pocht und prahlet?

\flagverse{3.} Was hilft ihm all sein Hab und Gut,\\
wann sich der Tod herfindet?\\
Da gilt kein Geld, kein hoher Mut,\\
all Hilf und Rat verschwindet.\\
Und wenn auch gleich sein Bruder wollt\\
ihm an die Seite treten,\\
doch kann ihn weder rotes Gold\\
noch Bruders Blut erbeten,\\
er muß dem Tod herhalten.

\flagverse{4.} Der Tod ist gar ein treuer Mann,\\
fragt nichts nach gutem Willen;\\
wann einer gleich gibt, was er kann,\\
noch läßt er sich nicht stillen.\\
Und sieht er auch schon manchem zu,\\
läßt ihn viel Jahr erlangen,\\
doch bricht er endlich solche Ruh,\\
er kommt einmal gegangen\\
und holt die alten Greisen.

\flagverse{5.} Denn solche Weisen müssen doch\\
sowohl als wie die Narren\\
sich lassen in des Grabes Loch\\
versenken und verscharren;\\
da kommt denn, was sie an sich bracht,\\
in andrer Leute Hände,\\
und also gehet ihre Pracht\\
und Herrlichkeit zu Ende,\\
viel anders, als sie wünschen.

\flagverse{6.} Dies ist ihr Herz, das ist ihr Sinn,\\
daß ihr Haus ewig bleibe,\\
ihr Ehr und Würd auch immerhin\\
sich mehr und wohl erkleibe;\\
noch dennoch aber können sie\\
nichts überall erhalten,\\
sie müssen fort und wie ein Vieh\\
hinunter und erkalten.\\
Das ist ein töricht Wesen.

\flagverse{7.} Doch gleichwohl wird es hoch gerühmt\\
mit Lippen der Nachkommen\\
und gar nicht, wie es sich geziemt,\\
zur Beßrung angenommen.\\
Sie liegen in der Höllen Grund\\
in einem bösen Schlafe,\\
der Tod, der nagt sie wie ein Hund\\
und wie ein Wolf die Schafe,\\
die keine Hilfe haben.

\flagverse{8.} Die Bösen sind des Teufels Beut\\
und müssen Marter leiden,\\
die Frommen wird der Herr mit Freud\\
im Himmelsreiche weiden.\\
Der Trotz der unverschämten Rott\\
muß brechen und vergehen,\\
wer aber treu bleibt seinem Gott,\\
der soll dort ewig stehen\\
im Chor der Auserwählten.

\flagverse{9.} Darum, mein allerliebstes Kind,\\
laß dich nicht irre machen,\\
ob einer reich wird und mit Sünd\\
erlangt viel teure Sachen;\\
denn wann er stirbt, bleibt alles hier,\\
er kann nichts mit sich nehmen.\\
Sein Herrlichkeit, sein Ehr und Zier\\
verschwindet wie ein Schemen\\
und will ihm nicht nachfolgen.

\flagverse{10.} Die Welt liebt ihren Kot und Stank,\\
hält viel von schnöden Dingen.\\
Und also gehn sie auch den Gang,\\
den ihre Väter gingen,\\
und sehen hinfort nimmermehr\\
das Licht, das uns ernähret;\\
kurz: Wann ein Mensch hat Würd und Ehr\\
und ist nicht fromm, so fähret\\
er wie ein Vieh von hinnen.

\end{verse}
\end{multicols}
%\attrib{\small{THZE}}

\index{Hört an ihr Völker}
\newpage
\subsection*{\centerline{Ich hab oft bei mir selbst gedacht}}
\addcontentsline{toc}{subsection}{Ich hab oft bei mir selbst gedacht}
%StartInfo%%%%%%%%%%%%%%%%%%%%%%%%%%%%%%%%%%%%%%%%%%%%%%%%%%%%%%%%%%%%%%%%%%%%
%  Autor:
%  Titel:
%  File:
%  Ref:
%  Mod:
%EndInfo%%%%%%%%%%%%%%%%%%%%%%%%%%%%%%%%%%%%%%%%%%%%%%%%%%%%%%%%%%%%%%%%%%%%%%
%\poemtitle{pt}
\begin{multicols}{2}
\settowidth{\versewidth}{dem nicht sein Angst, sein Schmerz und Weh}
\begin{verse}[\versewidth]

\flagverse{1.} Ich hab oft bei mir selbst gedacht,\\
wann ich den Lauf der Welt betracht,\\
ob auch das Leben dieser Erd\\
uns gut sei und des Wünschens wert,\\
und ob nicht der viel besser tu,\\
der sich fein zeitlich legt zur Ruh.

\flagverse{2.} Denn, Lieber, denk und sage mir:\\
Was für ein Stand ist wohl allhier,\\
dem nicht sein Angst, sein Schmerz und Weh\\
alltäglich überm Haupte steh?\\
Ist auch ein Ort, der Kummers frei\\
und ohne Klag und Sorgen sei?

\flagverse{3.} Sieh unsers ganzen Lebens Lauf:\\
Ist auch ein Tag von Jugend auf,\\
der nicht sein eigne Qual und Plag\\
auf seinem Rücken mit sich trag?\\
Ist nicht die Freude, die uns stillt,\\
auch selbst mit Jammer überfüllt?

\flagverse{4.} Hat einer Glück und gute Zeit,\\
hilf Gott, wie tobt und zürnt der Neid!\\
Hat einer Ehr und große Würd,\\
ach, mit was großer Last und Bürd\\
ist, der vor andern ist geehrt,\\
vor andern auch dabei beschwert!

\flagverse{5.} Ist einer heute gutes Muts,\\
ergötzt und freut sich seines Guts:\\
Eh ers vermeint, fährt sein Gewinn\\
zusamt dem guten Mute hin!\\
Wie plötzlich kommt ein Ungestüm\\
und wirft die großen Güter üm!

\flagverse{6.} Bist du denn fromm und fleuchst die Welt\\
und liebst Gott mehr als Gold und Geld,\\
so wird dein Ruhm, dein Schmuck und Kron\\
in aller Welt zu Spott und Hohn;\\
denn wer der Welt nicht heucheln kann,\\
den sieht die Welt für albern an.

\flagverse{7.} Nun, es ist wahr, es steht uns hier\\
die Trübsal täglich vor der Tür,\\
und findt ein jeder überall\\
des Kreuzes Not und bittre Gall.\\
Sollt aber drum der Christen Licht\\
ganz nichts mehr sein? Das glaub ich nicht.

\flagverse{8.} Ein Christe, der an Christo klebt,\\
und stets im Geist und Glauben lebt,\\
dem kann kein Unglück, keine Pein\\
im ganzen Leben schädlich sein;\\
gehts ihm nicht allzeit wie es soll,\\
so ist ihm dennoch allzeit wohl.

\flagverse{9.} Hat er nicht Gold, so hat er Gott,\\
fragt nicht nach böser Leute Spott,\\
verwirft mit Freuden und verlacht\\
der Welt verkehrten Stolz und Pracht.\\
Sein Ehr ist Hoffnung und Geduld,\\
sein Hoheit ist des Höchsten Huld.

\flagverse{10.} Es weiß ein Christ und bleibt dabei,\\
daß Gott sein Freund und Vater sei;\\
er hau, er brenn, er stech, er schneid,\\
hier ist nichts, das uns von ihm scheid,\\
je mehr er schlägt, je mehr er liebt,\\
bleibt fromm, ob er uns gleich betrübt.

\flagverse{11.} Laß alles fallen, wie es fällt:\\
Wer Christi Lieb im Herzen hält,\\
der ist ein Held und bleibt bestehn,\\
wann Erd und Himmel untergehn;\\
und wann ihn alle Welt verläßt,\\
hält Gottes Wort ihn steif und fest.

\flagverse{12.} Des Höchsten Wort dämpft alles Leid\\
und kehrts in lauter Lust und Freud;\\
es nimmt dem Unglück alles Gift,\\
daß, obs uns gleich verfolgt und trifft,\\
es dennoch unser Herze nie\\
in allzu große Trauer zieh.

\flagverse{13.} Ei nun, so mäßge deine Klag!\\
Ist dieses Leben voller Plag,\\
ists dennoch an der Christen Teil\\
auch voller Gottes Schutz und Heil.\\
Wer Gott vertraut und Christum ehrt,\\
der bleibt im Kreuz auch unversehrt.

\flagverse{14.} Gleichwie das Gold durchs Feuer geht\\
und in dem Ofen wohl besteht,\\
so bleibt ein Christ durch Gottes Gnad\\
im Elendsofen ohne Schad;\\
ein Kind bleibt seines Vaters Kind,\\
obs gleich des Vaters Zucht empfindt.

\flagverse{15.} Drum, liebes Herz, sei ohne Scheu\\
und sieh auf deines Vaters Treu!\\
Empfindst du auch hier seine Rut,\\
er meints nicht bös, es ist dir gut!\\
Gib dich getrost in seine Händ,\\
es nimmt zuletzt ein gutes End.

\flagverse{16.} Leb immerhin, so lang er will!\\
Ists Leben schwer, so sei du still,\\
es geht zuletzt in Freuden aus:\\
Im Himmel ist ein schönes Haus,\\
da, wer nach Christo hier gestrebt,\\
mit Christi Engeln ewig lebt!
   
\end{verse}
\end{multicols}
%\attrib{\small{THZE}}

\index{Ich hab oft bei mir selbst gedacht}
\newpage
\subsection*{\centerline{Ich weiß, mein Gott, daß all mein Tun}}
\addcontentsline{toc}{subsection}{Ich weiß mein Gott daß all mein Tun}
%StartInfo%%%%%%%%%%%%%%%%%%%%%%%%%%%%%%%%%%%%%%%%%%%%%%%%%%%%%%%%%%%%%%%%%%%%
%  Autor:
%  Titel:
%  File:
%  Ref:
%  Mod:
%EndInfo%%%%%%%%%%%%%%%%%%%%%%%%%%%%%%%%%%%%%%%%%%%%%%%%%%%%%%%%%%%%%%%%%%%%%%
%\poemtitle{pt}
\begin{multicols}{2}
\settowidth{\versewidth}{Ich weiß, mein Gott, das all mein Tun}
\begin{verse}[\versewidth]
%vertrauen auf Gottes Willen\\
%ich weiß, mein Gott, daß all mein Tun und Werk auf deinem Willen ruhn\\
%(Jer. 10, 23)

\flagverse{1.} Ich weiß, mein Gott, daß all mein Tun\\
und Werk in deinem Willen ruhn,\\
von dir kommt Glück und Segen;\\
was du regierst, das geht und steht\\
auf rechten, guten Wegen.

\flagverse{2.} Es steht in keines Menschen Macht,\\
daß sein Rat werd ins Werk gebracht\\
und seines Gangs sich freue;\\
des Höchsten Rat, der machts allein,\\
daß Menschenrat gedeihe.

\flagverse{3.} Oft denkt der Mensch in seinem Mut,\\
dies oder jenes sei ihm gut,\\
und ist doch weit gefehlet;\\
oft sieht er auch für schädlich an,\\
was doch Gott selbst erwählet.

\flagverse{4.} So fängt auch mancher weise Mann\\
ein gutes Werk zwar fröhlich an\\
und bringts doch nicht zum Stande;\\
er baut ein Schloß und festes Haus,\\
doch nur auf lauterm Sande.

\flagverse{5.} Wie mancher ist in seinem Sinn\\
fast über Berg und Spitzen hin,\\
und eh er sichs versiehet,\\
so liegt er da und hat sein Fuß\\
vergeblich sich bemühet.

\flagverse{6.} Drum, lieber Vater, der du Kron\\
und Zepter trägst in deinem Thron\\
und aus den Wolken blitzest,\\
vernimm mein Wort und höre mich\\
vom Stuhle, da du sitzest.

\flagverse{7.} Verleihe mir das edle Licht,\\
das sich von deinem Angesicht\\
in fromme Seelen strecket\\
und da der rechten Weisheit Kraft\\
durch deine Kraft erwecket.

\flagverse{8.} Gib mir Verstand aus deiner Höh,\\
auf daß ich ja nicht ruf und steh\\
auf meinem eignen Willen;\\
sei du mein Freund und treuer Rat,\\
was recht ist, zu erfüllen.

\flagverse{9.} Prüf alles wohl, und was mir gut,\\
das gib mir ein; was Fleisch und Blut\\
erwählet, das verwehre;\\
der Höchste Zweck, das beste Teil\\
sei deine Lieb und Ehre.

\flagverse{10.} Was dir gefällt, das laß auch mir,\\
o meiner Seelen Sonn und Zier,\\
gefallen und belieben;\\
was dir zuwider, laß mich nicht\\
im Werk und Tat verüben.

\flagverse{11.} Ists Werk von dir, so hilf zu Glück;\\
ists Menschentum, so treibs zurück\\
und ändre meine Sinnen;\\
was du nicht wirkst, pflegt von ihm selbst\\
in kurzem zu zerrinnen.

\flagverse{12.} Sollt aber dein und unser Feind\\
an dem, was dein Herz gut gemeint,\\
beginnen sich zu rächen:\\
Ist das mein Trost, daß seinen Zorn\\
du leichtlich könnest brechen.

\flagverse{13.} Tritt zu mir zu und mache leicht,\\
was mir sonst fast unmöglich deucht,\\
und bring zum guten Ende,\\
was du selbst angefangen hast,\\
durch Weisheit deiner Hände.

\flagverse{14.} Ist ja der Anfang etwas schwer,\\
und muß ich auch ins tiefe Meer\\
der bittern Sorgen treten,\\
so treib mich nur ohn Unterlaß\\
zu seufzen und zu beten.

\flagverse{15.} Wer fleißig betet und dir traut,\\
wird alles, da ihm sonst vor graut,\\
mit tapferm Mut bezwingen;\\
sein Sorgenstein wird in der Eil\\
in tausend Stücke springen.

\flagverse{16.} Der Weg zum Guten ist fast wild,\\
mit Dorn und Hecken ausgefüllt,\\
doch wer ihn freudig gehet,\\
kommt endlich, Herr, durch deinen Geist,\\
wo Freud und Wonne stehet.

\flagverse{17.} Du bist mein Vater, ich dein Kind,\\
was ich bei mir nicht hab und find,\\
hast du zu aller Gnüge,\\
so hilf nur, daß ich meinen Stand\\
wohl halt und herrlich siege.

\flagverse{18.} Dein soll sein aller Ruhm und Ehr,\\
ich will dein Tun je mehr und mehr\\
aus hocherfreuter Seelen\\
vor deinem Volk und aller Welt,\\
so lang ich leb, erzählen.

\end{verse}
\end{multicols}
%\attrib{\small{THZE}}

\index{Ich weiß, main Gott, daß all mein Tun}
\newpage
\subsection*{\centerline{Tapfre Leute soll man loben}}
\addcontentsline{toc}{subsection}{Tapfre Leute soll man loben}
%StartInfo%%%%%%%%%%%%%%%%%%%%%%%%%%%%%%%%%%%%%%%%%%%%%%%%%%%%%%%%%%%%%%%%%%%%
%  Autor:
%  Titel:
%  File:
%  Ref:
%  Mod:
%EndInfo%%%%%%%%%%%%%%%%%%%%%%%%%%%%%%%%%%%%%%%%%%%%%%%%%%%%%%%%%%%%%%%%%%%%%%
%\poemtitle{pt}
\begin{multicols}{2}
\settowidth{\versewidth}{Er, Herr Sturm, pflanzt Palmenbäume;}
\begin{verse}[\versewidth]
%an den Landphysikus Samuel Sturm in Lukau\\
%tapfere Leute soll man loben\\
%(Aus »Fünfzehn-ästiger Nieder-Lausitzer Palm-Baum« des Samuel Sturm, 1675)

\flagverse{1.} Tapfre Leute soll man loben,\\
und was Tugend hat erhoben,\\
hebt auch billig unser Fleiß.\\
Laß, was schnöd ist, unten liegen,\\
was die Welt hat überstiegen,\\
deme bleibt sein Ruhm und Preis.

\flagverse{2.} Also wer, was andre haben\\
von des edlen Himmels Gaben,\\
weiß gebührlich anzuziehn,\\
dem gebührt vor andern allen,\\
daß zu seinem Wohlgefallen\\
Harf und Saiten sich bemühn.
\end{verse}
\end{multicols}

\begin{center}
\settowidth{\versewidth}{Er, Herr Sturm, pflanzt Palmenbäume;}
\begin{verse}[\versewidth]  
\flagverse{3.} Er, Herr Sturm, pflanzt Palmenbäume;\\
billig, daß hier keiner säume,\\
ihm ein Ehr und Dank zu tun.\\
Ich kann nichts mehr als nur bitten,\\
daß er stets mög in der Mitten\\
aller Tugendpalmen ruhn.

\end{verse}
\end{center}
\attrib{\small{An den Landphysikus Samuel Sturm in Lukau - 1675}}

\index{Tapfre Leute soll man loben}
\newpage
\subsection*{\centerline{Voller Wunder, voller Kunst}}
\addcontentsline{toc}{subsection}{Voller Wunder voller Kunst}
%StartInfo%%%%%%%%%%%%%%%%%%%%%%%%%%%%%%%%%%%%%%%%%%%%%%%%%%%%%%%%%%%%%%%%%%%%
%  Autor:
%  Titel:
%  File:
%  Ref:
%  Mod:
%EndInfo%%%%%%%%%%%%%%%%%%%%%%%%%%%%%%%%%%%%%%%%%%%%%%%%%%%%%%%%%%%%%%%%%%%%%%
%\poemtitle{pt}
\begin{multicols}{2}
\settowidth{\versewidth}{Hier wächst ein geschickter Sohn,}
\begin{verse}[\versewidth]
%der wundervolle Ehestand\\
%voller Wunder, voller Kunst

\flagverse{1.} Voller Wunder, voller Kunst,\\
voller Weisheit, voller Kraft,\\
voller Hulde, Gnad und Gunst,\\
voller Labsal, Trost und Saft,\\
voller Wunder, sag ich noch,\\
ist der keuschen Liebe Joch.

\flagverse{2.} Die sich nach dem Angesicht\\
niemals hiebevor gekannt,\\
auch sonst im geringsten nicht\\
mit Gedanken zugewandt,\\
derer Herzen, derer Hand\\
knüpft Gott in ein Liebesband.

\flagverse{3.} Dieser Vater zeucht sein Kind,\\
jener seins dagegen auf,\\
beide treibt ihr sonder Wind,\\
ihre sondre Bahn und Lauf.\\
Aber wenn die Zeit nun dar,\\
wirds ein wohlgeratnes Paar.

\flagverse{4.} Hier wächst ein geschickter Sohn,\\
dort ein edle Tochter zu,\\
eines ist des andern Kron,\\
eines ist des andern Ruh,\\
eines ist des andern Licht,\\
wissens aber beide nicht.

\flagverse{5.} Bis solang es dem beliebt,\\
der die Welt im Schoße hält,\\
und zur rechten Stunde gibt\\
jedem, der ihm wohlgefällt;\\
da erscheint im Werk und Tat\\
der so tief verborgne Rat.

\flagverse{6.} Da wählt Ahasverus Blick\\
ihm die stille Esther aus,\\
den Tobias führt das Glück\\
in der frommen Sara Haus,\\
Davids bald gewandter Will\\
holt die klug Abigail.

\flagverse{7.} Jakob fleucht vor Esaus Schwert\\
und trifft seine Rahel an,\\
Joseph dient auf fremder Erd\\
und wird Asnath Herr und Mann,\\
Mose spricht bei Jethro ein,\\
da wird die Zipora sein.

\flagverse{8.} Jeder findet, jeder nimmt,\\
was der Höchst ihm ausersehn,\\
was im Himmel ist bestimmt,\\
pflegt auf Erden zu geschehn,\\
und was denn nun so geschicht,\\
das ist sehr wohl ausgericht.

\flagverse{9.} Öfters denkt man dies und dies\\
hätte können besser sein,\\
aber wie die Finsternis\\
nicht erreicht der Sonnen Schein,\\
also geht auch Menschensinn\\
hinter Gottes Weisheit hin.

\flagverse{10.} Laßt zusammen, was Gott fügt,\\
der weiß, wies am besten sei,\\
unser Denken fehlt und trügt,\\
sein Gedank ist mangelfrei.\\
Gottes Werk hat festen Fuß,\\
wann sonst alles fallen muß.

\flagverse{11.} Siehe frommen Kindern zu,\\
die im heilgen Stande stehn,\\
wie so wohl Gott ihnen tu,\\
wie so schön er lasse gehn\\
alle Taten ihrer Händ\\
auf ein gutes selges End.

\flagverse{12.} Ihrer Tugend werter Ruhm\\
steht in steter voller Blüt,\\
wann sonst aller Liebe Blum,\\
als ein Schatten, sich verzieht;\\
und wann aufhört alle Treu,\\
ist doch ihre Treue neu.

\flagverse{13.} Ihre Lieb ist immer frisch\\
und verjüngt sich fort und fort,\\
Liebe zieret ihren Tisch\\
und verzuckert alle Wort;\\
Liebe gibt dem Herzen Rast\\
in der Müh- und Sorgenlast.

\flagverse{14.} Gehts nicht allzeit wie es soll,\\
ist doch diese Liebe still,\\
hält sich in dem Kreuze wohl,\\
denkt, es sei des Herren Will,\\
und versichert sich mit Freud\\
einer künftig bessern Zeit.

\flagverse{15.} Unterdessen geht und fleußt\\
Gottes reicher Segenbach,\\
speist die Leiber, tränkt den Geist,\\
stärkt des Hauses Grund und Dach,\\
und was klein, gering und bloß,\\
macht er mächtig, viel und groß.

\flagverse{16.} Endlich wenn nun ganz vollbracht,\\
was Gott hier in dieser Welt\\
frommen Kindern zugedacht,\\
nimmt er sie ins Himmelszelt\\
und drückt sie mit großer Lust\\
selbst an seinen Mund und Brust.



\end{verse}
\end{multicols}


\begin{center}
\settowidth{\versewidth}{Der, vor dem die Welt erschrickt,}
\begin{verse}[\versewidth]

\flagverse{17.} Nun so bleibt ja voller Gunst,\\
voller Labsal, Trost und Saft,\\
voller Wunder, voller Kunst,\\
voller Weisheit, voller Kraft,\\
voller Wunder, sag ich noch,\\
bleibt der keuschen Liebe Joch.
  
\end{verse}
\end{center}



%\attrib{\small{THZE}}

\index{Voller Wunder, voller Kunst}
\newpage
\subsection*{\centerline{Weltskribenten und Poeten}}                  %witt:? --> Leben
\addcontentsline{toc}{subsection}{Weltskribenten und Poeten}                  %witt:? --> Leben}
%StartInfo%%%%%%%%%%%%%%%%%%%%%%%%%%%%%%%%%%%%%%%%%%%%%%%%%%%%%%%%%%%%%%%%%%%%
%  Autor:
%  Titel:
%  File:
%  Ref:
%  Mod:
%EndInfo%%%%%%%%%%%%%%%%%%%%%%%%%%%%%%%%%%%%%%%%%%%%%%%%%%%%%%%%%%%%%%%%%%%%%%
%\poemtitle{pt}
\begin{multicols}{2}
\settowidth{\versewidth}{Gottes Wort, das ists vor allen,}
\begin{verse}[\versewidth]
%ode: Weltskribenten und Poeten\\
%vorspruch zu: Michael Schirmer, Biblische Lieder und Lehrsprüche, Berlin 1650

\flagverse{1.} Weltskribenten und Poeten\\
haben ihren Glanz und Schein,\\
mögen auch zu lesen sein,\\
wenn wir leben außer Nöten;\\
in dem Unglück, Kreuz und Übel\\
ist nichts Bessers als die Bibel.

\flagverse{2.} Cato deuchte sich zu stellen\\
in der Angst mit Plato Buch,\\
aber Gottes Zorn und Fluch\\
drückt ihn gleichwohl bis zur Höllen;\\
sein verirrter blinder Sinn\\
ging und wußte nicht wohin.

\flagverse{3.} Was Homerus hat gesungen\\
und des Maro hoher Geist,\\
wird gerühmet und gepreist\\
und hat alle Welt durchdrungen;\\
aber wenn der Tod uns trifft,\\
was hilft da Homerus' Schrift?

\flagverse{4.} Gottes Wort, das ists vor allen,\\
so uns, wenn des Herz erschrickt,\\
wie ein kühler Tau erquickt,\\
daß wir nicht zu Boden fallen.\\
Wenn die ganze Welt verzagt,\\
steht und siegt, was Gott gesagt.

\flagverse{5.} Wenn die Scharen aller Teufel\\
sich empören und bemühn,\\
dich von Christo abzuziehn\\
und zu stürzen in den Zweifel,\\
und du sprichst nur: So spricht Gott!\\
Werden sie zu Schand und Spott.

\flagverse{6.} Darum liebt, ihr lieben Herzen,\\
gottes Schriften, die gewiß\\
in der Herzensfinsternis\\
besser sind als alle Kerzen;\\
hier sind Strahlen, hier ist Licht,\\
das durch alles Herzleid bricht.
\end{verse}
\end{multicols}

\begin{center}
\settowidth{\versewidth}{Der, vor dem die Welt erschrickt,}
\begin{verse}[\versewidth]


\flagverse{7.} Unser Schirmer wirds euch lehren,\\
wenn ihr, was sein heilger Fleiß\\
ihm zum Trost und Gott zum Preis\\
hier gesetzet, werdet hören.\\
Lobt das Werk und liebt den Mann,\\
der das gute Werk getan.

  
\end{verse}
\end{center}




%\attrib{\small{THZE}}

\index{Weltscribenten und Poeten}
\newpage
\subsection*{\centerline{Wie schön ists doch, Herr Jesu Christ}}
\addcontentsline{toc}{subsection}{Wie schön ists doch Herr Jesu Christ}
%StartInfo%%%%%%%%%%%%%%%%%%%%%%%%%%%%%%%%%%%%%%%%%%%%%%%%%%%%%%%%%%%%%%%%%%%%
%  Autor:
%  Titel:
%  File:
%  Ref:
%  Mod:
%EndInfo%%%%%%%%%%%%%%%%%%%%%%%%%%%%%%%%%%%%%%%%%%%%%%%%%%%%%%%%%%%%%%%%%%%%%%
%\poemtitle{pt}
\begin{multicols}{2}
\settowidth{\versewidth}{Wie schön ists doch, Herr Jesu Christ,}
\begin{verse}[\versewidth]
%trostgesang christlicher Eheleute\\
%wie schön ists doch, Herr Jesu Christ, im Stande heilger Ehe

\flagverse{1.} Wie schön ists doch, Herr Jesu Christ,\\
im Stande, da dein Segen ist,\\
im Stande heilger Ehe!\\
Wie steigt und neigt sich deine Gab\\
und alles Gut so mild herab\\
aus deiner heilgen Höhe,\\
wenn sich\\
an dich\\
fleißig halten\\
jung und Alten,\\
die im Orden\\
eines Lebens einig worden!

\flagverse{2.} Wenn Mann und Weib sich wohl begehn\\
und unverrückt beisammen stehn\\
im Bande reiner Treue:\\
Da geht das Glück in vollem Lauf,\\
da sieht man wie der Engel Hauf\\
im Himmel selbst sich freue.\\
Kein Sturm,\\
kein Wurm\\
kann zerschlagen,\\
kann zernagen\\
was Gott gibet\\
dem Paar, das in ihm sich liebet.

\flagverse{3.} Vor allen gibt er seine Gnad\\
in derer Schoß er früh und spat\\
sein hoch Geliebten heget:\\
Da spannt sein Arm sich täglich aus,\\
da faßt er uns und unser Haus\\
gleich als ein Vater pfleget.\\
Da muß\\
ein Fuß\\
nach dem andern\\
gehn und wandern,\\
bis sie kommen\\
in das Zelt und Sitz der Frommen.

\flagverse{4.} Der Man wird einem Baume gleich\\
an Ästen schön, an Zweigen reich,\\
das Weib gleich einem Reben,\\
der seine Träublein trägt und nährt\\
und sich je mehr und mehr vermehrt\\
mit Früchten, die da leben.\\
Wohl dir,\\
o Zier,\\
Mannes Sonne,\\
Hauses Wonne,\\
Ehrenkrone!\\
Gott denkt dein bei seinem Throne.

\flagverse{5.} Dich, dich hat er sich auserkorn,\\
daß aus dir ward herausgeborn\\
das Volk, das sein Reich bauet.\\
Sein Wunderwerk geht immer fort,\\
und seines Mundes starkes Wort\\
macht, daß dein Auge schauet\\
schöne\\
Söhne\\
und die Tocken,\\
die den Wocken\\
abespinnen\\
und mit Kunst die Zeit gewinnen.

\flagverse{6.} Sei gutes Muts! Wir sind es nicht,\\
die diesen Orden aufgericht,\\
es ist ein höhrer Vater,\\
der hat uns je und je geliebt\\
und bleibt, wenn unsre Sorg uns trübt,\\
der beste Freund und Rater.\\
Anfang,\\
Ausgang\\
aller Sachen,\\
die zu machen\\
wir gedenken,\\
wird er wohl und weislich lenken.

\flagverse{7.} Zwar bleibts nicht aus, es kommt ja wohl\\
ein Stündlein, da man Leides voll\\
die Tränen lässet schießen;\\
jedennoch wer sich in Geduld\\
ergibt, des Leid wird Gottes Huld\\
in großen Freuden schließen.\\
Sitze,\\
schwitze\\
nur ein wenig!\\
Unser König\\
wird behende\\
machen, daß die Angst sich wende.

\flagverse{8.} Wohlher, mein König, nah herzu,\\
gib Rat in Kreuz, in Nöten Ruh,\\
in Ängsten Trost und Freude!\\
Des sollst du haben Ruhm und Preis,\\
wir wollen singen bester Weis\\
und danken alle beide,\\
bis wir\\
bei dir,\\
deinen Willen\\
zu erfüllen,\\
deinen Namen\\
ewig loben werden. Amen.

\end{verse}
\end{multicols}
\attrib{\small{Trostgesang christlicher Eheleute}}

\index{Wie schön ists doch}
\newpage
\subsection*{\centerline{Wohl dem, der den Herren scheuet}}
\addcontentsline{toc}{subsection}{Wohl dem der den Herren scheuet}
%StartInfo%%%%%%%%%%%%%%%%%%%%%%%%%%%%%%%%%%%%%%%%%%%%%%%%%%%%%%%%%%%%%%%%%%%%
%  Autor:
%  Titel:
%  File:
%  Ref:
%  Mod:
%EndInfo%%%%%%%%%%%%%%%%%%%%%%%%%%%%%%%%%%%%%%%%%%%%%%%%%%%%%%%%%%%%%%%%%%%%%%
%\poemtitle{pt}
\begin{multicols}{2}
\settowidth{\versewidth}{Wohl dem, der den Herren scheuet}
\begin{verse}[\versewidth]
%der 112. Psalm\\
%wohl dem, der den Herren scheuet

\flagverse{1.} Wohl dem, der den Herren scheuet\\
und sich fürcht't vor seinem Gott,\\
selig, der sich herzlich freuet,\\
zu erfüllen sein Gebot!\\
Wer den Höchsten liebt und ehrt,\\
wird erfahren, wie sich mehrt\\
alles, was in seinem Leben\\
ihm vom Himmel ist gegeben.

\flagverse{2.} Seine Kinder werden stehen\\
wie die Rosen in der Blüt,\\
sein Geschlecht wird einhergehen\\
voller Gnad und Gottes Güt;\\
und was diesen Leib erhält,\\
wird der Herrscher aller Welt\\
reichlich und mit vollen Händen\\
ihnen in die Häuser senden.

\flagverse{3.} Das gerechte Tun der Frommen\\
steht gewiß und wanket nicht;\\
sollt auch gleich ein Wetter kommen,\\
bleibt doch Gott der Herr ihr Licht,\\
tröstet, stärket, schützt und macht,\\
daß nach ausgestandner Nacht\\
und nach hochbetrübtem Weinen\\
freud und Sonne wieder scheinen.

\flagverse{4.} Gottes Gnad, Huld und Erbarmen\\
bleibt den Frommen immer fest.\\
Wohl dem, der die Not der Armen\\
sich zu Herzen gehen läßt\\
und mit Liebe Gutes tut;\\
den wird Gott, das höchste Gut,\\
gnädiglich in seinen Armen\\
als ein liebster Vater wärmen.

\flagverse{5.} Wenn die schwarzen Wolken blitzen\\
vor dem Donner in der Luft,\\
wird er ohne Sorgen sitzen\\
wie ein Vöglein in der Kluft.\\
Er wird bleiben ewiglich,\\
auch wird sein Gedächtnis sich\\
hie und da auf allen Seiten\\
wie die edlen Zweig ausbreiten.

\flagverse{6.} Wenn das Unglück an will kommen,\\
das die rohen Sünder plagt,\\
bleibt der Mut ihm unbenommen\\
und das Herze unverzagt;\\
unverzagt, ohn Angst und Pein\\
bleibt das Herze, das sich fein\\
seinem Gott und Herren ergibet\\
und die, so verlassen, liebet.

\flagverse{7.} Wer betrübte gern erfreuet,\\
wird vom Höchsten wohl ergötzt,\\
was die milde Hand ausstreuet,\\
wird vom Himmel hoch ersetzt;\\
wer viel gibt, erlanget viel.\\
Was sein Herze wünscht und will,\\
das wird Gott mit gutem Willen\\
schon zu rechter Zeit erfüllen.

\flagverse{8.} Aber seines Feindes Freude\\
wird er untergehen sehn;\\
er, der Feind, vor großem Neide\\
wird zerbeißen seine Zähn,\\
er wird knirschen und mit Grimm\\
solches Glück mißgönnen ihm\\
und doch damit gar nichts wehren,\\
sondern sich nur selbst verzehren.

\end{verse}
\end{multicols}
%\attrib{\small{THZE}}

\index{Wohl dem, der den Herren scheuet}
\newpage
\subsection*{\centerline{Wohl dem Menschen, der nicht wandelt}}
\addcontentsline{toc}{subsection}{Wohl dem Menschen der nicht wandelt}
%StartInfo%%%%%%%%%%%%%%%%%%%%%%%%%%%%%%%%%%%%%%%%%%%%%%%%%%%%%%%%%%%%%%%%%%%%
%  Autor:
%  Titel:
%  File:
%  Ref:
%  Mod:
%EndInfo%%%%%%%%%%%%%%%%%%%%%%%%%%%%%%%%%%%%%%%%%%%%%%%%%%%%%%%%%%%%%%%%%%%%%%
%\poemtitle{pt}
\begin{multicols}{2}
\settowidth{\versewidth}{Wohl dem Menschen, der nicht wandelt}
\begin{verse}[\versewidth]
%der 1. Psalm\\
%wohl dem Menschen, der nicht wandelt

\flagverse{1.} Wohl dem Menschen, der nicht wandelt\\
in gottloser Leute Rat!\\
Wohl dem, der nicht unrecht handelt\\
noch tritt auf der Sünder Pfad;\\
der der Spötter Freundschaft fleucht\\
und von ihren Stühlen weicht,\\
der hingegen herzlich ehret\\
was uns Gott vom Himmel lehret.

\flagverse{2.} Wohl dem, der mit Lust und Freuden\\
das Gesetz des Höchsten treibt\\
und hie, als auf süßer Weiden,\\
tag und Nacht beständig bleibt;\\
dessen Segen wächst und blüht\\
wie ein Palmbaum, den man sieht\\
bei den Flüssen an der Seiten\\
seine frischen Zweig ausbreiten.

\flagverse{3.} Also, sag ich, wird auch grünen,\\
wer in Gottes Wort sich übt,\\
luft und Sonne wird ihm dienen,\\
bis er reiche Früchte gibt.\\
Seine Blätter werden alt\\
und doch niemals ungestalt.\\
Gott gibt Glück zu seinen Taten,\\
was er macht, muß wohl geraten.

\flagverse{4.} Aber wen die Sünd erfreuet,\\
mit dem gehts viel anders zu:\\
Er wird wie die Spreu zerstreuet\\
von dem Wind im schnellen Nu.\\
Wo der Herr sein Häuflein richt't,\\
da bleibt kein Gottloser nicht.\\
Summa: Gott liebt alle Frommen,\\
und wer bös ist, muß umkommen.

\end{verse}
\end{multicols}
%\attrib{\small{THZE}}

\index{Wohl dem Menschen}
\newpage
\subsection*{\centerline{Zweierlei bitt ich von dir}}
\addcontentsline{toc}{subsection}{Zweierlei bitt ich von dir}
%StartInfo%%%%%%%%%%%%%%%%%%%%%%%%%%%%%%%%%%%%%%%%%%%%%%%%%%%%%%%%%%%%%%%%%%%%
%  Autor:
%  Titel:
%  File:
%  Ref:
%  Mod:
%EndInfo%%%%%%%%%%%%%%%%%%%%%%%%%%%%%%%%%%%%%%%%%%%%%%%%%%%%%%%%%%%%%%%%%%%%%%
%\poemtitle{pt}
\begin{multicols}{2}
\settowidth{\versewidth}{Ach, mein Gott, mein Schatz, mein Licht,}
\begin{verse}[\versewidth]
%aus den Sprüchen Salomonis (30, 7-9)\\
%zweierlei bitt ich von dir

\flagverse{1.} Zweierlei bitt ich von dir,\\
zweierlei trag ich dir für,\\
dir, der alles reichlich gibt,\\
was uns dient und dir beliebt;\\
gib mein Bitten, das du weißt,\\
eh ich sterb und sich mein Geist\\
aus des Lebens Banden reißt.

\flagverse{2.} Gib, daß ferne von mir sei\\
lügen und Abgötterei.\\
Armut, das die Maße bricht,\\
und groß Reichtum gib mir nicht.\\
Allzu arm und allzu reich\\
ist nicht gut, stürzt beides gleich\\
unsre Seel ins Sündenreich.

\flagverse{3.} Laß mich aber, o mein Heil,\\
nehmen mein bescheiden Teil\\
und beschere mir zur Not\\
hier mein täglich Bißlein Brot.\\
Ein klein wenig, da der Mut\\
und ein gut Gewissen ruht,\\
ist fürwahr ein großes Gut.

\flagverse{4.} Sonsten möcht im Überfluß\\
ich empfinden Überdruß,\\
dich verleugnen, dir zum Spott\\
fragen: Wer ist Herr und Gott?\\
Denn das Herz in Frechheit voll\\
weiß oft nicht, wann ihm ist wohl,\\
wie es sich erheben soll.

\flagverse{5.} Wiederum, wenns stehet bloß\\
und die Armut wird zu groß,\\
wird es untreu, stiehlt und stellt\\
nach des Nächsten Gut und Geld,\\
tut Gewalt, braucht Ränk und List,\\
ist mit Unrecht ausgerüst't,\\
fragt gar nicht, was christlich ist.

\flagverse{6.} Ach, mein Gott, mein Schatz, mein Licht,\\
dieser keines ziemt mir nicht:\\
Beides schändet deine Ehr,\\
beides stürzt ins Höllenmeer.\\
Drum so gib mir Füll und Hüll\\
also, wie dein Herze will,\\
nicht zu wenig, nicht zu viel.

\end{verse}
\end{multicols}
%\attrib{\small{THZE}}

\index{Zweierlei bitt ich von dir}
\newpage

\newpage

\section*{\centerline{\LARGE MORGEN UND ABEND}}
\addcontentsline{toc}{section}{MORGEN UND ABEND}
\rule{\textwidth}{0.2pt}\vspace*{-\baselineskip}\vspace{3.2pt}
\rule{\textwidth}{1.2pt}\\[\baselineskip]

\index{ Gedichte über Morgen und Abend}

\centerline{\scshape Die güldne Sonne}
\vspace*{2\baselineskip}
\centerline{\scshape Lobet den Herren alle, die ihn fürchten!}
\vspace*{2\baselineskip}
\centerline{\scshape Wach auf, mein Herz, und singe!}
\vspace*{2\baselineskip}
\centerline{\scshape Der Tag mit seinem Lichte}
\vspace*{2\baselineskip}
\centerline{\scshape Nun ruhen alle Wälder}

\newpage

\subsection*{\centerline{Die güldne Sonne}}
\addcontentsline{toc}{subsection}{Die güldne Sonne}
%StartInfo%%%%%%%%%%%%%%%%%%%%%%%%%%%%%%%%%%%%%%%%%%%%%%%%%%%%%%%%%%%%%%%%%%%%
%  Autor:
%  Titel:
%  File:
%  Ref:
%  Mod:
%EndInfo%%%%%%%%%%%%%%%%%%%%%%%%%%%%%%%%%%%%%%%%%%%%%%%%%%%%%%%%%%%%%%%%%%%%%%
%\poemtitle{Die güldne Sonne}
\begin{multicols}{2}
\settowidth{\versewidth}{ein herzerquickendes liebliches Licht.}
\begin{verse}[\versewidth]

\flagverse{1.} Die güldne Sonne\\
voll Freud und Wonne\\
bringt unsern Grenzen\\
mit ihrem Glänzen\\
ein herzerquickendes liebliches Licht.\\
Mein Haupt Und Glieder,\\
die lagen darnieder,\\
aber nun steh ich,\\
bin munter und fröhlich,\\
schaue den Himmel mit meinem Gesicht.

\flagverse{2.} Mein Auge schauet,\\
was Gott gebauet\\
zu seinen Ehren\\
und uns zu lehren,\\
wie sein Vermögen sei mächtig und groß,\\
und wo die Frommen\\
dann sollen hinkommen,\\
wenn sie mit Frieden\\
von hinnen geschieden\\
aus dieser Erden vergänglichem Schoß.

\flagverse{3.} Lasset uns singen,\\
dem Schöpfer bringen\\
Güter und Gaben;\\
was wir nur haben,\\
alles sei Gotte zum Opfer gesetzt.\\
Die besten Güter\\
sind unsre Gemüter;\\
dankbare Lieder\\
sind Weihrauch und Widder,\\
an welchen er sich am meisten ergötzt.

\flagverse{4.} Abend und Morgen\\
sind seine Sorgen;\\
segnen und mehren,\\
Unglück verwehren\\
sind seine Werke und Taten allein.\\
Wann wir uns legen,\\
so ist er zugegen,\\
wann wir aufstehen,\\
so läßt er aufgehen\\
über uns seiner Barmherzigkeit Schein.

\flagverse{5.} Ich hab erhoben\\
zu dir hoch droben\\
all meine Sinnen;\\
laß mein Beginnen\\
ohn allen Anstoß und glücklich ergehn!\\
Laster und Schande,\\
des Luzifers Bande,\\
Fallen und Tücke\\
treib ferne zurücke,\\
laß mich auf deinen Geboten bestehn!

\vfill\null
\columnbreak

\flagverse{6.} Laß mich mit Freuden\\
ohn alles Neiden\\
sehen den Segen,\\
den du wirst legen\\
in meines Bruders und Nähesten Haus;\\
geiziges Brennen,\\
unchristliches Rennen\\
nach Gut mit Sünde,\\
das tilge geschwinde\\
von meinem Herzen und wirf es hinaus!

\flagverse{7.} Menschliches Wesen,\\
was ist's? Gewesen.\\
In einer Stunde\\
geht es zu Grunde,\\
sobald das Lüftlein des Todes drein bläst.\\
Alles in allen\\
muß brechen und fallen,\\
Himmel und Erden\\
die müssen das werden,\\
was sie vor ihrer Erschöpfung gewest.

\flagverse{8.} Alles vergehet,\\
Gott aber stehet\\
ohn alles Wanken;\\
seine Gedanken,\\
sein Wort und Willen hat ewigen Grund,\\
sein Heil und Gnaden,\\
die nehmen nicht Schaden,\\
heilen im Herzen\\
die tödlichen Schmerzen,\\
halten uns zeitlich und ewig gesund.

\flagverse{9.} Gott, meine Krone,\\
vergib und schone;\\
laß meine Schulden\\
in Gnad und Hulden\\
aus deinen Augen sein abegewandt.\\
Sonsten regiere,\\
mich lenke und führe,\\
wie dirs gefället.\\
Ich habe gestellet\\
alles in deine Beliebung und Hand.

\flagverse{10.} Willst du mir geben,\\
womit mein Leben\\
ich kann ernähren,\\
so laß mich hören\\
allzeit im Herzen dies heilige Wort:\\
Gott ist das Größte,\\
das Schönste und Beste,\\
Gott ist das Süßte\\
und Allergewißte,\\
aus allen Schätzen der edelste Hort.

\vfill\null
\columnbreak

\flagverse{11.} Willst du mich kränken,\\
mit Gallen tränken,\\
und soll von Plagen\\
ich auch was tragen:\\
Wohlan, so mach es, wie dir es beliebt.\\
Was Gut und tüchtig,\\
was schädlich und nichtig\\
meinem Gebeine,\\
das weißt du alleine,\\
hast niemals keinen zu sehre betrübt.

\flagverse{12.} Kreuz und Elende,\\
das nimmt ein Ende;\\
nach Meeresbrausen\\
und Windessausen\\
leuchtet der Sonnen gewünschtes Gesicht.\\
Freude die Fülle\\
und selige Stille\\
hab ich zu warten\\
im himmlischen Garten;\\
dahin sind meine Gedanken gericht.

\end{verse}
\end{multicols}
%\attrib{\small{THZE}}

\index{Die güldne Sonne}
\newpage
\subsection*{\centerline{Lobet den Herren alle, die ihn fürchten!}}
\addcontentsline{toc}{subsection}{Lobet den Herren alle die ihn fürchten!}
%StartInfo%%%%%%%%%%%%%%%%%%%%%%%%%%%%%%%%%%%%%%%%%%%%%%%%%%%%%%%%%%%%%%%%%%%%
%  Autor:
%  Titel:
%  File:
%  Ref:
%  Mod:
%EndInfo%%%%%%%%%%%%%%%%%%%%%%%%%%%%%%%%%%%%%%%%%%%%%%%%%%%%%%%%%%%%%%%%%%%%%%
%\poemtitle{pt}
\begin{multicols}{2}
\settowidth{\versewidth}{und Preis und Dank zu seinem Altar bringen!}
\begin{verse}[\versewidth]
%morgenlied\\
%lobet den Herren alle, die ihn fürchten!

\flagverse{1.} Lobet den Herren\\
alle, die ihn fürchten!\\
Laßt uns mit Freuden seinem Namen singen\\
und Preis und Dank zu seinem Altar bringen!\\
Lobet den Herren!

\flagverse{2.} Der unser Leben,\\
das er uns hat geben,\\
in dieser Nacht so väterlich bedecket\\
und aus dem Schlaf uns fröhlich auferwecket.\\
Lobet den Herren!

\flagverse{3.} Daß unsre Sinnen\\
wir noch brauchen können\\
und Händ und Füße, Zung und Lippen regen,\\
das haben wir zu danken seinem Segen.\\
Lobet den Herren!

\flagverse{4.} Daß Feuersflammen\\
uns nicht allzusammen\\
mit unsern Häusern unversehns gefressen,\\
das machts, daß wir in seinem Schoß gesessen.\\
Lobet den Herren!

\flagverse{5.} Daß Dieb und Räuber\\
unser Gut und Leiber\\
nicht angetast't und grausamlich verletzet,\\
dawider hat sein Engel sich gesetzet.\\
Lobet den Herren!

\flagverse{6.} O treuer Hüter,\\
brunnen aller Güter,\\
ach laß doch ferner über unser Leben\\
bei Tag und Nacht dein Hut und Güte schweben.\\
Lobet den Herren!

\flagverse{7.} Gib, daß wir heute,\\
Herr, durch dein Geleite\\
auf unsern Wegen unverhindert gehen\\
und überall in deiner Gnade stehen.\\
Lobet den Herren!

\flagverse{8.} Treib unsern Willen,\\
dein Wort zu erfüllen,\\
lehr uns verrichten heilige Geschäfte,\\
und wo wir schwach sind, da gib du uns Kräfte.\\
Lobet den Herren!

\flagverse{9.} Richt unsre Herzen,\\
daß wir ja nicht scherzen\\
mit deinen Strafen, sondern fromm zu werden\\
vor deiner Zukunft uns bemühn auf Erden.\\
Lobet den Herren!

\flagverse{10.} Herr, du wirst kommen\\
und alle deine Frommen,\\
die sich bekehren, gnädig dahin bringen,\\
da alle Engel ewig, ewig singen:\\
Lobet den Herren!

\end{verse}
\end{multicols}
%\attrib{\small{THZE}}

\index{Lobet den Herren}
\newpage
%wiit: Lobet den Herren, lobet den Herren, alle, die ihn ehren
\subsection*{\centerline{Wach auf, mein Herz, und singe!}}
\addcontentsline{toc}{subsection}{Wach auf mein Herz und singe!}
%StartInfo%%%%%%%%%%%%%%%%%%%%%%%%%%%%%%%%%%%%%%%%%%%%%%%%%%%%%%%%%%%%%%%%%%%%
%  Autor:
%  Titel:
%  File:
%  Ref:
%  Mod:
%EndInfo%%%%%%%%%%%%%%%%%%%%%%%%%%%%%%%%%%%%%%%%%%%%%%%%%%%%%%%%%%%%%%%%%%%%%%
%\poemtitle{pt}
\begin{multicols}{2}
\settowidth{\versewidth}{Heint, als die dunklen Schatten}
\begin{verse}[\versewidth]
%wach auf, mein Herz, und singe!

\flagverse{1.} Wach auf, mein Herz und singe\\
dem Schöpfer aller Dinge,\\
dem Geber aller Güter,\\
dem frommen Menschenhüter.

\flagverse{2.} Heint, als die dunklen Schatten\\
mich ganz umgeben hatten,\\
hat Satan mein begehret,\\
Gott aber hats gewehret.

\flagverse{3.} Ja, Vater, als er suchte,\\
daß er mich fressen möchte,\\
war ich in deinem Schoße,\\
dein Flügel mich beschlosse.

\flagverse{4.} Du sprachst: Mein Kind, nun liege\\
trotz dem, der dich betrüge,\\
schlaf wohl, laß dir nicht grauen,\\
du sollst die Sonne schauen.

\flagverse{5.} Dein Wort, das ist geschehen,\\
ich kann das Licht noch sehen,\\
für Not bin ich befreiet,\\
dein Schutz hat mich erneuet.

\flagverse{6.} Du willst ein Opfer haben:\\
Hier bring ich meine Gaben;\\
mein Weihrauch und mein Widder\\
sind mein Gebet und Lieder.

\flagverse{7.} Die wirst du nicht verschmähen,\\
du kannst ins Herze sehen;\\
denn du weißt, daß zur Gabe\\
ich ja nichts Bessers habe.

\flagverse{8.} So wollst du nun vollenden\\
dein Werk an mir und senden,\\
der mich an diesem Tage\\
auf seinen Händen trage.

\flagverse{9.} Sprich Ja zu meinen Taten,\\
hilf selbst das Beste raten,\\
den Anfang, Mitt und Ende,\\
ach Herr, zum besten wende.

\flagverse{10.} Mich segne, mich behüte,\\
mein Herz sei deine Hütte,\\
dein Wort sei meine Speise,\\
bis ich gen Himmel reise.

\end{verse}
\end{multicols}
%\attrib{\small{THZE}}

\index{Wach auf mein Herz und singe}
\newpage
\subsection*{\centerline{Der Tag mit seinem Lichte}}
\addcontentsline{toc}{subsection}{Der Tag mit seinem Lichte}
%StartInfo%%%%%%%%%%%%%%%%%%%%%%%%%%%%%%%%%%%%%%%%%%%%%%%%%%%%%%%%%%%%%%%%%%%%
%  Autor:
%  Titel:
%  File:
%  Ref:
%  Mod:
%EndInfo%%%%%%%%%%%%%%%%%%%%%%%%%%%%%%%%%%%%%%%%%%%%%%%%%%%%%%%%%%%%%%%%%%%%%%
%\poemtitle{Der Tag mit seinem Lichte fleucht hin}
\begin{multicols}{2}
\settowidth{\versewidth}{Wohlauf, wohlauf, mein Psalter,}
\begin{verse}[\versewidth]
%Abendsegen\\
%Der Tag mit seinem Lichte fleucht hin

\flagverse{1.} Der Tag mit seinem Lichte\\
fleucht hin und wird zunichte;\\
die Nacht kommt angegangen,\\
mit Ruhe zu umfangen\\
den matten Erdenkreis.\\
Der Tag, der ist geendet,\\
mein Herz zu dir sich wendet,\\
der Tag und Nacht geschaffen\\
zum Wachen und zum Schlafen,\\
will singen deinen Preis.

\flagverse{2.} Wohlauf, wohlauf, mein Psalter,\\
erhebe den Erhalter,\\
der mir an Leib und Seelen\\
viel mehr, als ich kann zählen,\\
hat heute Guts getan.\\
All Augenblick und Stunden\\
hat sich gar viel gefunden,\\
womit er sein Gemüte\\
und unerschöpfte Güte\\
mir klar gezeiget an.

\flagverse{3.} Gleichwie des Hirten Freude,\\
Ein Schäflein an der Weide,\\
sich unter seiner Treue\\
ohn alle Furcht und Scheue\\
ergetzet in dem Feld\\
und sich mit Blumen füllet,\\
den Durst mit Quellen stillet:\\
So hat mich heut geführet,\\
mit manchem Gut gezieret\\
der Hirt in aller Welt.

\flagverse{4.} Gott hat mich nicht verlassen,\\
ich aber hab ohn Maßen\\
mich nicht gescheut, mit Sünden\\
und Unrecht zu entzünden\\
das treue Vaterherz.\\
Ach Vater, laß nicht brennen\\
den Eifer, noch mich trennen\\
von deiner Hand und Seiten:\\
Mein Tun und Überschreiten\\
erweckt mir Reu und Schmerz.

\flagverse{5.} Erhöre, Herr, mein Beten\\
und laß mein Übertreten\\
zur Rechten und zur Linken\\
ins Meeres Tiefe sinken\\
und ewig untergehn;\\
laß aber, laß hergegen\\
sich deine Engel legen\\
um mich mit ihren Waffen!\\
Mit dir will ich entschlafen,\\
mit dir auch auferstehn.

\flagverse{6.} Darauf so laß ich nieder\\
mein Haupt und Augenlider,\\
will ruhen ohne Sorgen,\\
bis daß der güldne Morgen\\
mich wieder munter macht.\\
Dein Flügel wird mich decken,\\
so wird mich nicht erschrecken\\
der Feind mit tausend Listen,\\
der mich und alle Christen\\
verfolget Tag und Nacht.

\end{verse}
\end{multicols}

\begin{center}
\settowidth{\versewidth}{Der, vor dem die Welt erschrickt,}
\begin{verse}[\versewidth]

\flagverse{7.} Ich lieg hier oder stehe,\\
ich sitz auch oder gehe,\\
so bleib ich dir ergeben,\\
und du bist auch mein Leben:\\
Das ist ein wahres Wort.\\
Was ich beginn und mache,\\
ich schlaf ein oder wache,\\
wohn ich als wie im Schlosse\\
in deinem Arm und Schoße,\\
bin selig hier und dort.
  
\end{verse}
\end{center}


%\attrib{\small{Abendsegen}}

\index{Der Tag mit seinem Lichte}
\newpage
\subsection*{\centerline{Nun ruhen alle Wälder}}
\addcontentsline{toc}{subsection}{Nun ruhen alle Wälder}
%StartInfo%%%%%%%%%%%%%%%%%%%%%%%%%%%%%%%%%%%%%%%%%%%%%%%%%%%%%%%%%%%%%%%%%%%%
%  Autor:
%  Titel:
%  File:
%  Ref:
%  Mod:
%EndInfo%%%%%%%%%%%%%%%%%%%%%%%%%%%%%%%%%%%%%%%%%%%%%%%%%%%%%%%%%%%%%%%%%%%%%%
%\poemtitle{pt}
\begin{multicols}{2}
\settowidth{\versewidth}{Wo bist du, Sonne, blieben?}
\begin{verse}[\versewidth]

\flagverse{1.} Nun ruhen alle Wälder,\\
Vieh, Menschen, Städt und Felder,\\
es schläft die ganze Welt;\\
ihr aber, meine Sinnen,\\
auf auf, ihr sollt beginnen,\\
was eurem Schöpfer wohlgefällt.

\flagverse{2.} Wo bist du, Sonne, blieben?\\
Die Nacht hat dich vertrieben,\\
die Nacht, des Tages Feind;\\
fahr hin! Ein ander Sonne,\\
mein Jesus, meine Wonne,\\
gar hell in meinem Herzen scheint.

\flagverse{3.} Der Tag ist nun vergangen,\\
die güldnen Sterne prangen\\
am blauen Himmelssaal;\\
also werd ich auch stehen,\\
wenn mich wird heißen gehen\\
mein Gott aus diesem Jammertal.

\flagverse{4.} Der Leib eilt nun zur Ruhe,\\
legt ab das Kleid und Schuhe,\\
das Bild der Sterblichkeit;\\
die zieh ich aus. Dagegen\\
wird Christus mir anlegen\\
den Rock der Ehr und Herrlichkeit.

\flagverse{5.} Das Haupt, die Füß und Hände\\
sind froh, daß nun zu Ende\\
die Arbeit kommen sei;\\
herz, freu dich, du sollst werden\\
vom Elend dieser Erden\\
und von der Sünden Arbeit frei.

\flagverse{6.} Nun geht, ihr matten Glieder,\\
geht hin und legt euch nieder,\\
der Betten ihr begehrt;\\
es kommen Stund und Zeiten,\\
da man euch wird bereiten\\
zur Ruh ein Bettlein in der Erd.

\flagverse{7.} Mein Augen stehn verdrossen,\\
im Hui sind sie geschlossen,\\
wo bleibt denn Leib und Seel?\\
Nimm sie zu deinen Gnaden,\\
sei gut für allem Schaden,\\
du Aug und Wächter Israel.

\flagverse{8.} Breit aus die Flügel beide,\\
o Jesu, meine Freude,\\
und nimm dein Küchlein ein!\\
Will Satan mich verschlingen,\\
so laß die Englein singen:\\
Dies Kind soll unverletzet sein.

\end{verse}
\end{multicols}
%\attrib{\small{THZE}}

\begin{center}
\settowidth{\versewidth}{Der, vor dem die Welt erschrickt,}
\begin{verse}[\versewidth]

\flagverse{9.} Auch euch, ihr meine Lieben,\\
soll heute nicht betrüben\\
ein Unfall noch Gefahr.\\
Gott laß euch selig schlafen,\\
stell euch die güldnen Waffen\\
ums Bett und seiner Engel Schar.


  
\end{verse}
\end{center}




\index{Nun ruhen alle Wälder}
\newpage

\section*{\centerline{\LARGE STERBEN}}
\addcontentsline{toc}{section}{STERBEN}
\rule{\textwidth}{0.2pt}\vspace*{-\baselineskip}\vspace{3.2pt}
\rule{\textwidth}{1.2pt}\\[\baselineskip]

\index{ Gedichte von Tod und Sterben}

\centerline{\scshape Du bist zwar mein und bleibest mein}
\vspace*{2\baselineskip}
\centerline{\scshape Erhebe dich, betrübtes Herz}                %witt:? --> Sterben
\vspace*{2\baselineskip}
\centerline{\scshape Herr Gott, du bist ja für und für}
\vspace*{2\baselineskip}
\centerline{\scshape Herr Lindholtz legt sich hin}               %witt:? --> Sterben
\vspace*{2\baselineskip}
\centerline{\scshape Ich bin ein Gast auf Erden}
\vspace*{2\baselineskip}
\centerline{\scshape Ich weiß, daß mein Erlöser lebt}
\vspace*{2\baselineskip}
\centerline{\scshape Johannes sahe durch Gesicht}
\vspace*{2\baselineskip}
\centerline{\scshape Leid ist mirs in meinem Herzen}             %witt:?  --> Sterben
\vspace*{2\baselineskip}
\centerline{\scshape Liebes Kind, wenn ich bei mir}              %witt:? --> Sterben
\vspace*{2\baselineskip}
\centerline{\scshape Mein Gott, ich habe mir}
\vspace*{2\baselineskip}
\centerline{\scshape Mein herzer Vater, weint ihr noch?}
\vspace*{2\baselineskip}
\centerline{\scshape Nun sei getrost und unbetrübt}
\vspace*{2\baselineskip}
\centerline{\scshape O Tod, o Tod, du greulichs Bild}
\vspace*{2\baselineskip}
\centerline{\scshape O, wie so ein großes Gut}                   %witt:? --> Sterben
\vspace*{2\baselineskip}
\centerline{\scshape So geht der alte liebe Herr nun auch dahin} %witt:? --> Sterben
\vspace*{2\baselineskip}
\centerline{\scshape Was trauerst du, mein Angesicht}
\vspace*{2\baselineskip}
\centerline{\scshape Weint, und weint gleichwohl nicht zu sehr}  %witt:?--> Sterben
\vspace*{2\baselineskip}
\centerline{\scshape Wer selig stirbt, stirbt nicht}             %witt:? --> Sterben

\newpage

\subsection*{\centerline{Du bist zwar mein und bleibest mein}}
\addcontentsline{toc}{subsection}{Du bist zwar mein und bleibest mein}
%StartInfo%%%%%%%%%%%%%%%%%%%%%%%%%%%%%%%%%%%%%%%%%%%%%%%%%%%%%%%%%%%%%%%%%%%%
%  Autor:
%  Titel:
%  File:
%  Ref:
%  Mod:
%EndInfo%%%%%%%%%%%%%%%%%%%%%%%%%%%%%%%%%%%%%%%%%%%%%%%%%%%%%%%%%%%%%%%%%%%%%%
%\poemtitle{Du bist zwar mein und bleibest mein}
\begin{multicols}{2}
\settowidth{\versewidth}{Du bist zwar mein und bleibest mein}
\begin{verse}[\versewidth]

%Aauf den Tod des Sohnes des Archidiakonus Johann Berkow in Berlin (1650)

\flagverse{1.} Du bist zwar mein und bleibest mein\\
(Wer will mir anders sagen?),\\
doch bist du nicht nur mein allein;\\
der Herr von ewgen Tagen,\\
der hat das meiste Recht an dir,\\
der fordert und erhebt von mir\\
dich, o mein Sohn, mein Wille,\\
mein Herz und Wunsches Fülle.

\flagverse{2.} Ach, gült es Wünschens, wollt ich dich,\\
du Sternlein meiner Seelen,\\
vor allem Weltgut williglich\\
mir wünschen und erwählen;\\
ich wollte sagen: Bleib bei mir!\\
Du sollst sein meins Hauses Zier;\\
an dir will ich mein Lieben\\
bis in mein Sterben üben.

\flagverse{3.} So sagt mein Herz und meint es gut.\\
Gott aber meints noch besser.\\
Groß ist die Lieb in meinem Mut,\\
in Gott ist sie noch größer.\\
Ich bin ein Vater und nichts mehr,\\
Gott ist der Väter Haupt und Ehr,\\
ein Quell, da Alt und Jungen\\
in aller Welt entsprungen.

\flagverse{4.} Ich sehne mich nach meinem Sohn,\\
und der mir ihn gegeben\\
will, daß er nah an seinem Thron\\
im Himmel solle leben.\\
Ich sprech: Ach weh, mein Licht verschwindt!\\
Gott spricht: Willkommn, du liebes Kind,\\
dich will ich bei mir haben\\
und ewig reichlich laben.

\flagverse{5.} O süßer Rat, o schönes Wort\\
und heilger, als wir denken!\\
Bei Gott ist ja kein böser Ort,\\
kein Unglück und kein Kränken,\\
kein Angst, kein Mangel, kein Versehn,\\
bei Gott kann keinem Leid geschehn;\\
wen Gott versorgt und liebet,\\
wird nimmermehr betrübet.

\flagverse{6.} Wir Menschen sind ja auch bedacht,\\
die Unsrigen zu zieren;\\
wir gehn und sorgen Tag und Nacht,\\
wie wir sie wollen führen\\
in einen feinen selgen Stand,\\
und ist doch selten so bewandt\\
mit dem, wohin sie kommen,\\
als wir uns vorgenommen.

\flagverse{7.} Wie manches junges fromme Blut\\
wird jämmerlich verführet\\
durch bös Exempel, daß es tut,\\
was Christen nicht gebühret.\\
Da hats denn Gottes Zorn zu Lohn,\\
auf Erden nichts als Spott und Hohn,\\
der Vater muß mit Grämen\\
sich seines Kindes schämen.

\flagverse{8.} Ein solches darf ja ich nun nicht\\
an meinen Sohn erwarten;\\
der steht vor Gottes Angesicht\\
und geht in Christi Garten,\\
hat Freude, die ihn recht erfreut,\\
und ruht von allem Herzeleid;\\
er sieht und hört die Scharen,\\
die uns allhier bewahren.

\flagverse{9.} Er sieht und hört der Engel Mund,\\
sein Mündlein hilft selbst singen;\\
Weiß alle Weisheit aus dem Grund\\
und redt von solchen Dingen,\\
die unser keiner noch nicht weiß,\\
die auch durch unsern Fleiß und Schweiß\\
wir, weil wir sind auf Erden,\\
nicht ausstudieren werden.

\flagverse{10.} Ach, sollt ich doch von fernen stehn\\
und nur ein wenig hören,\\
wenn deine Sinnen sich erhöhn\\
und Gottes Namen ehren,\\
der heilig, heilig, heilig ist,\\
durch den du auch geheiligt bist:\\
Ich weiß, ich würde müssen\\
vor Freuden Tränen gießen.

\flagverse{11.} Ich würde sprechen: Bleib allhier!\\
Nun will ich nicht mehr klagen:\\
Ach, mein Sohn, wärst du noch bei mir!\\
Nein; sondern: Komm du Wagen\\
Eliä, hole mich geschwind\\
und bring mich dahin, da mein Kind\\
und so viel liebe Seelen\\
so schöne Ding erzählen.

\flagverse{12.} Nun, es sei ja und bleib also,\\
ich will dich nicht mehr weinen.\\
Du lebst und bist von Herzen froh,\\
siehst lauter Sonnen scheinen,\\
die Sonnen ewger Freud und Ruh;\\
hier leb und bleib nun immerzu,\\
ich will, wills Gott, mit andern\\
auch bald hernacher wandern.

\end{verse}
\end{multicols}
%\attrib{\small{THZE}}

\index{Du bist zwar mein und bleibest mein}
\newpage
\subsection*{\centerline{Erhebe dich, betrübtes Herz}}           %witt:? --> Sterben
\addcontentsline{toc}{subsection}{Erhebe dich betrübtes Herz}           %witt:? --> Sterben}
%StartInfo%%%%%%%%%%%%%%%%%%%%%%%%%%%%%%%%%%%%%%%%%%%%%%%%%%%%%%%%%%%%%%%%%%%%
%  Autor:
%  Titel:
%  File:
%  Ref:
%  Mod:
%EndInfo%%%%%%%%%%%%%%%%%%%%%%%%%%%%%%%%%%%%%%%%%%%%%%%%%%%%%%%%%%%%%%%%%%%%%%
%\poemtitle{Erhebe dich, betrübtes Herz}
\begin{multicols}{2}
\settowidth{\versewidth}{Was stürzt wohl eines Frommen Sinn?}
\begin{verse}[\versewidth]

%»Trost-Gesang über den unversehenen Todesfall des wohlseligen Herrn Johannes Bercovii« (1651)

\flagverse{1.} Erhebe dich, betrübtes Herz,\\
und laß die Sinnen überwärts\\
hin nach dem Himmel steigen:\\
Nimm Gott zu Trost, so wird dein Schmerz\\
alsbald sich merklich neigen.

\flagverse{2.} Dein Schad ist groß, das ist ja wahr,\\
doch ist ja auch bekannt und klar\\
des höchsten Vaters Gnade:\\
Die macht, daß uns des Unglücks Schar\\
nicht um ein Härlein schade.

\flagverse{3.} Der Fall, der unverhoffte Fall\\
schlägt uns nicht anders als der Schall\\
des Donners aus der Höhe:\\
Gott aber hilft, daß Fall und Knall\\
zum Glück und Guten gehe.

\flagverse{4.} Was stürzt wohl eines Frommen Sinn?\\
Wo kann ein Christ auch anders hin\\
als in den Himmel fallen?\\
Trost, Fried und Freud erhalten ihn,\\
Angst muß zurückeprallen.

\flagverse{5.} Was hat der Tod mit seiner Müh,\\
er komme spät an oder früh,\\
an gottergebnen Seelen?\\
Nimmt er sie bald, befreit er sie\\
vor langem sauren Quälen.

\flagverse{6.} Wer plötzlich stirbt und stirbt nur wohl,\\
der nimmt ein Ende, das man soll\\
gewünscht und selig preisen:\\
Ists Herze gut und glaubensvoll,\\
was schadt das schnelle Reisen?

\flagverse{7.} Was fragt ein Kämpfer nach der Zeit,\\
wenn er den Feind nur in dem Streit\\
hat ritterlich empfangen?\\
Wie mancher kann die Siegesbeut\\
im Augenblick erlangen.

\flagverse{8.} Ein solches Lob und edlen Lohn\\
hat auch fürwahr und trägt davon\\
der, den wir jetzt beweinen:\\
Er sieht nun selbst ein helle Kron\\
auf seinem Haupte scheinen.

\flagverse{9.} Er hat gesiegt, das ist gewiß.\\
Er ist durch Todes Finsternis\\
zu Gottes Licht gekommen.\\
Er lebt, obschon ein schneller Riß\\
ihn von uns hingenommen.

\flagverse{10.} Den schnellen Riß hat Gott getan,\\
der nichts als Gutes machen kann\\
im Himmel und auf Erden.\\
Was gut tut, hebts gleich traurig an,\\
muß doch zuletzt gut werden.

\flagverse{11.} Wir wünschen zwar, ach hätten wir\\
doch bei dem Bette sollen hier\\
in seinem Ende stehen\\
und hören gegen dir und mir\\
sein letztes Wort ergehen.

\flagverse{12.} Denkt aber, denkt, ob dies Gehör\\
uns mehr betrübt als tröstlich wär,\\
und gebt euch wohl zufrieden,\\
weil er in Gott zu Gottes Ehr\\
auf Gottes Wort verschieden.

\flagverse{13.} Hilf Gott! Sprach sein gottselger Mund,\\
das hörte Gott, und half zur Stund\\
ihn in die hohen Freuden,\\
da sich sein Aug und Herzensgrund\\
in reiner Wollust weiden.

\flagverse{14.} Da hat er nun all Hilf und Heil,\\
ist froh in seinem Erb und Teil,\\
wonach er hier gestrebet,\\
ruht fern vom Tod und Todespfeil,\\
in dem er ewig lebet.

\flagverse{15.} Nun darf sein Herz nicht traurig sein\\
und fühlt nicht mehr den schweren Stein\\
des Kummers wie hienieden,\\
da sein Fleiß in der Sorgen Pein\\
sich täglich mußt ermüden.

\flagverse{16.} Sein süßer Mund, des edle Zier\\
des Höchsten Weisheit für und für\\
so treulich hat gelehret,\\
der predigt, was kein Ohr allhier\\
bei uns je hat gehöret.

\flagverse{17.} Er predigt seines Gottes Ruhm\\
und füllt das güldne Heiligtum\\
und die so schönen Tore,\\
sein Name reucht gleich einer Blum\\
im heilgen Engelchore.

\flagverse{18.} Die Pflänzlein, die er vorgeschickt,\\
hat er auch schon mit Lust erblickt\\
und herzlich sich ergetzet,\\
nun ist sein Geist in ihm erquickt\\
und alles Leid ersetzet.

\flagverse{19.} Was wollt ihr nun mit eurem Leid,\\
ihr, die ihr ihm gewogen seid,\\
euch selbst nun ferner plagen?\\
Wems wohlgeht und sich glücklich freut,\\
den darf man nicht mehr klagen.

\flagverse{20.} Wischt eure Tränen vom Gesicht\\
und laßt des lieben Trostes Licht\\
in eure Herzen brechen,\\
so wird, der alles Herzleid bricht,\\
euch Herz und Mut einsprechen.

\flagverse{21.} Nehmt eure Zuflucht zu ihm zu,\\
und glaubt, daß er nichts anderes tu\\
als nur, was uns kann nützen:\\
Wer das behält, wird in der Ruh\\
und Gott im Schoße sitzen.

\flagverse{22.} Wer Gott vertraut, wird in der Tat\\
erfahren, daß des Höchsten Rat\\
ihn weislich werde führen\\
und hier und dort mit großer Gnad\\
und reichem Segen zieren.

\end{verse}
\end{multicols}
%\attrib{\small{THZE}}

\index{Erhebe dich, betrübtes Herz}
\newpage
\subsection*{\centerline{Herr Gott, du bist ja für und für}}
\addcontentsline{toc}{subsection}{Herr Gott du bist ja für und für}
%StartInfo%%%%%%%%%%%%%%%%%%%%%%%%%%%%%%%%%%%%%%%%%%%%%%%%%%%%%%%%%%%%%%%%%%%%
%  Autor:
%  Titel:
%  File:
%  Ref:
%  Mod:
%EndInfo%%%%%%%%%%%%%%%%%%%%%%%%%%%%%%%%%%%%%%%%%%%%%%%%%%%%%%%%%%%%%%%%%%%%%%
%\poemtitle{Herr Gott, du bist ja für und für die Zuflucht deiner Herde}
\begin{multicols}{2}
\settowidth{\versewidth}{Wir sind ein Kraut, das bald verdorrt,}
\begin{verse}[\versewidth]
%Der 90. Psalm


\flagverse{1.} Herr Gott, du bist ja für und für\\
die Zuflucht deiner Herde,\\
du bist gewesen, eh allhier\\
gelegt der Grund zur Erde;\\
und da noch kein Berg war bereit,\\
da warst du in der Ewigkeit,\\
o Anfang aller Dinge.

\flagverse{2.} Du läßt die Menschen in das Tor\\
des Todes häufig wandern\\
und sprichst: Kommt wieder, Menschen, vor\\
und folget jenen andern!\\
Denn dir sind, Höchster, tausend Jahr\\
als wie ein Tag, der gestern war\\
und nunmehr ist vergangen.

\flagverse{3.} Du läßt das schnöde Menschenheer\\
wie einen Strom verfließen\\
und wie die Schifflein auf dem Meer\\
bei gutem Wind hinschießen:\\
Gleichwie ein Schlaf und Traum bei Nacht,\\
der, wenn der Mensch vom Schlaf erwacht,\\
entfallen und vergessen.

\flagverse{4.} Wir sind ein Kraut, das bald verdorrt,\\
ein Gras, das jetzt aufgehet,\\
wird aber schnell von seinem Ort\\
entführet und verwehet;\\
so ist ein Mensch: Heut blühet er,\\
und morgen, wann ihn ungefähr\\
ein Wind rührt, liegt er nieder.

\flagverse{5.} Das macht, Herr, deines Zornes Grimm\\
daß wir sobald verschwinden;\\
dein Eifer stößt und wirft uns üm,\\
von wegen unsrer Sünden.\\
Die Sünden stellest du vor dich,\\
davon brennt und entrüstet sich\\
dein allzeit reines Herze.

\flagverse{6.} Das ist das Feur, das uns versehrt\\
das Mark in allen Beinen,\\
daher kommts, daß der Tod verzehrt\\
die Großen und die Kleinen;\\
drum fahren unsre Tage hin\\
wie ein Geschwätze durch den Sinn,\\
wenn wir die Zeit vertreiben.

\flagverse{7.} Wie lang hält doch das Leben aus?\\
Gar selten siebzig Jahre.\\
Wenns hoch kommt, werden achtzig draus,\\
und wenn man alle Ware,\\
die hier gewonnen, nimmt zuhauf\\
ists lauter Müh von Jugend auf\\
und lauter Angst gewesen.

\flagverse{8.} Wir rennen, laufen, sorgen viel,\\
und eh wir uns versehen,\\
da kommt der Tod, steckt uns das Ziel,\\
und da ists dann geschehen;\\
wie fliehen eilend und behend,\\
und ist doch niemand, der sein End\\
und Gottes Zorn bedenke.

\flagverse{9.} Lehr uns bedenken, frommer Gott,\\
das Elend dieser Erden,\\
auf daß wir, wann wir an den Tod\\
gedenken, klüger werden!\\
Ach Kehre wieder, kehr uns zu\\
dein Angesicht und steh in Ruh\\
mit deinen bösen Knechten!

\flagverse{10.} Erfüll uns früh mit deiner Gnad\\
am Leib und an der Seelen,\\
so wollen wir dir früh und spat\\
dein Lob mit Dank erzählen;\\
erfreu uns, o du höchste Freud,\\
und gib uns wieder gute Zeit\\
nach so viel bösen Tagen!

\flagverse{11.} Bisher hats lauter Kreuz geschneit,\\
laß nun die Sonne scheinen,\\
beschehr uns Freude nach dem Leid\\
und Lachen nach dem Weinen!\\
Laß deiner Werke süßen Schein,\\
Herr, deinen Knechten kundbar sein\\
und dein Ehr ihren Kindern!

\flagverse{12.} Bleib unser Gott und treuer Freund,\\
halt uns auf festem Fuße;\\
und wenn wir etwa irrig seind,\\
so gib, daß sich mit Buße\\
das Herze wieder zu dir wend;\\
auch fördre das Tun unser Händ\\
und segn all unsre Werke!

\end{verse}
\end{multicols}
%\attrib{\small{THZE}}

\index{Herr Gott, du bist ja für und für}
\newpage
\subsection*{\centerline{Herr Lindholtz legt sich hin}}          %witt:? --> Sterben
\addcontentsline{toc}{subsection}{Herr Lindholtz legt sich hin}          %witt:? --> Sterben}
%StartInfo%%%%%%%%%%%%%%%%%%%%%%%%%%%%%%%%%%%%%%%%%%%%%%%%%%%%%%%%%%%%%%%%%%%%
%  Autor:
%  Titel:
%  File:
%  Ref:
%  Mod:
%EndInfo%%%%%%%%%%%%%%%%%%%%%%%%%%%%%%%%%%%%%%%%%%%%%%%%%%%%%%%%%%%%%%%%%%%%%%
%\poemtitle{Herr Lindholtz legt sich hin}
%\begin{multicols}{2}
\settowidth{\versewidth}{Herr Lindholtz legt sich hin und schläft in Gottes Namen,}
\begin{verse}[\versewidth]
%Herr Lindholtz legt sich hin und schläft in Gottes Namen\\
%Auf den Tod des Kammergerichtsadvokaten in Berlin Christian Lindholtz (1659)

\flagverse{1.} Herr Lindholtz legt sich hin und schläft in Gottes Namen,\\
weiß nichts mehr von dem Leid und von dem großen Gramen,\\
das jetzt die Welt durchstreicht. Sein Grabmal deckt ihn zu;\\
der Himmel ist sein Sitz, die Erdgruft seine Ruh.

\flagverse{2.} O schweigt, o schweigt und ruht, ihr hochgeliebten Seinen!\\
Wer in der Freude lebt, den darf man nicht beweinen.\\
Wir schweben in der See, der Sturm trübt unsern Sinn:\\
Herr Lindholtz ist im Port. Gott helf uns allen hin.

\end{verse}
%\end{multicols}
%\attrib{\small{THZE}}

\index{Herr Lindholz legt sich hin}
\newpage
\subsection*{\centerline{Ich bin ein Gast auf Erden}}
\addcontentsline{toc}{subsection}{Ich bin ein Gast auf Erden}
%StartInfo%%%%%%%%%%%%%%%%%%%%%%%%%%%%%%%%%%%%%%%%%%%%%%%%%%%%%%%%%%%%%%%%%%%%
%  Autor:
%  Titel:
%  File:
%  Ref:
%  Mod:
%EndInfo%%%%%%%%%%%%%%%%%%%%%%%%%%%%%%%%%%%%%%%%%%%%%%%%%%%%%%%%%%%%%%%%%%%%%%
%\poemtitle{pt}
\begin{multicols}{2}
\settowidth{\versewidth}{Ich bin ein Gast auf Erden}
\begin{verse}[\versewidth]
%der 119. Psalm\\
%ich bin ein Gast auf Erden

\flagverse{1.} Ich bin ein Gast auf Erden\\
und hab hier keinen Stand,\\
der Himmel soll mir werden,\\
da ist mein Vaterland.\\
Hier reis ich aus und abe,\\
dort, in der ewgen Ruh,\\
ist Gottes Gnadengabe,\\
die schleußt all Arbeit zu.

\flagverse{2.} Was ist mein ganzes Wesen,\\
von meiner Jugend an,\\
als Müh und Not gewesen?\\
So lang ich denken kann,\\
hab ich so manchen Morgen,\\
so manche liebe Nacht\\
mit Kummer und mit Sorgen\\
des Herzens zugebracht.

\flagverse{3.} Mich hat auf meinen Wegen\\
manch harter Sturm erschreckt,\\
blitz, Donner, Wind und Regen\\
hat mir manch Angst erweckt,\\
verfolgung, Haß und Neiden,\\
ob ichs gleich nicht verschuldt,\\
hab ich doch müssen leiden\\
und tragen mit Geduld

\flagverse{4.} So gings den lieben Alten,\\
an derer Fuß und Pfad\\
wir uns noch täglich halten,\\
wanns fehlt am guten Rat:\\
Wie mußte doch sich schmiegen\\
der Vater Abraham,\\
eh als ihm sein Vergnügen\\
und rechte Wohnstatt kam!

\flagverse{5.} Wie manche schwere Bürde\\
trug Isaak, sein Sohn!\\
Und Jakob, dessen Würde\\
stieg bis zum Himmelsthron,\\
wie mußte der sich plagen,\\
in was für Weh und Schmerz,\\
in was für Furcht und Zagen\\
sank oft sein armes Herz!

\flagverse{6.} Die frommen heilgen Seelen,\\
die gingen fort und fort\\
und änderten mit Quälen\\
den erstbewohnten Ort;\\
sie zogen hin und wieder,\\
ihr Kreuz war immer groß,\\
bis daß der Tod sie nieder\\
legt in des Grabes Schoß.

\flagverse{7.} Ich habe mich ergeben\\
in gleiches Glück und Leid:\\
Was will ich besser leben\\
als solche großen Leut?\\
Es muß ja durchgedrungen,\\
es muß gelitten sein;\\
wer nicht hat wohl gerungen,\\
geht nicht zur Freud hinein.

\flagverse{8.} So will ich zwar nun treiben\\
mein Leben durch die Welt,\\
doch denk ich nicht zu bleiben\\
in diesem fremden Zelt.\\
Ich wandre meine Straßen,\\
die zu der Heimat führt,\\
da mich ohn alle Maßen\\
mein Vater trösten wird.

\flagverse{9.} Mein Heimat ist dort droben,\\
da aller Engel Schar\\
den großen Herrscher loben,\\
der alles ganz und gar\\
in seinen Händen träget\\
und für und für erhält,\\
auch alles hebt und leget,\\
nach dems ihm wohl gefällt.

\flagverse{10.} Zu dem steht mein Verlangen,\\
da wollt ich gerne hin;\\
die Welt bin ich durchgangen,\\
daß ichs fast müde bin.\\
Je länger ich hier walle,\\
je wen'ger find ich Lust,\\
die meinem Geist gefalle;\\
das meist ist Stank und Wust.

\flagverse{11.} Die Herberg ist zu böse,\\
der Trübsal ist zu viel:\\
Ach komm, mein Gott, und löse\\
mein Herz, wann dein Herz will;\\
komm, mach ein seligs Ende\\
an meiner Wanderschaft,\\
und was mich kränkt, das wende\\
durch deinen Arm und Kraft!

\flagverse{12.} Wo ich bisher gesessen,\\
ist nicht mein rechtes Haus;\\
wann mein Ziel ausgemessen,\\
so tret ich dann hinaus,\\
und was ich hier gebrauchet,\\
das leg ich alles ab;\\
und wenn ich ausgehauchet,\\
so scharrt man mich ins Grab.

\flagverse{13.} Du aber, meine Freude,\\
du meines Lebens Licht,\\
du zeuchst mich, wenn ich scheide,\\
hin vor dein Angesicht,\\
ins Haus der ewgen Wonne,\\
da ich stets freudenvoll\\
gleich als die helle Sonne\\
nebst andern leuchten soll.

\flagverse{14.} Da will ich immer wohnen,\\
und nicht nur als ein Gast,\\
bei denen, die mit Kronen\\
du ausgeschmücket hast;\\
da will ich herrlich singen\\
von deinem großen Tun\\
und frei von schnöden Dingen\\
in meinem Erbteil ruhn.

\end{verse}
\end{multicols}
\attrib{\small{Der 119. Psalm}}

\index{Ich bin ein Gast auf Erden}
\newpage
\subsection*{\centerline{Ich weiß, daß mein Erlöser lebt}}
\addcontentsline{toc}{subsection}{Ich weiß daß mein Erlöser lebt}
%StartInfo%%%%%%%%%%%%%%%%%%%%%%%%%%%%%%%%%%%%%%%%%%%%%%%%%%%%%%%%%%%%%%%%%%%%
%  Autor:
%  Titel:
%  File:
%  Ref:
%  Mod:
%EndInfo%%%%%%%%%%%%%%%%%%%%%%%%%%%%%%%%%%%%%%%%%%%%%%%%%%%%%%%%%%%%%%%%%%%%%%
%\poemtitle{pt}
\begin{multicols}{2}
\settowidth{\versewidth}{Mein Heiland lebt! Ob ich nun werd}
\begin{verse}[\versewidth]
%ich weiß, daß mein Erlöser lebt

\flagverse{1.} Ich weiß, daß mein Erlöser lebt,\\
das soll mir niemand nehmen!\\
Er lebt, und was ihm widerstrebt,\\
das muß sich endlich schämen.\\
Er lebt fürwahr, der starke Held,\\
sein Arm, der alle Feinde fällt,\\
hat auch den Tod bezwungen.

\flagverse{2.} Des bin ich herzlich hoch erfreut\\
und habe gar kein Scheuen\\
vor dem, der alles Fleisch zerstreut\\
gleich wie der Wind die Spreuen.\\
Nimmt er gleich mich und mein Gebein\\
und scharrt uns in die Gruft hinein,\\
was kann er damit schaden!

\flagverse{3.} Mein Heiland lebt! Ob ich nun werd\\
ins Todes Staub mich strecken,\\
so wird er mich doch aus der Erd\\
hernachmals auferwecken;\\
er wird mich reißen aus dem Grab\\
und aus dem Lager, da ich hab\\
ein kleines ausgeschlafen.

\flagverse{4.} Da werd ich eben diese Haut\\
und eben diese Glieder,\\
die jeder jetzo an mir schaut,\\
auch was sich hin und wieder\\
von Adern und Gelenken findt\\
und meinen Leib zusammenbindt,\\
ganz richtig wieder haben.

\flagverse{5.} Zwar alles, was der Mensche trägt,\\
das Fleisch und seine Knochen,\\
wird, wenn er sich hin sterben legt,\\
zermalmet und zerbrochen\\
von Maden, Motten und was mehr\\
gehöret zu der Würmer Heer;\\
doch solls nicht stets so bleiben.

\flagverse{6.} Es soll doch alles wieder stehn\\
in seinem vorgen Wesen,\\
was niederlag, wird Gott erhöhn,\\
was umkam, wird genesen.\\
Was die Verfaulung hat verheert\\
und die Verwesung hat gezehrt,\\
wird alles wiederkommen.

\flagverse{7.} Das hab ich je und je gegläubt\\
und fass ein fest Vertrauen.\\
Ich werde den, der ewig bleibt,\\
in meinem Fleische schauen;\\
ja, in dem Fleische, das hier stirbt\\
und in dem Stank und Kot verdirbt,\\
da werd ich Gott inn sehen.

\flagverse{8.} Ich selber werd in seinem Licht\\
ihn sehn und mich erquicken,\\
mein Auge wird sein Angesicht\\
mit großer Lust erblicken.\\
Ich werd ihn mir sehn, mir zur Freud,\\
und werd ihm dienen ohne Zeit,\\
ich selber und kein Fremder.

\end{verse}
\end{multicols}

\begin{center}
\settowidth{\versewidth}{Der, vor dem die Welt erschrickt,}
\begin{verse}[\versewidth]


\flagverse{9.} Trotz sei nun allem, was mir will\\
mein Herze blöde machen!\\
Wärs noch so mächtig groß und viel,\\
kann ich doch fröhlich lachen.\\
Man treib und spanne noch so hoch\\
Sarg, Grab und Tod, so bleibet doch\\
Gott, mein Erlöser, leben.

\end{verse}
\end{center}
%\attrib{\small{THZE}}

\index{Ich weiß, daß mein Erlöser lebt}
\newpage
\subsection*{\centerline{Johannes sahe durch Gesicht}}
\addcontentsline{toc}{subsection}{Johannes sahe durch Gesicht}
%StartInfo%%%%%%%%%%%%%%%%%%%%%%%%%%%%%%%%%%%%%%%%%%%%%%%%%%%%%%%%%%%%%%%%%%%%
%  Autor:
%  Titel:
%  File:
%  Ref:
%  Mod:
%EndInfo%%%%%%%%%%%%%%%%%%%%%%%%%%%%%%%%%%%%%%%%%%%%%%%%%%%%%%%%%%%%%%%%%%%%%%
%\poemtitle{pt}
\begin{multicols}{2}
\settowidth{\versewidth}{Wer, sprach Johannes, sind doch die,}
\begin{verse}[\versewidth]
%johannes sahe durch Gesicht ein edles Licht\\
%(Off. Joh. 7, 9 ff.)

\flagverse{1.} Johannes sahe durch Gesicht\\
ein edles Licht\\
und liebliches Gemälde:\\
Er sah ein Haufen Völker stehn,\\
sehr hell und schön,\\
im güldnen Himmelsfelde.\\
Ihr Herz und Mut\\
schwebt in dem Gut,\\
das hier kein Mann\\
bezahlen kann\\
mit allem Gut und Gelde.

\flagverse{2.} Sie trugen Palmen in der Hand;\\
ihr Ort und Stand\\
war vor des Lammes Throne,\\
ihr Mund war voller Lob und Preis,\\
die Kleider weiß,\\
ihr Lied, im höhren Tone,\\
klang süß und sang\\
des Höchsten Dank,\\
und dieser Stimm\\
half üm und üm\\
der Engel heilge Krone.

\flagverse{3.} Wer, sprach Johannes, sind doch die,\\
die ich allhie\\
in weißem Schmuck seh halten?\\
Es sind, antwortet aus der Schar,\\
die um ihn war,\\
der eine von den Alten:\\
Es sind, mein Sohn,\\
die sich den Hohn\\
und Spott der Welt\\
von Gottes Zelt\\
nicht lassen abehalten.

\flagverse{4.} Es sind die, so vor dieser Zeit\\
in großem Leid\\
auf Erden sich befunden,\\
die bei des Herren Jesu Ehr\\
und seiner Lehr\\
all Angst und Trübsalswunden,\\
zwar ohne Schuld,\\
doch mit Geduld,\\
durch Gott gekühlt,\\
recht wohl gefühlt\\
und fröhlich überwunden.

\flagverse{5.} Dieselben haben all ihr Kleid,\\
als treue Leut,\\
im Glaubensbad erkläret;\\
sie haben sich der Höllen List,\\
so viel der ist,\\
mit starkem Mut erwehret\\
und nicht geacht\\
der Erden Pracht,\\
des Lammes Blut\\
zu ihrem Gut\\
erwählet und begehret.

\flagverse{6.} Darum so stehen sie auch nun\\
und all ihr Tun\\
wo Gottes Tempel stehet;\\
der Tempel, da man Tag und Nacht\\
dem Höchsten wacht\\
und seinen Ruhm erhöhet;\\
da leben sie\\
ohn alle Müh,\\
ohn alle Qual\\
im Freudensaal,\\
der nimmermehr vergehet.

\flagverse{7.} Daselbst sitzt Gott in seinem Haus\\
und breitet aus\\
die Hütte seiner Güte\\
und deckt mit sanfter Wollust zu\\
in stiller Ruh\\
manch trauriges Gemüte.\\
Was Freude gibt,\\
dem Herzen liebt,\\
die Augen füllt,\\
das Sehnen stillt,\\
steht da in voller Blüte.

\flagverse{8.} Da ist kein Durst, kein Hungersnot,\\
das Himmelsbrot\\
läßt keinen Mangel leiden,\\
da scheint die Sonne keinem mehr\\
zu heiß und sehr,\\
ihr Glanz bringt lauter Freuden.\\
Die Himmelssonn\\
und Herzenswonn\\
ist unser Hirt,\\
der große Wirt\\
und Herr der ewgen Weiden.
\end{verse}
\end{multicols}
%\attrib{\small{THZE}}

\begin{center}
\settowidth{\versewidth}{Das Lamm wird weiden seine Herd,}
\begin{verse}[\versewidth]
\flagverse{9.} Das Lamm wird weiden seine Herd,\\
als sies begehrt,\\
auf Auen, die schön prangen;\\
es wird sie leiten zu dem Quell,\\
der frisch und hell,\\
das Heil draus zu erlangen;\\
und wird gewiß\\
nicht ruhen, bis\\
er uns erfrischt\\
und abgewischt\\
die Tränen unsrer Wangen.
 
\end{verse}
\end{center}




\index{Johannes sahe durch Gesicht}
\newpage
\subsection*{\centerline{Leid ist mirs in meinem Herzen}}        %witt:?  --> Sterben
\addcontentsline{toc}{subsection}{Leid ist mirs in meinem Herzen}        %witt:?  --> Sterben}
%StartInfo%%%%%%%%%%%%%%%%%%%%%%%%%%%%%%%%%%%%%%%%%%%%%%%%%%%%%%%%%%%%%%%%%%%%
%  Autor:
%  Titel:
%  File:
%  Ref:
%  Mod:
%EndInfo%%%%%%%%%%%%%%%%%%%%%%%%%%%%%%%%%%%%%%%%%%%%%%%%%%%%%%%%%%%%%%%%%%%%%%
%\poemtitle{pt}
\begin{multicols}{2}
\settowidth{\versewidth}{Muß das Leibchen gleich verwesen,}
\begin{verse}[\versewidth]
%leid ist mirs in meinem Herzen\\
%auf den Tod der kleinen Elisabeth Heintzelmann,
%Tochter des Diakons an St. Nikolai in Berlin Johannes H. (1659)

\flagverse{1.} Leid ist mirs in meinem Herzen\\
um die, so dir, liebes Kind,\\
mit so großem Weh und Schmerzen\\
um den Hals gefallen sind,\\
da du dich bei deinem Ende\\
gabst in deines Gottes Hände.

\flagverse{2.} Ach, es ist ein bittres Leiden\\
und ein rechter Myrrhentrank,\\
sich von seinen Kindern scheiden\\
durch den schweren Todesgang!\\
Hier geschieht ein Herzensbrechen,\\
das kein Mund recht kann aussprechen.

\flagverse{3.} Aber das, was wir beweinen,\\
weiß hievon ganz lauter nichts,\\
sondern sieht die Sonne scheinen\\
und den Glanz des ewgen Lichts,\\
singt und springt und hört die Scharen,\\
die hier seine Wächter waren.

\flagverse{4.} Muß das Leibchen gleich verwesen,\\
ists ihm doch ein schlechter Schad,\\
Gott wird schon zusammenlesen,\\
was der Tod zerstreuet hat;\\
treu ist er und fromm den Seinen,\\
trägt sich auch mit ihren Beinen.

\flagverse{5.} Diesem Herrn ist nichts verdorben;\\
wenn des Todes Nacht vorbei,\\
nimmt er das, was war gestorben,\\
und machts wieder ganz und neu.\\
Also werden wir zur Erden,\\
daß wir mögen himmlisch werden.

\flagverse{6.} Auf derwegen! Seid zufrieden,\\
Vaterherz und Muttergeist,\\
lasset schlafen, was geschieden\\
und zu Gott ist hingereist!\\
Was für Tränen ihr vergossen,\\
wollen sein mit Trost geschlossen.
\end{verse}
\end{multicols}
%\attrib{\small{THZE}}

\begin{center}
\settowidth{\versewidth}{Der, vor dem die Welt erschrickt,}
\begin{verse}[\versewidth]


\flagverse{7.} Wandelt eure Klag in Singen!\\
Ist doch nunmehr alles gut.\\
Trauern mag nicht wiederbringen,\\
was im Himmelsschoße ruht.\\
Aber wer getrost sich gibet,\\
ist bei Gott sehr hoch beliebet.
\end{verse}
\end{center}
  
%\attrib{\small{THZE}}

\index{Leid ist mirs in meinem Herzen}
\newpage
\subsection*{\centerline{Liebes Kind, wenn ich bei mir}}         %witt:? --> Sterben
\addcontentsline{toc}{subsection}{Liebes Kind wenn ich bei mir}         %witt:? --> Sterben}
%StartInfo%%%%%%%%%%%%%%%%%%%%%%%%%%%%%%%%%%%%%%%%%%%%%%%%%%%%%%%%%%%%%%%%%%%%
%  Autor:
%  Titel:
%  File:
%  Ref:
%  Mod:
%EndInfo%%%%%%%%%%%%%%%%%%%%%%%%%%%%%%%%%%%%%%%%%%%%%%%%%%%%%%%%%%%%%%%%%%%%%%
%\poemtitle{pt}
\begin{multicols}{2}
\settowidth{\versewidth}{Liebes Kind, wenn ich bei mir}
\begin{verse}[\versewidth]
%liebes Kind, wenn ich bei mir bedenke\\
%auf den Tod des kleinen Friedrich Ludwig Zarlang, Sohn des Berliner Bürgermeisters Z. (1660)

\flagverse{1.} Liebes Kind, wenn ich bei mir\\
deines schönen Leibes Zier\\
und der Seelen Schmuck bedenke,\\
weiß es Gott, wie ich mich kränke.

\flagverse{2.} Kein Smaragd mag je so schön\\
in dem feinen Golde stehn,\\
keine Rose mag im Lenzen\\
dir gleich, schöne Blume, glänzen.

\flagverse{3.} Dein Gebärde, dein Gesicht\\
und der beiden Augen Licht\\
war in Tugend ganz verhüllet\\
und mit guter Zucht erfüllet.

\flagverse{4.} Deine Liebe, deine Gunst\\
ging und hing nach lauter Kunst;\\
viel zu lernen, viel zu wissen,\\
war dein edler Geist geflissen.

\flagverse{5.} Auch war hier ein guter Grund,\\
da das ganze Werk auf stund,\\
nämlich Gott und sein Wort hören\\
und die heilge Bibel ehren.

\flagverse{6.} Wollte, wollte Gott, daß nur\\
deines Lebens schwache Schnur\\
etwas noch hier auf der Erden\\
hätten müssen länger werden.

\flagverse{7.} O wie manche große Freud,\\
o wie manch Ergötzlichkeit\\
würden wir von deinen Gaben\\
noch zuletzt genossen haben.

\flagverse{8.} Nun, mich jammerts; aber du,\\
liebes Kind, schweigst still dazu,\\
wohnst in Gottes Stadt und Mauern\\
kehrst dich nicht an unser Trauern.

\flagverse{9.} Deines Wesens hoher Stand\\
ist auch nun also bewandt,\\
daß, wers gut will mit dir meinen,\\
dich nicht dürfe mehr beweinen.

\flagverse{10.} Du bist ungleich besser dran,\\
als die Welt hier sinnen kann;\\
du hast mehr als wir dir gönnen,\\
mehr auch, als wir wünschen können.

\flagverse{11.} Es ist an dir ganz und gar,\\
was hier unvollkommen war;\\
was du hier hast angefangen,\\
hast du dort vollauf empfangen.

\flagverse{12.} Deine Seel hat Gottes Reich,\\
und du bist den Engeln gleich:\\
Alle Himmel hörst du singen\\
und du gehst in vollen Springen.

\end{verse}
\end{multicols}
%\attrib{\small{THZE}}

\begin{center}
\settowidth{\versewidth}{Der, vor dem die Welt erschrickt,}
\begin{verse}[\versewidth]

\flagverse{13.} Nun so lebe, wie du lebest!\\
Schweb in Freuden, wie du schwebest!\\
Balde, balde wirds geschehen,\\
daß du uns, wir dich dort sehen.

\end{verse}
\end{center}


\index{Liebes Kind, wenn ich bei mir}
\newpage
\subsection*{\centerline{Mein Gott, ich habe mir}}
\addcontentsline{toc}{subsection}{Mein Gott ich habe mir}
%StartInfo%%%%%%%%%%%%%%%%%%%%%%%%%%%%%%%%%%%%%%%%%%%%%%%%%%%%%%%%%%%%%%%%%%%%
%  Autor:
%  Titel:
%  File:
%  Ref:
%  Mod:
%EndInfo%%%%%%%%%%%%%%%%%%%%%%%%%%%%%%%%%%%%%%%%%%%%%%%%%%%%%%%%%%%%%%%%%%%%%%
%\poemtitle{pt}
\begin{multicols}{2}
\settowidth{\versewidth}{Wenn mein Geblüt entbrennt,}
\begin{verse}[\versewidth]
%der 39. Psalm\\
%mein Gott, ich habe mir gar fest gesetzet für

\flagverse{1.} Mein Gott ich habe mir\\
gar fest gesetzet für,\\
ich will mich fleißig hüten,\\
wenn meine Feinde wüten,\\
daß, wenn ich ja was spreche,\\
ich dein Gesetz nicht breche.

\flagverse{2.} Wenn mein Geblüt entbrennt,\\
so hab ich mich gewöhnt,\\
vor deinen Stuhl zu treten,\\
laß Herz und Zunge beten;\\
Herr, zeige deinem Knechte,\\
zu tun nach deinem Rechte.

\flagverse{3.} Herr, lehre mich doch wohl\\
bedenken, daß ich soll\\
einmal von dieser Erden\\
hinweg geraffet werden,\\
und daß mir deine Hände\\
gesetzet Zeit und Ende.

\flagverse{4.} Die Tage meiner Zeit\\
sind eine Hande breit,\\
und wenn man dies mein Bleiben\\
soll recht und wohl beschreiben,\\
so ists ein Nichts und bleibet\\
ein Stäublein, das zerstäubet.

\flagverse{5.} Ach, wie so gar nichts wert\\
sind Menschen auf der Erd,\\
die doch so sicher leben\\
und gar nicht Acht drauf geben,\\
daß all ihr Tun und Glücke\\
verschwind im Augenblicke.

\flagverse{6.} Sie gehen in der Welt\\
und suchen Gut und Geld,\\
der Schatten einen Schemen!\\
Und können nichts mitnehmen,\\
wenn nach der Menschen Weise\\
sie tun des Todes Reise.

\flagverse{7.} Sie schlafen ohne Ruh,\\
arbeiten immerzu,\\
sind Tag und Nacht geflissen,\\
und können doch nicht wissen,\\
wer, wenn sie niederliegen,\\
ihr Erbe werde kriegen.

\flagverse{8.} Nun, Herr, wo soll ich hin?\\
Wer tröstet meinen Sinn?\\
Ich komm an deine Pforten,\\
der du mit Werk und Worten\\
erfreuest, die dich scheuen\\
und dein allein sich freuen.

\flagverse{9.} Wenn sich mein Feind erregt\\
und mir viel Dampfs anlegt,\\
so will ich stille schweigen,\\
mein Herz zur Ruhe neigen;\\
du Richter aller Sachen,\\
du kannst und wirsts wohl machen.

\flagverse{10.} Wenn du dein Hand ausstreckst,\\
des Menschen Herz erschreckst,\\
wenn du die Sünd heimsuchest,\\
den Sünder schiltst und fluchest:\\
So geht in einer Stunde\\
all Herrlichkeit zugrunde.

\flagverse{11.} Der schönen Jugend Kranz,\\
der roten Wangen Glanz\\
wird wie ein Kleid verzehret,\\
so hier die Motten nähret.\\
Ach, wie gar nichts im Leben\\
sind die auf Erden schweben!

\flagverse{12.} Du aber, du mein Hort,\\
du bleibest fort und fort\\
mein Helfer, siehst mein Sehnen,\\
mein Angst und heiße Tränen,\\
erhörest meine Bitte,\\
wenn ich mein Herz ausschütte.

\flagverse{13.} Drum ruhet mein Gemüt\\
allein auf deiner Güt;\\
ich laß dein Herze sorgen,\\
als deme nicht verborgen,\\
wie meiner Feinde Tücke\\
du treiben sollst zurücke.

\flagverse{14.} Ich bin dein Knecht und Kind,\\
dein Erb und Hausgesind,\\
dein Pilgrim und dein Bürger,\\
der, wenn der Menschenwürger\\
mein Leben mir genommen,\\
zu dir gewiß wird kommen.

\flagverse{15.} Zur Welt muß ich hinaus,\\
der Himmel ist mein Haus,\\
da in den Engelscharen\\
mein Eltern und Vorfahren,\\
auch Schwestern, Freund und Brüder\\
jetzt singen ihre Lieder.

\flagverse{16.} Hie ist nur Qual und Pein,\\
dort, dort wird Freude sein!\\
Dahin, wenn es dein Wille,\\
ich fröhlich, sanft und stille\\
aus diesen Jammerjahren\\
zur Ruhe will abfahren.
   
\end{verse}
\end{multicols}
\attrib{\small{THZE}}

\index{Mein Gott, ich habe mir}
\newpage
\subsection*{\centerline{Mein herzer Vater, weint ihr noch?}}    %witt: Herzensvater
\addcontentsline{toc}{subsection}{Mein herzer Vater weint ihr noch?}    %witt: Herzensvater}
%StartInfo%%%%%%%%%%%%%%%%%%%%%%%%%%%%%%%%%%%%%%%%%%%%%%%%%%%%%%%%%%%%%%%%%%%%
%  Autor:
%  Titel:
%  File:
%  Ref:
%  Mod:
%EndInfo%%%%%%%%%%%%%%%%%%%%%%%%%%%%%%%%%%%%%%%%%%%%%%%%%%%%%%%%%%%%%%%%%%%%%%
%\poemtitle{pt}
\begin{multicols}{2}
\settowidth{\versewidth}{Nichts ist so schön und wohl bestellt,}
\begin{verse}[\versewidth]
%mein herzer Vater, weint ihr noch?\\
%Auf den Tod eines Kindes des Rektors Adam Spengler (1650)

\flagverse{1.} Mein herzer Vater, weint ihr noch?\\
Und ihr, die mich geboren?\\
Was grämt ihr euch? Was macht ihr doch?\\
Ich bin ja unverloren.\\
Ach, sollt ihr sehen, wie mirs geht,\\
und wie mich der so hoch erhöht,\\
der selbst so hoch erhoben;\\
ich weiß, ihr würdet anders tun\\
und meiner Seele süßes Ruhn\\
mit eurem Munde loben.

\flagverse{2.} Der saure Kampf, den ich dort hab\\
in eurer Welt empfunden,\\
der ist durch Gottes Gnad und Gab\\
all glücklich überwunden.\\
Es ging mir, wie es pflegt zu gehn\\
all denen, die bei Christo stehn\\
und von der Welt sich scheiden;\\
wer Christo folgt, der muß mit ihm\\
das Kreuz und alles Ungestüm\\
auf seinen Wegen leiden.

\flagverse{3.} Nun bin ich durch. Gott Lob und Dank!\\
Hier kommt ein ander Leben;\\
hier wird mir, was mein Leben lang\\
ich nicht gesehn, gegeben:\\
Ein ganzer Himmel voller Licht,\\
ein Licht, davon mein Angesicht\\
so schön wird als die Sonne;\\
hier ist ein ewges Freudenmeer,\\
wohin ich nur die Augen kehr,\\
ist alles voller Wonne.

\flagverse{4.} Nun lobt, ihr Menschen, wie ihr wollt,\\
des Erdenlebens Güte:\\
Was ist darinnen, das mir sollt\\
jetzt neigen mein Gemüte?\\
Was ist das Beste, das ihr liebt?\\
Was gibt die Erde, wenn sie gibt,\\
als Angst und bittre Schmerzen?\\
Was ist das güldne Gut und Geld?\\
Was bringt der Schein und Pracht der Welt\\
als Kummer eurer Herzen?

\flagverse{5.} Was ist der großen Leute Gunst\\
als Zunder großes Neides?\\
Was ist das Wissen vieler Kunst\\
als Ursprung vieles Leides?\\
Denn wer viel weiß, der grämt sich viel,\\
und welcher andre lehren will,\\
muß leiden und viel tragen.\\
Seht alles an, Ruhm, Lob und Ehr,\\
habt Freud und Lust, was habt ihr mehr\\
als endlich Weh und Klagen?

\flagverse{6.} Nichts ist so schön und wohl bestellt,\\
da man hier wohl auf stehe,\\
drum nimmt Gott, was ihm wohlgefällt,\\
bei Zeiten in die Höhe\\
und setzet es in seinen Schoß;\\
da ist es allen Kummers los,\\
darf nicht, wie ihr, sich kränken,\\
die ihr oft denket, wie doch wohl\\
dies oder jenes werden soll,\\
und könnets nicht erdenken.

\flagverse{7.} Wer selig stirbt, der schleußet zu\\
die schwarzen Jammertore,\\
hingegen schwingt er sich zur Ruh\\
im güldnen Engelchore,\\
legt Aschen weg, kriegt Freudenöl,\\
zeucht aus das Fleisch und schmückt die Seel\\
in reiner weißer Seiden;\\
er läßt die Erd und nimmet ein\\
die Lust, da Christi Schäfelein\\
in lauter Rosen weiden.

\flagverse{8.} So gebt, ihr Liebsten, euch doch schlecht\\
dahin in Gottes Willen;\\
sein Rat ist gut, sein Tun ist recht\\
und wird wohl wieder stillen\\
den Schmerzen, den er euch gemacht.\\
Und hiemit sei euch gute Nacht\\
von eurem Sohn gegönnet.\\
Es kommt die Zeit, da mich und euch\\
vereingen wird in seinem Reich,\\
der euch und mich getrennet.

\end{verse}
\end{multicols}
%\attrib{\small{THZE}}

\begin{center}
\settowidth{\versewidth}{Der, vor dem die Welt erschrickt,}
\begin{verse}[\versewidth]


\flagverse{9.} Da will ich eure Treu und Müh\\
und was ihr eurem Kranken\\
erwiesen habt, im Himmel hie,\\
sobald ihr kommt, verdanken.\\
Ich will erzählen, wie ihr habt\\
euch selbst betrübt und mich gelabt,\\
vor Christo und vor allen;\\
und für den heißen Tränenfluß\\
will ich mit mehr als einem Kuß\\
um euren Hals euch fallen.

\end{verse}
\end{center}


\index{Mein herzer Vater, weint ihr noch}
\newpage
\subsection*{\centerline{Nun sei getrost und unbetrübt}}
\addcontentsline{toc}{subsection}{Nun sei getrost und unbetrübt}
%StartInfo%%%%%%%%%%%%%%%%%%%%%%%%%%%%%%%%%%%%%%%%%%%%%%%%%%%%%%%%%%%%%%%%%%%%
%  Autor:
%  Titel:
%  File:
%  Ref:
%  Mod:
%EndInfo%%%%%%%%%%%%%%%%%%%%%%%%%%%%%%%%%%%%%%%%%%%%%%%%%%%%%%%%%%%%%%%%%%%%%%
%\poemtitle{pt}
\begin{multicols}{2}
\settowidth{\versewidth}{Erschrecke nicht vor deinem End,}
\begin{verse}[\versewidth]
%nun sei getrost und unbetrübt\\
%auf den Tod der Regina Leyser, geb. Calow, in Wittenberg (1664)

\flagverse{1.} Nun sei getrost und unbetrübt,\\
du mein Geist und Gemüte!\\
Dein Jesus lebt, der dich geliebt\\
eh, als dir dein Geblüte\\
und Fleisch und Haut ward zugericht;\\
der wird dich auch gewißlich nicht\\
an deinem Ende hassen.

\flagverse{2.} Erschrecke nicht vor deinem End,\\
es ist nichts Böses drinnen;\\
dein lieber Herr streckt seine Händ\\
und fordert dich von hinnen\\
aus soviel tausend Angst und Qual,\\
die du in diesem Jammertal\\
bisher hast ausgestanden.

\flagverse{3.} Zwar heißts ja Tod und Sterbensnot,\\
doch ist da gar kein Sterben;\\
denn Jesus ist des Todes Tod\\
und nimmt ihm das Verderben,\\
daß alle seine Stärk und Kraft\\
mir, wenn ich jetzt werd hingerafft,\\
nicht auf ein Härlein schade.

\flagverse{4.} Des Todes Kraft steht in der Sünd\\
und schnöden Missetaten,\\
darin ich armes Adamskind\\
so oft und viel geraten;\\
nun ist die Sünd in Jesu Blut\\
ersäuft, erstickt, getilgt und tut\\
fort gar nichts mehr zur Sachen.

\flagverse{5.} Die Sünd ist hin und ich bin rein;\\
trotz dem, der mir das nehme!\\
Hinfüro ist das Leben mein,\\
darf nicht, daß ich mich gräme\\
um einger Sünden Lohn und Sold;\\
wer ausgesöhnt, dem ist man hold\\
und tut ihm nichts zuwider.

\flagverse{6.} Ei nun, so nehm ich Gottes Gnad\\
und alle seine Freude\\
mit mir auf meinen letzten Pfad\\
und weiß von keinem Leide.\\
Der wilde Feind muß nur ein Schaf,\\
sein Ungestüm ein süßer Schlaf\\
und sanfte Ruhe werden.

\flagverse{7.} Du Jesu, allerliebster Freund,\\
bist selbst mein Licht und Leben:\\
Du hältst mich fest, und kann kein Feind\\
dich, wo du stehest, heben.\\
In dir steh ich, und du in mir;\\
und wie wir stehn, so bleiben wir\\
hier und dort ungeschieden.

\flagverse{8.} Mein Leib, der legt sich hin zur Ruh,\\
als der fast müde worden;\\
die Seele fährt dem Himmel zu\\
und mischt sich in den Orden\\
der auserwählten Gottesschar\\
und hält das ewge Jubeljahr\\
mit allen heilgen Engeln.

\flagverse{9.} Kommt dann der Tag, o höchster Fürst\\
der Kleinen und der Großen,\\
da du zum allerletzten wirst\\
in die Posaunen stoßen,\\
so soll denn Seel und Leib zugleich\\
mit dir in deines Vaters Reich\\
zu deiner Freud eingehen.

\flagverse{10.} Ists nun dein Will, so stell dich ein,\\
mich selig zu versetzen.\\
Ach, ewig bei und mit dir sein,\\
wie hoch muß das ergötzen!\\
Eröffne dich, du Todespfort,\\
auf daß an solchen schönen Ort\\
ich durch dich möge fahren!

\end{verse}
\end{multicols}
%\attrib{\small{THZE}}

\index{Nun sei getrost und unbetrübt}
\newpage
\subsection*{\centerline{O, wie so ein großes Gut}}              %witt:? --> Sterben
\addcontentsline{toc}{subsection}{O wie so ein großes Gut}              %witt:? --> Sterben}
%StartInfo%%%%%%%%%%%%%%%%%%%%%%%%%%%%%%%%%%%%%%%%%%%%%%%%%%%%%%%%%%%%%%%%%%%%
%  Autor:
%  Titel:
%  File:
%  Ref:
%  Mod:
%EndInfo%%%%%%%%%%%%%%%%%%%%%%%%%%%%%%%%%%%%%%%%%%%%%%%%%%%%%%%%%%%%%%%%%%%%%%
%\poemtitle{pt}
\begin{multicols}{2}
\settowidth{\versewidth}{Nun, der Gott, der sie gekränkt,}
\begin{verse}[\versewidth]
%o, wie so ein großes Gut ist es doch, im Frieden scheiden\\
%auf den Tod der Frau Ursula von der Linden (1661)

\flagverse{1.} O, wie so ein großes Gut\\
ist es doch, im Frieden scheiden\\
und mit wohlvergnügtem Mut\\
in Geduld den Tod erleiden!\\
Lasset uns loben, was jeder nur weiß:\\
Seliges Sterben hat dennoch den Preis.

\flagverse{2.} Dieses Gut, das herrlich prangt,\\
hat aus Gottes Hand und Throne,\\
mein Herr Linde, wohl erlangt\\
eures Hauses Ehr und Krone.\\
Ihre Begierde nach himmlischer Au\\
ist ihr erfüllet, der seligen Frau.

\flagverse{3.} Sie hat ja des Kreuzes Joch\\
auch zuweilen wohl genossen:\\
Wie gekränket war sie doch,\\
da ihr Berkow ward erschossen,\\
Berkow, das feine, geschickte Gemüt,\\
dessen Gedächtnis noch immerzu blüht!

\flagverse{4.} Nun, der Gott, der sie gekränkt,\\
hat sie wieder auch erfreuet\\
und euch ihr zum Mann geschenkt,\\
welches euch noch nie gereuet.\\
Jetzo genießt sie der ewigen Ehr\\
in Gottes Reiche. Was will sie doch mehr?

\end{verse}
\end{multicols}
%\attrib{\small{THZE}}

\index{O, wie so ein großes Gut}
\newpage
\subsection*{\centerline{O Tod, o Tod, du greulichs Bild}}
\addcontentsline{toc}{subsection}{O Tod o Tod du greulichs Bild}
%StartInfo%%%%%%%%%%%%%%%%%%%%%%%%%%%%%%%%%%%%%%%%%%%%%%%%%%%%%%%%%%%%%%%%%%%%
%  Autor:
%  Titel:
%  File:
%  Ref:
%  Mod:
%EndInfo%%%%%%%%%%%%%%%%%%%%%%%%%%%%%%%%%%%%%%%%%%%%%%%%%%%%%%%%%%%%%%%%%%%%%%
%\poemtitle{pt}
\begin{multicols}{2}
\settowidth{\versewidth}{Ich weiß, daß dir zerschlagen ist}
\begin{verse}[\versewidth]
%o Tod, o Tod, du greulichs Bild\\
%nach Paul Röbers »O Tod, o Tod, schreckliches Bild«

\flagverse{1.} O Tod, o Tod, du greulichs Bild\\
und Feind voll Zorns und Blitzen,\\
wie machst du dich so groß und wild\\
mit deiner Pfeile Spitzen?\\
Hier ist ein Herz, das dich nicht acht\\
und spottet deiner schnöden Macht\\
und der zerbrochnen Pfeile.

\flagverse{2.} Komm nur mit deinem Bogen bald\\
und ziele mir zum Herzen;\\
in deiner seltsamen Gestalt\\
versuchs mit Pein und Schmerzen:\\
Was wirst du damit richten aus?\\
Ich werde dir doch aus dem Haus\\
einmal gewiß entlaufen.

\flagverse{3.} Ich weiß, daß dir zerschlagen ist\\
dein Schloß und seine Riegel\\
durch meinen Heiland Jesum Christ;\\
der brach des Grabes Siegel\\
und führte dich zum Siegesschau,\\
auf daß uns nicht mehr vor dir grau;\\
ein Spott ist aus dir worden.

\flagverse{4.} Besiehe deinen Palast wohl\\
und deines Reiches Wesen,\\
obs noch anitzo sei so voll\\
als es zuvor gewesen:\\
Ist Moses nicht aus deiner Hand\\
entwischt und im gelobten Land\\
auf Tabor schön erschienen?

\flagverse{5.} Wo ist der alten Heilgen Zahl,\\
die auch daselbst begraben?\\
Sie sind erhöht im Himmelssaal,\\
da sie sich ewig laben.\\
Des starken Jesus Heldenhand\\
hat dir zersprengt all deine Band,\\
als er dein Kämpfer wurde.

\flagverse{6.} Was solls denn nun, o Jesu, sein,\\
daß mich der Tod so schrecket?\\
Hat doch Elisa Totenbein,\\
was tot war, auferwecket:\\
Viel mehr wirst du, den Trost hab ich,\\
zum Leben kräftig rüsten mich,\\
drum schlaf ich ein mit Freuden.
   
\end{verse}
\end{multicols}
%\attrib{\small{THZE}}

\index{O Tod, du greulichs Bild}
\newpage
\subsection*{\centerline{So geht der alte liebe Herr nun auch dahin}}    %witt:? --> Sterben
\addcontentsline{toc}{subsection}{So geht der alte liebe Herr nun auch dahin}    %witt:? --> Sterben}
%StartInfo%%%%%%%%%%%%%%%%%%%%%%%%%%%%%%%%%%%%%%%%%%%%%%%%%%%%%%%%%%%%%%%%%%%%
%  Autor:
%  Titel:
%  File:
%  Ref:
%  Mod:
%EndInfo%%%%%%%%%%%%%%%%%%%%%%%%%%%%%%%%%%%%%%%%%%%%%%%%%%%%%%%%%%%%%%%%%%%%%%
%\poemtitle{pt}
%\begin{multicols}{2}
\settowidth{\versewidth}{Die Kinder klagen ihn, ach Vater, unser Schutz!}
\begin{verse}[\versewidth]
%so geht der alte liebe Herr nun auch dahin\\
%»Auff das selige Absterben und Christliche Beerdigung des umb diese gantze Stadt viel Jahr lang wolverdienten Herrn Bürgermeisters, Herrn\\
%benedicti Reichardts« (Berlin 1667)

\flagverse{1.} So geht der alte liebe Herr nun auch dahin:\\
Nachdem er achtzig und was drüber ist erlebet.\\
Er geht zu Gott: Und legt und schlägt aus seinem Sinn\\
das, was noch, wies Gott weiß, uns überm Haupte schwebet.

\flagverse{2.} Die Kinder klagen ihn, ach Vater, unser Schutz!\\
Die Ehgenossin läßt die Tränen häufig fließen.\\
Was Kindeskinder sind, bedenken, was für Nutz\\
sie hiebevor gehabt und nun nicht mehr genießen.

\flagverse{3.} Und weinen bitterlich. Die werte Bürgerschaft\\
folgt ihrem Haupte nach und gibt ihm das Geleite\\
zu seinem Schlafgemach, dahin der Tod ihn rafft\\
gleich wie uns allzumal. Ich aber setz ihm heute

\flagverse{4.} zu Ehren diese Schrift: Ein Mann von alter Treu\\
und deutscher Redlichkeit, ein Mann von vielen Gaben\\
und großer Wissenschaft, ein Mann, der frisch und frei\\
das Recht geschützt, die Stadt regiert, wird jetzt begraben.\\

%»Zur Bezeugung Christlichen Mitleidens Gegen die gesambte Hochbetrübte Leidtragende setzte dieses Paulus Gerhardt.«

\end{verse}
%\end{multicols}
%\attrib{\small{Auff das selige Absterben und Christliche Beerdigung\\
%des umb diese gantze Stadt viel Jahr lang wolverdienten Herrn Bürgermeisters,\\ Herrn Benedicti Reichardts« (Berlin 1667)}}

\index{So geht der alte liebe Herr}
\newpage
\subsection*{\centerline{Was trauerst du, mein Angesicht?}}
\addcontentsline{toc}{subsection}{Was trauerst du mein Angesicht?}
%StartInfo%%%%%%%%%%%%%%%%%%%%%%%%%%%%%%%%%%%%%%%%%%%%%%%%%%%%%%%%%%%%%%%%%%%%
%  Autor:
%  Titel:
%  File:
%  Ref:
%  Mod:
%EndInfo%%%%%%%%%%%%%%%%%%%%%%%%%%%%%%%%%%%%%%%%%%%%%%%%%%%%%%%%%%%%%%%%%%%%%%
%\poemtitle{pt}
\begin{multicols}{2}
\settowidth{\versewidth}{Ja Herr, du tratst ihm an das Herz,}
\begin{verse}[\versewidth]

\flagverse{1.} Was trauerst du, mein Angesicht,\\
wann du den Tod hörst nennen?\\
Sei ohne Furcht: er schadt dir nicht,\\
lern ihn nur recht erkennen.\\
Kennst du den Tod,\\
so hats nicht Not,\\
all Angst wird sich zertrennen.

\flagverse{2.} Vors erste, zeuch die Larven ab\\
der alten roten Schlangen;\\
sieh an, daß sie kein Gift mehr hab,\\
es ist ihr abgefangen\\
durch Jesum Christ,\\
der vor uns ist\\
ins Grab und Tod gegangen.

\flagverse{3.} Ja Herr, du tratst ihm an das Herz,\\
brachst seines Stachels Spitzen;\\
nunmehr ist er ein lauter Scherz\\
und kann uns gar nicht ritzen;\\
dein edles Blut\\
dämpft seine Glut,\\
dein Flammen zwingt sein Hitzen.

\flagverse{4.} Die Sünde war des Todes Kraft,\\
die uns zum Sterben triebe,\\
nun ist die Sünd all abgeschafft\\
durch Christi Treu und Liebe;\\
ihr Ernst und Macht\\
ist matt gemacht;\\
trotz, daß sie uns betrübe.

\flagverse{5.} Die Sünd ist tot, Gott ist versöhnt,\\
durch seines Sohnes Dulden,\\
der Grimm ist hin, den wir verdient\\
mit unsers Lebens Schulden;\\
der vor war Feind,\\
ist nunmehr Freund\\
voll süßer Gnad und Hulden.

\flagverse{6.} Bist du denn Freund, so kannst du mich,\\
mein Gott, ja nicht umbringen;\\
dein Vaterherze lässet sich\\
zum Mord und Tod nicht dringen.\\
Wer sich befindt\\
dein Erb und Kind,\\
ist frei von bösen Dingen.

\flagverse{7.} Das aber, Vater, tust du wohl,\\
wann uns die Trübsal kränket,\\
wann wir des Lebens satt und voll\\
des Jammers, der uns tränket,\\
daß dann dein Hand\\
ans Vaterland\\
uns aus den Fluten lenket.

\flagverse{8.} Wann sich das starke Wetter regt,\\
davon die Höhen fallen,\\
wann deines Zornes Donner schlägt,\\
daß Berg und Tal erschallen:\\
So trittst du zu\\
und bringst zur Ruh\\
uns, die dir wohlgefallen.

\flagverse{9.} Wann unsre Feinde um uns her\\
uns bringen in die Mitten,\\
wann Ottern, Löwen, Wölf und Bär\\
ihr Gift auf uns ausschütten:\\
Nimmst du dein Schaf,\\
bringt's in den Schlaf\\
bei dir in deiner Hütten.

\flagverse{10.} Wann diese Welt gibt bösen Lohn\\
dem, der dich treulich ehret,\\
so sprichst du: Komm zu mir, mein Sohn,\\
hier hab ich, was dich nähret:\\
Lust, Ehr und Freud,\\
die keine Zeit\\
in Ewigkeit verzehret.

\flagverse{11.} Alsbald schließt uns der Engel Schar\\
mit Freud in ihrem Bogen\\
und nehmen unsrer Seele wahr,\\
die, wann sie ausgeflogen,\\
in ihre Hut\\
mit stillem Mut\\
zu Gott kommt angezogen.

\flagverse{12.} Der Herr empfänget seine Braut\\
und spricht: Sei mir willkommen!\\
Du bists, die ich mir anvertraut,\\
komm, wohne bei den Frommen,\\
die ich vor dir\\
anher zu mir\\
aus jener Welt genommen.

\flagverse{13.} Du hast behalten Glaub und Treu\\
im Herzen, da ich wohne:\\
So geb und leg ich dir nun bei\\
die schöne Freudenkrone.\\
Ich bin dein Heil,\\
dein Erb und Teil,\\
tritt her zu meinem Throne.

\flagverse{14.} Hier trockn ich deiner Augen Flut,\\
hier still ich deine Tränen,\\
hier setzt sich in dem höchsten Gut\\
dein Seufzen, Klag und Sehnen;\\
dein Jammermeer\\
wird niemand mehr,\\
als nur in Freud, erwähnen.

\flagverse{15.} Hier kleid ich meiner Christen Zahl\\
mit reiner weißer Seide;\\
hier springen sie im Himmelssaal,\\
und ist nicht, der sie neide;\\
hier ist kein Tod,\\
kein Kreuz und Not,\\
das gute Freunde scheide.

\flagverse{16.} Ach, Gott mein Herr, was will ich doch\\
mich vor dem Tode scheuen?\\
Er ists ja, der mich von dem Joch\\
des Elends will befreien:\\
Er nimmt mich aus\\
dem Marterhaus,\\
das kann mich nicht gereuen.

\flagverse{17.} Der Tod, der ist mein Rotes Meer,\\
dadurch auf trocknem Sande\\
dein Israel, das fromme Heer,\\
geht zum Gelobten Lande,\\
da Milch und Wein\\
stets fleußt herein\\
wie Ström in ihrem Rande.

\flagverse{18.} Er ist das güldne Himmelstor\\
und des Eliä Wagen,\\
darauf mich Gott zum Engelchor\\
gar bald wird lassen tragen,\\
wann er, der Letzt\\
und Erste, setzt\\
ein End an meinen Tagen.

\flagverse{19.} O süße Lust, o edle Ruh,\\
o frommer Seelen Freude,\\
komm, schleuß mir meine Augen zu,\\
daß ich mit Fried abscheide\\
hin, da mein Hirt\\
mich leiten wird\\
zur immergrünen Weide.

\flagverse{20.} Daselbst wird er mit vollem Maß,\\
was hier gefehlt, einbringen;\\
dafür wird ihm ohn Unterlaß\\
sein Halleluja klingen,\\
das will auch ich\\
ihm williglich\\
eins nach dem andern singen.

\end{verse}
\end{multicols}
\attrib{\small{THZE}}

\index{Was trauerst du, mein Angesicht?}
\newpage
\subsection*{\centerline{Weint, und weint gleichwohl nicht zu sehr}}     %witt:?--> Sterben
\addcontentsline{toc}{subsection}{Weint und weint gleichwohl nicht zu sehr}     %witt:?--> Sterben}
%StartInfo%%%%%%%%%%%%%%%%%%%%%%%%%%%%%%%%%%%%%%%%%%%%%%%%%%%%%%%%%%%%%%%%%%%%
%  Autor:
%  Titel:
%  File:
%  Ref:
%  Mod:
%EndInfo%%%%%%%%%%%%%%%%%%%%%%%%%%%%%%%%%%%%%%%%%%%%%%%%%%%%%%%%%%%%%%%%%%%%%%
%\poemtitle{pt}
\begin{multicols}{2}
\settowidth{\versewidth}{Weint, und weint gleichwohl nicht zu sehr,}
\begin{verse}[\versewidth]
%weint, und weint gleichwohl nicht zu sehr\\
%auf den Tod der kleinen Margaretha Zarlang\\
%an die Eltern (1667)

\flagverse{1.} Weint, und weint gleichwohl nicht zu sehr,\\
denn was euch abgestorben,\\
ist wohl daran und hat nunmehr\\
das beste Teil erworben!\\
Es ist hindurch ins Vaterland,\\
nachdem der harte schwere Stand,\\
der hier war, überstanden.

\flagverse{2.} Hier sind wir auf der wilden See\\
im Sturm und tiefen Fluten,\\
da gehts uns, daß vor Ach und Weh\\
das Herze möchte bluten.\\
Sobald der Mensch ins Leben tritt,\\
sobald kommt auch die Trübsal mit\\
und folgt ihm auf dem Fuße.

\flagverse{3.} Da ist kein Kind so zart und klein,\\
es muß sein Leiden tragen;\\
ein jedes hat sein Angst und Pein,\\
kanns oft nicht von sich sagen;\\
und wenns auch gleich noch etwas spricht,\\
so bleibt doch drum das Elend nicht\\
von seines Leibes Gliedern.

\flagverse{4.} Kommts auf die Bein und wächst herzu,\\
lernt schwarz und weiß verstehen,\\
so merkts, was man auf Erden tu,\\
wie Menschenwerke gehen,\\
sieht lauter Böses, gar nichts Guts,\\
darüber wird betrübtes Muts\\
und fängt sich an zu grämen.

\flagverse{5.} Hilft endlich Gott zur vollen Kraft\\
und reifen Mannesjahren,\\
tritts in den Stand, da man was schafft,\\
da kanns denn recht erfahren,\\
wie alles so voll Mühe sei;\\
und hat doch selten mehr dabei\\
als wenig gute Stunden.

\flagverse{6.} Das alles sieht der Vater an,\\
die Mutter nimmts zu Herzen,\\
und niemand ist, der helfen kann;\\
da kommen denn die Schmerzen,\\
die häufen sich ohn Unterlaß\\
und halten stets die Augen naß\\
bei Eltern und bei Kindern.

\flagverse{7.} Drum laßts Gott machen, wie er will!\\
Er weiß die besten Weisen.\\
Wer balde kommt zu seinem Ziel,\\
der darf nicht ferne reisen;\\
und wer bei Zeit wird ausgespannt,\\
der darf des Jammers schweren Stand\\
nicht allzu lange ziehen.

\flagverse{8.} Was unser Welt ist zugedacht,\\
darf euer Kind nicht schmecken;\\
es schläft und ruht, bis Gottes Macht\\
es wieder wird erwecken.\\
Und wann ihr kommt ins Himmels Saal,\\
so wird euch eurer Kinder Zahl\\
mit großer Lust empfangen.
\end{verse}
\end{multicols}

\begin{center}
\settowidth{\versewidth}{Der, vor dem die Welt erschrickt,}
\begin{verse}[\versewidth]


\flagverse{9.} So schlaf nun wohl, du herzes Kind,\\
doch tröste Gott die Deinen,\\
wann jetzt ihr Herz und Auge rinnt,\\
und kehr ihr bittres Weinen\\
zu seiner Zeit, die er bestellt,\\
auf Weis und Art, die ihm gefällt,\\
in Freud und süßes Singen.

  
\end{verse}
\end{center}

%\attrib{\small{THZE}}

\index{Weint und weint gleichwohl}
\newpage
\subsection*{\centerline{Wer selig stirbt, stirbt nicht}}                %witt:? --> Sterben
\addcontentsline{toc}{subsection}{Wer selig stirbt stirbt nicht}                %witt:? --> Sterben}
%StartInfo%%%%%%%%%%%%%%%%%%%%%%%%%%%%%%%%%%%%%%%%%%%%%%%%%%%%%%%%%%%%%%%%%%%%
%  Autor:
%  Titel:
%  File:
%  Ref:
%  Mod:
%EndInfo%%%%%%%%%%%%%%%%%%%%%%%%%%%%%%%%%%%%%%%%%%%%%%%%%%%%%%%%%%%%%%%%%%%%%%
%\poemtitle{pt}
\begin{multicols}{2}

\settowidth{\versewidth}{Und soll kein Kreuz, kein Schmerz, kein Leiden,}
\begin{verse}[\versewidth]
%wer selig stirbt, stirbt nicht\\
%auf den Tod des Rats Joh. Adam Preunel, gestorben in Berlin 1668, dessen letztes Wort war: Ego sum in Christo, et Christus est in me

\flagverse{1.} Wer selig stirbt, stirbt nicht!\\
Ein guter Tod gedeiht zum Leben\\
und macht die Seel in Freuden schweben\\
für Gottes Angesicht.\\
Laß alles fallen und vergehen,\\
wer Christo stirbt, bleibt ewig stehen.

\flagverse{2.} Da fehlts oft vielen an;\\
Herrn Preunel aber ists gelungen,\\
der hat mit Christo durchgedrungen,\\
ist nun sehr herrlich dran.\\
In Christo, sprach er, sei mein Ende,\\
dem geb ich mich in seine Hände.

\flagverse{3.} Herr Jesu, du bist mein!\\
Du hast dich selber mir geschenket.\\
Auch bin ich dir ganz eingesenket\\
und leb und sterbe dein.\\
Und soll kein Kreuz, kein Schmerz, kein Leiden,\\
ja uns soll auch der Tod nicht scheiden.

\flagverse{4.} Und damit ging er hin!\\
Heißt das nun nicht recht selig sterben?\\
Wer kann doch immermehr verderben\\
bei so gestaltem Sinn?\\
Wer hier in Christo wohl gewesen,\\
wird dort bei Christo wohl genesen.

\end{verse}
\end{multicols}

\begin{center}
\settowidth{\versewidth}{Der, vor dem die Welt erschrickt,}
\begin{verse}[\versewidth]

\flagverse{5.} Drum weinet nicht zu viel,\\
ihr, die Herr Preunel hat geliebet;\\
denn der, an dem ihr euch betrübet,\\
hat sein erwünschtes Ziel.\\
Laßt vielmehr diesen Seufzer hören:\\
Gott woll auch uns so sterben lehren!
  
\end{verse}
\end{center}



%\attrib{\small{THZE}}

\index{Wer selig stirbt, stirbt nicht}
\newpage

\backmatter% ------------------------------------------------------------------------
%\backmatter% --------------------------------------------------------------------
\section*{Anmerkungen zu den einzelnen Gedichten}
\section*{\centerline{***Noch nicht zugeordnet***}}


Paul Gerhardt hat seine Gedichte selbst nicht gesammelt herausgegeben.
Sofern sie nicht als Einzeldrucke publiziert wurden, erschienen sie
zumeist erstmals in den verschiedenen Ausgaben des von Johann Crüger
herausgegebenen Gesangbuchs \frqq Praxis pietatis melica\flqq , Berlin bei
Christoph Runge 1648, 1653, 1656 und 1661. Noch zu Gerhardts Lebzeiten
erschien außerdem eine Gesamtausgabe seiner Gedichte in zehn Lieferungen
zu je zwölf Gedichten, die von Johann Georg Ebeling herausgegeben wurde
(\frqq Geistliche Andachten\flqq , Berlin 1666/67). Inwieweit Gerhardt an dieser
Ausgabe mitgewirkt hat, ist nicht bekannt. – Die vorliegende Sammlung
umfaßt alle deutschen Gedichte Paul Gerhardts.
Vollständige Neuausgabe mit einer Biographie des Autors. Herausgegeben von Karl-Maria Guth. Berlin 2013.
Textgrundlage ist die Ausgabe: Paul Gerhardt: Dichtungen und Schriften.
Herausgegeben und textkritisch durchgesehen von Eberhard von Cranach-Sichart, München: Paul Müller, 1957. 

\subsection*{Wie soll ich dich empfangen}

Erstdruck 1653.

\subsection*{ Warum willst du draußen stehen}

Erstdruck 1653.

\subsection*{ Wir singen dir, Immanuel}

Erstdruck 1653.

\subsection*{ O Jesu Christ, dein Kripplein ist}

Erstdruck 1653.

\subsection*{ Fröhlich soll mein Herze springen}

Erstdruck 1653.

\subsection*{ Ich steh an deiner Krippen hier}

Erstdruck 1653.

\subsection*{ Schaut, schaut, was ist für Wunder dar?}

Erstdruck 1666 unter dem Titel \frqq Christ- Nacht-Liedlein\flqq .

\subsection*{ Kommt und laßt uns Christum ehren}

Erstdruck 1666 unter dem Titel \frqq Weihnacht- Gesang\flqq .

\subsection*{ Alle, die ihr, Gott zu ehren}

Entstanden früh. Erstdruck 1666 unter dem Titel \frqq Christ-Wiegenlied\flqq .

\subsection*{ Nun laßt uns gehn und treten}

Entstanden im 30-jähr. Krieg. Erstdruck 1653.

\subsection*{ Warum machet solche Schmerzen}

Erstdruck 1648.

\subsection*{ Ein Lämmlein geht und trägt die Schuld}

Erstdruck 1648.

\subsection*{ O Welt, sieh hier dein Leben}

Erstdruck 1648.

\subsection*{ O Mensch, beweine deine Sünd}

Erstdruck 1648 unter dem Titel \frqq Die Passion aus den vier Evangelisten\flqq .

\subsection*{ Siehe, mein getreuer Knecht}

Erstdruck 1653 unter dem Titel \frqq Das 53. Kapitel Esaiä\flqq .

\subsection*{ Hör an, mein Herz, die sieben Wort}

Erstdruck 1653.

\subsection*{ Als Gottes Lamm und Leue}

Erstdruck 1653.

\subsection*{ Passions-Salve an die leidenden Glieder Christi}

\subsection*{ 1. An die Füße - Sei mir tausendmal gegrüßet}

Erstdruck 1653. Nach einem Zyklus des Bernard de Clairvaux: \frqq Rhythmica
oratio ad unum quodlibet membrorum Christi patientis et a cruce
pendentis. 1. Salve, mundi salutare\flqq .

\subsection*{ 2. An die Knie - Gegrüßet seist du, meine Kron}

Erstdruck 1653 unter dem Titel \frqq An die leydende Knie des Herrn Christi\flqq .
Original: \frqq 2. Salve Jesu, rex sanctorum\flqq .

\subsection*{ 3. An die Hände - Sei wohl gegrüßet, guter Hirt}

Erstdruck 1653 unter dem Titel \frqq An die leydende Hände des Herrn
Christi\flqq . Original: \frqq 3. Salve Jesu, pastor bone\flqq .

\subsection*{ 4. An die Seite - Ich grüße dich, du frömmster Mann}

Erstdruck 1653 unter dem Titel \frqq An die leydende Seite des Herrn
Christi\flqq . Original: \frqq 4. Salve Jesu, summe bonus\flqq .

\subsection*{ 5. An die Brust - Gegrüßet seist du, Gott mein Heil}

Erstdruck 1653 unter dem Titel \frqq An die leydende Brust des Herrn
Christi\flqq . Original: \frqq 5. Salve, salus mea, Deus\flqq .

\subsection*{ 6. An das Herz - O Herz des Königs aller Welt}

Erstdruck 1653 unter dem Titel \frqq An das leydende Herz des Herrn Christi\flqq .
Original: \frqq 6. Summi Regis, cor, aveto\flqq .

\subsection*{ 7. An das Angesicht - O Haupt voll Blut und Wunden}

Erstdruck 1653 unter dem Titel \frqq An das leydende Angesicht Jesu Christi\flqq .
Original: \frqq 7. Salve, caput cruentatum\flqq .

\subsection*{ Also hat Gott die Welt geliebt}

Amazon.de Widgets 

Erstdruck 1661.

\subsection*{ Auf auf, mein Herz, mit Freuden}

Erstdruck 1648.

\subsection*{ Nun freut euch hier und überall}

Erstdruck 1653 unter dem Titel \frqq Die Auferstehung Christi\flqq .

\subsection*{ Sei fröhlich alles weit und breit}t

Erstdruck 1653.

\subsection*{ Zeuch ein zu deinen Toren}

Entstanden vor 1648. Erstdruck 1653.

\subsection*{ O du allersüß'ste Freude}

Erstdruck 1648.

\subsection*{ Gott Vater, sende deinen Geist}

Erstdruck 1648.

\subsection*{ Was alle Weisheit in der Welt}

Erstdruck 1653.

\subsection*{ Du Volk, das du getaufet bist}

Erstdruck 1667 unter dem Titel \frqq Von der heiligen Taufe\flqq .

\subsection*{ Herr Jesu, meine Liebe}

Erstdruck 1667 unter dem Titel \frqq Vom heiligen Abendmahl\flqq .

\subsection*{ Wach auf, mein Herz, und singe!}

Erstdruck 1648.

\subsection*{ Lobet den Herren alle, die ihn fürchten!}

Erstdruck 1653.

\subsection*{ Die güldne Sonne}

Erstdruck 1666 unter dem Titel \frqq Morgen- Segen\flqq .

\subsection*{ Nun ruhen alle Wälder}

Erstdruck 1648.

\subsection*{ Der Tag mit seinem Lichte}

Erstdruck 1666 unter dem Titel \frqq Abend- Segen\flqq .

\subsection*{ Geh aus, mein Herz, und suche Freud}

Erstdruck 1653.

\subsection*{ O Herrscher in dem Himmelszelt}

Erstdruck 1666 unter dem Titel \frqq Buß- und Bet-Gesang bei unzeitiger Nässe
und betrübtem Gewitter\flqq .

\subsection*{ Nun ist der Regen hin}

Erstdruck 1653 unter dem Titel \frqq Danklied vor einen gnädigen
Sonnenschein\flqq .

\subsection*{ Nun geht frisch drauf, es geht nach Haus}

Erstdruck 1653 unter dem Titel \frqq Danklied nach der Reise\flqq .

\subsection*{ Der aller Herz und Willen lenkt}

Erstdruck 1643. \frqq Oda: Hochzeitsgedicht für Joachim Fromm und Sabina
Barthold 1643\flqq .

\subsection*{ Ein Weib, das Gott den Herren liebt}

Erstdruck 1653 unter dem Titel \frqq Frauen- Lob. Aus den Sprüchen Salomonis
am 31. Kap.\flqq .

\subsection*{ Voller Wunder, voller Kunst}

Erstdruck 1666 unter dem Titel \frqq Der Wunder volle Ehestand\flqq .

\subsection*{ Wie schön ists doch, Herr Jesu Christ}

Erstdruck 1666 unter dem Titel \frqq Trost-Gesang christlicher Eheleute\flqq .

\subsection*{ Unter allen, die da leben}

Am Schluß der \frqq Vier geistlichen Lieder\flqq  von Joachim Pauli (Berlin o.J.,
wahrscheinlich 1665) von Paul Gerhardt angefügt \frqq Zur Bezeugung guter
Zuneigung gegen den Authore\flqq .

\subsection*{ Tapfere Leute soll man loben}

Aus \frqq Fünfzehn-ästiger Nieder-Lausitzer Palm-Baum\flqq  des Samuel Sturm,
1675.

\subsection*{ Herr, ich will gar gerne bleiben}

Erstdruck 1667 unter dem Titel \frqq Wahre Erniedrigung sein selbsten aus dem
Matthäo am 15.V.27. Ja Herr, aber doch essen die Hündlein von den
Brosamen, die von ihrer Herren Tische fallen\flqq . Nach den lat. Distichen
des Nathan Chyträus \frqq Sum canis indignus\flqq , 1568.

\subsection*{ Weg, mein Herz, mit den Gedanken}

Erstdruck 1648.

\subsection*{ Herr, höre, was mein Mund}

Erstdruck 1648.

\subsection*{ Nach dir, o Herr, verlanget mich}

Erstdruck 1648 unter dem Titel \frqq Der 25. Psalm\flqq .

\subsection*{ Zweierlei bitt ich von dir}

Erstdruck 1648.

\subsection*{ O Gott, mein Schöpfer, edler Fürst}

Erstdruck 1648 unter dem Titel \frqq Syrachs Gebätlein umb ein züchtiges und
mäßiges Leben\flqq .

\subsection*{ Ich erhebe, Herr, zu dir}

Erstdruck 1648 unter dem Titel \frqq Der 121. Psalm\flqq .

\subsection*{ Weltskribenten und Poeten}

Vorspruch zu Michael Schirmer, \frqq Bibl. Lieder und Lehrsprüche\flqq , Berlin
1650.

\subsection*{ Ich weiß, mein Gott, daß all mein Tun}

Erstdruck 1653.

\subsection*{ Ich danke dir demütiglich}

Erstdruck 1653. Nach Johann Arnds \frqq Paradies-gärtlein\flqq , Goslar 1621, III,
17: \frqq Gebet um zeitliche und ewige Wohlfahrt\flqq .

\subsection*{ O Jesu Christ, mein schönstes Licht}

Erstdruck 1653 unter dem Titel \frqq Um die Liebe Christi. Aus Herrn Johann
Arnds Gebät\flqq  (nach \frqq Paradiesgärtlein\flqq , Goslar 1621, II, 5: \frqq Gebet um die
Liebe Christi\flqq ).

\subsection*{ Wohl dem Menschen, der nicht wandelt}

Erstdruck 1653.

\subsection*{ Hört an, ihr Völker, hört doch an}

Erstdruck 1653.

\subsection*{ Wohl dem, der den Herren scheuet}

Erstdruck 1653.

\subsection*{ Herr, aller Weisheit Quell und Grund}

Entstanden wahrscheinlich früh. Erstdruck 1661 (nach Joh. Arnds
\frqq Paradiesgärtlein\flqq , Goslar 1621, I, 14: \frqq Umb Weisheit\flqq ).

\subsection*{ Jesu, allerliebster Bruder}

Entstanden wahrscheinlich früh. Erstdruck 1661 (nach Joh. Arnds
\frqq Paradiesgärtlein\flqq , Goslar 1621, I, 34: \frqq Umb Christliche beständige
Freundschaft\flqq ).

\subsection*{ Herr, du erforschest meinen Sinn}

Erstdruck 1666 unter dem Titel \frqq Der 139. Psalm Davids\flqq .

\subsection*{ Ist Ephraim nicht meine Kron?}

Entstanden 1641. Erstdruck 1653.

\subsection*{ Was soll ich doch, o Ephraim}

Entstanden 1641. Erstdruck 1653 unter dem Titel \frqq Aus dem 11. Kap.
Hoseä\flqq ,

\subsection*{ Kommt, ihr traurigen Gemüter}

Entstanden 1641. Erstdruck 1653 unter dem Titel \frqq Aus dem Hosea am 6.
Kap.\flqq .

\subsection*{ Was trotzest du, stolzer Tyrann?}

Entstanden vor 1648. Erstdruck 1666 unter dem Titel \frqq 52. Psalm Davids\flqq .

\subsection*{ Herr, der du vormals hast dein Land}

Entstanden vor 1648. Erstdruck 1653.

\subsection*{ Nicht so traurig, nicht so sehr}

Erstdruck 1648 unter dem Titel \frqq Christliche Zufriedenheit\flqq .

\subsection*{ Ich hab in Gottes Herz und Sinn}

Erstdruck 1648 unter dem Titel \frqq Christliche Ergebung in Gottes Willen\flqq .

\subsection*{ Ich hab oft bei mir selbst gedacht}

Erstdruck 1653.

\subsection*{ Du bist ein Mensch, das weißt du wohl}

Erstdruck 1653.

\subsection*{ Du liebe Unschuld du, wie schlecht wirst du geacht't}

Erstdruck 1653.

\subsection*{ Ich habs verdient, was will ich doch}

Erstdruck 1653 unter dem Titel \frqq Aus dem Micha am 7. Kap.\flqq .

\subsection*{ Ach treuer Gott, barmherzigs Herz}

Erstdruck 1653\par\noindent
(nach Joh. Arnds \frqq Paradiesgärtlein\flqq , III, 27: \frqq Gebet um
Geduld in großem Kreuz\flqq ).

\subsection*{ Barmherzger Vater, höchster Gott}

Erstdruck 1653 unter dem Titel \frqq Joh. Arnds Kreuzgebet\flqq  (nach Joh. Arnds
\frqq Paradies- gärtlein\flqq , III, 26).

\subsection*{ Was Gott gefällt, mein frommes Kind}

Erstdruck 1653.

\subsection*{ Schwing dich auf zu deinem Gott}

Erstdruck 1653 unter dem Titel \frqq Trost in schwerer Anfechtung\flqq , ohne
Strophe 3. 1666 vollständig.

\subsection*{ Ist Gott für mich, so trete}

Entstanden evtl. 1651. Erstdruck 1653.

\subsection*{ Warum sollt ich mich doch grämen?}

Amazon.de Widgets 

Erstdruck 1653.

\subsection*{\centerline{Befiehl du deine Wege}}

Erstdruck 1653.

\subsection*{ Noch dennoch mußt du drum nicht ganz}

Erstdruck 1653.

\subsection*{ Wie lang, o Herr, wie lange soll}

Erstdruck 1653 unter dem Titel \frqq Der 13. Psalm Davids\flqq .

\subsection*{ Gott ist mein Licht, der Herr mein Heil}

Erstdruck 1653.

\subsection*{ Wie der Hirsch im großen Dürsten}

Erstdruck 1653.

\subsection*{ Sei wohlgemut, o Christenseel}

Erstdruck 1653.

\subsection*{ Wer unterm Schirm des Höchsten sitzt}

Erstdruck 1653.

\subsection*{ Geduld ist euch vonnöten}

Erstdruck 1661 unter dem Titel \frqq Geduld ist euch noth\flqq .

\subsection*{ Ach Herr, wie lange willst du mein}

Erstdruck im Anhang einer Leichenrede für Chr. Ludw. von Thümen 1660,
Berlin o.J., unter dem Titel \frqq Der 13. Psalm Davids\flqq .

\subsection*{ Herr, was hast du im Sinn?}

Evtl. entstanden 1664 oder 1652 (auf die Erscheinung eines Kometen).
Erstdruck 1666 unter dem Titel \frqq Bey Erscheinung eines Cometen\flqq .

\subsection*{ Gib dich zufrieden und sei stille}

Erstdruck 1666 unter dem Titel \frqq Gib dich zufrieden\flqq .

\subsection*{ Meine Seele ist in der Stille}

Erstdruck 1666 unter dem Titel \frqq Der 62. Psalm Davids\flqq .

\subsection*{ Nun danket all und bringet Ehr}

Erstdruck 1648.

\subsection*{ Wie ist so groß und schwer die Last}

Entstanden vor 1648. Erstdruck 1653 unter dem Titel \frqq Schutz Gottes in
Kriegsläuft\flqq .

\subsection*{ Gott Lob! Nun ist erschollen}

Entstanden 1648 zum Westfälischen Frieden. Erstdruck 1653.

\subsection*{ Sollt ich meinen Gott nicht singen?}

Erstdruck 1653.

\subsection*{ Wer wohlauf ist und gesund}

Erstdruck 1653 unter dem Titel \frqq Danklied für Leibesgesundheit\flqq .

\subsection*{ Ich singe dir mit Herz und Mund}

Erstdruck 1653 ohne Strophe 4, 8, 9, 17. 1666 vollständig.

\subsection*{ Auf den Nebel folgt die Sonne}

Erstdruck 1653 unter dem Titel \frqq Ein schönes Danklied, welches nach
überstandenem Kummer zu singen\flqq .

\subsection*{ Der Herr, der aller Enden}

Erstdruck 1653 unter dem Titel \frqq Psal. 23\flqq .

\subsection*{ Ich preise dich und singe}

Erstdruck 1653 unter dem Titel \frqq Der 30. Psalm Davids\flqq .

\subsection*{ Ich will erhöhen immerfort}

Erstdruck 1653.

\subsection*{ Ich will mit Danken kommen}

Erstdruck 1653 unter dem Titel \frqq Der 111. Psalm\flqq .

\subsection*{ Das ist mir lieb, daß Gott, mein Hort}

Erstdruck 1653 unter dem Titel \frqq Der 116. Psalm Davids\flqq .

\subsection*{ Du meine Seele singe}

Erstdruck 1653.

\subsection*{ Herr, dir trau ich all mein Tage}

Erstdruck 1655 als Anhang einer Leichenpredigt P. Gerhardts auf den
Amtsschreiber Joachim Schröder zu Mittenwalde, unter dem Titel \frqq Der 71.
Psalm, Gesangsweise übersetzet, auff die Meloden: Du o schönes Welt
Gebäude\flqq .

\subsection*{ Wie ist es möglich, höchstes Licht?}

Erstdruck 1667 unter dem Titel \frqq Gott allein die Ehre\flqq .

\subsection*{ Merkt auf, merkt, Himmel, Erde}

Entstanden um 1648. Erstdruck 1666 unter dem Titel \frqq Das Lied Mosis, aus
dem 32. Capitel des fünften Buchs Mose\flqq .

\subsection*{ Ich, der ich oft in tiefes Leid}

Erstdruck 1666 unter dem Titel \frqq Der 145. Psalm Davids\flqq .

Amazon.de Widgets 

\subsection*{ Ich danke dir mit Freuden}

Erstdruck 1666/67 unter dem Titel \frqq Dank- Gebetlein Sirachs aus dem 51.
Cap.\flqq 

\subsection*{ Mein Gott, ich habe mir}

Erstdruck 1648 unter dem Titel \frqq Der 39. Psalm Davids\flqq 

\subsection*{ O Tod, o Tod, du greulichs Bild}

Erstdruck 1667 unter dem Titel \frqq Freudige Empfahung des Todes\flqq .
Bearbeitung von Paul Röbers Lied \frqq O Tod, o Tod, schreckliches Bild\flqq .

\subsection*{ Mein herzer Vater, weint ihr noch?}

Erstdruck 1650 mit der Leichenpredigt auf den Sohn des Rektors Adam
Spengler.

\subsection*{ Du bist zwar mein und bleibest mein}

Erstdruck 1650 im Anhang der Leichenpredigt von Georg Lilie auf
Constantin Andreas Berkow unter dem Titel \frqq Der betrübte Vater tröstet
sich über seinem nunmehr seligen Sohn\flqq .

\subsection*{ Nun, du lebest, unsre Krone}

Erstdruck 1650 im Anhang der Leichenpredigt auf den Hofkammergerichtsrat
Petrus Fritz (gest. 1648), unter dem Titel \frqq Trostgesang derer, so über
den Hintritt des sel. Herrn D. Fritzens betrübet worden\flqq .

\subsection*{ Erhebe dich, betrübtes Herz}

Erstdruck 1650 im Anhang der Leichenpredigt auf den Hofkammergerichtsrat
Petrus Fritz (gest. 1648), unter dem Titel \frqq Trostgesang derer, so über
den Hintritt des sel. Herrn D. Fritzens betrübet worden\flqq .

\subsection*{ Die Zeit ist nunmehr nah}

Entstanden im 30-jährigen Krieg oder anläßlich des Erscheinens eines
Kometen 1652. Erstdruck 1653.

\subsection*{ Leid ist mirs in meinem Herzen}

Erstdruck 1659 im Anhang der Leichenrede von Georg Lilie auf die Tochter
des Diakons Georg Heintzelmann, unter dem Titel \frqq Auf das zwar
frühzeitige aber dennoch selige Abscheiden des Tugend und Gottliebenden
Jungfräuleins Elisabeth Heintzelmans\flqq .

\subsection*{ Herr Lindholtz legt sich hin}

Erstdruck 1659 mit der Leichenpredigt von Chr. Nikolai auf Chr.
Lindholtz unter dem Titel \frqq Auff das selige Absterben Herrn Chr.
Lindholtzes\flqq .

\subsection*{ Liebes Kind, wenn ich bei mir}

Erstdruck 1660 mit dem Leich-Sermon P. Gerhardts unter dem Titel \frqq Auff
das frühzeitige doch wohlselige Absterben deß bald zur Vollkommenheit
gelangten Knabens Friedrich Ludowig Zarlanges\flqq .

\subsection*{ O, wie so ein großes Gut}

Erstdruck 1661 mit der Leichenrede Joh. Rosners auf Frau U. von der
Linden unter dem Titel \frqq Dem Herrn Land-Rentmeister Herrn Chr. von der
Linden. Als desselben hertzgeliebte Haus-Ehre Fr. Ursula Monsin selig im
Herrn entschlaffen\flqq .

\subsection*{ Nun sei getrost und unbetrübt}

Erstdruck 1664 mit der Leichenpredigt Joh. Leißners auf Frau Regina
Leyser, geb. Calow, unter dem Titel \frqq Fröhliche Ergebung zu einem seligen
Abschiede aus dieser müheseligen Welt\flqq .

\subsection*{ Hörst du hier die Ewigkeit?}

Erstdruck 1664 am Schluß von J. Paulis \frqq Vorschmack der Traurigen und
frölichen Ewigkeit\flqq .

\subsection*{ Herr Gott, du bist ja für und für}

Erstdruck 1667 unter dem Titel \frqq Vom Tod und Sterben. Aus den 90. Psal.
Davids\flqq .

\subsection*{ Ich bin ein Gast auf Erden}

Erstdruck 1666 unter dem Titel \frqq Auß dem 119. Psalm Davids\flqq .

\subsection*{ Was trauerst du, mein Angesicht}

Erstdruck 1666 unter dem Titel \frqq Christliche Todes-Freude\flqq .

\subsection*{ Ich weiß, daß mein Erlöser lebt}

Erstdruck 1667.

\subsection*{ Weint, und weint gleichwohl nicht zu sehr}

Erstdruck 1667 mit der Leichenpredigt Gerhardts für M. Zarlang mit der
Überschrift \frqq Bey frühzeitigem Absterben des frommen hertz-lieben
Töchterleins, Margritgen Zarlanges, an desselben hertzlichgekränckte und
hoch-betrübte Eltern\flqq .

\subsection*{ So geht der alte liebe Herr nun auch dahin}

Erstdruck 1667 unter dem Titel \frqq Auff das selige Absterben und christl.
Beerdigung des umb diese gantze Stadt viel Jahr lang wolverdienten Herrn
Bürgermeisters, Herrn Benedicti Reichardts\flqq .

\subsection*{ Wer selig stirbt, stirbt nicht}

Erstdruck 1668 mit der Leichenrede Gerhardts auf Joh. Adam Preunel.

\subsection*{ Johannes sahe durch Gesicht}

Erstdruck 1668 mit der Leichenrede Gerhardts auf Joh. Adam Preunel.



\renewcommand{\indexname}{
  \begin{center}
    {\LARGE INDEX}
  \end{center}
  \rule{\textwidth}{0.2pt}\vspace*{-\baselineskip}\vspace{3.2pt}
  \rule{\textwidth}{1.2pt}\\%[\baselineskip]
}
\printindex

%\addcontentsline{toc}{section}{INDEX}
\end{document}
