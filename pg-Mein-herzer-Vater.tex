%StartInfo%%%%%%%%%%%%%%%%%%%%%%%%%%%%%%%%%%%%%%%%%%%%%%%%%%%%%%%%%%%%%%%%%%%%
%  Autor:
%  Titel:
%  File:
%  Ref:
%  Mod:
%EndInfo%%%%%%%%%%%%%%%%%%%%%%%%%%%%%%%%%%%%%%%%%%%%%%%%%%%%%%%%%%%%%%%%%%%%%%
%\poemtitle{pt}
\begin{multicols}{2}
\settowidth{\versewidth}{Nichts ist so schön und wohl bestellt,}
\begin{verse}[\versewidth]
%mein herzer Vater, weint ihr noch?\\
%Auf den Tod eines Kindes des Rektors Adam Spengler (1650)

\flagverse{1.} Mein herzer Vater, weint ihr noch?\\
Und ihr, die mich geboren?\\
Was grämt ihr euch? Was macht ihr doch?\\
Ich bin ja unverloren.\\
Ach, sollt ihr sehen, wie mirs geht,\\
und wie mich der so hoch erhöht,\\
der selbst so hoch erhoben;\\
ich weiß, ihr würdet anders tun\\
und meiner Seele süßes Ruhn\\
mit eurem Munde loben.

\flagverse{2.} Der saure Kampf, den ich dort hab\\
in eurer Welt empfunden,\\
der ist durch Gottes Gnad und Gab\\
all glücklich überwunden.\\
Es ging mir, wie es pflegt zu gehn\\
all denen, die bei Christo stehn\\
und von der Welt sich scheiden;\\
wer Christo folgt, der muß mit ihm\\
das Kreuz und alles Ungestüm\\
auf seinen Wegen leiden.

\flagverse{3.} Nun bin ich durch. Gott Lob und Dank!\\
Hier kommt ein ander Leben;\\
hier wird mir, was mein Leben lang\\
ich nicht gesehn, gegeben:\\
Ein ganzer Himmel voller Licht,\\
ein Licht, davon mein Angesicht\\
so schön wird als die Sonne;\\
hier ist ein ewges Freudenmeer,\\
wohin ich nur die Augen kehr,\\
ist alles voller Wonne.

\flagverse{4.} Nun lobt, ihr Menschen, wie ihr wollt,\\
des Erdenlebens Güte:\\
Was ist darinnen, das mir sollt\\
jetzt neigen mein Gemüte?\\
Was ist das Beste, das ihr liebt?\\
Was gibt die Erde, wenn sie gibt,\\
als Angst und bittre Schmerzen?\\
Was ist das güldne Gut und Geld?\\
Was bringt der Schein und Pracht der Welt\\
als Kummer eurer Herzen?

\flagverse{5.} Was ist der großen Leute Gunst\\
als Zunder großes Neides?\\
Was ist das Wissen vieler Kunst\\
als Ursprung vieles Leides?\\
Denn wer viel weiß, der grämt sich viel,\\
und welcher andre lehren will,\\
muß leiden und viel tragen.\\
Seht alles an, Ruhm, Lob und Ehr,\\
habt Freud und Lust, was habt ihr mehr\\
als endlich Weh und Klagen?

\flagverse{6.} Nichts ist so schön und wohl bestellt,\\
da man hier wohl auf stehe,\\
drum nimmt Gott, was ihm wohlgefällt,\\
bei Zeiten in die Höhe\\
und setzet es in seinen Schoß;\\
da ist es allen Kummers los,\\
darf nicht, wie ihr, sich kränken,\\
die ihr oft denket, wie doch wohl\\
dies oder jenes werden soll,\\
und könnets nicht erdenken.

\flagverse{7.} Wer selig stirbt, der schleußet zu\\
die schwarzen Jammertore,\\
hingegen schwingt er sich zur Ruh\\
im güldnen Engelchore,\\
legt Aschen weg, kriegt Freudenöl,\\
zeucht aus das Fleisch und schmückt die Seel\\
in reiner weißer Seiden;\\
er läßt die Erd und nimmet ein\\
die Lust, da Christi Schäfelein\\
in lauter Rosen weiden.

\flagverse{8.} So gebt, ihr Liebsten, euch doch schlecht\\
dahin in Gottes Willen;\\
sein Rat ist gut, sein Tun ist recht\\
und wird wohl wieder stillen\\
den Schmerzen, den er euch gemacht.\\
Und hiemit sei euch gute Nacht\\
von eurem Sohn gegönnet.\\
Es kommt die Zeit, da mich und euch\\
vereingen wird in seinem Reich,\\
der euch und mich getrennet.

\end{verse}
\end{multicols}
%\attrib{\small{THZE}}

\begin{center}
\settowidth{\versewidth}{Der, vor dem die Welt erschrickt,}
\begin{verse}[\versewidth]


\flagverse{9.} Da will ich eure Treu und Müh\\
und was ihr eurem Kranken\\
erwiesen habt, im Himmel hie,\\
sobald ihr kommt, verdanken.\\
Ich will erzählen, wie ihr habt\\
euch selbst betrübt und mich gelabt,\\
vor Christo und vor allen;\\
und für den heißen Tränenfluß\\
will ich mit mehr als einem Kuß\\
um euren Hals euch fallen.
   
\end{verse}
\end{center}

