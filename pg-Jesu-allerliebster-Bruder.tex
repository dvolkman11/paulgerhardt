%StartInfo%%%%%%%%%%%%%%%%%%%%%%%%%%%%%%%%%%%%%%%%%%%%%%%%%%%%%%%%%%%%%%%%%%%%
%  Autor:
%  Titel:
%  File:
%  Ref:
%  Mod:
%EndInfo%%%%%%%%%%%%%%%%%%%%%%%%%%%%%%%%%%%%%%%%%%%%%%%%%%%%%%%%%%%%%%%%%%%%%%
%\poemtitle{pt}
\begin{multicols}{2}
\settowidth{\versewidth}{Wenn die Zung und Mund nur liebet,}
\begin{verse}[\versewidth]
%nach Johann Arnds »Paradiesgärtlein«, Goslar 1621, I, 34

\flagverse{1.} Jesu, allerliebster Bruder,\\
ders am besten mit mir meint,\\
du mein Anker, Mast und Ruder\\
und mein treuester Herzensfreund;\\
der du, ehe was geboren,\\
dir das Menschenvolk erkoren,\\
auch mich armen Erdengast\\
dir zur Lieb ersehen hast.

\flagverse{2.} Du bist ohne Falsch und Tücke,\\
dein Herz weiß von keiner List,\\
aber wenn ich nur erblicke\\
was hier auf Erden ist,\\
find ich alles voller Lügen:\\
Wer am besten kann betrügen,\\
wer am schönsten heucheln kann,\\
ist der allerbeste Mann.

\flagverse{3.} Ach, wie untreu und verlogen\\
ist die Liebe dieser Welt;\\
ist sie jemand wohl gewogen,\\
währts nicht länger als sein Geld.\\
Wenn das Glück uns fügt und grünet,\\
sind wir schön und hübsch bedienet,\\
kommt ein wenig Ungestüm,\\
kehrt sich alle Freundschaft üm.

\flagverse{4.} Treib, Herr, von mir und verhüte\\
solchen unbeständgen Sinn;\\
hätt ich aber mein Gemüte,\\
weil ich auch ein Mensche bin,\\
schon mit diesem Kot besprenget\\
und der Falschheit nachgehänget,\\
so erkenn ich meine Schuld,\\
bitt um Gnad und um Geduld.

\flagverse{5.} Laß mir ja nicht widerfahren,\\
was du Herr zur Straf und Last\\
denen, die mit falschen Waren\\
handeln, angedräuet hast,\\
da du sprichst, du wollest scheuen\\
und als Unflat von dir speien\\
aller Heuchler falschen Mut,\\
der Guts fürgibt und nicht tut.

\flagverse{6.} Gib mir ein beständges Herze\\
gegen alle meine Freund;\\
auch dann, wann mit Kreuz und Schmerze\\
sie von dir beleget seind,\\
daß ich mich nicht ihrer schäme,\\
sondern mich nach dir bequeme,\\
der du, da wir arm und bloß,\\
uns gesetzt in deinen Schoß.

\flagverse{7.} Gib mir auch nach deinem Willen\\
einen Freund, in dessen Treu\\
ich mein Herze möge stillen,\\
da mein Mund sich ohne Scheu\\
öffnen und erklären möge,\\
da ich alles abelege\\
(nach dem Maße, das mir gnügt),\\
was mir auf dem Herzen liegt.

\flagverse{8.} Laß mich Davids Glück erleben:\\
Gib mir einen Jonathan,\\
der mir sein Herz möge geben,\\
der auch, wenn nun jedermann\\
mir nichts Gutes mehr will gönnen,\\
sich nicht lasse von mir trennen,\\
sondern fest in Wohl und Weh\\
als ein Felsen bei mir steh.

\flagverse{9.} Herr, ich bitte dich, erwähle\\
mir aus aller Menschen Meng\\
eine fromme heilge Seele,\\
die an dir fein kleb und häng,\\
auch nach deinem Sinn und Geiste\\
mir stets Trost und Hilfe leiste:\\
Trost, der in der Not besteht,\\
Hilfe, die von Herzen geht.

\flagverse{10.} Wenn die Zung und Mund nur liebet,\\
ist die Liebe schlecht bestellt.\\
Wer mir gute Worte gibet\\
und den Haß im Herzen hält,\\
wer nur seinen Kuchen schmieret\\
und, wanns Bienlein nicht mehr führet,\\
alsdann geht er nach der Tür -\\
ei, der bleibe fern von mir.

\flagverse{11.} Hab ich Schwachheit und Gebrechen,\\
Herr, so lenke meinen Freund,\\
mich in Güte zu besprechen\\
und nicht als ein Leu und Feind.\\
Wer mich freundlich weiß zu schlagen,\\
ist, als der in Freudentagen\\
reichlich auf mein Haupt mir geußt\\
Balsam, der am Jordan fleußt.

\flagverse{12.} O, wie groß ist meine Habe,\\
o, wie köstlich ist mein Gut,\\
Jesu, wenn mit dieser Gabe\\
dein Hand meinen Willen tut,\\
daß mich meines Freundes Treue\\
und beständigs Herz erfreue!\\
Wer dich fürchtet, liebt und ehrt,\\
dem ist solch ein Schatz beschert.

\flagverse{13.} Gute Freunde sind wie Stäbe,\\
da der Menschen Gang sich hält,\\
daß der schwache Fuß sich hebe,\\
wann der Leib zu Boden fällt.\\
Wehe dem, der nicht zum Frommen\\
solches Stabes weiß zu kommen!\\
Der hat einen schweren Lauf;\\
wann er fällt, wer hilft ihn auf?

\flagverse{14.} Nun, Herr, laß dirs wohl gefallen,\\
bleib mein Freund bis in mein Grab!\\
Bleib mein Freund und unter allen\\
mein getreuster, stärkster Stab!\\
Wenn du dich mir wirst verbinden,\\
wird sich schon ein Herze finden,\\
das, durch deinen Geist gerührt,\\
mir was Gutes gönnen wird.

\end{verse}
\end{multicols}
%\attrib{\small{THZE}}
