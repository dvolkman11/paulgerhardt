%StartInfo%%%%%%%%%%%%%%%%%%%%%%%%%%%%%%%%%%%%%%%%%%%%%%%%%%%%%%%%%%%%%%%%%%%%
%  Autor:
%  Titel:
%  File:
%  Ref:
%  Mod:
%EndInfo%%%%%%%%%%%%%%%%%%%%%%%%%%%%%%%%%%%%%%%%%%%%%%%%%%%%%%%%%%%%%%%%%%%%%%
%\poemtitle{pt}
\begin{multicols}{2}
\settowidth{\versewidth}{Liebes Kind, wenn ich bei mir}
\begin{verse}[\versewidth]
%liebes Kind, wenn ich bei mir bedenke\\
%auf den Tod des kleinen Friedrich Ludwig Zarlang, Sohn des Berliner Bürgermeisters Z. (1660)

\flagverse{1.} Liebes Kind, wenn ich bei mir\\
deines schönen Leibes Zier\\
und der Seelen Schmuck bedenke,\\
weiß es Gott, wie ich mich kränke.

\flagverse{2.} Kein Smaragd mag je so schön\\
in dem feinen Golde stehn,\\
keine Rose mag im Lenzen\\
dir gleich, schöne Blume, glänzen.

\flagverse{3.} Dein Gebärde, dein Gesicht\\
und der beiden Augen Licht\\
war in Tugend ganz verhüllet\\
und mit guter Zucht erfüllet.

\flagverse{4.} Deine Liebe, deine Gunst\\
ging und hing nach lauter Kunst;\\
viel zu lernen, viel zu wissen,\\
war dein edler Geist geflissen.

\flagverse{5.} Auch war hier ein guter Grund,\\
da das ganze Werk auf stund,\\
nämlich Gott und sein Wort hören\\
und die heilge Bibel ehren.

\flagverse{6.} Wollte, wollte Gott, daß nur\\
deines Lebens schwache Schnur\\
etwas noch hier auf der Erden\\
hätten müssen länger werden.

\flagverse{7.} O wie manche große Freud,\\
o wie manch Ergötzlichkeit\\
würden wir von deinen Gaben\\
noch zuletzt genossen haben.

\flagverse{8.} Nun, mich jammerts; aber du,\\
liebes Kind, schweigst still dazu,\\
wohnst in Gottes Stadt und Mauern\\
kehrst dich nicht an unser Trauern.

\flagverse{9.} Deines Wesens hoher Stand\\
ist auch nun also bewandt,\\
daß, wers gut will mit dir meinen,\\
dich nicht dürfe mehr beweinen.

\flagverse{10.} Du bist ungleich besser dran,\\
als die Welt hier sinnen kann;\\
du hast mehr als wir dir gönnen,\\
mehr auch, als wir wünschen können.

\flagverse{11.} Es ist an dir ganz und gar,\\
was hier unvollkommen war;\\
was du hier hast angefangen,\\
hast du dort vollauf empfangen.

\flagverse{12.} Deine Seel hat Gottes Reich,\\
und du bist den Engeln gleich:\\
Alle Himmel hörst du singen\\
und du gehst in vollen Springen.

\end{verse}
\end{multicols}
%\attrib{\small{THZE}}

\begin{center}
\settowidth{\versewidth}{Der, vor dem die Welt erschrickt,}
\begin{verse}[\versewidth]

\flagverse{13.} Nun so lebe, wie du lebest!\\
Schweb in Freuden, wie du schwebest!\\
Balde, balde wirds geschehen,\\
daß du uns, wir dich dort sehen.

\end{verse}
\end{center}

