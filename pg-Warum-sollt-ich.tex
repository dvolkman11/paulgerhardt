%StartInfo%%%%%%%%%%%%%%%%%%%%%%%%%%%%%%%%%%%%%%%%%%%%%%%%%%%%%%%%%%%%%%%%%%%%
%  Autor:
%  Titel:
%  File:
%  Ref:
%  Mod:
%EndInfo%%%%%%%%%%%%%%%%%%%%%%%%%%%%%%%%%%%%%%%%%%%%%%%%%%%%%%%%%%%%%%%%%%%%%%
%\poemtitle{pt}
\begin{multicols}{2}
\settowidth{\versewidth}{Gut und Blut, Leib, Seel und Leben}
\begin{verse}[\versewidth]

\flagverse{1.} Warum sollt ich mich doch grämen?\\
Hab ich doch\\
Christum noch,\\
wer will mir den nehmen?\\
Wer will mir den Himmel rauben,\\
den mir schon\\
Gottes Sohn\\
beigelegt im Glauben?

\flagverse{2.} Nackend lag ich auf dem Boden,\\
da ich kam,\\
da ich nahm\\
meinen ersten Odem;\\
nackend werd ich auch hinziehen,\\
wann ich werd\\
von der Erd\\
als ein Schatten fliehen.

\flagverse{3.} Gut und Blut, Leib, Seel und Leben\\
ist nicht mein;\\
Gott allein\\
ist es, ders gegeben.\\
Will ers wieder zu sich kehren,\\
nehm ers hin!\\
Ich will ihn\\
dennoch fröhlich ehren.

\flagverse{4.} Schickt er mir ein Kreuz zu tragen,\\
dringt herein\\
Angst und Pein,\\
sollt ich drum verzagen?\\
Der es schickt, der wird es wenden!\\
Er weiß wohl,\\
wie er soll\\
all mein Unglück enden.

\flagverse{5.} Gott hat mich bei guten Tagen\\
oft ergötzt:\\
Sollt ich jetzt\\
auch nicht etwas tragen?\\
Fromm ist Gott und schärft mit Maßen\\
sein Gericht;\\
kann mich nicht\\
ganz und gar verlassen.

\flagverse{6.} Satan, Welt und ihre Rotten\\
können mir\\
nichts mehr hier\\
tun, als meiner spotten.\\
Laß sie spotten, laß sie lachen!\\
Gott, mein Heil,\\
wird in Eil\\
sie zu Schanden machen.

\flagverse{7.} Unverzagt und ohne Grauen\\
soll ein Christ,\\
wo er ist,\\
stets sich lassen schauen.\\
Wollt ihn auch der Tod aufreiben,\\
soll der Mut\\
dennoch gut\\
und fein stille bleiben.

\flagverse{8.} Kann uns doch kein Tod nicht töten,\\
sondern reißt\\
unsern Geist\\
aus viel tausend Nöten;\\
schleußt das Tor des bittern Leiden\\
und macht Bahn,\\
da man kann\\
gehn zur Himmelsfreuden.

\flagverse{9.} Allda will in süßen Schätzen\\
ich mein Herz\\
auf den Schmerz\\
ewiglich ergötzen.\\
Hier ist kein recht Gut zu finden:\\
Was die Welt\\
in sich hält,\\
muß im Hui verschwinden.

\flagverse{10.} Was sind dieses Lebens Güter?\\
Eine Hand\\
voller Sand,\\
Kummer der Gemüter.\\
Dort, dort sind die edlen Gaben,\\
da mein Hirt,\\
Christus, wird\\
mich ohn Ende laben.

\flagverse{11.} Herr, mein Hirt, Brunn aller Freuden,\\
du bist mein,\\
ich bin dein,\\
niemand kann uns scheiden:\\
Ich bin dein, weil du dein Leben\\
und dein Blut\\
mir zugut\\
in den Tod gegeben.

\flagverse{12.} Du bist mein, weil ich dich fasse\\
und dich nicht,\\
o mein Licht,\\
aus dem Herzen lasse.\\
Laß mich, laß mich hingelangen,\\
da du mich\\
und ich dich\\
lieblich werd umfangen.

\end{verse}
\end{multicols}
%\attrib{\small{THZE}}
