%StartInfo%%%%%%%%%%%%%%%%%%%%%%%%%%%%%%%%%%%%%%%%%%%%%%%%%%%%%%%%%%%%%%%%%%%%
%  Autor:
%  Titel:
%  File:
%  Ref:
%  Mod:
%EndInfo%%%%%%%%%%%%%%%%%%%%%%%%%%%%%%%%%%%%%%%%%%%%%%%%%%%%%%%%%%%%%%%%%%%%%%
%\poemtitle{pt}
\begin{multicols}{2}
\settowidth{\versewidth}{Wer, sprach Johannes, sind doch die,}
\begin{verse}[\versewidth]
%johannes sahe durch Gesicht ein edles Licht\\
%(Off. Joh. 7, 9 ff.)

\flagverse{1.} Johannes sahe durch Gesicht\\
ein edles Licht\\
und liebliches Gemälde:\\
Er sah ein Haufen Völker stehn,\\
sehr hell und schön,\\
im güldnen Himmelsfelde.\\
Ihr Herz und Mut\\
schwebt in dem Gut,\\
das hier kein Mann\\
bezahlen kann\\
mit allem Gut und Gelde.

\flagverse{2.} Sie trugen Palmen in der Hand;\\
ihr Ort und Stand\\
war vor des Lammes Throne,\\
ihr Mund war voller Lob und Preis,\\
die Kleider weiß,\\
ihr Lied, im höhren Tone,\\
klang süß und sang\\
des Höchsten Dank,\\
und dieser Stimm\\
half üm und üm\\
der Engel heilge Krone.

\flagverse{3.} Wer, sprach Johannes, sind doch die,\\
die ich allhie\\
in weißem Schmuck seh halten?\\
Es sind, antwortet aus der Schar,\\
die um ihn war,\\
der eine von den Alten:\\
Es sind, mein Sohn,\\
die sich den Hohn\\
und Spott der Welt\\
von Gottes Zelt\\
nicht lassen abehalten.

\flagverse{4.} Es sind die, so vor dieser Zeit\\
in großem Leid\\
auf Erden sich befunden,\\
die bei des Herren Jesu Ehr\\
und seiner Lehr\\
all Angst und Trübsalswunden,\\
zwar ohne Schuld,\\
doch mit Geduld,\\
durch Gott gekühlt,\\
recht wohl gefühlt\\
und fröhlich überwunden.

\flagverse{5.} Dieselben haben all ihr Kleid,\\
als treue Leut,\\
im Glaubensbad erkläret;\\
sie haben sich der Höllen List,\\
so viel der ist,\\
mit starkem Mut erwehret\\
und nicht geacht\\
der Erden Pracht,\\
des Lammes Blut\\
zu ihrem Gut\\
erwählet und begehret.

\flagverse{6.} Darum so stehen sie auch nun\\
und all ihr Tun\\
wo Gottes Tempel stehet;\\
der Tempel, da man Tag und Nacht\\
dem Höchsten wacht\\
und seinen Ruhm erhöhet;\\
da leben sie\\
ohn alle Müh,\\
ohn alle Qual\\
im Freudensaal,\\
der nimmermehr vergehet.

\flagverse{7.} Daselbst sitzt Gott in seinem Haus\\
und breitet aus\\
die Hütte seiner Güte\\
und deckt mit sanfter Wollust zu\\
in stiller Ruh\\
manch trauriges Gemüte.\\
Was Freude gibt,\\
dem Herzen liebt,\\
die Augen füllt,\\
das Sehnen stillt,\\
steht da in voller Blüte.

\flagverse{8.} Da ist kein Durst, kein Hungersnot,\\
das Himmelsbrot\\
läßt keinen Mangel leiden,\\
da scheint die Sonne keinem mehr\\
zu heiß und sehr,\\
ihr Glanz bringt lauter Freuden.\\
Die Himmelssonn\\
und Herzenswonn\\
ist unser Hirt,\\
der große Wirt\\
und Herr der ewgen Weiden.
\end{verse}
\end{multicols}
%\attrib{\small{THZE}}

\begin{center}
\settowidth{\versewidth}{Das Lamm wird weiden seine Herd,}
\begin{verse}[\versewidth]
\flagverse{9.} Das Lamm wird weiden seine Herd,\\
als sies begehrt,\\
auf Auen, die schön prangen;\\
es wird sie leiten zu dem Quell,\\
der frisch und hell,\\
das Heil draus zu erlangen;\\
und wird gewiß\\
nicht ruhen, bis\\
er uns erfrischt\\
und abgewischt\\
die Tränen unsrer Wangen.
 
\end{verse}
\end{center}



