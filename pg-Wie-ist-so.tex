%StartInfo%%%%%%%%%%%%%%%%%%%%%%%%%%%%%%%%%%%%%%%%%%%%%%%%%%%%%%%%%%%%%%%%%%%%
%  Autor:
%  Titel:
%  File:
%  Ref:
%  Mod:
%EndInfo%%%%%%%%%%%%%%%%%%%%%%%%%%%%%%%%%%%%%%%%%%%%%%%%%%%%%%%%%%%%%%%%%%%%%%
%\poemtitle{pt}
\begin{multicols}{2}
\settowidth{\versewidth}{Wir unsers Teils sind dir verpflicht't}
\begin{verse}[\versewidth]
%dankgebet in Kriegszeiten\\
%wie ist so groß und schwer die Last

\flagverse{1.} Wie ist so groß und schwer die Last,\\
die du uns aufgeleget hast,\\
o aller Götter Gott!\\
Gott, der du streng und eifrig bist\\
dem, der nicht fromm und heilig ist.

\flagverse{2.} Die Last, die ist die Kriegesflut,\\
so jetzt die Welt mit rotem Blut\\
und heißen Tränen füllt;\\
es ist das Feur, das hitzt und brennt,\\
so weit fast Sonn und Mond sich wendt.

\flagverse{3.} Groß ist die Last, doch ist dabei\\
dein starker Schutz und Vatertreu\\
uns gar nicht unbekannt;\\
du strafst, und mitten in dem Leid\\
erzeigst du Lieb und Freundlichkeit.

\flagverse{4.} Wir unsers Teils sind dir verpflicht't\\
dafür, daß du dein Heil und Licht\\
uns niemals ganz versagt;\\
viel andre hast du abgelohnt,\\
uns hast du ja noch oft verschont.

\flagverse{5.} Wie manchmal hat sich hier und dar\\
ein großes Wetter der Gefahr\\
um uns gezogen auf;\\
dein Hand, die Erd und Himmel trägt,\\
hat Sturm und Wetter beigelegt.

\flagverse{6.} Wie oftmals hat bei Tag und Nacht\\
der Feinde List und große Macht\\
uns, deine Herd, umringt;\\
du aber, o du treuer Hirt\\
hast unsern Wolf zurückgeführt.

\flagverse{7.} Viel unsrer Brüder sind geplagt,\\
von Haus und Hof dazu verjagt;\\
wir aber haben noch\\
beim Weinstock und beim Feigenbaum\\
ein jeder seinen Sitz und Raum.

\flagverse{8.} Sieh an, mein Herz, wie Stadt und Land\\
an vielen Orten ist gewandt\\
zum tiefen Untergang;\\
der Menschen Hütten sind verstört,\\
die Gotteshäuser umgekehrt.

\flagverse{9.} Bei uns ist ja noch Polizei,\\
auch leisten wir noch ohne Scheu\\
dem Herren seinen Dienst;\\
man lehrt und hört ja fort und fort\\
alltäglich bei uns Gottes Wort.

\flagverse{10.} Wer dieses nun will nicht verstehn,\\
läßts in die Luft und Winde gehn\\
und bei so hellem Licht\\
nicht Gottes Gnad und Güt erkennt,\\
der ist fürwahr durchaus verblendt.

\flagverse{11.} O frommer Gott, nimm von uns hin\\
solch Unvernunft, richt unsern Sinn,\\
daß wir zur Dankbarkeit\\
mit Lobgesang und süßem Ton\\
uns finden stets vor deinem Thron.

\flagverse{12.} Nicht unserm Werk, nicht unserm Tun,\\
allein dir, dir, o Gnadenbrunn,\\
gebührt all Ehr und Ruhm.\\
Wir haben Zorn und Tod verschuldt,\\
du zahlest uns mit Lieb und Huld.

\flagverse{13.} Laß diese Lieb, als eine Glut,\\
in uns entzünden Herz und Mut,\\
gib engelische Brunst,\\
daß alle unsre Äderlein\\
zu singen dir bereitet sein.

\flagverse{14.} Laß auch einmal nach so viel Leid\\
uns wieder scheinen unsre Freud,\\
des Friedens Angesicht,\\
das mancher Mensch noch nie einmal\\
geschaut in diesem Jammertal.

\flagverse{15.} Sind wirs nicht wert, so sieh doch an\\
die, so kein Unrecht je getan,\\
die kleinen Kinderlein;\\
solln sie denn in der Wiegen noch\\
mittragen solches schweres Joch?

\flagverse{16.} Erbarm dich, o barmherzigs Herz,\\
so vieler Seufzer, die der Schmerz\\
uns aus dem Herzen zwingt.\\
Du bist ja Gott und nicht ein Stein,\\
wie kannst du denn so harte sein?

\flagverse{17.} Wir sind an bösen Wunden krank,\\
voll Eiter, Striemen, Kot und Stank,\\
du Herr bist unser Arzt!\\
Geuß ein, geuß ein dein Gnadenöl,\\
so wird geheilet Leib und Seel.

\flagverse{18.} Nun, du wirsts tun, das glauben wir,\\
obgleich noch wenig scheinen für\\
die Mittel in der Welt.\\
Wenn alle Mittel stille stehn,\\
dann pflegt dein Helfen anzugehn.

\end{verse}
\end{multicols}
%\attrib{\small{THZE}}
