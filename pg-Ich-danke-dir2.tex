%StartInfo%%%%%%%%%%%%%%%%%%%%%%%%%%%%%%%%%%%%%%%%%%%%%%%%%%%%%%%%%%%%%%%%%%%%
%  Autor:
%  Titel:
%  File:
%  Ref:
%  Mod:
%EndInfo%%%%%%%%%%%%%%%%%%%%%%%%%%%%%%%%%%%%%%%%%%%%%%%%%%%%%%%%%%%%%%%%%%%%%%
%\poemtitle{pt}
\begin{multicols}{2}
\settowidth{\versewidth}{Ich danke dir mit Freuden,}
\begin{verse}[\versewidth]
%ich danke dir mit Freuden\\
%(Sir. 51)

\flagverse{1.} Ich danke dir mit Freuden,\\
mein König und mein Heil,\\
daß du manch schweres Leiden,\\
so mir zu meinem Teil\\
oft häufig zugedrungen,\\
durch deine Wunderhand\\
gewaltig hast bezwungen\\
und von mir abgewandt.

\flagverse{2.} Du hast in harten Zeiten\\
mir diese Gnad erteilt,\\
daß meiner Feinde Streiten\\
mein Leben nicht ereilt,\\
wenn sie an hohen Orten\\
mich, der ichs nicht gedacht,\\
mit bösen falschen Worten\\
sehr übel angebracht.

\flagverse{3.} Wenn sie wie wilde Leuen\\
die Zungen ausgestreckt\\
und mich mit ihrem Schreien\\
bis auf den Tod erschreckt,\\
so hat denn dein Erbarmen,\\
das alles lindern kann,\\
gewaltet und mir Armen\\
den treusten Dienst getan.

\flagverse{4.} Sie haben oft zusammen\\
sich wider mich gelegt\\
und wie die Feuerflammen\\
gefahr und Brand erregt:\\
Da hab ich denn gesessen\\
und Blut vor Angst geschwitzt,\\
als ob du mein vergessen,\\
und hast mich doch geschützt.

\flagverse{5.} Du hast mich aus dem Brande\\
und aus dem Feur gerückt,\\
und wenn der Höllen Bande\\
mich um und um bestrickt,\\
so hast du auf mein Bitten\\
dich, Herr, zu mir gesellt\\
und aus des Unglücks Mitten\\
mich frei ins Feld gestellt.

\flagverse{6.} Den Kläffer, der mit Lügen\\
gleich als mit Waffen kämpft\\
und nichts kann als betrügen,\\
den hast du oft gedämpft;\\
wenn er, gleich einem Drachen,\\
das Maul hoch aufgezerrt,\\
so hast du ihm den Rachen\\
durch deine Kraft gesperrt.

\flagverse{7.} Ich war nah am Verderben,\\
du nahmst mich in den Schoß;\\
es kam mit mir zum Sterben,\\
du aber sprachst mich los\\
und hieltest mich beim Leben\\
und gabst mir Rat und Tat,\\
die sonst kein Mensch zu geben\\
in seinen Mächten hat.

\flagverse{8.} Es war in allen Landen,\\
so weit die Wolken gehn,\\
kein einzger Freund vorhanden,\\
der bei mir wollte stehn;\\
da dacht ich an die Güte,\\
die du, Herr, täglich tust,\\
und hub Herz und Gemüte\\
zur Höhe, da du ruhst.

\end{verse}
\end{multicols}


\begin{center}
\settowidth{\versewidth}{Der, vor dem die Welt erschrickt,}
\begin{verse}[\versewidth]




\flagverse{9.} Ich rief mit vollem Munde,\\
du nahmest alles an\\
und halfst recht aus dem Grunde\\
so, daß ichs nimmer kann\\
nach Würden gnugsam loben:\\
Doch will ich Tag und Nacht\\
dich in dem Himmel droben\\
zu preisen sein bedacht.
  
\end{verse}
\end{center}


%\attrib{\small{THZE}}
