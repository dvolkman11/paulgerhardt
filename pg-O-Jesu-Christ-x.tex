%StartInfo%%%%%%%%%%%%%%%%%%%%%%%%%%%%%%%%%%%%%%%%%%%%%%%%%%%%%%%%%%%%%%%%%%%%
%  Autor:
%  Titel:
%  File:
%  Ref:
%  Mod:
%EndInfo%%%%%%%%%%%%%%%%%%%%%%%%%%%%%%%%%%%%%%%%%%%%%%%%%%%%%%%%%%%%%%%%%%%%%%
%\poemtitle{pt}
\begin{multicols}{2}
\settowidth{\versewidth}{gibst sich zum Dienst und wird ein Knecht der Sünder.}
\begin{verse}[\versewidth]
%o Jesu Christ, dein Kripplein ist mein Paradies

\flagverse{1.} O Jesu Christ,dein Kripplein ist\\
mein Paradies, da meine Seele weidet!\\
Hier ist der Ort, hier liegt das Wort,\\
mit unserm Fleisch persönlich angekleidet.

\flagverse{2.} Dem Meer und Wind gehorsam sind,\\
gibst sich zum Dienst und wird ein Knecht der Sünder.\\
Du, Gottes Sohn, wirst Erd und Ton,\\
gering und schwach wie wir und unsre Kinder.

\flagverse{3.} Du, höchstes Gut, hebst unser Blut\\
in deinen Thron hoch über alle Höhen.\\
Du, ewge Kraft, machst Brüderschaft\\
mit uns, die wie ein Dampf und Rauch vergehen.

\flagverse{4.} Was will uns nun zuwider tun\\
der Seelenfeind mit allem Gift und Gallen?\\
Was wirft er mir und andern für,\\
daß Adam ist, und wir mit ihm, gefallen?

\flagverse{5.} Schweig arger Feind! Da sitzt mein Freund,\\
mein Fleisch und Blut, hoch in dem Himmel droben;\\
was du gefällt, das hat der Held\\
aus Jakobs Stamm zu großer Ehr erhoben.

\flagverse{6.} Sein Licht und Heil macht alles heil;\\
der Himmelsschatz bringt allen Schaden wieder.\\
Der Freudenquell Immanuel\\
schlägt Teufel, Höll und all ihr Reich darnieder.

\flagverse{7.} Drum frommer Christ, wer du auch bist,\\
sei gutes Muts und laß dich nicht betrüben;\\
weil Gottes Kind dich ihm verbind't,\\
so kanns nicht anders sein, Gott muß dich lieben.

\flagverse{8.} Gedenke doch, wie herrlich hoch\\
er über alle Jammer dich geführet!\\
Der Engel Heer ist selbst nicht mehr\\
als eben du mit Seligkeit gezieret.

\flagverse{9.} Du siehest ja vor Augen da\\
dein Fleisch und Blut die Luft und Wolken lenken;\\
was will doch sich – ich frage dich –\\
erheben, dich in Angst und Furcht zu senken?

\flagverse{10.} Dein blöder Sinn geht oft dahin,\\
ruft Ach und Weh, läßt allen Trost verschwinden.\\
Komm her und richt dein Angesicht\\
zum Kripplein Christi, da, da wirst du's finden.

\flagverse{11.} Wirst du geplagt? Ei, unverzagt!\\
Dein Bruder wird dein Unglück nicht verschmähen;\\
sein Herz ist weich und gnadenreich,\\
kann unser Leid nicht ohne Tränen sehen.

\flagverse{12.} Tritt zu ihm zu! Such Hilf und Ruh!\\
Er wird's so machen, daß du ihm wirst danken.\\
Er weiß und kennt was beißt und brennt,\\
versteht wohl, wie zu Mute sei dem Kranken.

\flagverse{13.} Denn eben drum hat er den Grimm\\
des Kreuzes auch am Leibe wollen tragen,\\
daß seine Pein ihm möge sein\\
ein unverrückt Erinnrung unsrer Plagen.

\flagverse{14.} Mit einem Wort: Er ist die Pfort\\
zu dieses und des andern Lebens Freuden;\\
er macht behend ein seligs End\\
an alle dem, was fromme Herzen leiden.

\end{verse}
\end{multicols}

\begin{center}
\settowidth{\versewidth}{Der, vor dem die Welt erschrickt,}
\begin{verse}[\versewidth]

\flagverse{15.} Laß aller Welt ihr Gut und Geld\\
und siehe nur, daß dieser Schatz dir bleibe!\\
Wer den hier fest hält und nicht läßt,\\
den ehrt und krönt er dort an Seel und Leibe.

\end{verse}
\end{center}



%\attrib{\small{THZE}}
