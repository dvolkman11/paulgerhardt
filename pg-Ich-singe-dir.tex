%StartInfo%%%%%%%%%%%%%%%%%%%%%%%%%%%%%%%%%%%%%%%%%%%%%%%%%%%%%%%%%%%%%%%%%%%%
%  Autor:
%  Titel:
%  File:
%  Ref:
%  Mod:
%EndInfo%%%%%%%%%%%%%%%%%%%%%%%%%%%%%%%%%%%%%%%%%%%%%%%%%%%%%%%%%%%%%%%%%%%%%%
%\poemtitle{pt}
\begin{multicols}{2}
\settowidth{\versewidth}{Ich weiß, daß du der Brunn der Gnad}
\begin{verse}[\versewidth]
%ich singe dir mit Herz und Mund

\flagverse{1.} Ich singe dir mit Herz und Mund,\\
Herr, meines Herzens Lust,\\
ich sing und mach auf Erden kund,\\
was mir von dir bewußt.

\flagverse{2.} Ich weiß, daß du der Brunn der Gnad\\
und ewge Quelle seist,\\
daraus uns allen früh und spat\\
viel Heil und Gutes fleußt.

\flagverse{3.} Was sind wir doch? was haben wir\\
auf dieser ganzen Erd,\\
das uns, o Vater, nicht von dir\\
allein gegeben werd?

\flagverse{4.} Wer hat das schöne Himmelszelt\\
hoch über uns gesetzt?\\
Wer ist es, der uns unser Feld\\
mit Tau und Regen netzt?

\flagverse{5.} Wer wärmet uns in Kält und Frost?\\
Wer schützt uns vor dem Wind?\\
Wer macht es, daß man Öl und Most\\
zu seinen Zeiten findt?

\flagverse{6.} Wer gibt uns Leben und Geblüt?\\
Wer hält mit seiner Hand\\
den güldnen, werten, edlen Fried\\
in unserm Vaterland?

\flagverse{7.} Ach Herr, mein Gott, das kommt von dir!\\
Du, du mußt alles tun,\\
du hältst die Wacht an unsrer Tür\\
und läßt uns sicher ruhn.

\flagverse{8.} Du nährest uns von Jahr zu Jahr,\\
bleibst immer fromm und treu\\
und stehst uns, wann wir in Gefahr\\
geraten, treulich bei.

\flagverse{9.} Du strafst uns Sünder mit Geduld\\
und schlägst nicht allzu sehr,\\
ja endlich nimmst du unsre Schuld\\
und wirfst sie in das Meer.

\flagverse{10.} Wann unser Herze seufzt und schreit,\\
wirst du gar leicht erweicht,\\
und gibst uns, was uns hoch erfreut\\
und dir zu Ehren reicht.

\flagverse{11.} Du zählst, wie oft ein Christe wein\\
und was sein Kummer sei,\\
kein Zähr- und Tränlein ist so klein,\\
du hebst und legst es bei.

\flagverse{12.} Du füllst des Lebens Mangel aus\\
mit dem, was ewig steht,\\
und führst uns in das Himmelshaus,\\
wann uns die Erd entgeht.

\flagverse{13.} Wohlauf, mein Herze, sing und spring\\
und habe guten Mut,\\
dein Gott, der Ursprung aller Ding,\\
ist selbst und bleibt dein Gut.

\flagverse{14.} Er ist dein Schatz, dein Erb und Teil,\\
dein Glanz und Freudenlicht,\\
dein Schirm und Schild, dein Hilf und Heil,\\
schafft Rat und läßt dich nicht.

\flagverse{15.} Was kränkst du dich in deinem Sinn\\
und grämst dich Tag und Nacht?\\
Nimm deine Sorg und wirf sie hin\\
auf den, der dich gemacht!

\flagverse{16.} Hat er dich nicht von Jugend auf\\
versorget und ernährt?\\
Wie manches schweren Unglücks Lauf\\
hat er zurückgekehrt!

\flagverse{17.} Er hat noch niemals was versehn\\
in seinem Regiment,\\
nein, was er tut und läßt geschehn,\\
das nimmt ein gutes End.

\flagverse{18.} Ei nun, so laß ihn ferner tun\\
und red ihm nicht darein,\\
so wirst du hier im Frieden ruhn\\
und ewig fröhlich sein.
    
\end{verse}
\end{multicols}
%\attrib{\small{THZE}}
