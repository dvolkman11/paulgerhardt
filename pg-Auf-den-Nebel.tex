%StartInfo%%%%%%%%%%%%%%%%%%%%%%%%%%%%%%%%%%%%%%%%%%%%%%%%%%%%%%%%%%%%%%%%%%%%
%  Autor:
%  Titel:
%  File:
%  Ref:
%  Mod:
%EndInfo%%%%%%%%%%%%%%%%%%%%%%%%%%%%%%%%%%%%%%%%%%%%%%%%%%%%%%%%%%%%%%%%%%%%%%
%\poemtitle{Auf den Nebel folgt die Sonne}
\begin{multicols}{2}
\settowidth{\versewidth}{Der, vor dem die Welt erschrickt,}
\begin{verse}[\versewidth]

\flagverse{1.} Auf den Nebel folgt die Sonn,\\
auf das Trauern Freud und Wonn,\\
auf die schwere bittre Pein\\
stellt sich Trost und Labsal ein.\\
Meine Seele, die zuvor\\
sank bis zu dem Höllentor,\\
steigt nun bis zum Himmelschor.

\flagverse{2.} Der, vor dem die Welt erschrickt,\\
hat mir meinen Geist erquickt,\\
seine hohe starke Hand\\
reißt mich aus der Höllen Band;\\
alle seine Lieb und Güt\\
überschwemmt mir mein Gemüt\\
und erfrischt mir mein Geblüt.

\flagverse{3.} Hab ich vormals Angst gefühlt,\\
hat der Gram mein Herz zerwühlt,\\
hat der Kummer mich beschwert,\\
hat der Satan mich betört:\\
Ei, so bin ich nunmehr frei,\\
Heil und Rettung, Schutz und Treu\\
steht mir wieder treulich bei.

\flagverse{4.} Nun erfahr ich, schnöder Feind,\\
wie du's habst mit mir gemeint,\\
du hast wahrlich mich mit Macht\\
in dein Netz zu ziehn gedacht.\\
Hätt ich dir zuviel getraut,\\
hättst du, eh ich zugeschaut,\\
mir zu Fall ein Sieb gebaut.

\flagverse{5.} Ich erkenne deine List,\\
da du mit erfüllet bist;\\
du belügst mir meinen Gott\\
und machst seinen Ruhm zu Spott:\\
Wann er setzt, so wirfst du üm.\\
Wann er spricht, verkehrt dein Grimm\\
seine süße Vaterstimm.

\flagverse{6.} Hoff und wart ich alles Guts,\\
bin ich froh und gutes Muts,\\
rückst du mir aus meinem Sinn\\
alles gute Sinnen hin:\\
Gott ist, sprichst du, fern von dir,\\
alles Unglück bricht herfür,\\
steht und liegt vor deiner Tür.

\flagverse{7.} Heb dich weg, verlogner Mund!\\
Hie ist Gott und Gottes Grund,\\
hie ist Gottes Angesicht\\
und das schöne helle Licht\\
seines Segens, seiner Gnad;\\
all sein Wort und weiser Rat\\
steht vor mir in voller Tat.

\flagverse{8.} Gott läßt keinen traurig stehn,\\
noch mit Schimpf zurückegehn,\\
der sich ihm zu eigen schenkt\\
und ihn in sein Herze senkt;\\
wer auf Gott sein Hoffnung setzt,\\
findet endlich und zuletzt\\
was ihm Leib und Seel ergötzt.

\flagverse{9.} Kommts nicht heute, wie man will,\\
sei man nur ein wenig still:\\
Ist doch morgen auch ein Tag,\\
da die Wohlfahrt kommen mag.\\
Gottes Zeit hält ihren Schritt,\\
wann die kommt, kommt unsre Bitt\\
und die Freude reichlich mit.

\flagverse{10.} Ach, wie ofte dacht ich doch,\\
da mir noch des Trübsals Joch\\
auf dem Haupt und Halse saß\\
und das Leid mein Herze fraß:\\
Nun ist keine Hoffnung mehr,\\
auch kein Ruhen, bis ich kehr\\
in das schwarze Totenmeer.

\flagverse{11.} Aber mein Gott wandt es bald,\\
heilt und hielt mich dergestalt,\\
daß ich, was sein Arm getan,\\
nimmermehr gnug preisen kann;\\
da ich weder hie noch da\\
einen Weg zur Rettung sah,\\
hatt ich seine Hilfe nah.

\flagverse{12.} Als ich furchtsam und verzagt\\
mich selbst und mein Herze plagt,\\
als ich manche liebe Nacht\\
mich mit Wachen krank gemacht,\\
als mir aller Mut entfiel:\\
Tratst du, mein Gott, selbst ins Spiel,\\
gabst dem Unfall Maß und Ziel.

\flagverse{13.} Nun, so lang ich in der Welt\\
haben werde Haus und Zelt,\\
soll mir dieser Wunderschein\\
stets vor meinen Augen sein.\\
Ich will all mein Leben lang\\
meinem Gott mit Lobgesang\\
hiefür bringen Lob und Dank.

\flagverse{14.} Allen Jammer, allen Schmerz,\\
den des ewgen Vaters Herz\\
mir schon jetzo zugezählt\\
oder künftig auserwählt,\\
will ich hier in diesem Lauf\\
meines Lebens allzuhauf\\
frisch und freudig nehmen auf.

\end{verse}
\end{multicols}

\begin{center}
\settowidth{\versewidth}{Der, vor dem die Welt erschrickt,}
\begin{verse}[\versewidth]
  
\flagverse{15.} Ich will gehn in Angst und Not,\\
ich will gehn bis in den Tod,\\
ich will gehn ins Grab hinein\\
und doch allzeit fröhlich sein.\\
Wem der Stärkste bei will stehn,\\
wen der Höchste will erhöhn,\\
kann nicht ganz zugrunde gehn.

\end{verse}
\end{center}


%\attrib{\small{THZE}}
