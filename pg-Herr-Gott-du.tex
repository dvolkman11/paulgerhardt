%StartInfo%%%%%%%%%%%%%%%%%%%%%%%%%%%%%%%%%%%%%%%%%%%%%%%%%%%%%%%%%%%%%%%%%%%%
%  Autor:
%  Titel:
%  File:
%  Ref:
%  Mod:
%EndInfo%%%%%%%%%%%%%%%%%%%%%%%%%%%%%%%%%%%%%%%%%%%%%%%%%%%%%%%%%%%%%%%%%%%%%%
%\poemtitle{Herr Gott, du bist ja für und für die Zuflucht deiner Herde}
\begin{multicols}{2}
\settowidth{\versewidth}{Wir sind ein Kraut, das bald verdorrt,}
\begin{verse}[\versewidth]
%Der 90. Psalm


\flagverse{1.} Herr Gott, du bist ja für und für\\
die Zuflucht deiner Herde,\\
du bist gewesen, eh allhier\\
gelegt der Grund zur Erde;\\
und da noch kein Berg war bereit,\\
da warst du in der Ewigkeit,\\
o Anfang aller Dinge.

\flagverse{2.} Du läßt die Menschen in das Tor\\
des Todes häufig wandern\\
und sprichst: Kommt wieder, Menschen, vor\\
und folget jenen andern!\\
Denn dir sind, Höchster, tausend Jahr\\
als wie ein Tag, der gestern war\\
und nunmehr ist vergangen.

\flagverse{3.} Du läßt das schnöde Menschenheer\\
wie einen Strom verfließen\\
und wie die Schifflein auf dem Meer\\
bei gutem Wind hinschießen:\\
Gleichwie ein Schlaf und Traum bei Nacht,\\
der, wenn der Mensch vom Schlaf erwacht,\\
entfallen und vergessen.

\flagverse{4.} Wir sind ein Kraut, das bald verdorrt,\\
ein Gras, das jetzt aufgehet,\\
wird aber schnell von seinem Ort\\
entführet und verwehet;\\
so ist ein Mensch: Heut blühet er,\\
und morgen, wann ihn ungefähr\\
ein Wind rührt, liegt er nieder.

\flagverse{5.} Das macht, Herr, deines Zornes Grimm\\
daß wir sobald verschwinden;\\
dein Eifer stößt und wirft uns üm,\\
von wegen unsrer Sünden.\\
Die Sünden stellest du vor dich,\\
davon brennt und entrüstet sich\\
dein allzeit reines Herze.

\flagverse{6.} Das ist das Feur, das uns versehrt\\
das Mark in allen Beinen,\\
daher kommts, daß der Tod verzehrt\\
die Großen und die Kleinen;\\
drum fahren unsre Tage hin\\
wie ein Geschwätze durch den Sinn,\\
wenn wir die Zeit vertreiben.

\flagverse{7.} Wie lang hält doch das Leben aus?\\
Gar selten siebzig Jahre.\\
Wenns hoch kommt, werden achtzig draus,\\
und wenn man alle Ware,\\
die hier gewonnen, nimmt zuhauf\\
ists lauter Müh von Jugend auf\\
und lauter Angst gewesen.

\flagverse{8.} Wir rennen, laufen, sorgen viel,\\
und eh wir uns versehen,\\
da kommt der Tod, steckt uns das Ziel,\\
und da ists dann geschehen;\\
wie fliehen eilend und behend,\\
und ist doch niemand, der sein End\\
und Gottes Zorn bedenke.

\flagverse{9.} Lehr uns bedenken, frommer Gott,\\
das Elend dieser Erden,\\
auf daß wir, wann wir an den Tod\\
gedenken, klüger werden!\\
Ach Kehre wieder, kehr uns zu\\
dein Angesicht und steh in Ruh\\
mit deinen bösen Knechten!

\flagverse{10.} Erfüll uns früh mit deiner Gnad\\
am Leib und an der Seelen,\\
so wollen wir dir früh und spat\\
dein Lob mit Dank erzählen;\\
erfreu uns, o du höchste Freud,\\
und gib uns wieder gute Zeit\\
nach so viel bösen Tagen!

\flagverse{11.} Bisher hats lauter Kreuz geschneit,\\
laß nun die Sonne scheinen,\\
beschehr uns Freude nach dem Leid\\
und Lachen nach dem Weinen!\\
Laß deiner Werke süßen Schein,\\
Herr, deinen Knechten kundbar sein\\
und dein Ehr ihren Kindern!

\flagverse{12.} Bleib unser Gott und treuer Freund,\\
halt uns auf festem Fuße;\\
und wenn wir etwa irrig seind,\\
so gib, daß sich mit Buße\\
das Herze wieder zu dir wend;\\
auch fördre das Tun unser Händ\\
und segn all unsre Werke!

\end{verse}
\end{multicols}
%\attrib{\small{THZE}}
