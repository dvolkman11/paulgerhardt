%StartInfo%%%%%%%%%%%%%%%%%%%%%%%%%%%%%%%%%%%%%%%%%%%%%%%%%%%%%%%%%%%%%%%%%%%%
%  Autor:
%  Titel:
%  File:
%  Ref:
%  Mod:
%EndInfo%%%%%%%%%%%%%%%%%%%%%%%%%%%%%%%%%%%%%%%%%%%%%%%%%%%%%%%%%%%%%%%%%%%%%%
%\poemtitle{Herr, ich will gar gerne bleiben}
\begin{multicols}{2}
\settowidth{\versewidth}{Herr, ich will gar gerne bleiben,}
\begin{verse}[\versewidth]
%Selbsterniedrigung\\
%(Matth. 15, 21 u. Mark. 7, 28)\\
%nach den lateinischen Distichen des Nathan Chyträus »Sum canis indignus« 1568

\flagverse{1.} Herr, ich will gar gerne bleiben,\\
wie ich bin, dein armer Hund,\\
will auch anders nicht beschreiben\\
mich und meines Herzens Grund.\\
Denn ich fühle, was ich sei:\\
Alles Böse wohnt mir bei,\\
ich bin aller Schand ergeben,\\
unrein ist mein ganzes Leben.

\flagverse{2.} Hündisch ist mein Zorn und Eifer,\\
hündisch ist mein Neid und Haß,\\
hündisch ist mein Zank und Geifer,\\
hündisch ist mein Raub und Fraß;\\
ja, wenn ich mich recht genau,\\
als ich billig soll, beschau,\\
halt ich mich in vielen Sachen\\
ärger, als die Hund es machen.

\flagverse{3.} Ich will auch nicht mehr begehren,\\
als mir zukommt und gebührt,\\
wollst mich nur des Rechts gewähren,\\
das ein Hund im Hause führt!\\
Deine Heilgen, die sich dir\\
hier ergeben für und für,\\
mögen oben an der Spitzen\\
deiner Himmelstafel sitzen.

\flagverse{4.} Deine Kinder, die dich ehren\\
und in voller Tugend stehn,\\
mögen sich von Wollust nähren\\
und im Erbe sich erhöhn,\\
das du ihnen in dem Licht\\
deines Saals hast zugericht't,\\
ich will, wenn ich nur kann liegen\\
unterm Tisch, mir lassen gnügen.

\flagverse{5.} Ich will ins Verborgne kriechen,\\
da die Nacht den Tag verhüllt,\\
und hin nach der Erden riechen,\\
suchen, was den Hunger stillt;\\
ich will mit den Brosamlein,\\
die ich finde, friedlich sein\\
und mich freuen über allen,\\
was die Herren lassen fallen.

\flagverse{6.} Murren will ich auch und bellen,\\
aber gleichwohl weiter nicht,\\
als nur wenn in Sündenfällen\\
dir von mir ein Schimpf geschicht,\\
wenn mein Fleisch mich übereilt\\
und zur Buße, die uns heilt,\\
sich viel träger als zur Sünden\\
und zur Bosheit lässet finden.

\flagverse{7.} Dennoch will ohn alles Heucheln,\\
das so fest sonst in uns steckt,\\
ich dir auch hinwieder schmeicheln,\\
wenn ich deinen Zorn erweckt\\
und du meinen Übermut\\
strafest mit der scharfen Rut.\\
Ach Herr, schone, will ich sprechen,\\
laß mein Wort dein Herze brechen!

\flagverse{8.} Mache mich zum wackern Hüter,\\
dessen Augen offen sein,\\
wenn das schönste deiner Güter,\\
deine Kinder, schlafen ein.\\
Wenn das Haus zu Bette geht\\
und der Dieb mit Listen steht\\
nach des Nächsten Gut und Gelde,\\
ei, so gib, daß ich ihn melde!

\flagverse{9.} Mehre meinen kleinen Glauben\\
und wehr allem, das da will\\
dieses Schatzes mich berauben;\\
führe mich zum rechten Ziel!\\
Laß mich sein, o ewges Heil,\\
deines Hauses kleines Teil\\
auch den Kleinsten unter allen,\\
die nach deinem Reiche wallen.

\flagverse{10.} Hab ich dies, so ruht mein Wille,\\
denn ich habe selber dich,\\
dich, du unvermessne Fülle\\
dessen, was mich ewiglich\\
in dem Himmel laben soll.\\
Wohl mir, wohl und aber wohl!\\
Soll mich Gottes Fülle laben,\\
woran will ich Mangel haben?

\end{verse}
\end{multicols}
%\attrib{\small{THZE}}
