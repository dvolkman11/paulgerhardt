%StartInfo%%%%%%%%%%%%%%%%%%%%%%%%%%%%%%%%%%%%%%%%%%%%%%%%%%%%%%%%%%%%%%%%%%%%
%  Autor:
%  Titel:
%  File:
%  Ref:
%  Mod:
%EndInfo%%%%%%%%%%%%%%%%%%%%%%%%%%%%%%%%%%%%%%%%%%%%%%%%%%%%%%%%%%%%%%%%%%%%%%
%\poemtitle{pt}
\begin{multicols}{2}
\settowidth{\versewidth}{Nun geht frisch drauf, es geht nach Haus,}
\begin{verse}[\versewidth]

\flagverse{1.} Nun geht frisch drauf, es geht nach Haus,\\
ihr Rößlein, regt die Bein;\\
ich will dem, der uns ein und aus\\
begleitet, dankbar sein.

\flagverse{2.} Ich will ihm singen Lob und Preis,\\
so viel ich singen kann,\\
ich will sein Werk, so gut ichs weiß,\\
mit Freuden zeigen an.

\flagverse{3.} Es ist fürwahr nicht Menschenkunst,\\
auf sichern Wegen gehn,\\
führt uns nicht Gott und Gottes Gunst,\\
würds oftmals seltsam stehn.

\flagverse{4.} Wie manches Leid, wie manche Not,\\
wie manches Jammerheer\\
brächt uns in Angst, tät uns den Tod,\\
wo Gott nicht bei uns wär.

\flagverse{5.} Wie mancher Feind, wie mancher Dieb,\\
wo ihn nicht Gott gerührt,\\
hätt uns das Unsre, das uns lieb,\\
genommen und entführt.

\flagverse{6.} Wie mancher böser schwarzer Geist\\
hätt unser Leib und Seel,\\
wo uns der Herr nicht Gnad erweist,\\
erschreckt aus seiner Höhl.

\flagverse{7.} Es ist der alte große Drach\\
doch allzeit ohne Ruh,\\
wohin wir gehn, da geht er nach\\
und setzt uns heftig zu.

\flagverse{8.} Er sucht zu Haus, er sucht zu Feld,\\
er sucht zur See und Land,\\
er sucht uns in der ganzen Welt\\
mit unverdroßner Hand.

\flagverse{9.} Noch dennoch trifft er uns nicht an,\\
sein Anschlag geht zurück,\\
denn Gottes Schutz hegt unsre Bahn\\
für unsres Feindes Tück.

\flagverse{10.} Es zeucht der heilgen Engel Schar,\\
mit Waffen ausgerüst,\\
und wehren fleißig hie und dar\\
des Tausendkünstlers List.

\flagverse{11.} Es müssen ja noch immerfort\\
die Mahanaim gehn\\
und Gottes Volk auf Gottes Wort\\
zu Dienst und Willen stehn.

\flagverse{12.} Wenn Gott mir meiner Augen Licht\\
mit Licht erfüllen wollt,\\
als wie dem Jakob, der sich nicht\\
für Esau fürchten sollt:

\flagverse{13.} Ach, was für Wunder würd ich hier\\
auf meinen Reisen sehn,\\
wie schön, wie lieblich würde mir\\
in solchem Sehn geschehn.

\flagverse{14.} Nun, was den Augen nicht vergunnt,\\
das sieht mein Herz und Geist,\\
dem Gott der heilgen Weisheit Grund\\
in seinem Geiste weist.

\flagverse{15.} Es ist sein Wort, er hats gesagt:\\
Sein Heervolk sei bereit,\\
uns zu umlagern, wenn uns plagt\\
des Satans Neid und Streit.

\flagverse{16.} Was Gott geredt, das ist vollbracht,\\
mein Herz, sei wohlgemut\\
und laß ja nimmer aus der Acht,\\
was dein Gott an dir tut.

\flagverse{17.} Du siehst und greifst, wie gut er sei\\
dem, der ihn ehrt und liebt,\\
er ziert mit Lieb, er führt mit Treu\\
ein Herz, das ihm sich gibt.

\flagverse{18.} Er trägt uns, wie (wenn einher schlägt\\
blitz, Hagel, Sturm und Wind)\\
ein treuer frommer Vater trägt\\
sein kleines zartes Kind.

\flagverse{19.} Er deckt uns zu mit seiner Hand,\\
wie eine Mutter tut,\\
in derer Schoß das süßte Pfand\\
der keuschen Liebe ruht.

\flagverse{20.} Er räumt aus unsern Wegen weg\\
des Unglücks scharfen Stein\\
und schafft, daß unsre Bahn und Steg\\
fein schlicht und eben sein.

\flagverse{21.} Er führt uns über Berg und Tal,\\
und wenns nun rechte Zeit,\\
so führt er uns in seinen Saal\\
zur ewgen Himmelsfreud.

\flagverse{22.} Alsdann werd ich die letzte Reis\\
und schönste Heimfahrt tun\\
und nach dem sauren Erdenschweiß\\
in süßer Stille ruhn.

\end{verse}
\end{multicols}
\attrib{\small{THZE}}
