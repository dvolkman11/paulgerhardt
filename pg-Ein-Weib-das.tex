%StartInfo%%%%%%%%%%%%%%%%%%%%%%%%%%%%%%%%%%%%%%%%%%%%%%%%%%%%%%%%%%%%%%%%%%%%
%  Autor:
%  Titel:
%  File:
%  Ref:
%  Mod:
%EndInfo%%%%%%%%%%%%%%%%%%%%%%%%%%%%%%%%%%%%%%%%%%%%%%%%%%%%%%%%%%%%%%%%%%%%%%
%\poemtitle{Ein Weib, das Gott den Herren liebt}
\begin{multicols}{2}
\settowidth{\versewidth}{Sie ist ein Schifflein auf dem Meer,}
\begin{verse}[\versewidth]
%frauenlob\\
%ein Weib, das Gott den Herren liebt\\
%(Spr. Sal. 31, 10-30)

\flagverse{1.} Ein Weib, das Gott den Herren liebt\\
und sich stets in der Tugend übt,\\
ist viel mehr Lobs und Liebens wert\\
als alle Perlen auf der Erd.

\flagverse{2.} Ihr Mann darf mit dem Herzen frei\\
verlassen sich auf ihre Treu,\\
sein Haus ist voller Freud und Licht,\\
an Nahrung wirds ihm mangeln nicht.

\flagverse{3.} Sie tut ihm Liebes und kein Leid,\\
durchsüßet seine Lebenszeit,\\
sie nimmt sich seines Kummers an\\
mit Trost und Rat, so gut sie kann.

\flagverse{4.} Die Woll und Flachs sind ihre Lust,\\
was hierzu dien, ist ihr bewußt,\\
ihr Händlein greifet selber zu,\\
hat oftmals Müh und selten Ruh.

\flagverse{5.} Sie ist ein Schifflein auf dem Meer,\\
wann dieses kommt, so kommts nicht leer:\\
So schafft auch sie aus allem Ort\\
und setzet ihre Nahrung fort.

\flagverse{6.} Sie schläft mit Sorg, ist früh heraus,\\
gibt Futter, wo sie soll, im Haus\\
und speist die Dirnen, derer Hand\\
zu ihren Diensten ist gewandt.

\flagverse{7.} Sie gürtet ihre Lenden fest\\
und stärket ihre Arm aufs best,\\
ist froh, wenns wohl von statten geht,\\
worauf ihr Sinn und Herze steht.

\flagverse{8.} Wenn andre löschen Feur und Licht,\\
verlöscht doch ihre Leuchte nicht,\\
ihr Herze wachet Tag und Nacht\\
zu dem, der Tag und Nacht gemacht.

\flagverse{9.} Sie nimmt den Wocken, setzt sich hin\\
und schämt sich nicht, daß sie ihn spinn,\\
ihr Finger faßt die Spindel wohl\\
und macht sie schnell mit Garne voll.

\flagverse{10.} Sie hört gar leicht der Armen Bitt,\\
ist gütig, teilet gerne mit,\\
ihr Haus und alles Hausgesind\\
ist wohl verwahrt vor Schnee und Wind.

\flagverse{11.} Sie näht, sie sitzt, sie wirkt mit Fleiß,\\
macht Decken nach der Künstler Weis,\\
hält sich selbst sauber; weiße Seid\\
und Purpur ist ihr schönes Kleid.

\flagverse{12.} Ihr Mann ist in der Stadt berühmt,\\
bestellt sein Amt, wie sichs geziemt,\\
er geht, steht und sitzt obenan,\\
und was er tut, ist wohlgetan.

\flagverse{13.} Ihr Schmuck ist, daß sie reinlich ist,\\
ihr Ehr ist, daß sie ausgerüst\\
mit Fleiße, der gewiß zuletzt\\
den, der ihn liebet, hoch ergötzt.

\flagverse{14.} Sie öffnet ihren weisen Mund,\\
tut Kindern und Gesinde kund\\
des Höchsten Wort und lehrt sie fein\\
fromm, ehrbar und gehorsam sein.

\flagverse{15.} Sie schauet, wies im Hause steht\\
und wie es hier und dort ergeht,\\
sie ißt ihr Brot und sagt dabei,\\
wie so groß Unrecht Faulsein sei.

\flagverse{16.} Die Söhne, die ihr Gott beschert,\\
die halten sie hoch, lieb und wert,\\
ihr Mann, der lobt sie spat und früh\\
und preiset selig sich und sie.

\flagverse{17.} Viel Töchter bringen Geld und Gut,\\
sind zart am Leib und stolz am Mut,\\
du aber, meine Kron und Zier,\\
gehst wahrlich ihnen allen für.

\flagverse{18.} Was hilft der äußerliche Schein?\\
Was ists doch, schön und lieblich sein?\\
Ein Weib, Das Gott liebt, ehrt und scheut,\\
das soll man loben weit und breit.

\end{verse}
\end{multicols}

\begin{center}
\settowidth{\versewidth}{Der, vor dem die Welt erschrickt,}
\begin{verse}[\versewidth]

\flagverse{19.} Die Werke, die sie hier verricht,\\
sind wie ein schönes helles Licht,\\
sie dringen bis zur Himmelspfort\\
und werden leuchten hier und dort.
  
\end{verse}
\end{center}



%\attrib{\small{THZE}}
