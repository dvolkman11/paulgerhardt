%StartInfo%%%%%%%%%%%%%%%%%%%%%%%%%%%%%%%%%%%%%%%%%%%%%%%%%%%%%%%%%%%%%%%%%%%%
%  Autor:
%  Titel:
%  File:
%  Ref:
%  Mod: 05.11.2017 Spell check
%EndInfo%%%%%%%%%%%%%%%%%%%%%%%%%%%%%%%%%%%%%%%%%%%%%%%%%%%%%%%%%%%%%%%%%%%%%%
%\poemtitle{Herr, du erforschest meinen Sinn}
\begin{multicols}{2}
\settowidth{\versewidth}{Das ist mir kund. Und bleibet doch}
\begin{verse}[\versewidth]
%der 139. Psalm


\flagverse{1.} Herr, du erforschest meinen Sinn\\
und kennest, was ich hab und bin,\\
ja, was mir selbst verborgen ist,\\
das weißt du, der du alles bist.

\flagverse{2.} Ich sitz hier oder stehe auf,\\
ich lieg, ich geh auch oder lauf:\\
So bist du um und neben mir,\\
und ich bin allzeit hart bei dir.

\flagverse{3.} All die Gedanken meiner Seel,\\
und was sich in der Herzenshöhl\\
hier reget, hast du schon betracht,\\
eh ich einmal daran gedacht.

\flagverse{4.} Auf meiner Zunge ist kein Wort,\\
das du nicht hörest allsofort,\\
du schaffests, was ich red und tu,\\
und siehst all meinem Leben zu.

\flagverse{5.} Das ist mir kund. Und bleibet doch\\
mir solch Erkenntnis viel zu hoch,\\
es ist die Weisheit, die kein Mann\\
recht aus dem Grunde wissen kann.

\flagverse{6.} Wo soll ich, der du alles weißt,\\
mich wenden hin vor deinem Geist?\\
Wo soll ich deinem Angesicht\\
entgehen, daß michs sehe nicht?

\flagverse{7.} Führ ich gleich an des Himmels Dach\\
so bist du da, hältst Hut und Wach,\\
stieg ich zur Höll und wollte mir\\
da betten, find ich dich auch hier.

\flagverse{8.} Wollt ich der Morgenröten gleich\\
geflügelt ziehn, so weit das Reich\\
der wilden Fluten netzt das Land,\\
käm ich doch nie aus deiner Hand.

\flagverse{9.} Rief ich zu Hilf die finstre Nacht,\\
hätt ich doch damit nichts verbracht;\\
denn laß die Nacht sein wie sie mag,\\
so ist sie bei dir heller Tag.

\flagverse{10.} Dich blendt der dunkle Schatten nicht,\\
die Finsternis ist dir ein Licht,\\
dein Augenglanz ist klar und rein,\\
darf weder Sonn noch Mondenschein.

\flagverse{11.} Mein Eingeweid ist dir bekannt,\\
es liegt frei da in deiner Hand,\\
der du von Mutterleibe an\\
mir lauter Lieb und Guts getan.

\flagverse{12.} Du bists, der Fleisch, Gebein und Haut\\
so künstlich in mir aufgebaut;\\
all deine Werk sind Wunder voll,\\
und das weiß meine Seele wohl.

\flagverse{13.} Du sahest mich, da ich noch gar\\
fast nichts und unbereitet war,\\
warst selbst mein Meister über mir\\
und zogst mich aus der Tief herfür.

\flagverse{14.} Auch meiner Tag und Jahre Zahl,\\
Minuten, Stunden allzumal\\
hast du, als meiner Zeiten Lauf,\\
vor meiner Zeit geschrieben auf.

\flagverse{15.} Wie köstlich, herrlich, süß und schön\\
seh ich, mein Gott, da vor mir stehn\\
dein weises Denken, was du denkst,\\
wenn du uns deine Güter schenkst!

\flagverse{16.} Wie ist doch des so trefflich viel!\\
Wenn ich bisweilen zählen will,\\
so find ich da bei weitem mehr\\
als Staub im Feld und Sand am Meer.

\flagverse{17.} Was macht denn nun die wüste Rott,\\
die dich, o großer Wundergott,\\
so schändlich lästert und mit Schmach\\
dir so viel Übels redet nach?

\flagverse{18.} Ach, stopfe ihren schnöden Mund!\\
Steh auf und stürze sie zu Grund!\\
Denn weil sie deine Feinde seind,\\
bin ich auch ihnen herzlich feind.

\flagverse{19.} Ob sie gleich nun hinwieder sehr\\
mich hassen, tu ich doch nicht mehr,\\
als daß ich wider ihren Trutz\\
mich leg in deinen Schoß und Schutz.

\flagverse{20.} Erforsch, Herr, all mein Herz und Mut,\\
sieh, ob mein Weg sei recht und gut,\\
und führe mich bald himmelan\\
den ewgen Weg, die Freudenbahn.

\end{verse}
\end{multicols}
%\attrib{\small{THZE}}
