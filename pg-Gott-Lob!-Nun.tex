%StartInfo%%%%%%%%%%%%%%%%%%%%%%%%%%%%%%%%%%%%%%%%%%%%%%%%%%%%%%%%%%%%%%%%%%%%
%  Autor:
%  Titel:
%  File:
%  Ref:
%  Mod:
%EndInfo%%%%%%%%%%%%%%%%%%%%%%%%%%%%%%%%%%%%%%%%%%%%%%%%%%%%%%%%%%%%%%%%%%%%%%
%\poemtitle{Gott Lob! Nun ist erschollen das edle Fried- und Freudenwort}
\begin{multicols}{2}
\settowidth{\versewidth}{die Spieß und Schwerter und ihr Mord.}
\begin{verse}[\versewidth]
%Danklied für die Verkündigung des Friedens


\flagverse{1.} Gott Lob! Nun ist erschollen\\
das edle Fried- und Freudenwort,\\
daß nunmehr ruhen sollen\\
die Spieß und Schwerter und ihr Mord.\\
Wohlauf und nimm nun wieder\\
dein Saitenspiel hervor,\\
o Deutschland, und sing Lieder\\
im hohen vollen Chor.\\
Erhebe Dein Gemüte\\
zu deinem Gott und sprich:\\
Herr, deine Gnad und Güte\\
bleibt dennoch ewiglich!

\flagverse{2.} Wir haben nichts verdienet\\
als schwere Straf und großen Zorn,\\
weil stets noch bei uns grünet\\
der freche schnöde Sündendorn.\\
Wir sind fürwahr geschlagen\\
mit harter, scharfer Rut,\\
und dennoch muß man fragen:\\
Wer ist, der Buße tut?\\
Wir sind und bleiben böse,\\
Gott ist und bleibet treu,\\
hilft, daß sich bei uns löse\\
der Krieg und sein Geschrei.

\flagverse{3.} Sei tausendmal willkommen,\\
du teure werte Friedensgab!\\
Jetzt sehn wir, was für Frommen\\
dein Bei-uns-wohnen in sich hab;\\
in dir hat Gott versenket\\
all unser Glück und Heil.\\
Wer dich betrübt und kränket,\\
der drückt sich selbst den Pfeil\\
des Herzleids in das Herze\\
und löscht aus Unverstand\\
die güldne Freudenkerze\\
mit seiner eignen Hand.

\flagverse{4.} Das drückt uns niemand besser\\
in unser Herz und Seel hinein\\
als ihr zerstörten Schlösser\\
und Städte voller Schutt und Stein;\\
ihr vormals schönen Felder\\
mit frischer Saat bestreut,\\
jetzt aber lauter Wälder\\
und dürre wüste Heid;\\
ihr Gräber voller Leichen\\
und blutgen Heldenschweiß\\
der Helden, derengleichen\\
auf Erden man nicht weiß.

\flagverse{5.} Hier trübe deine Sinnen,\\
o Mensch, und laß die Tränenbach\\
aus beiden Augen rinnen,\\
geh in dein Herz und denke nach:\\
Was Gott bisher gesendet,\\
das hast du ausgelacht,\\
nun hat er sich gewendet\\
und väterlich bedacht,\\
vom Grimm und scharfen Dringen\\
zu deinem Heil zu ruhn,\\
ob er dich möchte zwingen\\
mit Lieb und Gutestun.

\flagverse{6.} Ach, laß dich doch erwecken,\\
wach auf, wach auf, du harte Welt,\\
eh als das harte Schrecken\\
dich schnell und plötzlich überfällt!\\
Wer aber Christum liebet,\\
sei unerschrocknes Muts,\\
der Friede, den er gibet,\\
bedeutet alles Guts.\\
Er will die Lehre geben:\\
Das Ende naht herzu,\\
da sollt ihr bei Gott leben\\
in ewgem Fried und Ruh.

\end{verse}
\end{multicols}
%\attrib{\small{THZE}}
