%StartInfo%%%%%%%%%%%%%%%%%%%%%%%%%%%%%%%%%%%%%%%%%%%%%%%%%%%%%%%%%%%%%%%%%%%%
%  Autor:
%  Titel:
%  File:
%  Ref:
%  Mod: 05.11.2017 Spall check
%EndInfo%%%%%%%%%%%%%%%%%%%%%%%%%%%%%%%%%%%%%%%%%%%%%%%%%%%%%%%%%%%%%%%%%%%%%%
%\poemtitle{Herr, dir trau ich all mein Tage}
\begin{multicols}{2}
\settowidth{\versewidth}{Herr, dir trau ich all mein Tage}
\begin{verse}[\versewidth]
%Der 71. Psalm\\
%Auf den Tod des Amtsschreibers Joachim Schröder zu Mittenwalde

\flagverse{1.} Herr, dir trau ich all mein Tage\\
laß mich nicht mit Schimpf bestehn.\\
Wie ich von dir glaub und sage,\\
also laß mirs auch ergehn.\\
Rette mich, laß deine Güte\\
mir erfrischen mein Gemüte,\\
neige deiner Ohren Treu\\
und vernimm mein Angstgeschrei!

\flagverse{2.} Sei mein Aufhalt, laß mich sitzen\\
bei dir, o mein starker Hort!\\
Laß mich deinen Schutz beschützen\\
und erfülle mir dein Wort,\\
da du selbsten meinem Leben\\
dich zum Fels und Burg gegeben.\\
Hilf mir aus des Heuchlers Band\\
und des Ungerechten Hand!

\flagverse{3.} Denn dich hab ich auserlesen\\
von der zarten Jugend an;\\
dein Arm ist mein Trost gewesen,\\
Herr, so lang ich denken kann;\\
auf dich hab ich mich erwogen,\\
alsbald du mich der entzogen,\\
der ich, ehe Nacht und Tag\\
mich erblickt, im Leibe lag.

\flagverse{4.} Von dir ist mein Ruhm, mein Sagen,\\
dein erwähn ich immerzu;\\
viel, die spotten meiner Plagen,\\
höhnen, was ich red und tu.\\
Aber du bist meine Stärke:\\
Wenn ich Angst und Trübsal merke,\\
lauf ich dich an. Gönne mir,\\
fröhlich stets zu sein in dir!

\flagverse{5.} Stoß mich nicht von deiner Seiten,\\
wenn mein hohes Alter kömmt,\\
da die schwachen Tritte gleiten\\
und man Trost vom Stecken nimmt;\\
da greif du mir an die Arme,\\
fall ich nieder, so erbarme\\
du dich, hilf mir in die Höh\\
und halt, bis ich wieder steh.

\flagverse{6.} Mach es nicht wie mirs die gönnen,\\
die mein abgesagte Feind,\\
auch mir, wo sie immer können,\\
mit Gewalt zuwider seind;\\
sprechen: Auf, laßt uns ihn fassen,\\
sein Gott hat ihn ganz verlassen,\\
jagt und schlagt ihn immerhin,\\
niemand schützt und rettet ihn!

\flagverse{7.} Ach, mein Helfer, sei nicht ferne,\\
komm und eile doch zu mir,\\
hilf mir, mein Gott, bald und gerne,\\
zeuch mich aus der Not herfür,\\
daß sich meine Feinde schämen\\
und vor Hohn und Schande grämen,\\
ich hingegen lustig sei\\
über mir erwiesne Treu.

\flagverse{8.} Mein Herz soll dir allzeit bringen\\
deines Lobs gebührlich Teil,\\
auch soll meine Zunge singen\\
täglich dein unzählig Heil.\\
Ich bin stark, hereinzugehen,\\
unerschrocken dazustehen\\
durch des großen Herrschers Kraft,\\
der die Erd und alles schafft.

\flagverse{9.} Herr, ich preise deine Tugend,\\
Wahrheit und Gerechtigkeit,\\
die mich noch in meiner Jugend\\
hoch ergötzet und erfreut;\\
hast mich als ein Kind ernähret,\\
deine Furcht dabei gelehret,\\
oftmals wunderlich bedeckt,\\
daß mein Feind mich nicht erschreckt.

\flagverse{10.} Fahre fort, o mein Erhalter,\\
fahre fort und laß mich nicht\\
in dem hohen grauen Alter,\\
wenn mir Lebenskraft gebricht;\\
laß mein Leben in dir leben,\\
bis ich Unterricht kann geben\\
Kindeskindern, daß dein Hand\\
ihnen gleichfalls sei bekannt.

\flagverse{11.} Gott, du bist sehr hoch zu loben,\\
dir ist nirgend etwas gleich,\\
weder hier bei uns noch droben\\
in dem Stern- und Engelreich.\\
Dein Tun ist nicht auszusprechen,\\
deinen Rat kann niemand brechen,\\
alles liegt dir in dem Schoß,\\
und dein Werk ist alles groß.

\flagverse{12.} Du ergibst mich großen Nöten,\\
gibst auch wieder große Freud,\\
heute läßt du mich ertöten,\\
morgen ist die Lebens Zeit,\\
da ermunterst du mich wieder\\
und erneuerst meine Glieder,\\
holst sie aus der Erdenkluft,\\
gibst dem Herzen wieder Luft.

\flagverse{13.} Such ich Trost und finde keinen?\\
Balde Werd ich wieder groß.\\
Dein Trost Trocknet mir mein Weinen,\\
das mir aus den Augen floß.\\
Ich selbst werde wie ganz neue,\\
sing und klinge deine Treue,\\
meines Lebens einzges Ziel,\\
auf der Harf und Psalterspiel.

\flagverse{14.} Ich bin durch und durch entzündet,\\
fröhlich ist, was in mir ist,\\
alle mein Geblüt empfindet\\
dein Heil, das du selber bist.\\
Ich steh im gewünschten Stande,\\
mein Feind ist voll Scham und Schande;\\
der mein Unglück hat gesucht,\\
leidet, was er mir geflucht.

\end{verse}
\end{multicols}
%\attrib{\small{THZE}}
