%StartInfo%%%%%%%%%%%%%%%%%%%%%%%%%%%%%%%%%%%%%%%%%%%%%%%%%%%%%%%%%%%%%%%%%%%%
%  Autor:
%  Titel:
%  File:
%  Ref:
%  Mod:
%EndInfo%%%%%%%%%%%%%%%%%%%%%%%%%%%%%%%%%%%%%%%%%%%%%%%%%%%%%%%%%%%%%%%%%%%%%%
%\poemtitle{pt}
\begin{multicols}{2}
\settowidth{\versewidth}{Siehe, mein getreuer Knecht,}
\begin{verse}[\versewidth]
%siehe, mein getreuer Knecht\\
%(Jes. 52, 13 ff. u. 53)

\flagverse{1.} Siehe, mein getreuer Knecht,\\
der wird weislich handeln,\\
ohne Tadel, schlecht und recht\\
auf der Erden wandeln;\\
sein getreuer, frommer Sinn\\
wird in Einfalt gehen,\\
und noch dennoch wird man ihn\\
an das Kreuz erhöhen.

\flagverse{2.} Hoch am Kreuze wird mein Sohn\\
große Marter leiden,\\
und viel werden ihn mit Hohn\\
als ein Scheusal meiden.\\
Aber also wird sein Blut\\
auf die Heiden springen\\
und das ewge wahre Gut\\
in ihr Herze bringen.

\flagverse{3.} Kön'ge werden ihren Mund\\
gegen ihn verhalten.\\
Und aus innerm Herzensgrund\\
ihre Hände falten.\\
Das verblend'te taube Heer\\
wird ihn sehn und hören\\
und mit Lust zu seiner Ehr\\
ihren Glauben mehren.

\flagverse{4.} Aber da, wo Gottes Licht\\
reichlich wird gespüret,\\
hält man sich mit nichten nicht\\
wie es sich gebühret:\\
Denn wer glaubt im Judenland\\
unsrer Predigt Worten?\\
Wem wird Gottes Arm bekannt\\
in Israels Orten?

\flagverse{5.} Niemand will fast seinen Preis\\
ihm hie lassen werden,\\
denn er schießt auf wie ein Reis\\
aus der dürren Erden,\\
krank, verdorret, ungestalt,\\
voller Blut und Schmerzen,\\
daher scheut ihn jung und alt\\
mit verwandtem Herzen.

\flagverse{6.} Ei, was hat er denn getan?\\
Was sind seine Schulden,\\
daß er da für jedermann\\
solche Schmach muß dulden?\\
Hat er etwann Gott betrübt\\
bei gesunden Tagen,\\
daß er ihm anitzo gibt\\
seinen Lohn mit Plagen?

\flagverse{7.} Nein, fürwahr! Wahrhaftig nein!\\
Er ist ohne Sünden.\\
Sondern was der Mensch für Pein\\
billig sollt empfinden,\\
was für Krankheit, Angst und Weh\\
uns von Recht gebühret,\\
das ist's was ihn in die Höh\\
an das Kreuz geführet.

\flagverse{8.} Daß ihn Gott so heftig schlägt,\\
tut er unsertwillen,\\
daß er solche Bürden trägt,\\
damit will er stillen\\
Gottes Zorn und großen Grimm,\\
daß wir Frieden haben\\
durch sein Leiden und in ihm\\
leib und Seele laben.

\flagverse{9.} Wir sinds, die wir in der Irr\\
als die Schafe gingen\\
und noch stets zur Höllentür\\
als die Tollen dringen.\\
Aber Gott, der fromm und treu,\\
nimmt, was wir verdienen\\
und legts seinem Sohne bei,\\
der muß uns versühnen.

\flagverse{10.} Nun, er tut es herzlich gern,\\
ach, des frommen Herzens!\\
Er nimmt an den Zorn des Herrn\\
mit viel tausend Schmerzen\\
und ist allzeit voll Geduld,\\
läßt kein Wörtlein hören\\
wider die, so ohne Schuld\\
ihn so hoch beschweren.

\flagverse{11.} Wie ein Lämmlein sich dahin\\
läßt zur Schlachtbank leiten\\
und hat in dem frommen Sinn\\
gar kein Widerstreiten,\\
läßt sich handeln, wie man will,\\
fangen, binden, zähmen\\
und dazu in großer Still\\
auch sein Leben nehmen.

\flagverse{12.} Also läßt auch Gottes Lamm\\
ohne Widersprechen\\
ihm sein Herz am Kreuzesstamm\\
unsertwegen brechen.\\
Er sinkt in den Tod hinab,\\
den er selbst doch bindet,\\
weil er sterbend Tod und Grab\\
mächtig überwindet.

\flagverse{13.} Er wird aus der Angst und Qual\\
endlich ausgerissen,\\
tritt den Feinden allzumal\\
ihren Kopf mit Füßen.\\
Wer will seines Lebens Läng\\
immer mehr ausrechnen?\\
Seiner Tag und Jahre Meng\\
ist nicht auszusprechen.

\flagverse{14.} Doch ist er wahrhaftig hier\\
für sein Volk gestorben\\
und hat völlig mir und dir\\
Heil und Gnad erworben,\\
kommt auch in das Grab hinein\\
herrlich eingehüllet,\\
wie die, so mit Reichtum sein\\
in der Welt erfüllet.

\flagverse{15.} Er wird als ein böser Mann\\
vor der Welt geplaget,\\
da er doch noch nie getan,\\
auch noch nie gesaget,\\
was da bös und unrecht wär;\\
er hat nie betrogen,\\
nie verletzet Gottes Ehr,\\
sein Mund nie gelogen.

\flagverse{16.} Ach, er ist für fremde Sünd\\
in den Tod gegeben,\\
auf daß du, o Menschenkind,\\
durch ihn möchtest leben,\\
daß er mehrte sein Geschlecht,\\
den gerechten Samen,\\
der Gott dient und Opfer brächt\\
seinem heilgen Namen.

\flagverse{17.} Denn das ist sein höchste Freud\\
und des Vaters Wille,\\
daß den Erdkreis weit und breit\\
sein Erkenntnis fülle,\\
damit der gerechte Knecht,\\
der vollkommne Sühner,\\
gläubig mach und recht gerecht\\
alle Sündendiener.

\flagverse{18.} Große Menge wird ihm Gott\\
zur Verehrung schenken,\\
darum, daß er sich mit Spott\\
für uns lassen kränken,\\
da er denen gleich gesetzt,\\
die sehr übertreten,\\
auch die, so ihn hoch verletzt,\\
bei Gott selbst verbeten.

\end{verse}
\end{multicols}
%\attrib{\small{THZE}}
