%StartInfo%%%%%%%%%%%%%%%%%%%%%%%%%%%%%%%%%%%%%%%%%%%%%%%%%%%%%%%%%%%%%%%%%%%%
%  Autor:
%  Titel:
%  File:
%  Ref:
%  Mod:
%EndInfo%%%%%%%%%%%%%%%%%%%%%%%%%%%%%%%%%%%%%%%%%%%%%%%%%%%%%%%%%%%%%%%%%%%%%%
%\poemtitle{pt}
\begin{multicols}{2}
\settowidth{\versewidth}{Nehmt weg das Stroh, nehmt weg das Heu,}
\begin{verse}[\versewidth]
 
\flagverse{1.} Ich steh an deiner Krippen hier,\\
o Jesulein, mein Leben;\\
ich komme, bring und schenke dir,\\
was du mir hast gegeben.\\
Nimm hin, es ist mein Geist und Sinn,\\
Herz, Seel und Mut, nimm alles hin\\
und laß dir's wohlgefallen.
 
\flagverse{2.} Du hast mit deiner Lieb erfüllt\\
mein Adern und Geblüte,\\
dein schöner Glanz, dein süßes Bild\\
liegt mir ganz im Gemüte.\\
Und wie mag es auch anders sein:\\
Wie könnt ich dich, mein Herzelein,\\
aus meinem Herzen lassen!
 
\flagverse{3.} Da ich noch nicht geboren war,\\
da bist du mir geboren\\
und hast mich dir zu eigen gar,\\
eh ich dich kannt, erkoren.\\
Eh ich durch deine Hand gemacht,\\
da hast du schon bei dir bedacht,\\
wie du mein wolltest werden.
 
\flagverse{4.} Ich lag in tiefster Todesnacht,\\
du warest meine Sonne,\\
die Sonne, die mir zugebracht\\
Licht, Leben, Freud und Wonne.\\
O Sonne, die das werte Licht\\
des Glaubens in mir zugericht't,\\
wie schön sind deine Strahlen!
 
\flagverse{5.} Ich sehe dich mit Freuden an\\
und kann mich nicht satt sehen,\\
und weil ich nun nicht weiter kann,\\
so tu ich, was geschehen.\\
O daß mein Sinn ein Abgrund wär\\
und meine Seel ein weites Meer,\\
daß ich dich möchte fassen!
 
\flagverse{6.} Vergönne mir, o Jesulein,\\
daß ich dein Mündlein küsse,\\
das Mündlein, das den süßen Wein,\\
auch Milch und Honigflüsse\\
weit übertrifft in seiner Kraft;\\
es ist voll Labsal, Stärk und Saft,\\
der Mark und Bein erquicket.
 
\flagverse{7.} Wenn oft mein Herz im Leibe weint\\
und keinen Trost kann finden,\\
da ruft mir's zu: Ich bin dein Freund,\\
ein Tilger deiner Sünden!\\
Was trauerst du, mein Brüderlein?\\
Du sollst ja guter Dinge sein,\\
ich zahle deine Schulden.
 
\flagverse{8.} Wer ist der Meister, der allhier\\
nach Würdigkeit ausstreichet\\
die Händlein, so dies Kindlein mir\\
anlachende zureichet?\\
Der Schnee ist hell, die Milch ist weiß,\\
verlieren doch beid ihren Preis,\\
wann diese Händlein blicken.
 
\flagverse{9.} Wo nehm ich Weisheit und Verstand,\\
mit Lobe zu erhöhen\\
die Äuglein, die so unverwandt\\
nach mir gerichtet stehen?\\
Der volle Mond ist schön und klar,\\
schön ist der güldnen Sterne Schar,\\
dies' Äuglein sind viel schöner.
 
\flagverse{10.} O daß doch ein so lieber Stern\\
soll in der Krippen liegen!\\
Für edle Kinder großer Herrn\\
gehören güldne Wiegen.\\
Ach, Heu und Stroh ist viel zu schlecht,\\
Samt, Seide, Purpur wären recht,\\
dies Kindlein drauf zu legen.
 
\flagverse{11.} Nehmt weg das Stroh, nehmt weg das Heu,\\
ich will mir Blumen holen,\\
daß meines Heilands Lager sei\\
auf lieblichen Violen.\\
Mit Rosen, Nelken, Rosmarin\\
aus schönen Gärten will ich ihn\\
von obenher bestreuen.
 
\flagverse{12.} Zur Seiten will ich hier und dar\\
viel weißer Lilien stecken,\\
die sollen seiner Äuglein Paar\\
im Schlafe sanft bedecken.\\
Doch liebt viel mehr das dürre Gras\\
dies Kindelein, als alles das,\\
was ich hier nenn und denke.
 
\flagverse{13.} Du fragest nicht nach Lust der Welt\\
noch nach des Leibes Freuden,\\
du hast dich bei uns eingestellt,\\
an unsrer Statt zu leiden,\\
suchst meiner Seelen Herrlichkeit,\\
durch dein selbsteignes Herzeleid,\\
das will ich dir nicht wehren.
 
\flagverse{14.} Eins aber, hoff ich, wirst du mir,\\
mein Heiland, nicht versagen:\\
Daß ich dich möge für und für\\
in, bei und an mir tragen.\\
So laß mich doch dein Kripplein sein;\\
komm, komm und lege bei mir ein\\
dich und all deine Freuden.

\end{verse}
\end{multicols}


\begin{center}
\settowidth{\versewidth}{Zwar sollt ich denken, wie gering,}
\begin{verse}[\versewidth]

\flagverse{15.} Zwar sollt ich denken, wie gering\\
ich dich bewirten werde,\\
du bist der Schöpfer aller Ding,\\
ich bin nur Staub und Erde.\\
Doch bist du so ein frommer Gast,\\
daß du noch nie verschmähest hast\\
den, der dich gerne siehet.

  
\end{verse}
\end{center}



%\attrib{\small{THZE}}
