%StartInfo%%%%%%%%%%%%%%%%%%%%%%%%%%%%%%%%%%%%%%%%%%%%%%%%%%%%%%%%%%%%%%%%%%%%
%  Autor:
%  Titel:
%  File:
%  Ref:
%  Mod:
%EndInfo%%%%%%%%%%%%%%%%%%%%%%%%%%%%%%%%%%%%%%%%%%%%%%%%%%%%%%%%%%%%%%%%%%%%%%
%\poemtitle{pt}
\begin{multicols}{2}
\settowidth{\versewidth}{Ich bin ein Gast auf Erden}
\begin{verse}[\versewidth]
%der 119. Psalm\\
%ich bin ein Gast auf Erden

\flagverse{1.} Ich bin ein Gast auf Erden\\
und hab hier keinen Stand,\\
der Himmel soll mir werden,\\
da ist mein Vaterland.\\
Hier reis ich aus und abe,\\
dort, in der ewgen Ruh,\\
ist Gottes Gnadengabe,\\
die schleußt all Arbeit zu.

\flagverse{2.} Was ist mein ganzes Wesen,\\
von meiner Jugend an,\\
als Müh und Not gewesen?\\
So lang ich denken kann,\\
hab ich so manchen Morgen,\\
so manche liebe Nacht\\
mit Kummer und mit Sorgen\\
des Herzens zugebracht.

\flagverse{3.} Mich hat auf meinen Wegen\\
manch harter Sturm erschreckt,\\
blitz, Donner, Wind und Regen\\
hat mir manch Angst erweckt,\\
verfolgung, Haß und Neiden,\\
ob ichs gleich nicht verschuldt,\\
hab ich doch müssen leiden\\
und tragen mit Geduld

\flagverse{4.} So gings den lieben Alten,\\
an derer Fuß und Pfad\\
wir uns noch täglich halten,\\
wanns fehlt am guten Rat:\\
Wie mußte doch sich schmiegen\\
der Vater Abraham,\\
eh als ihm sein Vergnügen\\
und rechte Wohnstatt kam!

\flagverse{5.} Wie manche schwere Bürde\\
trug Isaak, sein Sohn!\\
Und Jakob, dessen Würde\\
stieg bis zum Himmelsthron,\\
wie mußte der sich plagen,\\
in was für Weh und Schmerz,\\
in was für Furcht und Zagen\\
sank oft sein armes Herz!

\flagverse{6.} Die frommen heilgen Seelen,\\
die gingen fort und fort\\
und änderten mit Quälen\\
den erstbewohnten Ort;\\
sie zogen hin und wieder,\\
ihr Kreuz war immer groß,\\
bis daß der Tod sie nieder\\
legt in des Grabes Schoß.

\flagverse{7.} Ich habe mich ergeben\\
in gleiches Glück und Leid:\\
Was will ich besser leben\\
als solche großen Leut?\\
Es muß ja durchgedrungen,\\
es muß gelitten sein;\\
wer nicht hat wohl gerungen,\\
geht nicht zur Freud hinein.

\flagverse{8.} So will ich zwar nun treiben\\
mein Leben durch die Welt,\\
doch denk ich nicht zu bleiben\\
in diesem fremden Zelt.\\
Ich wandre meine Straßen,\\
die zu der Heimat führt,\\
da mich ohn alle Maßen\\
mein Vater trösten wird.

\flagverse{9.} Mein Heimat ist dort droben,\\
da aller Engel Schar\\
den großen Herrscher loben,\\
der alles ganz und gar\\
in seinen Händen träget\\
und für und für erhält,\\
auch alles hebt und leget,\\
nach dems ihm wohl gefällt.

\flagverse{10.} Zu dem steht mein Verlangen,\\
da wollt ich gerne hin;\\
die Welt bin ich durchgangen,\\
daß ichs fast müde bin.\\
Je länger ich hier walle,\\
je wen'ger find ich Lust,\\
die meinem Geist gefalle;\\
das meist ist Stank und Wust.

\flagverse{11.} Die Herberg ist zu böse,\\
der Trübsal ist zu viel:\\
Ach komm, mein Gott, und löse\\
mein Herz, wann dein Herz will;\\
komm, mach ein seligs Ende\\
an meiner Wanderschaft,\\
und was mich kränkt, das wende\\
durch deinen Arm und Kraft!

\flagverse{12.} Wo ich bisher gesessen,\\
ist nicht mein rechtes Haus;\\
wann mein Ziel ausgemessen,\\
so tret ich dann hinaus,\\
und was ich hier gebrauchet,\\
das leg ich alles ab;\\
und wenn ich ausgehauchet,\\
so scharrt man mich ins Grab.

\flagverse{13.} Du aber, meine Freude,\\
du meines Lebens Licht,\\
du zeuchst mich, wenn ich scheide,\\
hin vor dein Angesicht,\\
ins Haus der ewgen Wonne,\\
da ich stets freudenvoll\\
gleich als die helle Sonne\\
nebst andern leuchten soll.

\flagverse{14.} Da will ich immer wohnen,\\
und nicht nur als ein Gast,\\
bei denen, die mit Kronen\\
du ausgeschmücket hast;\\
da will ich herrlich singen\\
von deinem großen Tun\\
und frei von schnöden Dingen\\
in meinem Erbteil ruhn.

\end{verse}
\end{multicols}
\attrib{\small{Der 119. Psalm}}
