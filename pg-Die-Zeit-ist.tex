%StartInfo%%%%%%%%%%%%%%%%%%%%%%%%%%%%%%%%%%%%%%%%%%%%%%%%%%%%%%%%%%%%%%%%%%%%
%  Autor:
%  Titel:
%  File:
%  Ref:
%  Mod:
%EndInfo%%%%%%%%%%%%%%%%%%%%%%%%%%%%%%%%%%%%%%%%%%%%%%%%%%%%%%%%%%%%%%%%%%%%%%
%\poemtitle{Die Zeit ist nunmehr nah}
\begin{multicols}{2}
\settowidth{\versewidth}{Werd ich denn auch vor Freud}
\begin{verse}[\versewidth]

%Veranlaßt durch den Kometen des Jahres 1652
%
\flagverse{1.} Die Zeit ist nunmehr nah,\\
Herr Jesu, du bist da.\\
Die Wunder, die den Leuten\\
dein Ankunft sollen deuten,\\
die sind, wie wir gesehen,\\
in großer Zahl geschehen.

\flagverse{2.} Was soll ich denn nun tun?\\
Ich soll auf dem beruhn,\\
was du mir hast verheißen,\\
daß du mich wollest reißen\\
aus meines Grabes Kammer\\
und allem andern Jammer.

\flagverse{3.} Ach Jesu, wie so schön\\
wird mir's alsdann ergehn!\\
Du wirst mit tausend Blicken\\
mich durch und durch erquicken,\\
wenn ich hier von der Erde\\
mich zu dir schwingen werde.

\flagverse{4.} Ach, was wird doch dein Wort,\\
o süßer Seelenhort,\\
was wird doch sein dein Sprechen,\\
wenn dein Herz aus wird brechen\\
zu mir und meinen Brüdern\\
als deinen Leibesgliedern.

\flagverse{5.} Werd ich denn auch vor Freud\\
in solcher Gnadenzeit\\
den Augen ihre Zähren\\
und Tränen können wehren,\\
daß sie mir nicht mit Haufen\\
auf meine Wangen laufen?

\flagverse{6.} Was für ein schönes Licht\\
wird mir dein Angesicht,\\
das ich in jenem Leben\\
werd erstmal sehen, geben!\\
Wie wird mir deine Güte\\
entzücken mein Gemüte!

\flagverse{7.} Dein Augen, deinen Mund,\\
den Leib, der noch verwundt,\\
da wir so fest auf trauen,\\
das werd ich alles schauen,\\
auch innig herzlich grüßen\\
die Mal an Händ und Füßen.

\flagverse{8.} Dir ist allein bewußt\\
die ungefälschte Lust\\
und edle Seelenspeise\\
in deinem Paradeise.\\
Die kannst du wohl beschreiben,\\
ich kann nichts mehr als gläuben.

\flagverse{9.} Doch was ich hie gegläubt,\\
das steht gewiß und bleibt\\
mein Teil, dem gar nicht gleichen\\
die Güter aller Reichen;\\
all andres Gut vergehet,\\
mein Erbteil, das bestehet.

\flagverse{10.} Ach Herr, mein schönstes Gut,\\
wie wird sich all mein Blut\\
in allen Adern freuen\\
und auf das Neu erneuen,\\
wenn du mir wirst mit Lachen\\
die Himmelstür aufmachen!

\flagverse{11.} Komm her, komm und empfind,\\
o auserwähltes Kind,\\
komm, schmecke, was für Gaben\\
ich und mein Vater haben,\\
komm, wirst du sagen, weide\\
dein Herz in ewger Freude!

\flagverse{12.} Ach, du so arme Welt,\\
was ist dein Gold und Geld\\
hier gegen diese Kronen\\
und mehr als güldnen Thronen,\\
die Christus hingestellet\\
dem Volk, das ihm gefället.

\flagverse{13.} Hie ist der Engel Land,\\
der selgen Seelen Stand;\\
hie hör ich nichts als singen,\\
hie seh ich nichts als springen,\\
hie ist kein Kreuz, kein Leiden,\\
kein Tod, kein bittres Scheiden.

\flagverse{14.} Halt ein, mein schwacher Sinn,\\
halt ein! Wo denkst du hin?\\
Willst du, was grundlos, gründen?\\
Was unbegreiflich, finden?\\
Hier muß der Witz sich neigen\\
und alle Redner schweigen.

\flagverse{15.} Dich aber, meine Zier,\\
dich laß ich nicht von mir;\\
dein will ich stets gedenken,\\
Herr, der du mir wirst schenken\\
mehr als mit meiner Seelen\\
ich wünschen kann und zählen.

\flagverse{16.} Ach, wie ist mir so weh,\\
eh ich dich aus der Höh,\\
Herr, sehe zu uns kommen!\\
Ach, daß zum Heil der Frommen\\
du meinen Wunsch und Willen\\
noch möchtest heut erfüllen!

\flagverse{17.} Doch du weißt deine Zeit.\\
Mir ziemt nur, stets bereit\\
und fertig dazustehen\\
und so zum Herren zu gehen,\\
daß alle Stund und Tage\\
mein Herz mich zu dir trage.

\flagverse{18.} Dies gib, Herr, und verleih,\\
auf daß dein Huld und Treu\\
ohn Unterlaß mich wecke,\\
daß mich dein Tag nicht schrecke,\\
da unser Schreck auf Erden\\
soll Fried und Freude werden.

\end{verse}
\end{multicols}
%\attrib{\small{THZE}}
