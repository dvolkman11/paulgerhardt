%StartInfo%%%%%%%%%%%%%%%%%%%%%%%%%%%%%%%%%%%%%%%%%%%%%%%%%%%%%%%%%%%%%%%%%%%%
%  Autor:
%  Titel:
%  File:
%  Ref:
%  Mod:
%EndInfo%%%%%%%%%%%%%%%%%%%%%%%%%%%%%%%%%%%%%%%%%%%%%%%%%%%%%%%%%%%%%%%%%%%%%%
%\poemtitle{pt}
\begin{multicols}{2}
\settowidth{\versewidth}{Sollt ich meinem Gott nicht singen}
\begin{verse}[\versewidth]

\flagverse{1.} Sollt ich meinem Gott nicht singen\\
sollt ich ihm nicht dankbar sein?\\
Denn ich seh in allen Dingen,\\
wie so gut ers mit mir mein.\\
Ist doch nichts als lauter Lieben,\\
das sein treues Herze regt,\\
das ohn Ende hebt und trägt,\\
die in seinem Dienst sich üben.\\
Alles Ding währt seine Zeit,\\
Gottes Lieb in Ewigkeit.

\flagverse{2.} Wie ein Adler sein Gefieder\\
über seine Jungen streckt,\\
also hat auch hin und wieder\\
mich des Höchsten Arm bedeckt,\\
alsobald im Mutterleibe,\\
da er mir mein Wesen gab\\
und das Leben, das ich hab\\
und noch diese Stunde treibe.\\
Alles Ding währt seine Zeit,\\
Gottes Lieb in Ewigkeit.

\flagverse{3.} Sein Sohn ist ihm nicht zu teuer,\\
nein, er gibt ihn für mich hin,\\
daß er mich vom ewgen Feuer\\
durch sein teures Blut gewinn.\\
O du ungegründter Brunnen,\\
wie will doch mein schwacher Geist,\\
ob er sich gleich hoch befleißt,\\
deine Tief ergründen können?\\
Alles Ding währt seine Zeit,\\
Gottes Lieb in Ewigkeit.

\flagverse{4.} Seinen Geist, den edlen Führer,\\
gibt er mir in seinem Wort,\\
daß er werde mein Regierer\\
durch die Welt zur Himmelspfort,\\
daß er mir mein Herz erfülle\\
mit dem hellen Glaubenslicht,\\
das des Todes Macht zerbricht\\
und die Hölle selbst macht stille.\\
Alles Ding währt seine Zeit,\\
Gottes Lieb in Ewigkeit.

\flagverse{5.} Meiner Seele Wohlergehen\\
hat er ja recht wohl bedacht;\\
will dem Leibe Not zustehen,\\
nimmt ers gleichfalls wohl in Acht.\\
Wann mein Können, mein Vermögen\\
nichts vermag, nichts helfen kann,\\
kommt mein Gott und hebt mir an,\\
sein Vermögen beizulegen.\\
Alles Ding währt seine Zeit,\\
Gottes Lieb in Ewigkeit.

\vfill\null
\columnbreak

\flagverse{6.} Himmel, Erd und ihre Heere\\
hat er mir zum Dienst bestellt;\\
wo ich nur mein Aug hinkehre,\\
find ich, was mich nährt und hält:\\
Tier und Kräuter und Getreide\\
in den Gründen, in der Höh,\\
in den Büschen, in der See,\\
überall ist meine Weide.\\
Alles Ding währt seine Zeit,\\
Gottes Lieb in Ewigkeit.

\flagverse{7.} Wenn ich schlafe, wacht sein Sorgen\\
und ermuntert mein Gemüt,\\
daß ich alle lieben Morgen\\
schaue neue Lieb und Güt.\\
Wäre mein Gott nicht gewesen,\\
hätte mich sein Angesicht\\
nicht geleitet, wär ich nicht\\
aus so mancher Angst genesen.\\
Alles Ding währt seine Zeit,\\
Gottes Lieb in Ewigkeit.

\flagverse{8.} Wie so manche schwere Plage\\
wird vom Satan umgeführt,\\
die mich doch mein Lebetage\\
niemals noch bisher gerührt.\\
Gottes Engel, den er sendet,\\
hat das Böse, was der Feind\\
anzurichten war gemeint,\\
in die Ferne weggewendet.\\
Alles Ding währt seine Zeit,\\
Gottes Lieb in Ewigkeit.

\flagverse{9.} Wie ein Vater seinem Kinde\\
sein Herz niemals ganz entzeucht,\\
ob es gleich bisweilen Sünde\\
tut und aus der Bahne weicht:\\
Also hält auch mein Verbrechen\\
mir mein frommer Gott zu gut,\\
will mein Fehlen mit der Rut\\
und nicht mit dem Schwerte rächen.\\
Alles Ding währt seine Zeit,\\
Gottes Lieb in Ewigkeit.

\flagverse{10.} Seine Strafen, seine Schläge,\\
ob sie mir gleich bitter seind,\\
dennoch, wenn ichs recht erwäge,\\
sind es Zeichen, daß mein Freund,\\
der mich liebet, mein gedenke\\
und mich von der schnöden Welt,\\
die uns hart gefangen hält,\\
durch das Kreuze zu ihm lenke.\\
Alles Ding währt seine Zeit,\\
Gottes Lieb in Ewigkeit.

\vfill\null
\columnbreak

\flagverse{11.} Das weiß ich fürwahr und lasse\\
mirs nicht aus dem Sinne gehn:\\
Christenkreuz hat seine Maße\\
und muß endlich stille stehn;\\
wenn der Winter ausgeschneiet,\\
tritt der schöne Sommer ein:\\
Also wird auch nach der Pein,\\
wers erwarten kann, erfreuet.\\
Alles Ding währt seine Zeit,\\
Gottes Lieb in Ewigkeit.

\flagverse{12.} Weil dann weder Ziel noch Ende\\
sich in Gottes Liebe findt,\\
ei, so heb ich meine Hände\\
zu dir, Vater, als dein Kind;\\
bitte, wollst mir Gnade geben,\\
dich aus aller meiner Macht\\
zu umfangen Tag und Nacht\\
hier in meinem ganzen Leben,\\
bis ich dich nach dieser Zeit\\
lob und lieb in Ewigkeit.
   
\end{verse}
\end{multicols}
%\attrib{\small{THZE}}
