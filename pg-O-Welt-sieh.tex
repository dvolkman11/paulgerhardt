%StartInfo%%%%%%%%%%%%%%%%%%%%%%%%%%%%%%%%%%%%%%%%%%%%%%%%%%%%%%%%%%%%%%%%%%%%
%  Autor:
%  Titel:
%  File:
%  Ref:
%  Mod:
%EndInfo%%%%%%%%%%%%%%%%%%%%%%%%%%%%%%%%%%%%%%%%%%%%%%%%%%%%%%%%%%%%%%%%%%%%%%
%\poemtitle{pt}
\begin{multicols}{2}
\settowidth{\versewidth}{Tritt her und schau mit Fleiße:}
\begin{verse}[\versewidth]
%o Welt, sieh hier dein Leben

\flagverse{1.} O Welt, sieh hier dein Leben\\
am Stamm des Kreuzes schweben!\\
Dein Heil sinkt in den Tod!\\
Der große Fürst der Ehren\\
läßt willig sich beschweren\\
mit Schlägen, Hohn und großem Spott.

\flagverse{2.} Tritt her und schau mit Fleiße:\\
Sein Leib ist ganz mit Schweiße\\
des Blutes überfüllt;\\
aus seinem edlen Herzen\\
vor unerschöpften Schmerzen\\
ein Seufzer nach dem andern quillt.

\flagverse{3.} Wer hat dich so geschlagen,\\
mein Heil, und dich mit Plagen\\
so übel zugericht't?\\
Du bist ja nicht ein Sünder\\
wie wir und unsre Kinder,\\
von Übeltaten weißt du nicht.

\flagverse{4.} Ich, ich und meine Sünden,\\
die sich wie Körnlein finden\\
des Sandes an dem Meer,\\
die haben dir erreget\\
das Elend, das dich schläget,\\
und das betrübte Marterheer.

\flagverse{5.} Ich bins, ich sollte büßen,\\
an Händen und an Füßen\\
gebunden, in der Höll;\\
die Geißeln und die Banden\\
und was du ausgestanden,\\
das hat verdienet meine Seel.

\flagverse{6.} Du nimmst auf deinen Rücken\\
die Lasten, die mich drücken\\
viel sehrer als ein Stein.\\
Du wirst ein Fluch, dagegen\\
verehrst du mir den Segen;\\
dein Schmerzen muß mein Labsal sein.

\flagverse{7.} Du setzest dich zum Bürgen,\\
ja lässest dich gar würgen\\
für mich und meine Schuld;\\
mir lässest du dich krönen\\
mit Dornen, die dich höhnen,\\
und leidest alles mit Geduld.

\flagverse{8.} Du springst ins Todes Rachen,\\
mich frei und los zu machen\\
von solchem Ungeheur.\\
Mein Sterben nimmst du abe,\\
vergräbst es in dem Grabe,\\
o unerhörtes Liebesfeur!

\flagverse{9.} Ich bin, mein Heil, verbunden\\
all Augenblick und Stunden\\
dir überhoch und sehr.\\
Was Leib und Seel vermögen,\\
das soll ich billig legen\\
allzeit an deinen Dienst und Ehr.

\flagverse{10.} Nun, ich kann nicht viel geben\\
in diesem armen Leben;\\
eins aber will ich tun:\\
Es soll dein Tod und Leiden\\
bis Leib und Seele scheiden,\\
mir stets in meinem Herzen ruhn.

\flagverse{11.} Ich wills vor Augen setzen,\\
mich stets daran ergötzen,\\
ich sei auch, wo ich sei;\\
es soll mir sein ein Spiegel\\
der Unschuld und ein Siegel\\
der Lieb und unverfälschten Treu.

\flagverse{12.} Wie heftig unsre Sünden\\
den frommen Gott entzünden,\\
wie Rach und Eifer gehn,\\
wie grausam seine Ruten,\\
wie zornig seine Fluten,\\
will ich aus diesem Leiden sehn.

\flagverse{13.} Ich will daraus studieren,\\
wie ich mein Herz soll zieren\\
mit stillem, sanften Mut,\\
und wie ich die soll lieben,\\
die mich doch sehr betrüben\\
mit Werken, so die Bosheit tut.

\flagverse{14.} Wenn böse Zungen stechen,\\
mir Glimpf und Namen brechen.\\
So will ich zähmen mich;\\
das Unrecht will ich dulden,\\
dem Nächsten seine Schulden\\
verzeihen gern und williglich.

\flagverse{15.} Ich will mich mit dir schlagen\\
ans Kreuz und dem absagen,\\
was meinem Fleisch gelüst't.\\
Was deine Augen hassen,\\
das will ich fliehn und lassen,\\
so viel mir immer möglich ist.

\flagverse{16.} Dein Seufzen und dein Stöhnen\\
und die viel tausend Tränen,\\
die dir geflossen zu,\\
die sollen mich am Ende\\
in deinen Schoß und Hände\\
begleiten zu der ewgen Ruh.

\end{verse}
\end{multicols}
%\attrib{\small{THZE}}
