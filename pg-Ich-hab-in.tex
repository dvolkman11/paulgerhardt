%StartInfo%%%%%%%%%%%%%%%%%%%%%%%%%%%%%%%%%%%%%%%%%%%%%%%%%%%%%%%%%%%%%%%%%%%%
%  Autor:
%  Titel:
%  File:
%  Ref:
%  Mod:
%EndInfo%%%%%%%%%%%%%%%%%%%%%%%%%%%%%%%%%%%%%%%%%%%%%%%%%%%%%%%%%%%%%%%%%%%%%%
%\poemtitle{pt}
\begin{multicols}{2}
\settowidth{\versewidth}{Wenn er mich auch gleich wirft ins Meer,}
\begin{verse}[\versewidth]
%ich hab in Gottes Herz und Sinn mein Herz und Sinn ergeben

\flagverse{1.} Ich hab in Gottes Herz und Sinn\\
mein Herz und Sinn ergeben:\\
Was böse scheint, ist mir Gewinn,\\
der Tod selbst ist mein Leben.\\
Ich bin ein Sohn des, der den Thron\\
des Himmels aufgezogen;\\
ob er gleich schlägt und Kreuz auflegt,\\
bleibt doch sein Herz gewogen.

\flagverse{2.} Das kann mir fehlen nimmermehr,\\
mein Vater muß mich lieben!\\
Wenn er mich auch gleich wirft ins Meer,\\
so will er mich nur üben\\
und mein Gemüt in seiner Güt\\
gewöhnen fest zu stehen;\\
halt ich den Stand, weiß seine Hand\\
mich wieder zu erhöhen.

\flagverse{3.} Ich bin ja von mir selber nicht\\
entsprungen noch formieret,\\
mein Gott ists, der mich zugericht,\\
an Leib und Seel gezieret,\\
der Seelen Sitz mit Sinn und Witz,\\
den Leib mit Fleisch und Beinen:\\
Wer so viel tut, des Herz und Mut\\
kanns nimmer böse meinen.

\flagverse{4.} Woher wollt ich mein Aufenthalt\\
auf dieser Erd erlangen?\\
Ich wäre längsten tot und kalt,\\
wo mich nicht Gott umfangen\\
mit seinem Arm, der alles warm,\\
gesund und fröhlich machet;\\
was er nicht hält, das bricht und fällt,\\
was er erfreut, das lachet.

\flagverse{5.} Zudem ist Weisheit und Verstand\\
bei ihm ohn alle Maßen,\\
Zeit, Ort und Stund ist ihm bekannt,\\
zu tun und auch zu lassen.\\
Er weiß, wenn Freud, er weiß, wenn Leid\\
uns, seinen Kindern, diene;\\
und was er tut, ist alles gut,\\
obs noch so traurig schiene.

\flagverse{6.} Du denkest zwar, wenn du nicht hast,\\
was Fleisch und Blut begehret,\\
als sei mit einer großen Last\\
dein Glück und Heil beschweret,\\
hast spät und früh viel Sorg und Müh,\\
an deinen Wunsch zu kommen,\\
und denkest nicht, daß, was geschicht,\\
gescheh zu deinem Frommen.

\flagverse{7.} Fürwahr, der dich geschaffen hat\\
und sich zur Ehr erbauet,\\
der hat schon längst in seinem Rat\\
ersehen und beschauet\\
aus wahrer Treu, was dienlich sei\\
dir und den Deinen alle;\\
laß ihm doch zu, daß er nur tu\\
das, was ihm wohlgefalle.

\flagverse{8.} Wanns Gott gefällt, so kanns nicht sein,\\
er wird dich letzt erfreuen:\\
Was du jetzt nennest Kreuz und Pein,\\
wird dir zum Trost gedeihen.\\
Wart in Geduld: Die Gnad und Huld\\
wird sich doch endlich finden;\\
all Angst und Qual wird auf einmal\\
gleichwie ein Dampf verschwinden.

\flagverse{9.} Das Feld kann ohne Ungestüm\\
gar keine Früchte tragen:\\
So fällt auch Menschenwohlfahrt üm\\
bei lauter guten Tagen.\\
Die Aloe bringt bittres Weh,\\
macht gleichwohl rote Wangen:\\
So muß ein Herz durch Angst und Schmerz\\
zu seinem Heil gelangen.

\flagverse{10.} Ei nun, mein Gott, so fall ich dir\\
getrost in deine Hände;\\
nimm mich und mach es du mit mir\\
bis an mein letztes Ende\\
wie du wohl weißt, daß meinem Geist\\
dadurch sein Nutz entstehe\\
und deine Ehr je mehr und mehr\\
sich in ihr selbst erhöhe.

\flagverse{11.} Willst du mir geben Sonnenschein,\\
so nehm ichs an mit Freuden,\\
solls aber Kreuz und Unglück sein,\\
will ichs geduldig leiden.\\
Soll mir allhier des Lebens Tür\\
noch ferner offen stehen:\\
Wie du mich führst und führen wirst,\\
so will ich gern mitgehen.

\flagverse{12.} Soll ich denn auch des Todes Weg\\
und finstre Straßen reisen:\\
Wohlan, so tret ich Bahn und Steg,\\
den mir dein Augen weisen.\\
Du bist mein Hirt, der alles wird\\
zu solchem Ende kehren,\\
daß ich einmal in deinem Saal\\
dich ewig möge ehren.

\end{verse}
\end{multicols}
%\attrib{\small{THZE}}
