%StartInfo%%%%%%%%%%%%%%%%%%%%%%%%%%%%%%%%%%%%%%%%%%%%%%%%%%%%%%%%%%%%%%%%%%%%
%  Autor: Dietmar Volkmann
%  Titel: Paul Gerhardt: Ach Herr, wie lange
%  File:  pg-Ach-Herr-wie.tex
%  Ref:   paulgerhardt
%  Mod:   
%EndInfo%%%%%%%%%%%%%%%%%%%%%%%%%%%%%%%%%%%%%%%%%%%%%%%%%%%%%%%%%%%%%%%%%%%%%%
%ANM:\poemtitle{Ach Herr, wie lange willst du mein so ganz und gar vergessen?}
\begin{multicols}{2}
\settowidth{\versewidth}{Ach Herr, wie lange willst du mein}
\begin{verse}[\versewidth]
%ANM:Der 13. Psalm\\
%ANM:Gedichtet zur Leichenfeier des 1660 verstorbenen Rittmeisters Christoph Ludwig von Thümen\\
\flagverse{1.} Ach Herr, wie lange willst du mein\\
so ganz und gar vergessen?\\
Wie lange soll der Sorgen Stein\\
mich und mein Herze pressen?\\
Wie lange soll dein Angesicht\\
sich von mir wenden? Willst du nicht\\
dich meiner mehr erbarmen?

\flagverse{2.} Wie lange soll ich armes Kind\\
der Seelen Ruh entbehren?\\
Wie lange soll der Sturm und Wind\\
der Herzensangst gewähren?\\
Wie lange soll mein stolzer Feind,\\
ders niemals gut, stets böse meint,\\
sich über mich erheben?

\flagverse{3.} Ach, schaue doch, mein Gott und Hort,\\
von deiner heilgen Hütte\\
und höre meiner Klage Wort\\
und hochbetrübte Bitte;\\
gib meinen Augen Kraft und Macht\\
und laß des Todes finstre Nacht\\
mich nicht so bald befallen!

\flagverse{4.} Sonst würde meiner Feinde Mund\\
des Ruhms kein Ende machen;\\
sie würden mein, als der zu Grund\\
und Boden gangen, lachen:\\
Da liegt der, würden sie mit Freud\\
herprahlen, der uns jederzeit\\
so viel zu schaffen machte!

\flagverse{5.} Ich kenne sie und weiß gar wohl,\\
was sie im Schilde führen,\\
ihr Herz ist aller Bosheit voll,\\
läßt sich nichts Guts regieren.\\
Du aber bist der fromme Mann,\\
Herr, mein Gott, der nicht lassen kann\\
die, so sich zu dir halten.

\flagverse{6.} Des tröst ich mich und hoffe drauf,\\  % KORR: Komma?
du wirst auch mir fromm bleiben\\
und aller bösen Tücke Lauf\\
gewaltig hintertreiben.\\
Mein Herze freut sich, wenns bedenkt,\\
wie gern du stets dein Heil geschenkt\\
dem, der sich dir vertrauet.

\end{verse}
\end{multicols}

\begin{center}
\settowidth{\versewidth}{Ach Herr, wie lange willst du mein}
\begin{verse}[\versewidth]
\flagverse{7.} Das tu ich, Herr; ich traue dir:\\
du bist mein einzge Freude,\\
bewehrest mich, tust wohl an mir\\
und führst mich aus dem Leide.\\
Dafür will ich mein Leben lang\\
dir manchen schönen Lobgesang\\
zum Dank und Opfer bringen.
\end{verse}
\end{center}

%\end{verse}
%\end{multicols}
%\attrib{\small{1660}}
