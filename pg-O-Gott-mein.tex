%StartInfo%%%%%%%%%%%%%%%%%%%%%%%%%%%%%%%%%%%%%%%%%%%%%%%%%%%%%%%%%%%%%%%%%%%%
%  Autor:
%  Titel:
%  File:
%  Ref:
%  Mod:
%EndInfo%%%%%%%%%%%%%%%%%%%%%%%%%%%%%%%%%%%%%%%%%%%%%%%%%%%%%%%%%%%%%%%%%%%%%%
%\poemtitle{pt}
\begin{multicols}{2}
\settowidth{\versewidth}{der nichts mehr schmeckt, nichts sieht und hört,}
\begin{verse}[\versewidth]
%sirachs Gebet um fleckenlosen Wandel (Sirach 23, 1-6)\\
%o Gott, mein Schöpfer, edler Fürst

\flagverse{1.} O Gott, mein Schöpfer, edler Fürst\\
und Vater meines Lebens,\\
wo du mein Leben nicht regierst,\\
so leb ich hier vergebens.\\
Ja lebendig bin ich auch tot,\\
der Sünden ganz ergeben,\\
wer sich wälzt\\
in dem Sündenkot,\\
der hat das rechte Leben\\
noch niemals recht gesehen.

\flagverse{2.} Darnum so wende deine Gnad\\
zu deinem armen Kinde\\
und gib mir allzeit guten Rat,\\
zu meiden Schand und Sünde;\\
behüte meines Mundes Tür,\\
daß mir ja nicht entfahre\\
ein solches Wort,\\
dadurch ich dir\\
und deiner frommen Schare\\
verdrießlich sei und schade.

\flagverse{3.} Bewahr, o Vater, mein Gehör\\
auf dieser schnöden Erde\\
vor allem, dadurch deine Ehr\\
und Reich beschimpfet werde;\\
laß mich der Lästrer Gall und Gift\\
ja nimmermehr berühren,\\
denn wen ein solcher\\
Unflat trifft,\\
den pflegt er zu verführen,\\
auch wohl gar umzukehren.

\flagverse{4.} Regiere meiner Augen Licht,\\
daß sie nichts Arges treiben,\\
ein unverschämtes Angesicht\\
laß ferne von mir bleiben;\\
was ehrbar ist, was Zucht erhält,\\
wonach die Englein trachten,\\
was dir beliebt\\
und wohlgefällt,\\
das laß auch mich hochachten,\\
all Üppigkeit verlachen.

\flagverse{5.} Gib, daß ich mich nicht lasse ein\\
zum Schlemmen und zum Prassen,\\
laß deine Lust mein eigen sein,\\
die andre fliehn und hassen.\\
Die Lust, die unser Fleisch ergötzt,\\
die zeucht uns nach der Höllen,\\
und was die Welt\\
für Freude schätzt,\\
pflegt Seel und Geist zu fällen\\
und ewiglich zu quälen.

\flagverse{6.} O selig ist, der stets sich nährt\\
mit Himmels Speis und Tränken,\\
der nichts mehr schmeckt,\\
nichts sieht und hört,\\
auch nichts begehrt zu denken,\\
als nur was zu dem Leben bringt,\\
da man bei Gotte lebet\\
und bei der Schar, die fröhlich singt\\
und in der Wollust schwebet,\\
die keine Zeit aufhebet.

\end{verse}
\end{multicols}
%\attrib{\small{THZE}}
