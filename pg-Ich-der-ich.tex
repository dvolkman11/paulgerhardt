%StartInfo%%%%%%%%%%%%%%%%%%%%%%%%%%%%%%%%%%%%%%%%%%%%%%%%%%%%%%%%%%%%%%%%%%%%
%  Autor:
%  Titel:
%  File:
%  Ref:
%  Mod:
%EndInfo%%%%%%%%%%%%%%%%%%%%%%%%%%%%%%%%%%%%%%%%%%%%%%%%%%%%%%%%%%%%%%%%%%%%%%
%\poemtitle{pt}
\begin{multicols}{2}
\settowidth{\versewidth}{Die Welt, die deucht uns schön und groß}
\begin{verse}[\versewidth]
%der 145. Psalm\\
%ich, der ich oft in tiefes Leid und große Not muß gehen

\flagverse{1.} Ich, der ich oft in tiefes Leid\\
und große Not muß gehen,\\
will dennoch Gott mit großer Freud\\
und Herzenslust erhöhen.\\
Mein Gott, du König, höre mich,\\
ich will ohn alles Ende dich\\
und deinen Namen loben.

\flagverse{2.} Ich will dir mit der Morgenröt\\
ein täglich Opfer bringen,\\
so oft die liebe Sonn aufgeht,\\
so ofte will ich singen\\
dem großen Namen deiner Macht,\\
das soll auch in der späten Nacht\\
mein Werk sein und Geschäfte.

\flagverse{3.} Die Welt, die deucht uns schön und groß\\
und was für Gut und Gaben\\
sie trägt in ihrem Arm und Schoß,\\
das will ein jeder haben:\\
Und ist doch alles lauter Nichts;\\
eh als mans recht genießt, zerbrichts\\
und geht im Hui zugrunde.

\flagverse{4.} Gott ist alleine groß und schön,\\
unmöglich auszuloben\\
auch denen, die doch allzeit stehn\\
vor seinem Throne droben.\\
Laß sprechen, wer nur sprechen kann,\\
doch wird kein Engel noch kein Mann\\
des Höchsten Größ aussprechen.

\flagverse{5.} Die Alten, die nun nicht mehr sind,\\
die haben ihn gepreiset;\\
so hat ein jeder auch sein Kind\\
zu solchem Dienst geweiset;\\
die Kinder werden auch nicht ruhn\\
und werden doch, o Gott, dein Tun\\
und Werk nicht ganz auspreisen.

\flagverse{6.} Wie mancher hat vor mir dein Heil\\
und Lob mit Fleiß getrieben;\\
und siehe, mir ist doch mein Teil\\
zu loben übrig blieben.\\
Ich will von deiner Wundermacht\\
und der so herrlich schönen Pracht\\
bis an mein Ende reden.

\flagverse{7.} Und was ich rede, wird von mir\\
manch frommes Herze lernen,\\
man wird dich heben für und für\\
hoch über allen Sternen;\\
dein Herrlichkeit und starke Hand\\
wird in der ganzen Welt bekannt\\
und hoch berufen werden.

\flagverse{8.} Wer ist so gnädig als wie du?\\
Wer kann so viel erdulden?\\
Wer sieht mit solcher Langmut zu\\
so vielen schweren Schulden,\\
die aus der ganzen weiten Welt\\
ohn Unterlaß bis an das Zelt\\
des hohen Himmels steigen?

\flagverse{9.} Es muß ein treues Herze sein,\\
das uns so hoch kann lieben,\\
da wir doch in den Tag hinein,\\
was gar nicht gut ist, üben.\\
Gott muß nichts anders sein als gut,\\
daher fließt seiner Güte Flut\\
auf alle seine Werke.

\flagverse{10.} Drum, Herr, so sollen dir auch nun\\
all deine Werke danken,\\
voraus die Heilgen, deren Tun\\
sich hält in deinen Schranken,\\
die sollen deines Reichs Gewalt\\
und schöne Regimentsgestalt\\
mit vollem Munde rühmen.

\flagverse{11.} Sie sollen rühmen, daß der Ruhm\\
durch alle Welt erklinge,\\
daß jedermann zum Heiligtum\\
dir Dienst und Opfer bringe;\\
dein Reich, das ist ein ewges Reich,\\
dein Herrschaft ist dir selber gleich,\\
der du kein End erreichest.

\flagverse{12.} Der Herr ist bis in unsern Tod\\
beständig bei uns allen,\\
erleichtert unsers Kreuzes Not\\
und hält uns, wenn wir fallen;\\
er steuert manches Unglücks Lauf\\
und hilft uns wieder freundlich auf,\\
wenn wir ganz hingeschlagen.

\flagverse{13.} Herr, aller Augen sind nach dir\\
und deinem Stuhl gekehret;\\
denn du bists auch, der alles hier\\
so väterlich ernähret;\\
du tust auf deine milde Hand,\\
machst froh und satt, was auf dem Land,\\
im Meer und Lüften lebet.

\flagverse{14.} Du meinst es gut und tust uns Guts,\\
auch da wirs oft nicht denken,\\
wie mancher ist betrübtes Muts\\
und frißt sein Herz mit Kränken,\\
besorgt und fürcht sich Tag und Nacht,\\
gott hab ihn gänzlich aus der Acht\\
gelassen und vergessen.

\flagverse{15.} Nein, Gott vergißt der Seinen nicht,\\
er ist uns viel zu treue,\\
sein Herz ist stets dahin gericht,\\
daß er uns letzt erfreue.\\
Gehts gleich bisweilen etwas schlecht,\\
ist er doch heilig und gerecht\\
in allen seinen Wegen.

\flagverse{16.} Der Herr ist nah und stets bereit\\
eim jeden, der ihn ehret,\\
und wer nur ernstlich zu ihm schreit,\\
der wird gewiß erhöret.\\
Gott weiß wohl, wer ihm günstig sei,\\
und deme steht er dann auch bei,\\
wann ihn die Angst nun treibet.

\flagverse{17.} Den Frommen wird nichts abgesagt;\\
gott tut, was sie begehren,\\
er mißt das Unglück, das sie plagt,\\
und zählt all ihre Zähren\\
und reißt sie endlich aus der Last;\\
den aber, der sie kränkt und haßt,\\
den stürzt er ganz zu Boden.

\flagverse{18.} Dies alles und was sonsten mehr\\
man kann für Lob erzwingen,\\
das soll mein Mund zu Ruhm und Ehr\\
des Höchsten täglich singen:\\
Und also tut auch immerfort\\
was lebt und webt an jedem Ort:\\
Das wird Gott wohlgefallen.

\end{verse}
\end{multicols}
%\attrib{\small{THZE}}
