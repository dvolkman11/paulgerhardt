%StartInfo%%%%%%%%%%%%%%%%%%%%%%%%%%%%%%%%%%%%%%%%%%%%%%%%%%%%%%%%%%%%%%%%%%%%
%  Autor:
%  Titel:
%  File:
%  Ref:
%  Mod:
%EndInfo%%%%%%%%%%%%%%%%%%%%%%%%%%%%%%%%%%%%%%%%%%%%%%%%%%%%%%%%%%%%%%%%%%%%%%
%\poemtitle{pt}
\begin{multicols}{2}
\settowidth{\versewidth}{Merkt auf, merkt, Himmel, Erde,}
\begin{verse}[\versewidth]
%merkt auf, merkt, Himmel, Erde\\
%danklied Moses vor seinem Tode

\flagverse{1.} Merkt auf, merkt, Himmel, Erde,\\
und du, o Meeresgrund,\\
was ich jetzt singen werde\\
aus Gottes heilgem Mund!\\
Es fließe meine Lehre,\\
wie Tau und Regen fleußt;\\
wer Ohren hat, der höre\\
des Höchsten Wort und Geist.

\flagverse{2.} Es läßt der Herr euch weisen\\
sein Ehr und Namenszier;\\
die soll und will ich preisen,\\
das tut auch ihr mit mir.\\
Er ist ein Gott der Götter,\\
ein Tröster in der Not,\\
ein Fels, ein einzger Retter\\
und selbst des Todes Tod.

\flagverse{3.} Sein Tun ist lauter Güte,\\
sein Werk ist rein und klar,\\
treu ist er am Gemüte,\\
im Wort und Reden wahr;\\
viel heilger als die Engel,\\
die doch nur recht getan,\\
frei aller Fehl und Mängel,\\
fern von der Unrechtsbahn.

\flagverse{4.} Er ist gerecht. Wir alle\\
sind schändlich angesteckt\\
mit Adams Sünd und Falle,\\
der täglich in uns heckt\\
viel böse schwere Taten,\\
die unserm großen Gott,\\
des kein Mensch kann entraten,\\
geraten nur zum Spott.

\flagverse{5.} Die ungeratnen Kinder,\\
die fallen von ihm ab\\
und werden freche Sünder,\\
vergessen aller Gab\\
und so viel tausend Güter\\
und so viel süßer Gnad,\\
die ihnen Gott, ihr Hüter,\\
so oft erwiesen hat.

\flagverse{6.} Dankst du denn solchermaßen,\\
du toll und töricht Volk,\\
dem, der dir regnen lassen\\
dein Manna aus der Wolk\\
und aus des Himmels Kammer\\
dir Speisen zugeschickt,\\
damit in deinem Jammer\\
dein Herze würd erquickt?

\flagverse{7.} Woher hast du dein Leben\\
und deines Leibes Bild?\\
Wer hat das Blut gegeben,\\
das dir die Adern füllt?\\
Ists nicht dein Herr, dein Schöpfer,\\
dein Vater und dein Licht,\\
der dich, gleich als ein Töpfer,\\
von Erde zugericht?

\flagverse{8.} Gedenk und geh zurücke\\
in die vergangnen Jahr;\\
erwäge, was für Glücke\\
gott deiner Väter Schar\\
erzeigt in schweren Zeiten!\\
Das ist den Alten kund,\\
die werden dir andeuten\\
den rechten wahren Grund.

\flagverse{9.} Er stieß die wilden Heiden\\
mit seiner starken Hand\\
aus ihrer fetten Weiden\\
und gab das schöne Land\\
des Israels Geschlechte\\
zu seines Namens Ruhm\\
und Jakob, seinem Knechte,\\
zum Erb und Eigentum.

\flagverse{10.} Er fand ihn, wo es heulet,\\
in dürrer Wüstenei,\\
er fand ihn und erteilet\\
ihm alle Vatertreu;\\
er lehret ihn, was tauge\\
und er selbst Tugend heiß,\\
er hielt ihn wie ein Auge\\
und sparte keinen Fleiß.

\flagverse{11.} Gleichwie ein Adler sitzet\\
auf seiner zarten Brut\\
und gar genau beschützet,\\
was ihm am Herzen ruht;\\
er dehnt die starken Flügel,\\
wenn er sich hoch erschwingt\\
und über Tal und Hügel\\
sein edle Jungen bringt:

\flagverse{12.} So hat sich auch gebreitet\\
des Höchsten Lieb und Gnad\\
auf Jakob, den er leitet,\\
auf daß ihm ja kein Schad\\
hier oder da anstieße;\\
er hub, er trug mit Fleiß,\\
bewahrt ihm Gang und Füße\\
auf seiner ganzen Reis.

\flagverse{13.} Er, sein Gott, tats alleine\\
und sonst kein andrer Gott;\\
es gaben Feld und Steine\\
öl, Honig, Wasser, Brot\\
ohn alle seine Mühe;\\
er hatte guten Mut\\
beim Fett der Schaf und Kühe\\
und trank gut Traubenblut.

\flagverse{14.} Da er nun wohl gegessen,\\
vergaß er Gottes Heil,\\
und da er des vergessen,\\
da ward er frech und geil;\\
da seine Not gestillet,\\
beschimpft er Gottes Ehr,\\
und da der Leib gefüllet,\\
da ward das Herze leer.

\flagverse{15.} Leer ward es an dem Guten,\\
des Bösen ward es voll,\\
ließ Götzenopfer bluten\\
und dient, als wär er toll,\\
den schändlichen Feldteufeln;\\
und den, an dessen Macht\\
die Teufel selbst nicht zweifeln,\\
den ließ er aus der Acht.

\flagverse{16.} Er ließ den ewgen Retter\\
und gab sich in den Schirm\\
der neuerdachten Götter,\\
hielt Bestien und Gewürm\\
und Bilder von Metallen,\\
von Holz, von Stein und Ton,\\
den Heiden zu gefallen,\\
für seiner Seelen Kron.

\flagverse{17.} Als das nun der erkannte,\\
der Herz und Nieren kennt,\\
da wuchs sein Zorn und brannte,\\
gleichwie ein Feuer brennt;\\
und die er vor so schöne\\
geliebt an seinem Teil\\
als Töchter und als Söhne,\\
die wurden ihm ein Greul.

\flagverse{18.} Ich will mich, sprach er, wenden\\
von dieser schnöden Art,\\
die so abscheulich schänden\\
mich, der ich nichts gespart\\
an meiner Treu und Güte;\\
ich habe recht geliebt,\\
dafür wird mein Gemüte\\
gekränket und betrübt.

\flagverse{19.} Sie reizen mich mit Sünden:\\
Was gilts, es soll einmal\\
sich wieder etwas finden\\
zu ihrem Zorn und Qual!\\
Es werden Völker kommen,\\
die blind sind als ein Stein;\\
die sollen meine frommen\\
und liebsten Kinder sein.

\flagverse{20.} Mein Feuer ist entstanden\\
und brennet lichterloh\\
in meines Volkes Landen,\\
die sind ihm wie das Stroh.\\
Es wird weit um sich greifen\\
bis zu der Höllen Grund\\
und alle Frucht abstreifen,\\
die auf der Erden stund.

\flagverse{21.} Ich will mit meinen Pfeilen\\
sie treiben in den Tod;\\
es soll sie übereilen\\
schwert, Pest und Hungersnot.\\
Ich will viel Tiere schicken\\
und strenges Schlangengift,\\
das soll zermartern, drücken\\
und fressen, wen es trifft.

\flagverse{22.} Ich will sie recht belohnen,\\
mein Zorn soll gleich ergehn,\\
auch derer nicht verschonen,\\
die jung, gerad und schön;\\
ich will sie all zerstäuben\\
und fragen hier und dort:\\
Wo ist dann nun ihr Bleiben?\\
Welch ist ihr Sitz und Ort?

\flagverse{23.} Doch muß ich gleichwohl scheuen\\
den ungereimten Wahn\\
der Feinde, die sich freuen,\\
als hätten sies getan.\\
Sie bleiben wie die Narren\\
bei ihrem Gaukelspiel\\
und ziehn am Torheitskarren,\\
ich tu auch, was ich will.

\flagverse{24.} O, daß mein Volk verstünde\\
das edle schöne Gut,\\
das, wenns nun seine Sünde\\
bereut und Buße tut,\\
ihm nachmals wird begegnen!\\
Denn was ich jetzt geflucht,\\
das will ich wieder segnen,\\
sobald es Gnade sucht.

\flagverse{25.} Mein Volk kommt aus dem Weinen,\\
sein Feind kommt aus der Ruh,\\
ihr tausend flieht vor einem,\\
wie geht das immer zu?\\
Ihr Herr, ihr Fels und Leben,\\
ist weg aus ihrem Zelt,\\
er hat sie übergeben\\
zur Flucht ins freie Feld.

\flagverse{26.} Seid froh, ihr treuen Knechte\\
des Gottes Israel,\\
des Arm und starke Rechte\\
euch schützt an Leib und Seel,\\
habt fröhliches Vertrauen\\
und Glauben, der da siegt:\\
So wird Gott wieder bauen,\\
was jetzt darnieder liegt.

\end{verse}
\end{multicols}
%\attrib{\small{THZE}}

\begin{center}
\settowidth{\versewidth}{Der, vor dem die Welt erschrickt,}
\begin{verse}[\versewidth]

\flagverse{27.} Er wird am Feinde rächen,\\
was uns zuviel geschehn,\\
uns wird er Trost zusprechen,\\
uns wieder lassen sehn\\
die Sonne seiner Gnaden:\\
Die wird in kurzer Zeit\\
des Landes Klag und Schaden\\
verkehrn in Glück und Freud.

\end{verse}
\end{center}

