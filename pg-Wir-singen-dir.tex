%StartInfo%%%%%%%%%%%%%%%%%%%%%%%%%%%%%%%%%%%%%%%%%%%%%%%%%%%%%%%%%%%%%%%%%%%%
%  Desc:  Wir singen dir, Immanuel - Paul Gerhard
%  Desc:  Input 
%  Desc:  Zweispaltiger Satz mit \verse
%  Tags:  GERHARD VERSE INCLUDE
%  File:  pg-wir-singen-dir.tex
%  Autor: dv
%  Ref:   
%  Mod:   16.11.2016/dv/initial
%  Mod:   
%EndInfo%%%%%%%%%%%%%%%%%%%%%%%%%%%%%%%%%%%%%%%%%%%%%%%%%%%%%%%%%%%%%%%%%%%%%%

%\poemtitle{Wir singen dir, Immanuel}
\begin{multicols}{2}
\settowidth{\versewidth}{Hast du doch selbst dich schwach gemacht,}                                                  
\begin{verse}[\versewidth]
  
  \flagverse{1.} Wir singen dir, Immanuel,\\
  du Lebensfürst und Gnadenquell,\\
  du Himmelsblum und Morgenstern,\\
  du Jungfraunsohn, Herr aller Herrn!\\
  Halleluja!

  \flagverse{2.} Wir singen dir in deinem Heer\\
  aus aller Kraft Lob, Preis und Ehr,\\
  daß du, o lang gewünschter Gast,\\
  dich nunmehr eingestellet hast.\\
  Halleluja!

  \flagverse{3.} Vom Anfang, da die Welt gemacht,\\
  hat so manch Herz nach dir gewacht;\\
  dich hat gehofft so lange Jahr\\
  der Väter und Propheten Schar.\\
  Halleluja!

  \flagverse{4.} Vor andern hat dein hoch begehrt\\
  der Hirt und König deiner Herd,\\
  der Mann, der dir so wohl gefiel,\\
  wann er dir sang auf Saitenspiel.\\
  Halleluja!

  \flagverse{5.} Ach, daß der Herr aus Zion käm\\
  und unsre Bande von uns nähm!\\
  Ach, Daß die Hilfe bräch herein,\\
  so würde Jakob fröhlich sein.\\
  Halleluja!

  \flagverse{6.} Nun du bist hier, da liegest du,\\
  hältst in dem Kripplein deine Ruh;\\
  bist klein und machst doch alles groß,\\
  bekleidst die Welt und kommst doch bloß.\\
  Halleluja!
  
  \flagverse{7.} Du kehrst in fremder Hausung ein,\\
  und sind doch alle Himmel dein;\\
  trinkst Milch aus deiner Mutter Brust\\
  und bist doch selbst der Engel Lust.\\
  Halleluja!

  \flagverse{8.} Du hast dem Meer sein Ziel gesteckt\\
  und wirst mit Windeln zugedeckt;\\
  bist Gott und liegst auf Heu und Stroh,\\
  wirst Mensch und bist doch A und O.\\
  Halleluja!

  \flagverse{9.} Du bist der Ursprung aller Freud\\
  und duldest so viel Herzeleid;\\
  bist aller Heiden Trost und Licht,\\
  suchst selber Trost und findst ihn nicht.\\
  Halleluja!

  \flagverse{10.} Du bist der süße Menschenfreund,\\
  doch sind dir so viel Menschen feind;\\
  Herodis Heer hält dich für Greul\\
  und bist doch nichts als lauter Heil.\\
  Halleluja!

  \flagverse{11.} Ich aber, dein geringster Knecht,\\
  ich sag es frei und mein es recht:\\
  Ich liebe dich, doch nicht so viel,\\
  als ich dich gerne lieben will.\\
  Halleluja!

  \flagverse{12.} Der Will ist da, die Kraft ist klein;\\
  doch wird dir nicht zuwider sein\\
  mein armes Herz, und was es kann,\\
  wirst du in Gnaden nehmen an.\\
  Halleluja!

  \flagverse{13.} Hast du doch selbst dich schwach gemacht,\\
  erwähltest, was die Welt veracht't;\\
  warst arm und dürftig, nahmst vorlieb\\
  da, wo der Mangel dich hintrieb.\\
  Halleluja!

  \flagverse{14.} Du schliefst ja auf der Erden Schoß;\\
  so war das Kripplein auch nicht groß;\\
  der Stall, das Heu, das dich umfing,\\
  war alles schlecht und sehr gering.\\
  Halleluja!

  \flagverse{15.} Darum so hab ich guten Mut:\\
  Du Wirst auch halten mich für gut.\\
  O Jesulein, Dein frommer Sinn\\
  macht, daß ich so voll Trostes bin.\\
  Halleluja!

  \flagverse{16.}Bin ich gleich sünd- und lastervoll,\\
  hab ich gelebt nicht, wie ich soll,\\
  ei, kommst du doch deswegen her,\\
  daß sich der Sünder zu dir kehr.\\
  Halleluja!

  \flagverse{17.} Hätt ich nicht auf mir Sündenschuld,\\
  hätt ich kein Teil an deiner Huld;\\
  vergeblich wärst du mir geborn,\\
  wenn ich nicht wär in Gottes Zorn.\\
  Halleluja!

  \flagverse{18.} So faß ich dich nun ohne Scheu,\\
  du machst mich alles Jammers frei;\\
  du trägst den Zorn, du würgst den Tod,\\
  verkehrst in Freud all Angst und Not.\\
  Halleluja!

  \flagverse{19.} Du bist mein Haupt, hinwiederum\\
  bin ich dein Glied und Eigentum\\
  und will, so viel dein Geist mir gibt,\\
  stets dienen dir, wie dir's beliebt.\\
  Halleluja!

  \flagverse{20.} Ich will dein Halleluja hier\\
  mit Freuden singen für und für\\
  und dort in deinem Ehrensaal\\
  solls schallen ohne Zeit und Zahl.\\
  Halleluja!

\end{verse}
\end{multicols}
%\attrib{\small{1653}}
