%StartInfo%%%%%%%%%%%%%%%%%%%%%%%%%%%%%%%%%%%%%%%%%%%%%%%%%%%%%%%%%%%%%%%%%%%%
%  Autor:
%  Titel:
%  File:
%  Ref:
%  Mod:
%EndInfo%%%%%%%%%%%%%%%%%%%%%%%%%%%%%%%%%%%%%%%%%%%%%%%%%%%%%%%%%%%%%%%%%%%%%%
%\poemtitle{Ein Lämmlein geht und trägt die Schuld}
\begin{multicols}{2}
\settowidth{\versewidth}{Ein Lämmlein geht und trägt die Schuld}
\begin{verse}[\versewidth]
 
\flagverse{1.} Ein Lämmlein geht und trägt die Schuld\\
der Welt und ihrer Kinder;\\
es geht und büßet in Geduld\\
die Sünden aller Sünder.\\
Es geht dahin, wird matt und krank,\\
ergibt sich auf die Würgebank,\\
verzeiht sich allen Freuden;\\
es nimmet an Schmach, Hohn und Spott,\\
Angst, Wunden, Striemen, Kreuz und Tod\\
und spricht: Ich wills gern leiden.
 
\flagverse{2.} Das Lämmlein ist der große Freund\\
und Heiland meiner Seelen;\\
den, den hat Gott zum Sündenfeind\\
und Sühner wollen wählen.\\
\flqq Geh hin, mein Kind, und nimm dich an\\
der Kinder, die ich ausgetan\\
zur Straf und Zornesruten;\\
die Straf ist schwer, der Zorn ist groß;\\
du kannst und sollst sie machen los\\
durch Sterben und durch Bluten.\frqq
 
\flagverse{3.} \flqq Ja, Vater, ja von Herzensgrund,\\
leg auf, ich will dirs tragen.\\
Mein Wollen hängt an deinem Mund;\\
mein Wirken ist dein Sagen.\frqq\\
O Wunder Lieb, o Liebesmacht,\\%                        O Wunder Lieb ... ?
du kannst, was nie kein Mensch gedacht,\\
Gott seinen Sohn abzwingen.\\
O Liebe, Liebe, Du bist stark,\\
du strecktest den ins Grab und Sarg,\\
vor dem die Felsen springen.
%                                       Die Liebe ist stärker als Gott?
 
\flagverse{4.} Du marterst ihn am Kreuzesstamm\\
mit Nägeln und mit Spießen;\\
du schlachtest ihn als wie ein Lamm,\\
machst Herz und Adern fließen:\\
Das Herze mit der Seufzer Kraft,\\
die Adern mit dem edlen Saft\\
des purpurroten Blutes.\\
O süßes Lamm, was soll ich dir\\
erweisen dafür, daß du mir\\
erweisest so viel Gutes?
 
\flagverse{5.} Mein Lebetage will ich dich\\
aus meinem Sinn nicht lassen;\\
dich will ich stets, gleich wie du mich,\\
mit Liebesarmen fassen.\\
Du sollst sein meines Herzens Licht,\\
und wenn mein Herz in Stücken bricht,\\
sollst du mein Herze bleiben.\\
Ich will mich dir, mein höchster Ruhm,\\
hiermit zu deinem Eigentum\\
beständiglich verschreiben.
 
\flagverse{6.} Ich will von deiner Lieblichkeit\\
bei Nacht und Tage singen,\\
mich selbst auch dir nach Möglichkeit\\
zum Freudenopfer bringen.\\
Mein Bach des Lebens soll sich dir\\
und deinem Namen für und für\\
in Dankbarkeit ergießen;\\
und was du mir zu gut getan,\\
das will ich stets, so tief ich kann,\\
in mein Gedächtnis schließen.
 
\flagverse{7.} Erweitre dich, mein Herzensschrein,\\
du sollst ein Schatzhaus werden\\
der Schätze, die viel größer sein\\
als Himmel, Meer und Erden.\\
Weg mit dem Gold Arabia!\\
Weg Kalmus, Myrrhen, Kassia!\\
Ich hab ein Bessers funden:\\
Mein großer Schatz, Herr Jesu Christ,\\
ist dieses, was geflossen ist\\
aus deines Leibes Wunden.
 
\flagverse{8.} Das soll und will ich mir zu nutz\\
zu allen Zeiten machen;\\
im Streite soll es sein mein Schutz,\\
in Traurigkeit mein Lachen,\\
in Fröhlichkeit mein Saitenspiel,\\
und wenn mir nichts mehr schmecken will,\\
soll mich dies Manna speisen.\\
Im Durst solls sein mein Wasserquell,\\
in Einsamkeit mein Sprachgesell\\
zu Haus und auch auf Reisen.
 
\flagverse{9.} Was schadet mir des Todes Gift?\\
Dein Blut, Das ist mein Leben.\\
Wenn mich der Sonnen Hitze trifft,\\
so kann mirs Schatten geben.\\
Setzt mir der Wehmut Schmerzen zu,\\
so find ich bei dir meine Ruh\\
als auf dem Bett ein Kranker.\\
Und wenn des Kreuzes Ungestüm\\
mein Schifflein treibet üm und üm,\\
so bist du dann mein Anker.
 
\flagverse{10}. Wenn endlich ich soll treten ein\\
in deines Reiches Freuden,\\
so soll dies Blut mein Purpur sein,\\
ich will mich darin kleiden;\\
es soll sein meines Hauptes Kron,\\
in welcher ich will vor dem Thron\\
des höchsten Vaters gehen\\
und dir, dem er mich anvertraut,\\
als eine wohlgeschmückte Braut\\
an deiner Seite stehen.

\end{verse}
\end{multicols}
%\attrib{\small{THZE}}
