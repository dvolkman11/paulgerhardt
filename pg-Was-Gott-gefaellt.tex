%StartInfo%%%%%%%%%%%%%%%%%%%%%%%%%%%%%%%%%%%%%%%%%%%%%%%%%%%%%%%%%%%%%%%%%%%%
%  Autor:
%  Titel:
%  File:
%  Ref:
%  Mod:
%EndInfo%%%%%%%%%%%%%%%%%%%%%%%%%%%%%%%%%%%%%%%%%%%%%%%%%%%%%%%%%%%%%%%%%%%%%%
%\poemtitle{pt}
\begin{multicols}{2}
\settowidth{\versewidth}{Was Gott gefällt, mein frommes Kind,}
\begin{verse}[\versewidth]
%was Gott gefällt, mein frommes Kind, nimm fröhlich an!

\flagverse{1.} Was Gott gefällt, mein frommes Kind,\\
nimm fröhlich an! Stürmt gleich der Wind\\
und braust, daß alles kracht und bricht,\\
so sei getrost, denn dir geschicht,\\
was Gott gefällt.

\flagverse{2.} Der beste Will ist Gottes Will,\\
auf diesem ruht man sanft und still,\\
da gib dich allzeit frisch hinein,\\
begehre nichts, als nur allein,\\
was Gott gefällt.

\flagverse{3.} Der klügste Sinn ist Gottes Sinn,\\
was Menschen sinnen, fället hin,\\
wird plötzlich kraftlos, müd und laß,\\
tut oft, was bös, und selten das,\\
was Gott gefällt.

\flagverse{4.} Der frömmste Mut ist Gottes Mut,\\
der niemand Arges gönnt und tut,\\
er segnet, wenn uns schilt und flucht\\
die böse Welt, die nimmer sucht,\\
was Gott gefällt.

\flagverse{5.} Das treuste Herz ist Gottes Herz,\\
treibt alles Unglück hinterwärts,\\
beschirmt und schützet Tag und Nacht\\
den, der stets hoch und herrlich acht,\\
was Gott gefällt.

\flagverse{6.} Ach könnt ich singen, wie ich wohl\\
im Herzen wünsch und billig soll,\\
so wollt ich öffnen meinen Mund\\
und singen jetzo diese Stund,\\
was Gott gefällt.

\flagverse{7.} Ich wollt erzählen seinen Rat\\
und übergroße Wundertat,\\
das süße Heil, die ewge Kraft,\\
die allenthalben wirkt und schafft,\\
was Gott gefällt.

\flagverse{8.} Er ist der Herrscher in der Höh,\\
auf ihm steht unser Wohl und Weh,\\
er trägt die Welt in seiner Hand,\\
hinwieder trägt uns See und Land,\\
was Gott gefällt.

\flagverse{9.} Er hält der Elemente Lauf,\\
und damit hält er uns auch auf,\\
gibt Sommer, Winter, Tag und Nacht\\
und alles, davon lebt und lacht,\\
was Gott gefällt.

\flagverse{10.} Sein Heer, die Sterne, Sonn und Mond\\
gehn ab und zu, wie sie gewohnt,\\
die Erd ist fruchtbar, bringt herfür\\
Korn, Öl und Most, Brot, Wein und Bier,\\
was Gott gefällt.

\flagverse{11.} Sein ist die Weisheit und Verstand,\\
ihm ist bewußt und wohlbekannt\\
sowohl wer Böses tut und übt\\
als auch wer Gutes tut und liebt,\\
was Gott gefällt.

\flagverse{12.} Sein Häuflein ist ihm lieb und wert;\\
sobald es sich zu Sünden kehrt,\\
so winkt er mit der Vaterrut\\
und locket, bis man wieder tut,\\
was Gott gefällt.

\flagverse{13.} Was unserm Herzen dienlich sei,\\
das weiß sein Herz, ist fromm dabei,\\
der keinem jemals Guts versagt,\\
der Guts gesucht, dem nachgejagt,\\
was Gott gefällt.

\flagverse{14.} Ist dem also, so mag die Welt\\
behalten, was ihr wohlgefällt;\\
du aber, mein Herz, halt genehm\\
und nimm fürlieb mit Gott und dem,\\
was Gott gefällt.

\flagverse{15.} Laß andre sich mit stolzem Mut\\
erfreuen über großes Gut,\\
du aber nimm des Kreuzes Last\\
und sei geduldig, wenn du hast,\\
was Gott gefällt.

\flagverse{16.} Lebst du in Sorg und großem Leid,\\
hast lauter Gram und Herzeleid,\\
ei, sei zufrieden; trägst du doch\\
in diesem sauren Lebensjoch,\\
was Gott gefällt.

\flagverse{17.} Mußt du viel leiden hie und dort,\\
so bleibe fest an deinem Hort,\\
denn alle Welt und Kreatur\\
ist unter Gott, kann nichts als nur,\\
was Gott gefällt.

\flagverse{18.} Wirst du veracht't von jedermann,\\
höhnt dich dein Feind und speit dich an:\\
Sei wohlgemut, denn Jesus Christ\\
erhöhet dich, weil in dir ist,\\
was Gott gefällt.

\flagverse{19.} Glaub, Hoffnung, Sanftmut und Geduld\\
erhalten Gottes Gnad und Huld;\\
die schleuß in deines Herzens Schrein,\\
so wird dein ewges Erbe sein,\\
was Gott gefällt.

\flagverse{20.} Dein Erb ist in dem Himmelsthron,\\
hier ist dein Zepter, Reich und Kron,\\
hier wirst du schmecken, hören, sehn,\\
hier wird ohn Ende dir geschehn,\\
was Gott gefällt.

\end{verse}
\end{multicols}
%\attrib{\small{THZE}}
