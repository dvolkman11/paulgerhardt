%StartInfo%%%%%%%%%%%%%%%%%%%%%%%%%%%%%%%%%%%%%%%%%%%%%%%%%%%%%%%%%%%%%%%%%%%%
%  Autor:
%  Titel:
%  File:
%  Ref:
%  Mod:
%EndInfo%%%%%%%%%%%%%%%%%%%%%%%%%%%%%%%%%%%%%%%%%%%%%%%%%%%%%%%%%%%%%%%%%%%%%%
%\poemtitle{pt}
\begin{multicols}{2}
\settowidth{\versewidth}{Wie schön ists doch, Herr Jesu Christ,}
\begin{verse}[\versewidth]
%trostgesang christlicher Eheleute\\
%wie schön ists doch, Herr Jesu Christ, im Stande heilger Ehe

\flagverse{1.} Wie schön ists doch, Herr Jesu Christ,\\
im Stande, da dein Segen ist,\\
im Stande heilger Ehe!\\
Wie steigt und neigt sich deine Gab\\
und alles Gut so mild herab\\
aus deiner heilgen Höhe,\\
wenn sich\\
an dich\\
fleißig halten\\
jung und Alten,\\
die im Orden\\
eines Lebens einig worden!

\flagverse{2.} Wenn Mann und Weib sich wohl begehn\\
und unverrückt beisammen stehn\\
im Bande reiner Treue:\\
Da geht das Glück in vollem Lauf,\\
da sieht man wie der Engel Hauf\\
im Himmel selbst sich freue.\\
Kein Sturm,\\
kein Wurm\\
kann zerschlagen,\\
kann zernagen\\
was Gott gibet\\
dem Paar, das in ihm sich liebet.

\flagverse{3.} Vor allen gibt er seine Gnad\\
in derer Schoß er früh und spat\\
sein hoch Geliebten heget:\\
Da spannt sein Arm sich täglich aus,\\
da faßt er uns und unser Haus\\
gleich als ein Vater pfleget.\\
Da muß\\
ein Fuß\\
nach dem andern\\
gehn und wandern,\\
bis sie kommen\\
in das Zelt und Sitz der Frommen.

\flagverse{4.} Der Man wird einem Baume gleich\\
an Ästen schön, an Zweigen reich,\\
das Weib gleich einem Reben,\\
der seine Träublein trägt und nährt\\
und sich je mehr und mehr vermehrt\\
mit Früchten, die da leben.\\
Wohl dir,\\
o Zier,\\
Mannes Sonne,\\
Hauses Wonne,\\
Ehrenkrone!\\
Gott denkt dein bei seinem Throne.

\flagverse{5.} Dich, dich hat er sich auserkorn,\\
daß aus dir ward herausgeborn\\
das Volk, das sein Reich bauet.\\
Sein Wunderwerk geht immer fort,\\
und seines Mundes starkes Wort\\
macht, daß dein Auge schauet\\
schöne\\
Söhne\\
und die Tocken,\\
die den Wocken\\
abespinnen\\
und mit Kunst die Zeit gewinnen.

\flagverse{6.} Sei gutes Muts! Wir sind es nicht,\\
die diesen Orden aufgericht,\\
es ist ein höhrer Vater,\\
der hat uns je und je geliebt\\
und bleibt, wenn unsre Sorg uns trübt,\\
der beste Freund und Rater.\\
Anfang,\\
Ausgang\\
aller Sachen,\\
die zu machen\\
wir gedenken,\\
wird er wohl und weislich lenken.

\flagverse{7.} Zwar bleibts nicht aus, es kommt ja wohl\\
ein Stündlein, da man Leides voll\\
die Tränen lässet schießen;\\
jedennoch wer sich in Geduld\\
ergibt, des Leid wird Gottes Huld\\
in großen Freuden schließen.\\
Sitze,\\
schwitze\\
nur ein wenig!\\
Unser König\\
wird behende\\
machen, daß die Angst sich wende.

\flagverse{8.} Wohlher, mein König, nah herzu,\\
gib Rat in Kreuz, in Nöten Ruh,\\
in Ängsten Trost und Freude!\\
Des sollst du haben Ruhm und Preis,\\
wir wollen singen bester Weis\\
und danken alle beide,\\
bis wir\\
bei dir,\\
deinen Willen\\
zu erfüllen,\\
deinen Namen\\
ewig loben werden. Amen.

\end{verse}
\end{multicols}
\attrib{\small{Trostgesang christlicher Eheleute}}
