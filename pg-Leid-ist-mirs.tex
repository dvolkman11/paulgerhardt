%StartInfo%%%%%%%%%%%%%%%%%%%%%%%%%%%%%%%%%%%%%%%%%%%%%%%%%%%%%%%%%%%%%%%%%%%%
%  Autor:
%  Titel:
%  File:
%  Ref:
%  Mod:
%EndInfo%%%%%%%%%%%%%%%%%%%%%%%%%%%%%%%%%%%%%%%%%%%%%%%%%%%%%%%%%%%%%%%%%%%%%%
%\poemtitle{pt}
\begin{multicols}{2}
\settowidth{\versewidth}{Muß das Leibchen gleich verwesen,}
\begin{verse}[\versewidth]
%leid ist mirs in meinem Herzen\\
%auf den Tod der kleinen Elisabeth Heintzelmann,
%Tochter des Diakons an St. Nikolai in Berlin Johannes H. (1659)

\flagverse{1.} Leid ist mirs in meinem Herzen\\
um die, so dir, liebes Kind,\\
mit so großem Weh und Schmerzen\\
um den Hals gefallen sind,\\
da du dich bei deinem Ende\\
gabst in deines Gottes Hände.

\flagverse{2.} Ach, es ist ein bittres Leiden\\
und ein rechter Myrrhentrank,\\
sich von seinen Kindern scheiden\\
durch den schweren Todesgang!\\
Hier geschieht ein Herzensbrechen,\\
das kein Mund recht kann aussprechen.

\flagverse{3.} Aber das, was wir beweinen,\\
weiß hievon ganz lauter nichts,\\
sondern sieht die Sonne scheinen\\
und den Glanz des ewgen Lichts,\\
singt und springt und hört die Scharen,\\
die hier seine Wächter waren.

\flagverse{4.} Muß das Leibchen gleich verwesen,\\
ists ihm doch ein schlechter Schad,\\
Gott wird schon zusammenlesen,\\
was der Tod zerstreuet hat;\\
treu ist er und fromm den Seinen,\\
trägt sich auch mit ihren Beinen.

\flagverse{5.} Diesem Herrn ist nichts verdorben;\\
wenn des Todes Nacht vorbei,\\
nimmt er das, was war gestorben,\\
und machts wieder ganz und neu.\\
Also werden wir zur Erden,\\
daß wir mögen himmlisch werden.

\flagverse{6.} Auf derwegen! Seid zufrieden,\\
Vaterherz und Muttergeist,\\
lasset schlafen, was geschieden\\
und zu Gott ist hingereist!\\
Was für Tränen ihr vergossen,\\
wollen sein mit Trost geschlossen.
\end{verse}
\end{multicols}
%\attrib{\small{THZE}}

\begin{center}
\settowidth{\versewidth}{Der, vor dem die Welt erschrickt,}
\begin{verse}[\versewidth]


\flagverse{7.} Wandelt eure Klag in Singen!\\
Ist doch nunmehr alles gut.\\
Trauern mag nicht wiederbringen,\\
was im Himmelsschoße ruht.\\
Aber wer getrost sich gibet,\\
ist bei Gott sehr hoch beliebet.
\end{verse}
\end{center}
  
%\attrib{\small{THZE}}
