%StartInfo%%%%%%%%%%%%%%%%%%%%%%%%%%%%%%%%%%%%%%%%%%%%%%%%%%%%%%%%%%%%%%%%%%%%
%  Autor:
%  Titel:
%  File:
%  Ref:
%  Mod:
%EndInfo%%%%%%%%%%%%%%%%%%%%%%%%%%%%%%%%%%%%%%%%%%%%%%%%%%%%%%%%%%%%%%%%%%%%%%
%\poemtitle{pt}
\begin{multicols}{2}
\settowidth{\versewidth}{Der wird zu Schanden, der dich schändt}
\begin{verse}[\versewidth]
%der 25. Psalm\\
%nach dir, o Herr, verlanget mich

\flagverse{1.} Nach dir, o Herr, verlanget mich,\\
du bist mein Gott, ich hoff auf dich,\\
ich hoff und bin der Zuversicht,\\
du werdest mich beschämen nicht.

\flagverse{2.} Der wird zu Schanden, der dich schändt\\
und sein Gemüte von dir wendt,\\
der aber, der sich dir ergibt\\
und dich recht liebt, bleibt unbetrübt.

\flagverse{3.} Herr, nimm dich meiner Seelen an\\
und führe sie die rechte Bahn,\\
laß deine Wahrheit leuchten mir\\
im Steige, der uns bringt zu dir.

\flagverse{4.} Denn du bist ja mein einigs Licht,\\
sonst weiß ich keinen Helfer nicht,\\
ich harre dein bei Tag und Nacht:\\
Was ists, das dich so säumend macht?

\flagverse{5.} Ach wende, Herr, dein Augen ab\\
von dem, wo ich geirret hab.\\
Was denkst du an den Sündenlauf,\\
den ich geführt von Jugend auf?

\flagverse{6.} Gedenk an deine Gütigkeit\\
und an die große Süßigkeit,\\
damit dein Herz zu trösten pflegt\\
das, was sich dir zu Füßen legt.

\flagverse{7.} Der Herr ist fromm und herzlich gut\\
dem, der sich prüft und Buße tut,\\
wer seinen Bund und Zeugnis hält,\\
der wird erhalten, wenn er fällt.

\flagverse{8.} Ein Herz, das Gott von Herzen scheut,\\
das wird in seinem Leid erfreut,\\
und wenn die Not am tiefsten steht,\\
so wird sein Kreuz zur Wonn erhöht.

\flagverse{9.} Nun, Herr, ich bin dir wohlbekannt,\\
mein Geist, der schwebt in deiner Hand,\\
du siehst, wie meine Seele tränt\\
und sich nach deiner Hilfe sehnt.

\flagverse{10.} Die Angst, die mir mein Herze dringt\\
und daraus soviel Seufzer zwingt,\\
ist groß; du aber bist der Mann,\\
dem nichts zu groß entstehen kann.

\flagverse{11.} Drum steht mein Auge stets nach dir\\
und trägt dir mein Begehren für.\\
Ach laß doch, wie du pflegst zu tun,\\
dein Aug auf meinen Augen ruhn.

\flagverse{12.} Wenn ich dein darf, so wende nicht\\
von mir dein Aug und Angesicht,\\
laß deiner Antwort Gegenschein\\
mit meinem Beten stimmen ein.

\flagverse{13.} Die Welt ist falsch, du bist mein Freund,\\
ders treulich und von Herzen meint,\\
der Menschen Gunst steht nur im Mund,\\
du aber liebst von Herzensgrund.

\flagverse{14.} Zerreiß die Netz, heb auf die Strick\\
und brich des Feindes List und Tück,\\
und wenn mein Unglück ist vorbei,\\
so gib, daß ich auch dankbar sei.

\flagverse{15.} Laß mich in deiner Furcht bestehn,\\
fein schlecht und recht stets einhergehn;\\
gib mir die Einfalt, die dich ehrt\\
und lieber duldet als beschwert.

\flagverse{16.} Regier und führe mich zu dir,\\
auch andre Christen neben mir,\\
nimm, was dir mißfällt, von uns hin,\\
gib neue Herzen, neuen Sinn.

\end{verse}
\end{multicols}
%\attrib{\small{THZE}}

\begin{center}
\settowidth{\versewidth}{Der, vor dem die Welt erschrickt,}
\begin{verse}[\versewidth]
  
\flagverse{17.} Wasch ab all unsern Sündenkot,\\
erlös aus aller Angst und Not,\\
und führ uns bald mit Gnaden ein\\
zum ewgen Fried und Freudenschein.
   
\end{verse}
\end{center}


