%StartInfo%%%%%%%%%%%%%%%%%%%%%%%%%%%%%%%%%%%%%%%%%%%%%%%%%%%%%%%%%%%%%%%%%%%%
%  Autor:
%  Titel:
%  File:
%  Ref:
%  Mod:
%EndInfo%%%%%%%%%%%%%%%%%%%%%%%%%%%%%%%%%%%%%%%%%%%%%%%%%%%%%%%%%%%%%%%%%%%%%%
%\poemtitle{Gedult ist euch vonnöten}
\begin{multicols}{2}
\settowidth{\versewidth}{Geduld ist Fleisch und Blute}
\begin{verse}[\versewidth]

%(Hebr. 10, 35-375)

\flagverse{1.} Geduld ist euch vonnöten,\\
wann Sorge, Gram und Leid\\
und was euch mehr will töten,\\
euch in das Herze schneidt,\\
o auserwählte Zahl!\\
Soll euch kein Tod nicht töten,\\
ist euch Geduld vonnöten:\\
Das sag ich noch einmal.

\flagverse{2.} Geduld ist Fleisch und Blute\\
ein herbes, bittres Kraut;\\
wenn unsers Kreuzes Rute\\
uns nur ein wenig draut,\\
erschrickt der zarte Sinn.\\
Im Glück ist er verwegen,\\
kommt aber Sturm und Regen,\\
fällt Herz und Mut dahin.

\flagverse{3.} Geduld ist schwer zu leiden,\\
dieweil wir irdisch seind\\
und nur in lautern Freuden\\
bei Gott zu sein vermeint.\\
Der doch sich klar erklärt:\\
Ich strafe, die ich liebe,\\
und die ich hoch betrübe,\\
die halt ich hoch und wert.

\flagverse{4.} Geduld ist Gottes Gabe\\
und seines Geistes Gut,\\
der zeucht und löst uns abe,\\
sobald er in uns ruht,\\
der edle werte Gast,\\
erlöst uns von dem Zagen\\
und hilft uns treulich tragen\\
die große Bürd und Last.

\flagverse{5.} Geduld kommt aus dem Glauben\\
und hängt an Gottes Wort;\\
das läßt sie ihr nicht rauben,\\
das ist ihr Heil und Hort,\\
das ist ihr hoher Wall,\\
da hält sie sich verborgen,\\
läßt Gott den Vater sorgen\\
und fürchtet keinen Fall.

\flagverse{6.} Geduld setzt ihr Vertrauen\\
auf Christi Tod und Schmerz,\\
macht Satan ihr ein Grauen,\\
so faßt sie hier ein Herz\\
und spricht: Zürn immerhin,\\
du wirst mich doch nicht fressen,\\
ich bin zu hoch gesessen,\\
weil ich in Christo bin!

\flagverse{7.} Geduld ist wohl zufrieden\\
mit Gottes weisem Rat,\\
läßt sich nicht leicht ermüden\\
durch Aufschub seiner Gnad,\\
hält frisch und fröhlich aus,\\
läßt sich getrost beschweren\\
und denkt: Wer wills ihm wehren?\\
Ist er doch Herr im Haus.

\flagverse{8.} Geduld kann lange warten,\\
vertreibt die lange Weil\\
in Gottes schönem Garten,\\
durchsucht zu ihrem Heil\\
das Paradies der Schrift\\
und schützt sich früh und späte\\
mit eifrigem Gebete\\
vor Satans List und Gift.

\flagverse{9.} Geduld tut Gottes Willen,\\
erfüllet sein Gebot\\
und weiß sich wohl zu stillen\\
in aller Feinde Spott.\\
Es lache, wems beliebt:\\
Wird sie doch nicht zuschanden,\\
es ist bei ihr vorhanden\\
ein Herz, das nichts drauf gibt.

\flagverse{10.} Geduld dient Gott zu Ehren\\
und läßt sich nimmermehr\\
von seiner Liebe kehren;\\
und schlüg er noch so sehr,\\
so ist sie doch bedacht,\\
sein heilge Hand zu loben,\\
spricht: Der im Himmel droben\\
hat alles wohl gemacht.

\flagverse{11.} Geduld erhält das Leben,\\
vermehrt der Jahre Zahl,\\
vertreibt und dämpft darneben\\
manch Angst und Herzensqual;\\
ist wie ein schönes Licht,\\
davon, wer an ihr hanget,\\
mit Gottes Hilf erlanget\\
ein fröhlichs Angesicht.

\flagverse{12.} Geduld macht große Freude,\\
bringt aus dem Himmelsthron\\
ein schönes Halsgeschmeide,\\
dem Haupt ein edle Kron\\
und königlichen Hut;\\
stillt die betrübten Tränen\\
und füllt das heiße Sehnen\\
mit rechtem guten Gut.

\flagverse{13.} Geduld ist mein Verlangen\\
und meines Herzens Lust,\\
nach der ich oft gegangen:\\
Das ist dir wohl bewußt,\\
Herr voller Gnad und Huld,\\
ach, gib mir und gewähre\\
mein Bitten! Ich begehre\\
nichts andres als Geduld.

\flagverse{14.} Geduld ist meine Bitte,\\
die ich sehr oft und viel\\
aus dieser Leibeshütte\\
zu dir, Herr, schicken will.\\
Kommt dann der letzte Zug,\\
so gib durch deine Hände\\
auch ein geduldigs Ende!\\
So hab ich alles gnug.

\end{verse}
\end{multicols}
%\attrib{\small{THZE}}
