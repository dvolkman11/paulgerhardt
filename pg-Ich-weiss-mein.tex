%StartInfo%%%%%%%%%%%%%%%%%%%%%%%%%%%%%%%%%%%%%%%%%%%%%%%%%%%%%%%%%%%%%%%%%%%%
%  Autor:
%  Titel:
%  File:
%  Ref:
%  Mod:
%EndInfo%%%%%%%%%%%%%%%%%%%%%%%%%%%%%%%%%%%%%%%%%%%%%%%%%%%%%%%%%%%%%%%%%%%%%%
%\poemtitle{pt}
\begin{multicols}{2}
\settowidth{\versewidth}{Ich weiß, mein Gott, das all mein Tun}
\begin{verse}[\versewidth]
%vertrauen auf Gottes Willen\\
%ich weiß, mein Gott, daß all mein Tun und Werk auf deinem Willen ruhn\\
%(Jer. 10, 23)

\flagverse{1.} Ich weiß, mein Gott, daß all mein Tun\\
und Werk in deinem Willen ruhn,\\
von dir kommt Glück und Segen;\\
was du regierst, das geht und steht\\
auf rechten, guten Wegen.

\flagverse{2.} Es steht in keines Menschen Macht,\\
daß sein Rat werd ins Werk gebracht\\
und seines Gangs sich freue;\\
des Höchsten Rat, der machts allein,\\
daß Menschenrat gedeihe.

\flagverse{3.} Oft denkt der Mensch in seinem Mut,\\
dies oder jenes sei ihm gut,\\
und ist doch weit gefehlet;\\
oft sieht er auch für schädlich an,\\
was doch Gott selbst erwählet.

\flagverse{4.} So fängt auch mancher weise Mann\\
ein gutes Werk zwar fröhlich an\\
und bringts doch nicht zum Stande;\\
er baut ein Schloß und festes Haus,\\
doch nur auf lauterm Sande.

\flagverse{5.} Wie mancher ist in seinem Sinn\\
fast über Berg und Spitzen hin,\\
und eh er sichs versiehet,\\
so liegt er da und hat sein Fuß\\
vergeblich sich bemühet.

\flagverse{6.} Drum, lieber Vater, der du Kron\\
und Zepter trägst in deinem Thron\\
und aus den Wolken blitzest,\\
vernimm mein Wort und höre mich\\
vom Stuhle, da du sitzest.

\flagverse{7.} Verleihe mir das edle Licht,\\
das sich von deinem Angesicht\\
in fromme Seelen strecket\\
und da der rechten Weisheit Kraft\\
durch deine Kraft erwecket.

\flagverse{8.} Gib mir Verstand aus deiner Höh,\\
auf daß ich ja nicht ruf und steh\\
auf meinem eignen Willen;\\
sei du mein Freund und treuer Rat,\\
was recht ist, zu erfüllen.

\flagverse{9.} Prüf alles wohl, und was mir gut,\\
das gib mir ein; was Fleisch und Blut\\
erwählet, das verwehre;\\
der Höchste Zweck, das beste Teil\\
sei deine Lieb und Ehre.

\flagverse{10.} Was dir gefällt, das laß auch mir,\\
o meiner Seelen Sonn und Zier,\\
gefallen und belieben;\\
was dir zuwider, laß mich nicht\\
im Werk und Tat verüben.

\flagverse{11.} Ists Werk von dir, so hilf zu Glück;\\
ists Menschentum, so treibs zurück\\
und ändre meine Sinnen;\\
was du nicht wirkst, pflegt von ihm selbst\\
in kurzem zu zerrinnen.

\flagverse{12.} Sollt aber dein und unser Feind\\
an dem, was dein Herz gut gemeint,\\
beginnen sich zu rächen:\\
Ist das mein Trost, daß seinen Zorn\\
du leichtlich könnest brechen.

\flagverse{13.} Tritt zu mir zu und mache leicht,\\
was mir sonst fast unmöglich deucht,\\
und bring zum guten Ende,\\
was du selbst angefangen hast,\\
durch Weisheit deiner Hände.

\flagverse{14.} Ist ja der Anfang etwas schwer,\\
und muß ich auch ins tiefe Meer\\
der bittern Sorgen treten,\\
so treib mich nur ohn Unterlaß\\
zu seufzen und zu beten.

\flagverse{15.} Wer fleißig betet und dir traut,\\
wird alles, da ihm sonst vor graut,\\
mit tapferm Mut bezwingen;\\
sein Sorgenstein wird in der Eil\\
in tausend Stücke springen.

\flagverse{16.} Der Weg zum Guten ist fast wild,\\
mit Dorn und Hecken ausgefüllt,\\
doch wer ihn freudig gehet,\\
kommt endlich, Herr, durch deinen Geist,\\
wo Freud und Wonne stehet.

\flagverse{17.} Du bist mein Vater, ich dein Kind,\\
was ich bei mir nicht hab und find,\\
hast du zu aller Gnüge,\\
so hilf nur, daß ich meinen Stand\\
wohl halt und herrlich siege.

\flagverse{18.} Dein soll sein aller Ruhm und Ehr,\\
ich will dein Tun je mehr und mehr\\
aus hocherfreuter Seelen\\
vor deinem Volk und aller Welt,\\
so lang ich leb, erzählen.

\end{verse}
\end{multicols}
  %\attrib{\small{THZE}}
