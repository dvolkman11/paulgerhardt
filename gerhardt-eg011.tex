%StartInfo%%%%%%%%%%%%%%%%%%%%%%%%%%%%%%%%%%%%%%%%%%%%%%%%%%%%%%%%%%%%%%%%%%%%
%  Desc:  Wie soll ich dich empfangen - Paul Gerhard
%  Desc:  Zweispaltiger Satz mit \verse
%  Tags:  GERHARD VERSE EG011
%  File:  gerhardt-eg011
%  Autor: dv
%  Ref:   
%  Mod:   12.10.2016/dv/initial
%  Mod:   12.10.2016/dv/pdfflatex compiled
%EndInfo%%%%%%%%%%%%%%%%%%%%%%%%%%%%%%%%%%%%%%%%%%%%%%%%%%%%%%%%%%%%%%%%%%%%%%


\documentclass[fontsize=11pt]{scrartcl}                                                                      
\usepackage[utf8]{inputenc}
\usepackage[T1]{fontenc}
\usepackage[ngerman]{babel}
\usepackage[centering,includeheadfoot,margin=2cm]{geometry}
\usepackage{lmodern}
\usepackage{multicol}
\usepackage{verse}
\usepackage{attrib}                                                                                                      

% Parameter um die Strophen auf eine Seite zu setzen:
\parindent0em
\setlength{\voffset}{-1.5cm}
\setlength{\hoffset}{-0.5cm}
\setlength{\textheight}{27cm}
\setlength{\textwidth}{15cm}
\setlength{\columnwidth}{7cm}
\setlength{\columnsep}{1cm}
\setlength{\vleftskip}{0.1cm}
\setlength{\oddsidemargin}{0.1cm}
\setlength{\vgap}{0.1cm}
%\setlength{\stanzaskip}{1.3cm}

\begin{document}                                                                                                        

\poemtitle{Wie soll ich dich empfangen}
\begin{multicols}{2}
\settowidth{\versewidth}{Wie soll ich dich empfangen}                                                  
\begin{verse}[\versewidth]                                                                                              
  \flagverse{1.} Wie soll ich dich empfangen\\
  und wie begegn' ich dir?\\
  O aller Welt Verlangen!\\
  O meiner Seelen Zier!\\
  O Jesu, Jesu setze\\
  mir selbst die Fackel bei,\\
  damit, was dich ergötze,\\
  mir kund und wissend sei.

  \flagverse{2.} Dein Zion streut dir Palmen\\
  und grüne Zweige hin,\\
  und ich will dir mit Psalmen\\
  ermuntern meinen Sinn.\\
  Mein Herze soll dir grünen\\
  in stetem Lob und Preis\\
  und deinem Namen dienen,\\
  so gut es kann und weiß.
  
  \flagverse{3.} Was hast du unterlassen\\
  zu meinem Trost und Freud'?\\
  Als Leib und Seele saßen,\\
  in ihrem größten Leid,\\
  als mir das Reich genommen,\\
  da Fried' und Freude lacht,\\
  da bist du, mein Heil, kommen,\\
  und hast mich froh gemacht.
  
  \flagverse{4.} Ich lag in schweren Banden,\\
  du kommst und machst mich los;\\
  ich stand in Spott und Schanden,\\
  du kommst und machst mich groß,\\
  und hebst mich hoch zu Ehren\\
  und schenkst mir großes Gut,\\
  das sich nicht läßt verzehren,\\
  wie irdisch Reichtum tut.

  \flagverse{5.} Nichts, nichts hat dich getrieben\\
  zu mir vom Himmelszelt\\
  als das geliebte Lieben,\\
  damit du alle Welt\\
  in ihren tausend Plagen\\
  und großen Jammerlast,\\
  die kein Mund kann aussagen,\\
  so fest umfangen hast.

  \flagverse{6.} Das schreib' dir in dein Herze,\\
  du hochbetrübtes Heer,\\
  bei denen Gram und Schmerze\\
  sich häuft je mehr und mehr;\\
  seid unverzagt, ihr habet\\
  die Hilfe vor der Tür!\\
  Der eure Herzen labet\\
  und tröstet, steht allhier.

  \flagverse{7.} Ihr dürft euch nicht bemühen\\
  noch sorgen Tag und Nacht,\\
  wie ihr ihn wollet ziehen\\
  mit eures Armes Macht.\\
  Er kommt, er kommt mit Willen,\\
  ist voller Lieb und Lust,\\
  all Angst und Not zu stillen,\\
  die ihm an euch bewußt.

  \flagverse{8.} Auch dürft ihr nicht erschrecken\\
  vor eurer Sündenschuld.\\
  Nein! Jesus will sie decken\\
  mit seiner Lieb' und Huld!\\
  Er kommt, er kommt den Sündern\\
  zum Trost und wahren Heil,\\
  schafft, daß bei Gottes Kindern\\
  verbleib' ihr Erb' und Teil.

  \flagverse{9.} Was fragt ihr nach dem Schreien\\
  der Feind' und ihrer Tück'?\\
  Ihr Herr wird sie zerstreuen\\%EG Der Herr (?)
  in einem Augenblick.\\
  Er kommt, er kommt, ein König,\\
  dem wahrlich alle Feind'\\
  auf Erden viel zu wenig\\
  zum Widerstande seind.

  \flagverse{10.} Er kommt zum Weltgerichte,\\
  zum Fluch dem, der ihm flucht,\\
  mit Gnad' und süßem Lichte\\
  dem, der ihn liebt und sucht.\\
  Ach komm, ach komm, o Sonne\\%im EG nach Sonne ein Komma (?)
  und hol uns allzumahl\\
  zum ew'gen Licht und Wonne\\
  in deinen Freudensaal.

\end{verse}
\end{multicols}
\attrib{\small{Paul Gerhard 1653}}
\end{document}                                                                                                          
  
