%StartInfo%%%%%%%%%%%%%%%%%%%%%%%%%%%%%%%%%%%%%%%%%%%%%%%%%%%%%%%%%%%%%%%%%%%%
%  Autor:
%  Titel:
%  File:
%  Ref:
%  Mod:
%EndInfo%%%%%%%%%%%%%%%%%%%%%%%%%%%%%%%%%%%%%%%%%%%%%%%%%%%%%%%%%%%%%%%%%%%%%%
%\poemtitle{pt}
\begin{multicols}{2}
\settowidth{\versewidth}{Herr, mein Gott, da ich Kranker}
\begin{verse}[\versewidth]
%der 30. Psalm\\
%ich preise dich und singe, Herr

\flagverse{1.} Ich preise dich und singe,\\
Herr, deine Wundergnad,\\
die mir so große Dinge\\
bisher erwiesen hat;\\
denn das ist meine Pflicht,\\
in meinem ganzen Leben\\
dir Lob und Dank zu geben,\\
mehr hab und kann ich nicht.

\flagverse{2.} Du hast mein Herz erhöhet\\
aus mancher tiefen Not,\\
den aber, der da gehet\\
und suchet meinen Tod\\
und tut mir Herzleid an,\\
den hast du weggeschlagen,\\
daß er sich meiner Plagen\\
mit nicht erfreuen kann.

\flagverse{3.} Herr, mein Gott, da ich Kranker\\
vom Bette zu dir schrei,\\
da ward mein Heil mein Anker\\
und stund mir treulich bei;\\
da andre fuhren hin\\
zur finstern Todeshöhle,\\
da hieltst du meine Seele\\
und mich noch, wo ich bin.

\flagverse{4.} Ihr Heiligen, lobsinget\\
und danket eurem Herrn,\\
der, wenn die Not herdringet,\\
bald hört und herzlich gern\\
uns Gnad und Hilfe gibt;\\
rühmt den, des Hand uns träget\\
und, wenn er uns ja schläget,\\
nicht allzusehr betrübt.

\flagverse{5.} Gott hat ja Vaterhände\\
und strafet mit Geduld,\\
sein Zorn nimmt bald ein Ende,\\
sein Herz ist voller Huld\\
und gönnt uns lauter Guts.\\
Den Abend währt das Weinen,\\
des Morgens macht das Scheinen\\
der Sonn uns gutes Muts.

\flagverse{6.} Ich sprach zur guten Stunde,\\
da mirs noch wohl erging:\\
Ich steh auf festem Grunde,\\
acht alles Kreuz gering;\\
ich werde nimmermehr,\\
das weiß ich, niederliegen;\\
denn Gott der kann nicht trügen,\\
der liebt mich gar zu sehr.

\flagverse{7.} Als aber dein Gesichte,\\
ach Gott, sich von mir wandt,\\
da war mein Trost zunichte,\\
da lag mein Heldenstand;\\
es war mir angst und bang,\\
ich führte schwere Klagen\\
mit Zittern und mit Zagen:\\
Herr, mein Gott, wie so lang?

\flagverse{8.} Hast du dir vorgenommen,\\
mein ewger Feind zu sein?\\
Was werden dir denn frommen\\
die ausgedorrten Bein\\
und der elende Staub,\\
zu welchem in der Erden\\
wir werden, wenn wir werden\\
des blassen Todes Raub?

\flagverse{9.} So lang ichs Leben habe,\\
lobsing ich deiner Ehr,\\
dort aber in dem Grabe,\\
gedenk ich dein nicht mehr;\\
drum eil und hilf mir auf\\
und gib mir Kraft und Leben;\\
dafür will ich dir geben\\
meins ganzen Lebens Lauf.

\flagverse{10.} Nun wohl, ich bin erhöret,\\
mein Seufzen ist erfüllt,\\
mein Kreuz ist umgekehret,\\
mein Herzleid ist gestillt,\\
mein Grämen hat ein End;\\
es ist von meinem Herzen\\
der bittern Sorgen Schmerzen\\
durch dich, Herr, abgewendt.

\flagverse{11.} Du hast mit mir gehandelt\\
noch besser, als ich will;\\
mein Klagen ist verwandelt\\
in eines Reigens Spiel,\\
und für das Trauerkleid,\\
in dem ich vor gestöhnet,\\
da hast du mich gekrönet\\
mit süßer Lust und Freud.

\flagverse{12.} Auf daß zu deiner Ehre\\
mein Ehre sich erhüb\\
und nimmer stille wäre,\\
bis daß ich deine Lieb\\
und ungezählte Zahl\\
der großen Wunderdinge\\
mit ewgen Freuden singe\\
im güldnen Himmelssaal.
   
\end{verse}
\end{multicols}
%\attrib{\small{THZE}}
