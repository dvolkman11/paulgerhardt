%StartInfo%%%%%%%%%%%%%%%%%%%%%%%%%%%%%%%%%%%%%%%%%%%%%%%%%%%%%%%%%%%%%%%%%%%%
%  Autor:
%  Titel:
%  File:
%  Ref:
%  Mod:
%EndInfo%%%%%%%%%%%%%%%%%%%%%%%%%%%%%%%%%%%%%%%%%%%%%%%%%%%%%%%%%%%%%%%%%%%%%%
%\poemtitle{pt}
\begin{multicols}{2}
\settowidth{\versewidth}{dem nicht sein Angst, sein Schmerz und Weh}
\begin{verse}[\versewidth]

\flagverse{1.} Ich hab oft bei mir selbst gedacht,\\
wann ich den Lauf der Welt betracht,\\
ob auch das Leben dieser Erd\\
uns gut sei und des Wünschens wert,\\
und ob nicht der viel besser tu,\\
der sich fein zeitlich legt zur Ruh.

\flagverse{2.} Denn, Lieber, denk und sage mir:\\
Was für ein Stand ist wohl allhier,\\
dem nicht sein Angst, sein Schmerz und Weh\\
alltäglich überm Haupte steh?\\
Ist auch ein Ort, der Kummers frei\\
und ohne Klag und Sorgen sei?

\flagverse{3.} Sieh unsers ganzen Lebens Lauf:\\
Ist auch ein Tag von Jugend auf,\\
der nicht sein eigne Qual und Plag\\
auf seinem Rücken mit sich trag?\\
Ist nicht die Freude, die uns stillt,\\
auch selbst mit Jammer überfüllt?

\flagverse{4.} Hat einer Glück und gute Zeit,\\
hilf Gott, wie tobt und zürnt der Neid!\\
Hat einer Ehr und große Würd,\\
ach, mit was großer Last und Bürd\\
ist, der vor andern ist geehrt,\\
vor andern auch dabei beschwert!

\flagverse{5.} Ist einer heute gutes Muts,\\
ergötzt und freut sich seines Guts:\\
Eh ers vermeint, fährt sein Gewinn\\
zusamt dem guten Mute hin!\\
Wie plötzlich kommt ein Ungestüm\\
und wirft die großen Güter üm!

\flagverse{6.} Bist du denn fromm und fleuchst die Welt\\
und liebst Gott mehr als Gold und Geld,\\
so wird dein Ruhm, dein Schmuck und Kron\\
in aller Welt zu Spott und Hohn;\\
denn wer der Welt nicht heucheln kann,\\
den sieht die Welt für albern an.

\flagverse{7.} Nun, es ist wahr, es steht uns hier\\
die Trübsal täglich vor der Tür,\\
und findt ein jeder überall\\
des Kreuzes Not und bittre Gall.\\
Sollt aber drum der Christen Licht\\
ganz nichts mehr sein? Das glaub ich nicht.

\flagverse{8.} Ein Christe, der an Christo klebt,\\
und stets im Geist und Glauben lebt,\\
dem kann kein Unglück, keine Pein\\
im ganzen Leben schädlich sein;\\
gehts ihm nicht allzeit wie es soll,\\
so ist ihm dennoch allzeit wohl.

\flagverse{9.} Hat er nicht Gold, so hat er Gott,\\
fragt nicht nach böser Leute Spott,\\
verwirft mit Freuden und verlacht\\
der Welt verkehrten Stolz und Pracht.\\
Sein Ehr ist Hoffnung und Geduld,\\
sein Hoheit ist des Höchsten Huld.

\flagverse{10.} Es weiß ein Christ und bleibt dabei,\\
daß Gott sein Freund und Vater sei;\\
er hau, er brenn, er stech, er schneid,\\
hier ist nichts, das uns von ihm scheid,\\
je mehr er schlägt, je mehr er liebt,\\
bleibt fromm, ob er uns gleich betrübt.

\flagverse{11.} Laß alles fallen, wie es fällt:\\
Wer Christi Lieb im Herzen hält,\\
der ist ein Held und bleibt bestehn,\\
wann Erd und Himmel untergehn;\\
und wann ihn alle Welt verläßt,\\
hält Gottes Wort ihn steif und fest.

\flagverse{12.} Des Höchsten Wort dämpft alles Leid\\
und kehrts in lauter Lust und Freud;\\
es nimmt dem Unglück alles Gift,\\
daß, obs uns gleich verfolgt und trifft,\\
es dennoch unser Herze nie\\
in allzu große Trauer zieh.

\flagverse{13.} Ei nun, so mäßge deine Klag!\\
Ist dieses Leben voller Plag,\\
ists dennoch an der Christen Teil\\
auch voller Gottes Schutz und Heil.\\
Wer Gott vertraut und Christum ehrt,\\
der bleibt im Kreuz auch unversehrt.

\flagverse{14.} Gleichwie das Gold durchs Feuer geht\\
und in dem Ofen wohl besteht,\\
so bleibt ein Christ durch Gottes Gnad\\
im Elendsofen ohne Schad;\\
ein Kind bleibt seines Vaters Kind,\\
obs gleich des Vaters Zucht empfindt.

\flagverse{15.} Drum, liebes Herz, sei ohne Scheu\\
und sieh auf deines Vaters Treu!\\
Empfindst du auch hier seine Rut,\\
er meints nicht bös, es ist dir gut!\\
Gib dich getrost in seine Händ,\\
es nimmt zuletzt ein gutes End.

\flagverse{16.} Leb immerhin, so lang er will!\\
Ists Leben schwer, so sei du still,\\
es geht zuletzt in Freuden aus:\\
Im Himmel ist ein schönes Haus,\\
da, wer nach Christo hier gestrebt,\\
mit Christi Engeln ewig lebt!

\end{verse}
\end{multicols}
%\attrib{\small{THZE}}
