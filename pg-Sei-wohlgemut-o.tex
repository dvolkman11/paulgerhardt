%StartInfo%%%%%%%%%%%%%%%%%%%%%%%%%%%%%%%%%%%%%%%%%%%%%%%%%%%%%%%%%%%%%%%%%%%%
%  Autor:
%  Titel:
%  File:
%  Ref:
%  Mod:
%EndInfo%%%%%%%%%%%%%%%%%%%%%%%%%%%%%%%%%%%%%%%%%%%%%%%%%%%%%%%%%%%%%%%%%%%%%%
%\poemtitle{pt}
\begin{multicols}{2}
\settowidth{\versewidth}{ein frommer Mensch, der nicht geschwebt}
\begin{verse}[\versewidth]
%der 73. Psalm\\
%sei wohlgemut, o Christenseel

\flagverse{1.} Sei wohlgemut, o Christenseel,\\
im Hochmut deiner Feinde;\\
es hat das rechte Israel\\
noch dennoch Gott zum Freunde,\\
wer glaubt und hofft, der wird geliebt\\
von dem, der unsern Herzen gibt\\
Trost, Friede, Freud und Leben.

\flagverse{2.} Zwar tut es weh und ärgert sehr,\\
wenn man vor Augen siehet,\\
wie dieser Welt gottloses Heer\\
so schön und herrlich blühet;\\
sie sind in keiner Todesfahr,\\
erleben hier so manches Jahr\\
und stehen wie Paläste.

\flagverse{3.} Sie haben Glück und wissen nicht,\\
wie Armen sei zu Mute;\\
Gold ist ihr Gott, Geld ist ihr Licht.\\
Sind stolz bei großem Gute;\\
sie reden hoch, und das gilt schlecht:\\
Was andre sagen, ist nicht recht,\\
es ist ihn'n viel zu wenig.

\flagverse{4.} Des Pöbelvolks unweiser Hauf\\
ist auch auf Ihrer Seite;\\
sie sperren Maul und Nasen auf\\
und sprechen: Das sind Leute!\\
Das sind ohn allen Zweifel die,\\
die Gott vor allen andern hie\\
zu Kindern auserkoren.

\flagverse{5.} Was sollte doch der große Gott\\
nach jenen andern fragen,\\
die sich mit Armut, Kreuz und Not\\
bis in die Grube tragen?\\
Wem hier des Glückes Gunst und Schein\\
nicht leuchtet, kann kein Christe sein,\\
er ist gewiß verstoßen.

\flagverse{6.} Solls denn, mein Gott, vergebens sein\\
daß dich mein Herze liebet?\\
Ich liebe dich und leide Pein,\\
bin dein und doch betrübet.\\
Ich hätte bald auch so gedacht\\
wie jene Rotte, die nichts acht't\\
als was vor Augen pranget.

\flagverse{7.} Sieh aber, sieh, in solchem Sinn\\
wär ich zu weit gekommen,\\
ich hätte bloß verdammt dahin\\
die ganze Schar der Frommen;\\
denn hat auch je einmal gelebt\\
ein frommer Mensch, der nicht geschwebt\\
in großem Kreuz und Leiden?

\flagverse{8.} Ich dachte hin, ich dachte her,\\
ob ich es möcht ergründen,\\
es war mir aber viel zu schwer,\\
den rechten Schluß zu finden,\\
bis daß ich ging ins Heiligtum\\
und merkte, wie du, unser Ruhm,\\
die Bösen führst zu Ende.

\flagverse{9.} Ihr Gang ist schlüpfrig, glatt ihr Pfad,\\
ihr Tritt ist ungewisse;\\
du suchst sie heim nach ihrer Tat\\
und stürzest ihre Füße.\\
Im Hui ist alles umgewendt,\\
da nehmen sie ein plötzlich End\\
und fahren hin mit Schrecken.

\flagverse{10.} Heut grünen sie gleich wie ein Baum,\\
ihr Herz ist froh und lachet,\\
und morgen sind sie wie ein Traum,\\
von dem der Mensch aufwachet,\\
ein bloßer Schatt, ein totes Bild,\\
das weder Hand noch Augen füllt,\\
verschwindt im Augenblicke.

\flagverse{11.} Es mag drum sein; es wäre gleich\\
mein Kreuz so lang ich lebe,\\
ich habe gnug am Himmelreich,\\
dahin ich täglich strebe.\\
Hält mich die Welt gleich als ein Tier,\\
ei, lebst du, Gott, doch über mir,\\
du bist mein Ehr und Krone.

\flagverse{12.} Du heilest meines Herzens Stich\\
mit deiner süßen Liebe\\
und wehrst dem Unglück, daß es mich\\
nicht allzu hoch betrübe;\\
du leitest mich mit deiner Hand\\
und wirst mich endlich in den Stand\\
der rechten Ehren setzen.

\flagverse{13.} Wenn ich nur dich, o starker Held,\\
behalt in meinem Leide,\\
so acht ichs nicht, wenn gleich zerfällt\\
das große Weltgebäude.\\
Du bist mein Himmel, und dein Schoß\\
bleibt allezeit mein Burg und Schloß,\\
wann diese Erd entweichet.

\flagverse{14.} Wann mir gleich Leib und Seel verschmacht,\\
so kann ich doch nicht sterben,\\
denn du bist meines Lebens Macht\\
und läßt mich nicht verderben.\\
Was frag ich nach dem Erb und Teil\\
auf dieser Welt? Du, du, mein Heil,\\
du bist mein Teil und Erbe.

\flagverse{15.} Das kann die gottvergessne Rott\\
mit Wahrheit nimmer sagen;\\
sie weicht von dir und wird zum Spott,\\
verdirbt in großen Plagen.\\
Mir aber ists, wie dir bewußt,\\
die größte Freud und höchste Lust,\\
daß ich mich zu dir halte.

\flagverse{16.} So will ich nun die Zuversicht\\
auf dich beständig setzen,\\
es werde mich dein Angesicht\\
zu rechter Zeit ergötzen.\\
Indessen will ich stille ruhn\\
und deiner weisen Hände Tun\\
mit meinem Munde preisen.

\end{verse}
\end{multicols}
%\attrib{\small{THZE}}
