%StartInfo%%%%%%%%%%%%%%%%%%%%%%%%%%%%%%%%%%%%%%%%%%%%%%%%%%%%%%%%%%%%%%%%%%%%
%  Autor:
%  Titel:
%  File:
%  Ref:
%  Mod:
%EndInfo%%%%%%%%%%%%%%%%%%%%%%%%%%%%%%%%%%%%%%%%%%%%%%%%%%%%%%%%%%%%%%%%%%%%%%
%\poemtitle{pt}
\begin{multicols}{2}
\settowidth{\versewidth}{Herr, ich bin nichts! Du aber bist}
\begin{verse}[\versewidth]
%wie ist es möglich, höchstes Licht?

\flagverse{1.} Wie ist es möglich, höchstes Licht,\\
daß, weil vor deinem Angesicht\\
doch alles muß erblassen,\\
ich und mein armes Fleisch und Blut\\
dir zu entgegen eingen Mut\\
und Herze sollten fassen?

\flagverse{2.} Was bin ich mehr als Erd und Staub?\\
Was ist mein Leib als Gras und Laub?\\
Was taugt mein ganzes Leben?\\
Was kann ich, wenn ich alles kann?\\
Was hab und trag ich um und an,\\
als was du mir gegeben?

\flagverse{3.} Ich bin ein arme Mad und Wurm,\\
ein Strohhalm, den ein kleiner Sturm\\
gar leichtlich hin kann treiben;\\
wenn deine Hand, die alles trägt,\\
mich nur ein wenig trifft und schlägt,\\
so weiß ich nicht zu bleiben.

\flagverse{4.} Herr, ich bin nichts! Du aber bist\\
der Mann, der alles hat und ist,\\
in dir steht all mein Wesen;\\
wo du mit deiner Hand mich schreckst,\\
und nicht mit Huld und Gnaden deckst,\\
so mag ich nicht genesen.

\flagverse{5.} Du bist getreu, ich ungerecht,\\
du fromm, ich gar ein böser Knecht\\
und muß mich wahrlich schämen,\\
daß ich bei solchem schnöden Stand\\
aus deiner milden Vaterhand\\
ein einzges Gut sollt nehmen.

\flagverse{6.} Ich habe dir von Jugend an\\
nichts andres als Verdruß getan,\\
bin Sünden voll geboren;\\
und wo du nicht durch deine Treu\\
mich wieder machest los und frei,\\
so wär ich gar verloren.

\flagverse{7.} Drum sei das Rühmen fern von mir,\\
was dir gebührt, das geb ich dir,\\
du bist allein zu ehren.\\
Ach laß, Herr Jesu, meinen Geist\\
und was aus meinem Geiste fleußt,\\
zu dir sich allzeit kehren!

\flagverse{8.} Auch wenn ich gleich was wohl gemacht,\\
so hab ichs doch nicht selbst verbracht,\\
aus dir ist es entsprungen;\\
dir sei auch dafür Ehr und Dank,\\
mein Heiland, all mein Leben lang\\
und Lob und Preis gesungen.

\end{verse}
\end{multicols}
%\attrib{\small{THZE}}
