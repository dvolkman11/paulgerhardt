%StartInfo%%%%%%%%%%%%%%%%%%%%%%%%%%%%%%%%%%%%%%%%%%%%%%%%%%%%%%%%%%%%%%%%%%%%
%  Autor:
%  Titel:
%  File:
%  Ref:
%  Mod:
%EndInfo%%%%%%%%%%%%%%%%%%%%%%%%%%%%%%%%%%%%%%%%%%%%%%%%%%%%%%%%%%%%%%%%%%%%%%
%\poemtitle{Du bist ein Mensch, das weißt du wohl}
\begin{multicols}{2}
\settowidth{\versewidth}{Du bist ein Mensch, das weißt du wohl,}
\begin{verse}[\versewidth]

\flagverse{1.} Du bist ein Mensch, das weißt du wohl,\\
was strebst du denn nach Dingen,\\
die Gott, der Höchst, alleine soll\\
und kann zu Werke bringen?\\
Du fährst mit deinem Witz und Sinn\\
durch so viel tausend Sorgen hin\\
und denkst: Wie wills auf Erden\\
doch endlich mit mir werden?

\flagverse{2.} Es ist umsonst. Du wirst fürwahr\\
mit allem deinem Dichten\\
auch nicht ein einzges kleinstes Haar\\
in aller Welt ausrichten,\\
und dient dein Gram sonst nirgend zu,\\
als daß du dich aus deiner Ruh\\
in Angst und Schmerzen stürzest\\
und selbst das Leben kürzest.

\flagverse{3.} Willst du was tun, was Gott gefällt\\
und dir zum Heil gedeihet,\\
so wirf dein Sorgen auf den Held,\\
den Erd und Himmel scheuet,\\
und gib dein Leben, Tun und Stand\\
nur fröhlich hin in Gottes Hand,\\
so wird er deinen Sachen\\
ein fröhlich Ende machen.

\flagverse{4.} Wer, hat gesorgt, da deine Seel\\
im Anfang deiner Tage\\
noch in der Mutterleibeshöhl\\
und finsterm Kerker lage?\\
Wer hat allda dein Heil bedacht?\\
Was tat da aller Menschen Macht,\\
da Geist und Sinn und Leben\\
dir ward ins Herz gegeben?

\flagverse{5.} Durch wessen Kunst steht dein Gebein\\
in ordentlicher Fülle?\\
Wer gab den Augen Licht und Schein,\\
dem Leibe Haut und Hülle?\\
Wer zog die Adern hie und dort\\
ein jed an ihre Stell und Ort?\\
Wer setzte hin und wieder\\
so viel und schöne Glieder?

\flagverse{6.} Wo war dein Herz, Will und Verstand,\\
da sich des Himmels Decken\\
erstreckten über See und Land\\
und aller Erden Ecken?\\
Wer brachte Sonn und Mond herfür?\\
Wer machte Kräuter, Bäum und Tier\\
und hieß sie deinen Willen\\
und Herzenslust erfüllen?

\flagverse{7.} Heb auf dein Haupt, schau überall\\
hier unten und dort oben,\\
wie Gottes Sorg auf allen Fall\\
vor dir sich hab erhoben:\\
Dein Brot, dein Wasser und dein Kleid\\
war eher noch als du bereit,\\
die Milch, die du erst nahmest,\\
war auch schon, da du kamest.

\flagverse{8.} Die Windeln, die dich allgemach\\
umfingen in der Wiegen,\\
dein Bettlein, Kammer, Stub und Dach\\
und wo du solltest liegen,\\
das war ja alles zugericht't,\\
eh als dein Aug und Angesicht\\
eröffnet ward und sahe,\\
was in der Welt geschahe.

\flagverse{9.} Noch dennoch soll dein Angesicht\\
dein ganzes Leben führen;\\
du traust und glaubest weiter nicht,\\
als was dein Augen spüren;\\
was du beginnst, da soll allein\\
dein Kopf dein Licht und Meister sein,\\
was der nicht auserkoren,\\
das hältst du als verloren.

\flagverse{10.} Nun siehe doch, wie viel und oft\\
ist schändlich umgeschlagen,\\
was du gewiß und fest gehofft\\
mit Händen zu erjagen.\\
Hingegen, wie so manchesmal\\
ist das geschehn, das überall\\
kein Mensch, kein Rat, kein Sinnen\\
ihm hat ersinnen können!

\flagverse{11.} Wie oft bist du in große Not\\
durch eignen Willen kommen,\\
da dein verblendter Sinn den Tod\\
fürs Leben angenommen;\\
und hätte Gott dein Werk und Tat\\
ergehen lassen nach dem Rat,\\
in dem du's angefangen,\\
du wärst zugrunde gangen.

\flagverse{12.} Der aber, der uns ewig liebt,\\
macht gut, was wir verwirren,\\
erfreut, wo wir uns selbst betrübt,\\
und führt uns, wo wir irren;\\
und dazu treibt ihn sein Gemüt\\
und die so reine Vatergüt,\\
in der uns arme Sünder\\
er trägt als seine Kinder.

\flagverse{13.} Ach, wie so oftmals schweigt er still\\
und tut doch, was uns nützet,\\
da unterdessen unser Will\\
und Herz in Ängsten sitzet,\\
sucht hier und da und findet nichts,\\
will sehn und mangelt doch des Lichts,\\
will aus der Angst sich winden\\
und kann den Weg nicht finden.

\flagverse{14.} Gott aber geht gerade fort\\
auf seinen weisen Wegen,\\
er geht und bringt uns an den Ort,\\
da Wind und Sturm sich legen.\\
Hernachmals, wann das Werk geschehn,\\
so kann alsdann der Mensche sehn,\\
was der, so ihn regieret,\\
in seinem Rat geführet.

\flagverse{15.} Drum, liebes Herz, sei wohlgemut\\
und laß von Sorg und Grämen!\\
Gott hat ein Herz, das nimmer ruht,\\
dein Bestes fürzunehmen.\\
Er kanns nicht lassen, glaube mir,\\
sein Eingeweid ist gegen dir\\
und uns hier allzusammen\\
voll allzu süßer Flammen.

\flagverse{16.} Er hitzt und brennt für Gnad und Treu,\\
und also kannst du denken,\\
wie seinem Mut zu Mute sei,\\
wenn wir uns oftmals kränken\\
mit so vergebner Sorgenbürd,\\
als ob er uns nun gänzlich würd\\
aus lauter Zorn und Hassen\\
ganz hilf- und trostlos lassen.

\flagverse{17.} Das schlag hinweg und laß dich nicht\\
so liederlich betören;\\
obgleich nicht allzeit das geschicht,\\
was Freude kann vermehren,\\
so wird doch wahrlich das geschehn,\\
was Gott dein Vater ausersehn;\\
was er dir zu will kehren,\\
das wird kein Mensche wehren.

\flagverse{18.} Tu als sein Kind und lege dich\\
in deines Vaters Arme,\\
bitt ihn und flehe, bis er sich\\
dein, wie er pflegt, erbarme:\\
So wird er dich durch seinen Geist\\
auf Wegen, die du jetzt nicht weißt,\\
nach wohlgehaltnem Ringen\\
aus allen Sorgen bringen.

\end{verse}
\end{multicols}
%\attrib{\small{THZE}}
