%StartInfo%%%%%%%%%%%%%%%%%%%%%%%%%%%%%%%%%%%%%%%%%%%%%%%%%%%%%%%%%%%%%%%%%%%%
%  Autor:
%  Titel:
%  File:
%  Ref:
%  Mod:
%EndInfo%%%%%%%%%%%%%%%%%%%%%%%%%%%%%%%%%%%%%%%%%%%%%%%%%%%%%%%%%%%%%%%%%%%%%%
%\poemtitle{pt}
\begin{multicols}{2}

\settowidth{\versewidth}{Und soll kein Kreuz, kein Schmerz, kein Leiden,}
\begin{verse}[\versewidth]
%wer selig stirbt, stirbt nicht\\
%auf den Tod des Rats Joh. Adam Preunel, gestorben in Berlin 1668, dessen letztes Wort war: Ego sum in Christo, et Christus est in me

\flagverse{1.} Wer selig stirbt, stirbt nicht!\\
Ein guter Tod gedeiht zum Leben\\
und macht die Seel in Freuden schweben\\
für Gottes Angesicht.\\
Laß alles fallen und vergehen,\\
wer Christo stirbt, bleibt ewig stehen.

\flagverse{2.} Da fehlts oft vielen an;\\
Herrn Preunel aber ists gelungen,\\
der hat mit Christo durchgedrungen,\\
ist nun sehr herrlich dran.\\
In Christo, sprach er, sei mein Ende,\\
dem geb ich mich in seine Hände.

\flagverse{3.} Herr Jesu, du bist mein!\\
Du hast dich selber mir geschenket.\\
Auch bin ich dir ganz eingesenket\\
und leb und sterbe dein.\\
Und soll kein Kreuz, kein Schmerz, kein Leiden,\\
ja uns soll auch der Tod nicht scheiden.

\flagverse{4.} Und damit ging er hin!\\
Heißt das nun nicht recht selig sterben?\\
Wer kann doch immermehr verderben\\
bei so gestaltem Sinn?\\
Wer hier in Christo wohl gewesen,\\
wird dort bei Christo wohl genesen.

\end{verse}
\end{multicols}

\begin{center}
\settowidth{\versewidth}{Der, vor dem die Welt erschrickt,}
\begin{verse}[\versewidth]

\flagverse{5.} Drum weinet nicht zu viel,\\
ihr, die Herr Preunel hat geliebet;\\
denn der, an dem ihr euch betrübet,\\
hat sein erwünschtes Ziel.\\
Laßt vielmehr diesen Seufzer hören:\\
Gott woll auch uns so sterben lehren!
  
\end{verse}
\end{center}



%\attrib{\small{THZE}}
