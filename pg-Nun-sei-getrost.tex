%StartInfo%%%%%%%%%%%%%%%%%%%%%%%%%%%%%%%%%%%%%%%%%%%%%%%%%%%%%%%%%%%%%%%%%%%%
%  Autor:
%  Titel:
%  File:
%  Ref:
%  Mod:
%EndInfo%%%%%%%%%%%%%%%%%%%%%%%%%%%%%%%%%%%%%%%%%%%%%%%%%%%%%%%%%%%%%%%%%%%%%%
%\poemtitle{pt}
\begin{multicols}{2}
\settowidth{\versewidth}{Erschrecke nicht vor deinem End,}
\begin{verse}[\versewidth]
%nun sei getrost und unbetrübt\\
%auf den Tod der Regina Leyser, geb. Calow, in Wittenberg (1664)

\flagverse{1.} Nun sei getrost und unbetrübt,\\
du mein Geist und Gemüte!\\
Dein Jesus lebt, der dich geliebt\\
eh, als dir dein Geblüte\\
und Fleisch und Haut ward zugericht;\\
der wird dich auch gewißlich nicht\\
an deinem Ende hassen.

\flagverse{2.} Erschrecke nicht vor deinem End,\\
es ist nichts Böses drinnen;\\
dein lieber Herr streckt seine Händ\\
und fordert dich von hinnen\\
aus soviel tausend Angst und Qual,\\
die du in diesem Jammertal\\
bisher hast ausgestanden.

\flagverse{3.} Zwar heißts ja Tod und Sterbensnot,\\
doch ist da gar kein Sterben;\\
denn Jesus ist des Todes Tod\\
und nimmt ihm das Verderben,\\
daß alle seine Stärk und Kraft\\
mir, wenn ich jetzt werd hingerafft,\\
nicht auf ein Härlein schade.

\flagverse{4.} Des Todes Kraft steht in der Sünd\\
und schnöden Missetaten,\\
darin ich armes Adamskind\\
so oft und viel geraten;\\
nun ist die Sünd in Jesu Blut\\
ersäuft, erstickt, getilgt und tut\\
fort gar nichts mehr zur Sachen.

\flagverse{5.} Die Sünd ist hin und ich bin rein;\\
trotz dem, der mir das nehme!\\
Hinfüro ist das Leben mein,\\
darf nicht, daß ich mich gräme\\
um einger Sünden Lohn und Sold;\\
wer ausgesöhnt, dem ist man hold\\
und tut ihm nichts zuwider.

\flagverse{6.} Ei nun, so nehm ich Gottes Gnad\\
und alle seine Freude\\
mit mir auf meinen letzten Pfad\\
und weiß von keinem Leide.\\
Der wilde Feind muß nur ein Schaf,\\
sein Ungestüm ein süßer Schlaf\\
und sanfte Ruhe werden.

\flagverse{7.} Du Jesu, allerliebster Freund,\\
bist selbst mein Licht und Leben:\\
Du hältst mich fest, und kann kein Feind\\
dich, wo du stehest, heben.\\
In dir steh ich, und du in mir;\\
und wie wir stehn, so bleiben wir\\
hier und dort ungeschieden.

\flagverse{8.} Mein Leib, der legt sich hin zur Ruh,\\
als der fast müde worden;\\
die Seele fährt dem Himmel zu\\
und mischt sich in den Orden\\
der auserwählten Gottesschar\\
und hält das ewge Jubeljahr\\
mit allen heilgen Engeln.

\flagverse{9.} Kommt dann der Tag, o höchster Fürst\\
der Kleinen und der Großen,\\
da du zum allerletzten wirst\\
in die Posaunen stoßen,\\
so soll denn Seel und Leib zugleich\\
mit dir in deines Vaters Reich\\
zu deiner Freud eingehen.

\flagverse{10.} Ists nun dein Will, so stell dich ein,\\
mich selig zu versetzen.\\
Ach, ewig bei und mit dir sein,\\
wie hoch muß das ergötzen!\\
Eröffne dich, du Todespfort,\\
auf daß an solchen schönen Ort\\
ich durch dich möge fahren!

\end{verse}
\end{multicols}
%\attrib{\small{THZE}}
