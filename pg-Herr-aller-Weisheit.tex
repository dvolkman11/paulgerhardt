%StartInfo%%%%%%%%%%%%%%%%%%%%%%%%%%%%%%%%%%%%%%%%%%%%%%%%%%%%%%%%%%%%%%%%%%%%
%  Autor:
%  Titel:
%  File:
%  Ref:
%  Mod:
%EndInfo%%%%%%%%%%%%%%%%%%%%%%%%%%%%%%%%%%%%%%%%%%%%%%%%%%%%%%%%%%%%%%%%%%%%%%
%\poemtitle{Herr, aller Weisheit Quell und Grund}
\begin{multicols}{2}
\settowidth{\versewidth}{Ja, Herr, ich bin gar viel zu schlecht,}
\begin{verse}[\versewidth]
%Herr, aller Weisheit Quell und Grund\\
%(in Anlehnung an Sprüche Sal. 7-9)\\
%nach Johann Arnds »Paradiesgärtlein«, Goslar 1621, I, 14

\flagverse{1.} Herr, aller Weisheit Quell und Grund,\\
dir ist all mein Vermögen kund,\\
wo du nicht hilfst und deine Gunst,\\
ist all mein Tun und Werk umsunst.

\flagverse{2.} Ich leider als ein Sündenkind\\
bin von Natur zum Guten blind,\\
mein Herze, wann dirs dienen soll,\\
ist ungeschickt und Torheit voll.

\flagverse{3.} Ja, Herr, ich bin gar viel zu schlecht,\\
zu handeln dein Gesetz und Recht,\\
was meinem Nächsten nütz im Land,\\
ist mir verdeckt und unbekannt.

\flagverse{4.} Mein Leben ist sehr kurz und schwach,\\
ein Lüftlein, das bald lässet nach;\\
was in der Welt zu prangen pflegt,\\
das ist mir wenig beigelegt.

\flagverse{5.} Wann ich auch gleich vollkommen wär,\\
hätt aller Gaben Ruhm und Ehr\\
und sollt entraten deines Lichts,\\
so wär ich doch ein lauter Nichts.

\flagverse{6.} Was hilfts, wann einer gleich viel weiß,\\
und hat zuvorderst nicht mit Fleiß\\
gelernet deine Furcht und Dienst,\\
der hat mehr Schaden als Gewinst.

\flagverse{7.} Das Wissen, das ein Mensche führt,\\
wird leichthin in ihm selbst verirrt;\\
wann unsre Kunst am meisten kann,\\
so stößt sie aller Enden an.

\flagverse{8.} Wie mancher stürzet seine Seel\\
durch Klugheit, wie Ahitophel,\\
und nimmt, weil er dich nicht recht kennt,\\
durch seinen Witz ein schrecklich End!

\flagverse{9.} O Gott, mein Vater, kehre dich\\
zu meiner Bitt und höre mich:\\
Nimm solche Torheit von mir hin\\
und gib mir einen bessern Sinn!

\flagverse{10.} Gib mir die Weisheit, die du liebst\\
und denen, die dich lieben, gibst,\\
die Weisheit, die vor deinem Thron\\
allstets erscheint in ihrer Kron.

\flagverse{11.} Ich lieb ihr liebes Angesicht,\\
sie ist meins Herzens Freud und Licht,\\
sie ist die Schönste, die mich hält\\
und meinen Augen wohlgefällt.

\flagverse{12.} Sie ist hochedel, auserkorn,\\
von dir, o Höchster, selbst geborn,\\
sie ist der hellen Sonnen gleich,\\
an Tugend und an Gaben reich.

\flagverse{13.} Ihr Mund ist süß und tröstet schön,\\
wenn uns die Augen übergehn;\\
wenn uns der Kummer niederdrückt,\\
so ist sies, die das Herz erquickt.

\flagverse{14.} Sie ist voll Ehr und Herrlichkeit,\\
bewehrt vorm Tod und großem Leid;\\
wer fleißig um sie kämpft und wirbt,\\
der bleibet lebend, wenn er stirbt.

\flagverse{15.} Sie ist des Schöpfers nächster Rat,\\
von Worten mächtig und von Tat;\\
durch sie erfährt die blinde Welt,\\
was Gott gedenkt in seinem Zelt.

\flagverse{16.} Denn welcher Mensch weiß Gottes Rat?\\
Wer ists, der je erfunden hat\\
den Schluß, den er im Himmel schleußt,\\
den Weg, den er uns laufen heißt?

\flagverse{17.} Die Seele wohnet in der Erd\\
und wird durch ihre Last beschwert;\\
die Sinnen, hin und her zerstreut,\\
sind ja von Irrtum nicht befreit.

\flagverse{18.} Wer will erforschen, was Gott setzt,\\
und sagen, was sein Herz ergötzt?\\
Es sei denn, der du ewig lebst,\\
daß du uns deine Weisheit gebst.

\flagverse{19.} Drum sende sie von deinem Thron\\
und gib sie deinem Kind und Sohn!\\
Ach, schütt und geuß sie reichlich aus\\
in meines Herzens armes Haus!

\flagverse{20.} Befiehl ihr, daß sie mit mir sei\\
und, wo ich gehe, stehe bei;\\
bin ich in Arbeit, helfe sie\\
mir tragen meine schwere Müh!

\flagverse{21.} Gib mir durch ihre weise Hand\\
die recht Erkenntnis und Verstand,\\
daß ich an dir alleine kleb\\
und nur nach deinem Willen leb!

\flagverse{22.} Gib mir durch sie Geschicklichkeit,\\
zur Wahrheit laß mich sein bereit,\\
daß ich nicht mach aus sauer süß,\\
noch aus dem Lichte Finsternis!

\flagverse{23.} Gib Lieb und Lust zu deinem Wort,\\
hilf, daß ich bleib an meinem Ort\\
und mich zur frommen Schar gesell,\\
in ihrem Rat mein Wesen stell!

\flagverse{24.} Gib auch, daß ich gern jedermann\\
mit Rat und Tat, so gut ich kann,\\
aus rechter unverfälschter Treu\\
zu helfen allzeit willig sei.

\end{verse}
\end{multicols}
%\attrib{\small{THZE}}

\begin{center}
\settowidth{\versewidth}{Auf daß in allem, was ich tu,}
\begin{verse}[\versewidth]

\flagverse{25.} Auf daß in allem, was ich tu,\\
in deiner Lieb ich nehme zu;\\
denn wer sich nicht der Weisheit gibt,\\
der bleibt von dir auch ungeliebt.

  
\end{verse}
\end{center}


%\attrib{\small{THZE}}
