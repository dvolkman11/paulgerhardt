%StartInfo%%%%%%%%%%%%%%%%%%%%%%%%%%%%%%%%%%%%%%%%%%%%%%%%%%%%%%%%%%%%%%%%%%%%
%  Autor:
%  Titel:
%  File:
%  Ref:
%  Mod:
%EndInfo%%%%%%%%%%%%%%%%%%%%%%%%%%%%%%%%%%%%%%%%%%%%%%%%%%%%%%%%%%%%%%%%%%%%%%
%\poemtitle{pt}
\begin{multicols}{2}
\settowidth{\versewidth}{Nichts anders, traun, als daß die Schar}
\begin{verse}[\versewidth]
%o Herrscher in dem Himmelszelt\\
%»Buß- und Betgesang bei unzeitiger Nässe und betrübtem Gewitter«

\flagverse{1.} O Herrscher in dem Himmelszelt,\\
was ist es doch, das unser Feld\\
und was es uns hervorgebracht,\\
so ungestalt und traurig macht?

\flagverse{2.} Nichts anders, traun, als daß die Schar\\
der Menschen sich so ganz und gar\\
bis in den tiefsten Grund verkehrt\\
und täglich ihre Schuld vermehrt.

\flagverse{3.} Die, so, als Gottes Eigentum,\\
stets preisen sollten Gottes Ruhm\\
und lieben seines Wortes Kraft,\\
sind gleich der blinden Heidenschaft.

\flagverse{4.} Drum wird uns auch der Himmel blind,\\
des Firmamentes Glanz verschwind't,\\
wir warten, wann der Tag anbricht,\\
aufs Tageslicht und kommt doch nicht.

\flagverse{5.} Man zankt noch immer fort und fort,\\
es bleibet Krieg an allem Ort,\\
in allen Winkeln Haß und Neid,\\
in allen Ständen Streitigkeit.

\flagverse{6.} Drum strecken auch all Element\\
hier wider uns aus ihre Händ,\\
Angst kommt uns aus der Tief und See\\
Angst kommt uns aus der Luft und Höh.

\flagverse{7.} Es ist ein hochbetrübte Zeit;\\
man plagt und jagt die armen Leut,\\
eh als es Zeit, zur Grube zu\\
und gönnet ihnen keine Ruh.

\flagverse{8.} Drum trauert auch der Freudenquell,\\
die Sonn, und scheint uns nicht so hell;\\
die Wolken gießen allzumal\\
die Tränen ohne Maß und Zahl.

\flagverse{9.} Ach, wein auch du, o Menschenkind,\\
und traure über deine Sünd;\\
halt doch von deinen Lastern ein\\
und mache dich durch Buße rein.

\flagverse{10.} Fall auf die Knie, fall in die Arm\\
des Herrn, daß sich sein Herz erbarm\\
und der so wohl verdienten Rach\\
in Gnaden bald ein Ende mach!

\flagverse{11.} Er ist ja fromm und bleibet fromm,\\
begehrt nichts mehr, als daß man komm\\
und mit geneigter Furcht und Scheu\\
ihn bitt um Gnad und Vatertreu.

\flagverse{12.} Ach Vater, Vater, höre doch\\
und lös uns aus dem Sündenjoch\\
und zeuch uns aus der Welt herfür\\
und kehr uns selbsten du zu dir!

\flagverse{13.} Erweiche unsern harten Mut\\
und mach uns Böse fromm und gut;\\
wen du bekehrst, der wird bekehrt,\\
und wer dich hört, der wird erhört.

\flagverse{14.} Laß deine Augen freundlich sein\\
und nimm mit gnädgen Ohren ein\\
das Angstgeschrei, das von der Erd\\
aus unserm Herzen zu dir fährt.

\flagverse{15.} Reiß weg das schwarze Zorngewand,\\
erquicke uns und unser Land\\
und der so schönen Früchte Kranz\\
mit süßem, warmem Sonnenglanz.

\flagverse{16.} Verleih uns bis in unsern Tod\\
alltäglich unser liebes Brot\\
und dermaleinst nach dieser Zeit\\
das süße Brot der Ewigkeit!

\end{verse}
\end{multicols}
%\attrib{\small{THZE}}
