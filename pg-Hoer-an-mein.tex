%StartInfo%%%%%%%%%%%%%%%%%%%%%%%%%%%%%%%%%%%%%%%%%%%%%%%%%%%%%%%%%%%%%%%%%%%%
%  Autor:
%  Titel:
%  File:
%  Ref:
%  Mod:
%EndInfo%%%%%%%%%%%%%%%%%%%%%%%%%%%%%%%%%%%%%%%%%%%%%%%%%%%%%%%%%%%%%%%%%%%%%%
%\poemtitle{Hör an, mein Herz, die sieben Wort,}
\begin{multicols}{2}
\settowidth{\versewidth}{Hör an, mein Herz, die sieben Wort}
\begin{verse}[\versewidth]
%die sieben Worte Christi am Kreuz\\
%hör an, mein Herz, die sieben Wort

\flagverse{1.} Hör an, mein Herz, die sieben Wort,\\
die Jesus ausgesprochen,\\
da ihm durch Qual und blutgen Mord\\
sein Herz am Kreuz gebrochen.\\
Tu auf den Schrein und schleuß sie ein\\
als edle hohe Gaben,\\
so wirst du Freud in schwerem Leid\\
und Trost im Kreuze haben.

\flagverse{2.} Sein allererste Sorge war,\\
zu schützen, die ihn hassen,\\
bat, daß sein Gott der bösen Schar\\
wollt ihre Sünd erlassen.\\
Vergib, vergib, sprach er aus Lieb,\\
o Vater, ihnen allen!\\
Ihr keiner ist, der säh und wüßt,\\
in was für Tat sie fallen.

\flagverse{3.} Lehrt uns hiemit, wie schön es sei,\\
die lieben, die uns kränken,\\
und ihnen ohne Heuchelei\\
all ihre Fehler schenken.\\
Er zeigt zugleich, wie gnadenreich\\
und fromm sei sein Gemüte,\\
daß auch sein Feind, der's böse meint,\\
bei ihm nichts find als Güte.

\flagverse{4.} Drauf spricht er seine Mutter an,\\
die bei Johanne stunde,\\
tröst't sie am Kreuz, so gut er kann,\\
mit seinem schwachen Munde:\\
Sieh hier dein Sohn! Weib, der wird schon\\
mein Amt bei dir verwalten.\\
Und, Jünger, sieh, hie stehet, die\\
du sollst als Mutter halten.

\flagverse{5.} Ach, treues Herz, so sorgest du\\
für alle deine Frommen.\\
Du siehst und schauest fleißig zu,\\
wie sie in Trübsal kommen,\\
trittst auch mit Rat und treuer Tat\\
zu ihnen auf die Seiten;\\
du bringst sie fort, gibst ihnen Ort\\
und Raum bei guten Leuten.

\flagverse{6.} Die dritte Red hast du getan\\
dem, der dich, Herr, gebeten:\\
Gedenk und nimm dich meiner an,\\
wenn du nun wirst eintreten\\
in deinen Thron und Ehr und Kron\\
als Himmelsfürst aufsetzen!\\
Ich will gewiß im Paradies,\\
sprachst du, dich heut ergötzen.

\vfill\null
\columnbreak


\flagverse{7.} O süßes Wort, o Freudenstimm!\\
Was will uns nun erschrecken?\\
Laß gleich den Tod mit großem Grimm\\
hergehn aus allen Ecken;\\
stürmt er gleich sehr, was kann er mehr,\\
als Leib und Seele scheiden?\\
Indessen schwing ich mich und spring\\
ins Paradies der Freuden.

\flagverse{8.} Nun wohl der Schächer wird mit Freud\\
aus Christi Wort erfüllet,\\
er aber selbst fängt an und schreit,\\
gleich als ein Leue brüllet:\\
Eli, mein Gott! Welch Angst und Not\\
muß ich, dein Kind, ausstehen!\\
Ich ruf, und du schweigst still dazu,\\
läß'st mich zu Grunde gehen.

\flagverse{9.} Nimm dies zur Folge, frommes Kind,\\
wann Gott sich grausam stellet,\\
schau, daß du, wenn sich Trübsal find't,\\
nicht werdest umgefället.\\
Halt steif und fest: Der dich jetzt läßt,\\
wird dich gar bald erfreuen,\\
sei du nur treu und halt dabei\\
stark an mit gläubgem Schreien.

\flagverse{10.} Der Herr fährt fort, ruft laut und hell,\\
klagt, wie ihn heftig dürste:\\
Mich dürstet, sprach der ewge Quell\\
und edle Lebensfürste.\\
Was meint er hier? Er zeiget dir,\\
wie matt er sich getragen\\
an deiner Last, die du ihm hast\\
gemacht in Sündentagen.

\flagverse{11.} Er deutet auch darneben an,\\
wie ihn so hoch verlange,\\
daß dies sein Kreuz bei jedermann\\
frucht bring und wohl verfange.\\
Das merk mit Fleiß, wer sich im Schweiß\\
der Seelenangst muß quälen:\\
Das ewge Licht schleußt keinen nicht\\
vom Teil und Heil der Seelen.

\flagverse{12.} Als nun des Todes finstre Nacht\\
begunnt hereinzudringen,\\
sprach Gottes Sohn: Es ist vollbracht\\
das, was ich soll vollbringen.\\
Was hier und dar die heilge Schar\\
der Väter und Propheten\\
hat aufgesetzt, wie man zuletzt\\
mich kreuzgen würd und töten.

\vfill\null
\columnbreak

\flagverse{13.} Ist's dann vollbracht, was willst du nun\\
dich so vergeblich plagen,\\
als müßt ein Mensch mit seinem Tun\\
die Sündenschuld abtragen?\\
Es ist vollbracht! Das nimm in Acht,\\
du darfst hie nichts zu geben,\\
als daß du gläubst und gläubig bleibst\\
in deinem ganzen Leben.

\flagverse{14.} Nun endlich redt er noch einmal,\\
schreit auf ohn alle Maßen:\\
Mein Vater, nimm in deinen Saal\\
das, was ich jetzt muß lassen:\\
Nimm meinen Geist, der hier sich reißt\\
aus meinem kalten Herzen!\\
Und hiermit wird der große Hirt\\
entbunden aller Schmerzen.

\end{verse}
\end{multicols}


\begin{center}
\settowidth{\versewidth}{Der, vor dem die Welt erschrickt,}
\begin{verse}[\versewidth]
\begin{verbatim}

\end{verbatim}

\flagverse{15.} O wollte Gott, daß ich mein End\\
auch also möchte enden\\
und meinen Geist in Gottes Händ\\
und treuen Schoß hinsenden!\\
Ach laß, mein Hort, dein letztes Wort\\
mein letztes Wort auch werden!\\
So werd ich schön und selig gehn\\
zum Vater von der Erden.
  
  
\end{verse}
\end{center}




%\attrib{\small{THZE}}
