%StartInfo%%%%%%%%%%%%%%%%%%%%%%%%%%%%%%%%%%%%%%%%%%%%%%%%%%%%%%%%%%%%%%%%%%%%
%  Autor:
%  Titel:
%  File:
%  Ref:
%  Mod:
%EndInfo%%%%%%%%%%%%%%%%%%%%%%%%%%%%%%%%%%%%%%%%%%%%%%%%%%%%%%%%%%%%%%%%%%%%%%
%\poemtitle{Ach treuer Gott, barmherzigs Herz}
\begin{multicols}{2}
\settowidth{\versewidth}{Denn das ist allzeit dein Gebrauch:}
\begin{verse}[\versewidth]
%ach treuer Gott, barmherzigs Herz\\
%nach Johann Arnds »Paradiesgärtlein« III, 27

\flagverse{1.} Ach treuer Gott, barmherzigs Herz,\\
des Güte sich nicht endet,\\
ich weiß, daß mir dies Kreuz und Schmerz\\
dein Vaterhand zusendet.\\
Ja, Herr, ich weiß, daß diese Last\\
du mir aus Lieb erteilet hast\\
und gar aus keinem Hasse.

\flagverse{2.} Denn das ist allzeit dein Gebrauch:\\
Wer Kind ist, muß was leiden;\\
und wen du liebst, den stäupst du auch,\\
schickst Trauern vor den Freuden,\\
führst uns zur Höllen, tust uns weh\\
und führst uns wieder in die Höh,\\
und so geht eins ums ander.

\flagverse{3.} Du führst ja wohl recht wunderlich\\
die, so dein Herz ergötzen:\\
Was Leben soll, muß erstlich sich\\
ins Todes Höhle setzen;\\
was steigen soll zur Ehr empor,\\
liegt auf der Erd und muß sich vor\\
im Kot und Staube wälzen.

\flagverse{4.} Das hat, Herr, dein geliebter Sohn\\
selbst wohl erfahrn auf Erden;\\
denn eh er kam zum Ehrenthron,\\
mußt er gekreuzigt werden.\\
Er ging durch Trübsal, Angst und Not,\\
ja durch den herben bittern Tod\\
drang er zur Himmelsfreude.

\flagverse{5.} Hat nun dein Sohn, der fromm und recht,\\
so willig sich ergeben,\\
was will ich armer Sündenknecht\\
dir viel zuwider streben?\\
Er ist der Spiegel der Geduld,\\
und wer sich sehnt nach seiner Huld,\\
der muß ihm endlich werden.

\flagverse{6.} Ach, liebster Vater, wie so schwer\\
ists der Vernunft, zu glauben,\\
daß du demselben, den du sehr\\
schlägst, solltest günstig bleiben!\\
Wie macht doch Kreuz so lange Zeit!\\
Wie schwerlich will sich Lieb und Leid\\
zusammen lassen reimen!

\flagverse{7.} Was ich nicht kann, das gib du mir,\\
o höchstes Gut der Frommen!\\
Gib, daß mir nicht des Glaubens Zier\\
durch Trübsal werd entnommen!\\
Erhalte mich, o starker Hort!\\
Befestge mich in deinem Wort,\\
behüte mich vor Murren!

\flagverse{8.} Bin ich ja schwach, laß deine Treu\\
mir an die Seite treten,\\
hilf, daß ich unverdrossen sei\\
zum Rufen, Seufzen, Beten!\\
So lang ein Herze hofft und gläubt\\
und im Gebet beständig bleibt,\\
so lang ists unbezwungen.

\flagverse{9.} Greif mich auch nicht zu heftig an,\\
damit ich nicht vergehe!\\
Du weißt wohl, was ich tragen kann,\\
wies um mein Leben stehe;\\
ich bin ja weder Stahl noch Stein:\\
Wie balde geht ein Wind herein,\\
so fall ich hin und sterbe.

\flagverse{10.} Ach Jesu, der du worden bist\\
mein Heil mit deinem Blute,\\
du weißt gar wohl, was Kreuze ist\\
und wie dem sei zu Mute,\\
den Kreuz und großes Unglück plagt;\\
drum wirst du, was mein Herze klagt,\\
gar gern zu Herzen fassen.

\flagverse{11.} Ich weiß, du wirst in deinem Sinn\\
mit mir Mitleiden haben\\
und mich, wie ichs jetzt dürftig bin,\\
mit Gnad und Hilfe laben.\\
Ach stärke meine schwache Hand,\\
ach heil und bring in bessern Stand\\
das Straucheln meiner Füße!

\flagverse{12.} Sprich meiner Seel ein Herze zu\\
und tröste mich aufs beste,\\
denn du bist ja der Müden Ruh,\\
der Schwachen Turm und Feste,\\
ein Schatten für der Sonnen Hitz,\\
ein Hütte, da ich sicher sitz\\
in Sturm und Ungewitter.

\flagverse{13.} Und weil ich ja nach deinem Rat\\
hie soll ein wenig leiden,\\
so laß mich auch in deiner Gnad\\
als wie ein Schäflein weiden,\\
daß ich im Glauben die Geduld\\
und durch Geduld die edle Huld\\
nach schwerer Prob erhalte.

\flagverse{14.} O heilger Geist, du Freudenöl,\\
das Gott vom Himmel schicket,\\
erfreue mich, gib meiner Seel\\
was Mark und Bein erquicket!\\
Du bist der Geist der Herrlichkeit,\\
weißt, was für Freud und Seligkeit\\
mein in dem Himmel warte.

\flagverse{15.} Ach laß mich schauen, wie so schön\\
und lieblich sei das Leben,\\
das denen, die durch Trübsal gehn,\\
du dermaleinst wirst geben.\\
Ein Leben, gegen welches hier\\
die ganze Welt mit ihrer Zier\\
durchaus nicht zu vergleichen.

\flagverse{16.} Daselbst wirst du in ewger Lust\\
aufs süß'ste mit mir handeln:\\
Mein Kreuz, das dir und mir bewußt,\\
in Freud und Ehre wandeln;\\
da wird mein Weinen lauter Wein,\\
mein Ächzen lauter Jauchzen sein!\\
Das glaub ich. Hilf mir! Amen.

\end{verse}
\end{multicols}
%\attrib{\small{THZE}}
