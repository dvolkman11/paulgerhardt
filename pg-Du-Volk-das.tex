%StartInfo%%%%%%%%%%%%%%%%%%%%%%%%%%%%%%%%%%%%%%%%%%%%%%%%%%%%%%%%%%%%%%%%%%%%
%  Autor:
%  Titel:
%  File:
%  Ref:
%  Mod:
%EndInfo%%%%%%%%%%%%%%%%%%%%%%%%%%%%%%%%%%%%%%%%%%%%%%%%%%%%%%%%%%%%%%%%%%%%%%
%\poemtitle{Du Volk, das du getaufet bist}
\begin{multicols}{2}
\settowidth{\versewidth}{Dein Leib und Seel war mit der Sünd}
\begin{verse}[\versewidth]
%du Volk, das du getaufet bist – Nimms wohl in acht

\flagverse{1.} Du Volk, das du getaufet bist\\
und deinen Gott erkennest,\\
auch nach dem Namen Jesu Christ\\
dich und die Deinen nennest,\\
nimms wohl in Acht und denke dran,\\
wie viel dir Gutes sei getan\\
am Tage deiner Taufe.

\flagverse{2.} Du warst, noch eh du wurdst geborn\\
und eh du Milch gesogen,\\
verdammt, verstoßen und verlorn,\\
darum daß du gezogen\\
aus deiner Eltern Fleisch und Blut\\
ein Art, die sich vom höchsten Gut,\\
dem ewgen Gott, stets wendet.

\flagverse{3.} Dein Leib und Seel war mit der Sünd\\
als einem Gift durchkrochen,\\
und du warst nicht mehr Gottes Kind,\\
nachdem der Bund gebrochen,\\
den unser Schöpfer aufgericht,\\
da er uns seines Bildes Licht\\
und herrlichs Kleid erteilet.

\flagverse{4.} Der Zorn, der Fluch, der ewge Tod,\\
und was in diesen allen\\
enthalten ist für Angst und Not,\\
das war auf dich gefallen;\\
du warst des Satans Sklav und Knecht,\\
der hielt dich fest nach seinem Recht\\
in seinem Reich gefangen.

\flagverse{5.} Das alles hebt auf einmal auf\\
und schlägt und drückt es nieder\\
das Wasserbad der heilgen Tauf,\\
ersetzt dagegen wieder,\\
was Adam hat verderbt gemacht\\
und was wir selbsten durchgebracht\\
bei unserm bösen Wesen.

\flagverse{6.} Es macht dies Bad von Sünden los\\
und gibt die rechte Schöne.\\
Die Satans Kerker vor beschloß,\\
die werden frei und Söhne\\
des, der da trägt die höchste Kron;\\
der läßt sie, was sein einger Sohn\\
ererbt, auch mit ihm erben.

\flagverse{7.} Was von Natur vermaledeit\\
und mit dem Fluch umfangen,\\
das wird hier in der Tauf erneut,\\
den Segen zu erlangen.\\
Hier stirbt der Tod und würgt nicht mehr,\\
hier bricht die Höll, und all ihr Heer\\
muß uns zu Füßen liegen.

\flagverse{8.} Hier ziehn wir Jesum Christum an\\
und decken unsre Schanden\\
mit dem, was er für uns getan\\
und willig ausgestanden;\\
hier wäscht uns sein hochteures Blut\\
und macht uns heilig, fromm und gut\\
in seines Vaters Augen.

\flagverse{9.} O großes Werk! O heilges Bad,\\
o Wasser, dessengleichen\\
man in der ganzen Welt nicht hat,\\
kein Sinn kann dich erreichen!\\
Du hast recht eine Wunderkraft,\\
und die hat der, so alles schafft,\\
dir durch sein Wort geschenket.

\flagverse{10.} Du bist kein schlichtes Wasser nicht,\\
wies unsre Brunnen geben:\\
Was Gott mit seinem Munde spricht,\\
das hast du in dir leben.\\
Du bist ein Wasser, das den Geist\\
des Allerhöchsten in sich schleußt\\
und seinen großen Namen.

\flagverse{11.} Das halt, o Mensch, in allem wert\\
und danke für die Gaben,\\
die dein Gott dir darin beschert\\
und die uns alle laben,\\
wenn nichts mehr sonst uns laben will,\\
die laß, bis daß des Todes Ziel\\
dich trifft, nicht ungepreiset.

\flagverse{12.} Brauch alles wohl, und weil du bist\\
nun rein in Christo worden,\\
so leb und tu auch als ein Christ\\
und halte Christi Orden,\\
bis daß dort in der ewgen Freud\\
er dir das Ehr- und Freudenkleid\\
um deine Seele lege!

\end{verse}
\end{multicols}
%\attrib{\small{THZE}}
