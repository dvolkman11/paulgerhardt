%StartInfo%%%%%%%%%%%%%%%%%%%%%%%%%%%%%%%%%%%%%%%%%%%%%%%%%%%%%%%%%%%%%%%%%%%%
%  Autor:
%  Titel:
%  File:
%  Ref:
%  Mod:
%EndInfo%%%%%%%%%%%%%%%%%%%%%%%%%%%%%%%%%%%%%%%%%%%%%%%%%%%%%%%%%%%%%%%%%%%%%%
%\poemtitle{Herr, höre, was mein Mund spricht}
\begin{multicols}{2}
\settowidth{\versewidth}{Herr, höre, was mein Mund }
\begin{verse}[\versewidth]
%Der 143. Psalm


\flagverse{1.} Herr, höre, was mein Mund\\
aus innerm Herzensgrund\\
ohn alle Falschheit spricht,\\
wend, Herr, dein Angesicht,\\
vernimm meine Bitte!

\flagverse{2.} Ich bitte nicht um Gut,\\
das auf der Welt beruht,\\
auch endlich mit der Welt\\
bricht und zu Boden fällt\\
und mag gar nicht retten.

\flagverse{3.} Der Schatz, den ich begehr,\\
ist deine Gnad, o Herr,\\
die Gnade, die dein Sohn,\\
mein Heil und Gnadenthron,\\
mir sterbend erworben.

\flagverse{4.} Du bist rein und gerecht,\\
ich bin ein böser Knecht,\\
ich bin in Sünden tot,\\
du bist der fromme Gott,\\
der Sünde vergibet.

\flagverse{5.} Laß deine Frömmigkeit\\
sein meinen Trost und Freud,\\
laß über meine Schuld\\
dein edle Lieb und Huld\\
sich reichlich ergießen.

\flagverse{6.} Betrachte, wer ich bin,\\
im Hui fahr ich dahin,\\
zerbrechlich wie ein Glas,\\
vergänglich wie ein Gras,\\
ein Wind kann mich fällen.

\flagverse{7.} Willst du nichts sehen an\\
als was ein Mensch getan,\\
so wird kein Menschenkind\\
von wegen seiner Sünd\\
im Himmel bestehen.

\flagverse{8.} Sieh an, wie Jesus Christ\\
für mich gegeben ist,\\
der hat, was ich nicht kann,\\
erfüllt und gnug getan\\
im Leben und im Leiden.

\flagverse{9.} Du liebest Reu und Schmerz,\\
schau her, hier ist mein Herz,\\
das seine Sünd erkennt\\
und wie im Feuer brennt\\
vor Angst, Leid und Sorgen.

\flagverse{10.} Ich lechze wie ein Land,\\
dem deine milde Hand\\
den Regen lang entzeucht,\\
bis Saft und Kraft entweicht\\
und alles verdorret.

\flagverse{11.} Gleich wie auch auf der Heid\\
ein Hirsch begehrlich schreit\\
nach frischem Wasserquell,\\
so ruf ich laut und hell\\
nach dir, o mein Leben.

\flagverse{12.} Erquicke mein Gebein,\\
geuß Trost und Labsal ein\\
und sprich mir freundlich zu,\\
daß meine Seele ruh\\
im Schoß deiner Liebe.

\flagverse{13.} Gib mir getrosten Mut,\\
wenn meiner Sünden Flut\\
aufsteiget in die Höh,\\
ersäuf all Angst und Weh\\
im Meer deiner Gnaden.

\flagverse{14.} Treib weg den bösen Feind,\\
der mich zu stürzen meint,\\
du bist mein Hirt, und ich\\
will bleiben ewiglich\\
ein Schaf deiner Weide.

\flagverse{15.} So lang auf dieser Erd\\
ich Atem holen werd,\\
o Herr, so will ich dein\\
und deines Willens sein\\
gehorsamer Diener.

\flagverse{16.} Ich will dir dankbar sein,\\
doch ist mein Können klein,\\
allein in deiner Kraft,\\
die Tun und Wollen schafft,\\
steht all mein Vermögen.

\flagverse{17.} Drum sende deinen Geist,\\
der deinen Kindern weist\\
den Weg, der dir gefällt;\\
wer den bewahrt und hält,\\
wird nimmermehr fehlen.

\flagverse{18.} Ich richte mich nach dir,\\
du sollst mir gehen für.\\
Du sollst mir schließen auf\\
die Bahn im Tugendlauf,\\
ich will treulich folgen.

\end{verse}
\end{multicols}


\begin{center}
\settowidth{\versewidth}{Und wenn des Himmels Pfort}
\begin{verse}[\versewidth]

\flagverse{19.} Und wenn des Himmels Pfort\\
ich werd ergreifen dort,\\
so will im Engelheer\\
ich ewig deiner Ehr\\
in Freuden lobsingen.  
  
\end{verse}
\end{center}

%\attrib{\small{THZE}}
