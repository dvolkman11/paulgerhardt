%StartInfo%%%%%%%%%%%%%%%%%%%%%%%%%%%%%%%%%%%%%%%%%%%%%%%%%%%%%%%%%%%%%%%%%%%%
%  Autor:
%  Titel:
%  File:
%  Ref:
%  Mod:
%EndInfo%%%%%%%%%%%%%%%%%%%%%%%%%%%%%%%%%%%%%%%%%%%%%%%%%%%%%%%%%%%%%%%%%%%%%%
%\poemtitle{pt}
\begin{multicols}{2}
\settowidth{\versewidth}{Ja Herr, du tratst ihm an das Herz,}
\begin{verse}[\versewidth]

\flagverse{1.} Was trauerst du, mein Angesicht,\\
wann du den Tod hörst nennen?\\
Sei ohne Furcht: er schadt dir nicht,\\
lern ihn nur recht erkennen.\\
Kennst du den Tod,\\
so hats nicht Not,\\
all Angst wird sich zertrennen.

\flagverse{2.} Vors erste, zeuch die Larven ab\\
der alten roten Schlangen;\\
sieh an, daß sie kein Gift mehr hab,\\
es ist ihr abgefangen\\
durch Jesum Christ,\\
der vor uns ist\\
ins Grab und Tod gegangen.

\flagverse{3.} Ja Herr, du tratst ihm an das Herz,\\
brachst seines Stachels Spitzen;\\
nunmehr ist er ein lauter Scherz\\
und kann uns gar nicht ritzen;\\
dein edles Blut\\
dämpft seine Glut,\\
dein Flammen zwingt sein Hitzen.

\flagverse{4.} Die Sünde war des Todes Kraft,\\
die uns zum Sterben triebe,\\
nun ist die Sünd all abgeschafft\\
durch Christi Treu und Liebe;\\
ihr Ernst und Macht\\
ist matt gemacht;\\
trotz, daß sie uns betrübe.

\flagverse{5.} Die Sünd ist tot, Gott ist versöhnt,\\
durch seines Sohnes Dulden,\\
der Grimm ist hin, den wir verdient\\
mit unsers Lebens Schulden;\\
der vor war Feind,\\
ist nunmehr Freund\\
voll süßer Gnad und Hulden.

\flagverse{6.} Bist du denn Freund, so kannst du mich,\\
mein Gott, ja nicht umbringen;\\
dein Vaterherze lässet sich\\
zum Mord und Tod nicht dringen.\\
Wer sich befindt\\
dein Erb und Kind,\\
ist frei von bösen Dingen.

\flagverse{7.} Das aber, Vater, tust du wohl,\\
wann uns die Trübsal kränket,\\
wann wir des Lebens satt und voll\\
des Jammers, der uns tränket,\\
daß dann dein Hand\\
ans Vaterland\\
uns aus den Fluten lenket.

\flagverse{8.} Wann sich das starke Wetter regt,\\
davon die Höhen fallen,\\
wann deines Zornes Donner schlägt,\\
daß Berg und Tal erschallen:\\
So trittst du zu\\
und bringst zur Ruh\\
uns, die dir wohlgefallen.

\flagverse{9.} Wann unsre Feinde um uns her\\
uns bringen in die Mitten,\\
wann Ottern, Löwen, Wölf und Bär\\
ihr Gift auf uns ausschütten:\\
Nimmst du dein Schaf,\\
bringt's in den Schlaf\\
bei dir in deiner Hütten.

\flagverse{10.} Wann diese Welt gibt bösen Lohn\\
dem, der dich treulich ehret,\\
so sprichst du: Komm zu mir, mein Sohn,\\
hier hab ich, was dich nähret:\\
Lust, Ehr und Freud,\\
die keine Zeit\\
in Ewigkeit verzehret.

\flagverse{11.} Alsbald schließt uns der Engel Schar\\
mit Freud in ihrem Bogen\\
und nehmen unsrer Seele wahr,\\
die, wann sie ausgeflogen,\\
in ihre Hut\\
mit stillem Mut\\
zu Gott kommt angezogen.

\flagverse{12.} Der Herr empfänget seine Braut\\
und spricht: Sei mir willkommen!\\
Du bists, die ich mir anvertraut,\\
komm, wohne bei den Frommen,\\
die ich vor dir\\
anher zu mir\\
aus jener Welt genommen.

\flagverse{13.} Du hast behalten Glaub und Treu\\
im Herzen, da ich wohne:\\
So geb und leg ich dir nun bei\\
die schöne Freudenkrone.\\
Ich bin dein Heil,\\
dein Erb und Teil,\\
tritt her zu meinem Throne.

\flagverse{14.} Hier trockn ich deiner Augen Flut,\\
hier still ich deine Tränen,\\
hier setzt sich in dem höchsten Gut\\
dein Seufzen, Klag und Sehnen;\\
dein Jammermeer\\
wird niemand mehr,\\
als nur in Freud, erwähnen.

\flagverse{15.} Hier kleid ich meiner Christen Zahl\\
mit reiner weißer Seide;\\
hier springen sie im Himmelssaal,\\
und ist nicht, der sie neide;\\
hier ist kein Tod,\\
kein Kreuz und Not,\\
das gute Freunde scheide.

\flagverse{16.} Ach, Gott mein Herr, was will ich doch\\
mich vor dem Tode scheuen?\\
Er ists ja, der mich von dem Joch\\
des Elends will befreien:\\
Er nimmt mich aus\\
dem Marterhaus,\\
das kann mich nicht gereuen.

\flagverse{17.} Der Tod, der ist mein Rotes Meer,\\
dadurch auf trocknem Sande\\
dein Israel, das fromme Heer,\\
geht zum Gelobten Lande,\\
da Milch und Wein\\
stets fleußt herein\\
wie Ström in ihrem Rande.

\flagverse{18.} Er ist das güldne Himmelstor\\
und des Eliä Wagen,\\
darauf mich Gott zum Engelchor\\
gar bald wird lassen tragen,\\
wann er, der Letzt\\
und Erste, setzt\\
ein End an meinen Tagen.

\flagverse{19.} O süße Lust, o edle Ruh,\\
o frommer Seelen Freude,\\
komm, schleuß mir meine Augen zu,\\
daß ich mit Fried abscheide\\
hin, da mein Hirt\\
mich leiten wird\\
zur immergrünen Weide.

\flagverse{20.} Daselbst wird er mit vollem Maß,\\
was hier gefehlt, einbringen;\\
dafür wird ihm ohn Unterlaß\\
sein Halleluja klingen,\\
das will auch ich\\
ihm williglich\\
eins nach dem andern singen.

\end{verse}
\end{multicols}
\attrib{\small{THZE}}
