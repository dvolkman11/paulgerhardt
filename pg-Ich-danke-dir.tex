%StartInfo%%%%%%%%%%%%%%%%%%%%%%%%%%%%%%%%%%%%%%%%%%%%%%%%%%%%%%%%%%%%%%%%%%%%
%  Autor:
%  Titel:
%  File:
%  Ref:
%  Mod:
%EndInfo%%%%%%%%%%%%%%%%%%%%%%%%%%%%%%%%%%%%%%%%%%%%%%%%%%%%%%%%%%%%%%%%%%%%%%
%\poemtitle{pt}
\begin{multicols}{2}
\settowidth{\versewidth}{Ich bitte, was ich bitten kann,}
\begin{verse}[\versewidth]
%ich danke dir demütiglich\\
  %nach Johann Arnds »Paradiesgärtlein«, Goslar 1621, III, 17:
  %»Gebet um zeitliche und ewige Wohlfahrt«

\flagverse{1.} Ich danke dir demütiglich,\\
o Gott, mein Vater, daß du dich\\
von deinem Zorn gewendet\\
und deinen Sohn\\
zur Freud und Kron\\
uns in die Welt gesendet.

\flagverse{2.} Er ist gekommen, hat sein Blut\\
vergossen und in solcher Flut\\
all unser Sünd ersticket.\\
Wer ihn nur faßt,\\
wird aller Last\\
benommen und erquicket.

\flagverse{3.} Ich bitte, was ich bitten kann,\\
herzlieber Vater, nimm mich an\\
in diesen edlen Orden,\\
der durch dies Blut\\
gerecht und gut\\
und ewig selig worden.

\flagverse{4.} Laß meines Glaubens Aug und Hand\\
ergreifen dieses werte Pfand\\
und nimmermehr verlieren;\\
laß dieses Licht\\
mein Angesicht\\
zum ewgen Lichte führen!

\flagverse{5.} Bereite meiner Seelen Haus,\\
wirf allen Kot und Unflat aus,\\
bau in mir deine Hütte,\\
daß deine Güt\\
in mein Gemüt\\
all ihre Lieb ausschütte!

\flagverse{6.} Wann ich dich hab, ist alles mein;\\
du kannst nicht ohne Gaben sein,\\
hast tausend Weg und Weisen,\\
dein arme Herd\\
auf dieser Erd\\
zu nähren und zu speisen.

\flagverse{7.} Gib mir, daß ich an meinem Ort\\
allstets dich fürcht in deinem Wort\\
und meinen Stand so führe,\\
daß Glaub und Treu\\
stets bei mir sei\\
und all mein Leben ziere!

\flagverse{8.} Gib nur ein gnügsam Herz und Sinn!\\
Denn das ist ja ein großer Gwinn,\\
in steter Andacht liegen\\
und, wenn Gott gibt\\
was ihm beliebt,\\
ihm lassen gern genügen.

\flagverse{9.} Das Wen'ge, das durch Gottes Hand\\
ein Frommer und Gerechter hat,\\
ist vielmal mehr geehret\\
als alles Geld,\\
davon die Welt\\
mit frechem Herzen zehret.

\flagverse{10.} Die Frommen sind dir, Herr, bewußt;\\
du bist ihr und sie deine Lust\\
und werden nicht zuschanden,\\
kommt teure Zeit,\\
findt sich bereit\\
ihr Brot in allen Landen.

\flagverse{11.} Gott hat den, der ihn fürchtet, lieb,\\
sieht zu, daß ihn kein Unfall trüb,\\
hat Lust zu seinen Wegen;\\
und wenn er fällt,\\
steht Gott und hält\\
ihn fest in seinem Segen.

\flagverse{12.} Des Höchsten Auge sieht auf die,\\
so auf ihn hoffen spat und früh,\\
daß er sie schütz und rette\\
aus aller Not,\\
wann sie der Tod\\
auch selbst verschlungen hätte.

\flagverse{13.} Herr, du kannst nichts als Güte sein,\\
du wollest deiner Güte Schein\\
uns und all denen gönnen,\\
die sich mit Mund\\
und Herzensgrund\\
allein zu dir bekennen!

\flagverse{14.} Insonderheit nimm wohl in Acht\\
den Fürsten, den du uns gemacht\\
zu unsers Landes Krone,\\
laß immerzu\\
sein Fried und Ruh\\
auf seinem Stuhl und Throne.

\flagverse{15.} Halt unser liebes Vaterland\\
in deinem Schoß und starker Hand!\\
Behüt uns allzusammen\\
vor falscher Lehr\\
und Feindes Heer,\\
vor Pest und Feuersflammen.

\flagverse{16.} Nimm all der Meinen eben wahr,\\
treib, Herr, die böse Höllenschar\\
von Jungen und von Alten,\\
daß deine Herd\\
hie zeitlich werd\\
und ewig dort erhalten.

\end{verse}
\end{multicols}
%\attrib{\small{THZE}}
