%StartInfo%%%%%%%%%%%%%%%%%%%%%%%%%%%%%%%%%%%%%%%%%%%%%%%%%%%%%%%%%%%%%%%%%%%%
%  Autor:
%  Titel:
%  File:
%  Ref:
%  Mod:
%EndInfo%%%%%%%%%%%%%%%%%%%%%%%%%%%%%%%%%%%%%%%%%%%%%%%%%%%%%%%%%%%%%%%%%%%%%%
%\poemtitle{Du meine Seele, singe}
\begin{multicols}{2}
\settowidth{\versewidth}{Ihr Menschen laßt euch lehren,}
\begin{verse}[\versewidth]
%der 146. Psalm


\flagverse{1.} Du meine Seele, singe,
wohlauf, und singe schön\\
dem, welchem alle Dinge\\
zu Dienst und Willen stehn.\\
Ich will den Herren droben\\
hier preisen auf der Erd,\\
ich will ihn herzlich loben,\\
so lang ich leben werd.

\flagverse{2.} Ihr Menschen laßt euch lehren,\\
es wird sehr nützlich sein:\\
Laßt euch doch nicht betören\\
die Welt mit ihrem Schein.\\
Verlasse sich ja keiner\\
auf Fürstenmacht und –gunst,\\
weil sie wie unser einer\\
nichts sind, als nur ein Dunst.

\flagverse{3.} Was Mensch ist, muß erblassen\\
und sinken in den Tod;\\
er muß den Geist auslassen,\\
selbst werden Erd und Kot.\\
Allda ist's dann geschehen\\
mit seinem klugen Rat\\
und ist frei klar zu sehen,\\
wie schwach sei Menschentat.

\flagverse{4.} Wohl dem, der einzig schauet\\
nach Jakobs Gott und Heil;\\
wer dem sich anvertrauet,\\
der hat das beste Teil,\\
das höchste Gut erlesen,\\
den schönsten Schatz geliebt,\\
sein Herz und ganzes Wesen\\
bleibt ewig unbetrübt.

\flagverse{5.} Hier sind die starken Kräfte,\\
die unerschöpfte Macht,\\
das weisen die Geschäfte,\\
die seine Hand gemacht:\\
Der Himmel und die Erde\\
mit ihrem ganzen Heer,\\
der Fisch unzählig Herde\\
im großen wilden Meer.

\flagverse{6.} Hier sind die treuen Sinnen,\\
die niemand Unrecht tun,\\
all denen Gutes gönnen,\\
die in der Treu beruhn.\\
Gott hält sein Wort mit Freuden,\\
und was er spricht, geschicht,\\
und wer Gewalt muß leiden,\\
den schützt er im Gericht.

\flagverse{7.} Er weiß viel tausend Weisen,\\
zu retten aus dem Tod,\\
ernährt und gibet Speisen\\
zur Zeit der Hungersnot,\\
macht schöne rote Wangen\\
oft bei geringem Mahl,\\
und die da sind gefangen,\\
die reißt er aus der Qual.

\flagverse{8.} Er ist das Licht der Blinden,\\
erleuchtet ihr Gesicht,\\
und die sich schwach befinden,\\
die stellt er aufgericht:\\
Er liebet alle Frommen,\\
und die ihm günstig seind,\\
die finden, wenn sie kommen,\\
an ihm den besten Freund.

\flagverse{9.} Er ist der Fremden Hütte,\\
die Waisen nimmt er an,\\
erfüllt der Witwen Bitte,\\
wird selbst ihr Trost und Mann;\\
die aber, die ihn hassen,\\
bezahlet er mit Grimm,\\
ihr Haus und wo sie saßen,\\
das wirft er üm und üm.

\flagverse{10.} Ach, ich bin viel zu wenig,\\
zu rühmen seinen Ruhm!\\
Der Herr allein ist König,\\
ich eine welke Blum.\\
Jedoch weil ich gehöre\\
gen Zion in sein Zelt,\\
ist's billig, daß ich mehre\\
sein Lob vor aller Welt.

\end{verse}
\end{multicols}
%\attrib{\small{THZE}}
