%StartInfo%%%%%%%%%%%%%%%%%%%%%%%%%%%%%%%%%%%%%%%%%%%%%%%%%%%%%%%%%%%%%%%%%%%%
%  Autor:
%  Titel:
%  File:
%  Ref:
%  Mod:
%EndInfo%%%%%%%%%%%%%%%%%%%%%%%%%%%%%%%%%%%%%%%%%%%%%%%%%%%%%%%%%%%%%%%%%%%%%%
%\poemtitle{Erhebe dich, betrübtes Herz}
\begin{multicols}{2}
\settowidth{\versewidth}{Was stürzt wohl eines Frommen Sinn?}
\begin{verse}[\versewidth]

%»Trost-Gesang über den unversehenen Todesfall des wohlseligen Herrn Johannes Bercovii« (1651)

\flagverse{1.} Erhebe dich, betrübtes Herz,\\
und laß die Sinnen überwärts\\
hin nach dem Himmel steigen:\\
Nimm Gott zu Trost, so wird dein Schmerz\\
alsbald sich merklich neigen.

\flagverse{2.} Dein Schad ist groß, das ist ja wahr,\\
doch ist ja auch bekannt und klar\\
des höchsten Vaters Gnade:\\
Die macht, daß uns des Unglücks Schar\\
nicht um ein Härlein schade.

\flagverse{3.} Der Fall, der unverhoffte Fall\\
schlägt uns nicht anders als der Schall\\
des Donners aus der Höhe:\\
Gott aber hilft, daß Fall und Knall\\
zum Glück und Guten gehe.

\flagverse{4.} Was stürzt wohl eines Frommen Sinn?\\
Wo kann ein Christ auch anders hin\\
als in den Himmel fallen?\\
Trost, Fried und Freud erhalten ihn,\\
Angst muß zurückeprallen.

\flagverse{5.} Was hat der Tod mit seiner Müh,\\
er komme spät an oder früh,\\
an gottergebnen Seelen?\\
Nimmt er sie bald, befreit er sie\\
vor langem sauren Quälen.

\flagverse{6.} Wer plötzlich stirbt und stirbt nur wohl,\\
der nimmt ein Ende, das man soll\\
gewünscht und selig preisen:\\
Ists Herze gut und glaubensvoll,\\
was schadt das schnelle Reisen?

\flagverse{7.} Was fragt ein Kämpfer nach der Zeit,\\
wenn er den Feind nur in dem Streit\\
hat ritterlich empfangen?\\
Wie mancher kann die Siegesbeut\\
im Augenblick erlangen.

\flagverse{8.} Ein solches Lob und edlen Lohn\\
hat auch fürwahr und trägt davon\\
der, den wir jetzt beweinen:\\
Er sieht nun selbst ein helle Kron\\
auf seinem Haupte scheinen.

\flagverse{9.} Er hat gesiegt, das ist gewiß.\\
Er ist durch Todes Finsternis\\
zu Gottes Licht gekommen.\\
Er lebt, obschon ein schneller Riß\\
ihn von uns hingenommen.

\flagverse{10.} Den schnellen Riß hat Gott getan,\\
der nichts als Gutes machen kann\\
im Himmel und auf Erden.\\
Was gut tut, hebts gleich traurig an,\\
muß doch zuletzt gut werden.

\flagverse{11.} Wir wünschen zwar, ach hätten wir\\
doch bei dem Bette sollen hier\\
in seinem Ende stehen\\
und hören gegen dir und mir\\
sein letztes Wort ergehen.

\flagverse{12.} Denkt aber, denkt, ob dies Gehör\\
uns mehr betrübt als tröstlich wär,\\
und gebt euch wohl zufrieden,\\
weil er in Gott zu Gottes Ehr\\
auf Gottes Wort verschieden.

\flagverse{13.} Hilf Gott! Sprach sein gottselger Mund,\\
das hörte Gott, und half zur Stund\\
ihn in die hohen Freuden,\\
da sich sein Aug und Herzensgrund\\
in reiner Wollust weiden.

\flagverse{14.} Da hat er nun all Hilf und Heil,\\
ist froh in seinem Erb und Teil,\\
wonach er hier gestrebet,\\
ruht fern vom Tod und Todespfeil,\\
in dem er ewig lebet.

\flagverse{15.} Nun darf sein Herz nicht traurig sein\\
und fühlt nicht mehr den schweren Stein\\
des Kummers wie hienieden,\\
da sein Fleiß in der Sorgen Pein\\
sich täglich mußt ermüden.

\flagverse{16.} Sein süßer Mund, des edle Zier\\
des Höchsten Weisheit für und für\\
so treulich hat gelehret,\\
der predigt, was kein Ohr allhier\\
bei uns je hat gehöret.

\flagverse{17.} Er predigt seines Gottes Ruhm\\
und füllt das güldne Heiligtum\\
und die so schönen Tore,\\
sein Name reucht gleich einer Blum\\
im heilgen Engelchore.

\flagverse{18.} Die Pflänzlein, die er vorgeschickt,\\
hat er auch schon mit Lust erblickt\\
und herzlich sich ergetzet,\\
nun ist sein Geist in ihm erquickt\\
und alles Leid ersetzet.

\flagverse{19.} Was wollt ihr nun mit eurem Leid,\\
ihr, die ihr ihm gewogen seid,\\
euch selbst nun ferner plagen?\\
Wems wohlgeht und sich glücklich freut,\\
den darf man nicht mehr klagen.

\flagverse{20.} Wischt eure Tränen vom Gesicht\\
und laßt des lieben Trostes Licht\\
in eure Herzen brechen,\\
so wird, der alles Herzleid bricht,\\
euch Herz und Mut einsprechen.

\flagverse{21.} Nehmt eure Zuflucht zu ihm zu,\\
und glaubt, daß er nichts anderes tu\\
als nur, was uns kann nützen:\\
Wer das behält, wird in der Ruh\\
und Gott im Schoße sitzen.

\flagverse{22.} Wer Gott vertraut, wird in der Tat\\
erfahren, daß des Höchsten Rat\\
ihn weislich werde führen\\
und hier und dort mit großer Gnad\\
und reichem Segen zieren.

\end{verse}
\end{multicols}
%\attrib{\small{THZE}}
