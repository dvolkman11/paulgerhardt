\backmatter% --------------------------------------------------------------------
\section*{Anmerkungen zu den einzelnen Gedichten}
\section*{\centerline{***Noch nicht zugeordnet***}}


Paul Gerhardt hat seine Gedichte selbst nicht gesammelt herausgegeben.
Sofern sie nicht als Einzeldrucke publiziert wurden, erschienen sie
zumeist erstmals in den verschiedenen Ausgaben des von Johann Crüger
herausgegebenen Gesangbuchs \frqq Praxis pietatis melica\flqq , Berlin bei
Christoph Runge 1648, 1653, 1656 und 1661. Noch zu Gerhardts Lebzeiten
erschien außerdem eine Gesamtausgabe seiner Gedichte in zehn Lieferungen
zu je zwölf Gedichten, die von Johann Georg Ebeling herausgegeben wurde
(\frqq Geistliche Andachten\flqq , Berlin 1666/67). Inwieweit Gerhardt an dieser
Ausgabe mitgewirkt hat, ist nicht bekannt. – Die vorliegende Sammlung
umfaßt alle deutschen Gedichte Paul Gerhardts.
Vollständige Neuausgabe mit einer Biographie des Autors. Herausgegeben von Karl-Maria Guth. Berlin 2013.
Textgrundlage ist die Ausgabe: Paul Gerhardt: Dichtungen und Schriften.
Herausgegeben und textkritisch durchgesehen von Eberhard von Cranach-Sichart, München: Paul Müller, 1957. 

\subsection*{Wie soll ich dich empfangen}

Erstdruck 1653.

\subsection*{ Warum willst du draußen stehen}

Erstdruck 1653.

\subsection*{ Wir singen dir, Immanuel}

Erstdruck 1653.

\subsection*{ O Jesu Christ, dein Kripplein ist}

Erstdruck 1653.

\subsection*{ Fröhlich soll mein Herze springen}

Erstdruck 1653.

\subsection*{ Ich steh an deiner Krippen hier}

Erstdruck 1653.

\subsection*{ Schaut, schaut, was ist für Wunder dar?}

Erstdruck 1666 unter dem Titel \frqq Christ- Nacht-Liedlein\flqq .

\subsection*{ Kommt und laßt uns Christum ehren}

Erstdruck 1666 unter dem Titel \frqq Weihnacht- Gesang\flqq .

\subsection*{ Alle, die ihr, Gott zu ehren}

Entstanden früh. Erstdruck 1666 unter dem Titel \frqq Christ-Wiegenlied\flqq .

\subsection*{ Nun laßt uns gehn und treten}

Entstanden im 30-jähr. Krieg. Erstdruck 1653.

\subsection*{ Warum machet solche Schmerzen}

Erstdruck 1648.

\subsection*{ Ein Lämmlein geht und trägt die Schuld}

Erstdruck 1648.

\subsection*{ O Welt, sieh hier dein Leben}

Erstdruck 1648.

\subsection*{ O Mensch, beweine deine Sünd}

Erstdruck 1648 unter dem Titel \frqq Die Passion aus den vier Evangelisten\flqq .

\subsection*{ Siehe, mein getreuer Knecht}

Erstdruck 1653 unter dem Titel \frqq Das 53. Kapitel Esaiä\flqq .

\subsection*{ Hör an, mein Herz, die sieben Wort}

Erstdruck 1653.

\subsection*{ Als Gottes Lamm und Leue}

Erstdruck 1653.

\subsection*{ Passions-Salve an die leidenden Glieder Christi}

\subsection*{ 1. An die Füße - Sei mir tausendmal gegrüßet}

Erstdruck 1653. Nach einem Zyklus des Bernard de Clairvaux: \frqq Rhythmica
oratio ad unum quodlibet membrorum Christi patientis et a cruce
pendentis. 1. Salve, mundi salutare\flqq .

\subsection*{ 2. An die Knie - Gegrüßet seist du, meine Kron}

Erstdruck 1653 unter dem Titel \frqq An die leydende Knie des Herrn Christi\flqq .
Original: \frqq 2. Salve Jesu, rex sanctorum\flqq .

\subsection*{ 3. An die Hände - Sei wohl gegrüßet, guter Hirt}

Erstdruck 1653 unter dem Titel \frqq An die leydende Hände des Herrn
Christi\flqq . Original: \frqq 3. Salve Jesu, pastor bone\flqq .

\subsection*{ 4. An die Seite - Ich grüße dich, du frömmster Mann}

Erstdruck 1653 unter dem Titel \frqq An die leydende Seite des Herrn
Christi\flqq . Original: \frqq 4. Salve Jesu, summe bonus\flqq .

\subsection*{ 5. An die Brust - Gegrüßet seist du, Gott mein Heil}

Erstdruck 1653 unter dem Titel \frqq An die leydende Brust des Herrn
Christi\flqq . Original: \frqq 5. Salve, salus mea, Deus\flqq .

\subsection*{ 6. An das Herz - O Herz des Königs aller Welt}

Erstdruck 1653 unter dem Titel \frqq An das leydende Herz des Herrn Christi\flqq .
Original: \frqq 6. Summi Regis, cor, aveto\flqq .

\subsection*{ 7. An das Angesicht - O Haupt voll Blut und Wunden}

Erstdruck 1653 unter dem Titel \frqq An das leydende Angesicht Jesu Christi\flqq .
Original: \frqq 7. Salve, caput cruentatum\flqq .

\subsection*{ Also hat Gott die Welt geliebt}

Amazon.de Widgets 

Erstdruck 1661.

\subsection*{ Auf auf, mein Herz, mit Freuden}

Erstdruck 1648.

\subsection*{ Nun freut euch hier und überall}

Erstdruck 1653 unter dem Titel \frqq Die Auferstehung Christi\flqq .

\subsection*{ Sei fröhlich alles weit und breit}t

Erstdruck 1653.

\subsection*{ Zeuch ein zu deinen Toren}

Entstanden vor 1648. Erstdruck 1653.

\subsection*{ O du allersüß'ste Freude}

Erstdruck 1648.

\subsection*{ Gott Vater, sende deinen Geist}

Erstdruck 1648.

\subsection*{ Was alle Weisheit in der Welt}

Erstdruck 1653.

\subsection*{ Du Volk, das du getaufet bist}

Erstdruck 1667 unter dem Titel \frqq Von der heiligen Taufe\flqq .

\subsection*{ Herr Jesu, meine Liebe}

Erstdruck 1667 unter dem Titel \frqq Vom heiligen Abendmahl\flqq .

\subsection*{ Wach auf, mein Herz, und singe!}

Erstdruck 1648.

\subsection*{ Lobet den Herren alle, die ihn fürchten!}

Erstdruck 1653.

\subsection*{ Die güldne Sonne}

Erstdruck 1666 unter dem Titel \frqq Morgen- Segen\flqq .

\subsection*{ Nun ruhen alle Wälder}

Erstdruck 1648.

\subsection*{ Der Tag mit seinem Lichte}

Erstdruck 1666 unter dem Titel \frqq Abend- Segen\flqq .

\subsection*{ Geh aus, mein Herz, und suche Freud}

Erstdruck 1653.

\subsection*{ O Herrscher in dem Himmelszelt}

Erstdruck 1666 unter dem Titel \frqq Buß- und Bet-Gesang bei unzeitiger Nässe
und betrübtem Gewitter\flqq .

\subsection*{ Nun ist der Regen hin}

Erstdruck 1653 unter dem Titel \frqq Danklied vor einen gnädigen
Sonnenschein\flqq .

\subsection*{ Nun geht frisch drauf, es geht nach Haus}

Erstdruck 1653 unter dem Titel \frqq Danklied nach der Reise\flqq .

\subsection*{ Der aller Herz und Willen lenkt}

Erstdruck 1643. \frqq Oda: Hochzeitsgedicht für Joachim Fromm und Sabina
Barthold 1643\flqq .

\subsection*{ Ein Weib, das Gott den Herren liebt}

Erstdruck 1653 unter dem Titel \frqq Frauen- Lob. Aus den Sprüchen Salomonis
am 31. Kap.\flqq .

\subsection*{ Voller Wunder, voller Kunst}

Erstdruck 1666 unter dem Titel \frqq Der Wunder volle Ehestand\flqq .

\subsection*{ Wie schön ists doch, Herr Jesu Christ}

Erstdruck 1666 unter dem Titel \frqq Trost-Gesang christlicher Eheleute\flqq .

\subsection*{ Unter allen, die da leben}

Am Schluß der \frqq Vier geistlichen Lieder\flqq  von Joachim Pauli (Berlin o.J.,
wahrscheinlich 1665) von Paul Gerhardt angefügt \frqq Zur Bezeugung guter
Zuneigung gegen den Authore\flqq .

\subsection*{ Tapfere Leute soll man loben}

Aus \frqq Fünfzehn-ästiger Nieder-Lausitzer Palm-Baum\flqq  des Samuel Sturm,
1675.

\subsection*{ Herr, ich will gar gerne bleiben}

Erstdruck 1667 unter dem Titel \frqq Wahre Erniedrigung sein selbsten aus dem
Matthäo am 15.V.27. Ja Herr, aber doch essen die Hündlein von den
Brosamen, die von ihrer Herren Tische fallen\flqq . Nach den lat. Distichen
des Nathan Chyträus \frqq Sum canis indignus\flqq , 1568.

\subsection*{ Weg, mein Herz, mit den Gedanken}

Erstdruck 1648.

\subsection*{ Herr, höre, was mein Mund}

Erstdruck 1648.

\subsection*{ Nach dir, o Herr, verlanget mich}

Erstdruck 1648 unter dem Titel \frqq Der 25. Psalm\flqq .

\subsection*{ Zweierlei bitt ich von dir}

Erstdruck 1648.

\subsection*{ O Gott, mein Schöpfer, edler Fürst}

Erstdruck 1648 unter dem Titel \frqq Syrachs Gebätlein umb ein züchtiges und
mäßiges Leben\flqq .

\subsection*{ Ich erhebe, Herr, zu dir}

Erstdruck 1648 unter dem Titel \frqq Der 121. Psalm\flqq .

\subsection*{ Weltskribenten und Poeten}

Vorspruch zu Michael Schirmer, \frqq Bibl. Lieder und Lehrsprüche\flqq , Berlin
1650.

\subsection*{ Ich weiß, mein Gott, daß all mein Tun}

Erstdruck 1653.

\subsection*{ Ich danke dir demütiglich}

Erstdruck 1653. Nach Johann Arnds \frqq Paradies-gärtlein\flqq , Goslar 1621, III,
17: \frqq Gebet um zeitliche und ewige Wohlfahrt\flqq .

\subsection*{ O Jesu Christ, mein schönstes Licht}

Erstdruck 1653 unter dem Titel \frqq Um die Liebe Christi. Aus Herrn Johann
Arnds Gebät\flqq  (nach \frqq Paradiesgärtlein\flqq , Goslar 1621, II, 5: \frqq Gebet um die
Liebe Christi\flqq ).

\subsection*{ Wohl dem Menschen, der nicht wandelt}

Erstdruck 1653.

\subsection*{ Hört an, ihr Völker, hört doch an}

Erstdruck 1653.

\subsection*{ Wohl dem, der den Herren scheuet}

Erstdruck 1653.

\subsection*{ Herr, aller Weisheit Quell und Grund}

Entstanden wahrscheinlich früh. Erstdruck 1661 (nach Joh. Arnds
\frqq Paradiesgärtlein\flqq , Goslar 1621, I, 14: \frqq Umb Weisheit\flqq ).

\subsection*{ Jesu, allerliebster Bruder}

Entstanden wahrscheinlich früh. Erstdruck 1661 (nach Joh. Arnds
\frqq Paradiesgärtlein\flqq , Goslar 1621, I, 34: \frqq Umb Christliche beständige
Freundschaft\flqq ).

\subsection*{ Herr, du erforschest meinen Sinn}

Erstdruck 1666 unter dem Titel \frqq Der 139. Psalm Davids\flqq .

\subsection*{ Ist Ephraim nicht meine Kron?}

Entstanden 1641. Erstdruck 1653.

\subsection*{ Was soll ich doch, o Ephraim}

Entstanden 1641. Erstdruck 1653 unter dem Titel \frqq Aus dem 11. Kap.
Hoseä\flqq ,

\subsection*{ Kommt, ihr traurigen Gemüter}

Entstanden 1641. Erstdruck 1653 unter dem Titel \frqq Aus dem Hosea am 6.
Kap.\flqq .

\subsection*{ Was trotzest du, stolzer Tyrann?}

Entstanden vor 1648. Erstdruck 1666 unter dem Titel \frqq 52. Psalm Davids\flqq .

\subsection*{ Herr, der du vormals hast dein Land}

Entstanden vor 1648. Erstdruck 1653.

\subsection*{ Nicht so traurig, nicht so sehr}

Erstdruck 1648 unter dem Titel \frqq Christliche Zufriedenheit\flqq .

\subsection*{ Ich hab in Gottes Herz und Sinn}

Erstdruck 1648 unter dem Titel \frqq Christliche Ergebung in Gottes Willen\flqq .

\subsection*{ Ich hab oft bei mir selbst gedacht}

Erstdruck 1653.

\subsection*{ Du bist ein Mensch, das weißt du wohl}

Erstdruck 1653.

\subsection*{ Du liebe Unschuld du, wie schlecht wirst du geacht't}

Erstdruck 1653.

\subsection*{ Ich habs verdient, was will ich doch}

Erstdruck 1653 unter dem Titel \frqq Aus dem Micha am 7. Kap.\flqq .

\subsection*{ Ach treuer Gott, barmherzigs Herz}

Erstdruck 1653\par\noindent
(nach Joh. Arnds \frqq Paradiesgärtlein\flqq , III, 27: \frqq Gebet um
Geduld in großem Kreuz\flqq ).

\subsection*{ Barmherzger Vater, höchster Gott}

Erstdruck 1653 unter dem Titel \frqq Joh. Arnds Kreuzgebet\flqq  (nach Joh. Arnds
\frqq Paradies- gärtlein\flqq , III, 26).

\subsection*{ Was Gott gefällt, mein frommes Kind}

Erstdruck 1653.

\subsection*{ Schwing dich auf zu deinem Gott}

Erstdruck 1653 unter dem Titel \frqq Trost in schwerer Anfechtung\flqq , ohne
Strophe 3. 1666 vollständig.

\subsection*{ Ist Gott für mich, so trete}

Entstanden evtl. 1651. Erstdruck 1653.

\subsection*{ Warum sollt ich mich doch grämen?}

Amazon.de Widgets 

Erstdruck 1653.

\subsection*{\centerline{Befiehl du deine Wege}}

Erstdruck 1653.

\subsection*{ Noch dennoch mußt du drum nicht ganz}

Erstdruck 1653.

\subsection*{ Wie lang, o Herr, wie lange soll}

Erstdruck 1653 unter dem Titel \frqq Der 13. Psalm Davids\flqq .

\subsection*{ Gott ist mein Licht, der Herr mein Heil}

Erstdruck 1653.

\subsection*{ Wie der Hirsch im großen Dürsten}

Erstdruck 1653.

\subsection*{ Sei wohlgemut, o Christenseel}

Erstdruck 1653.

\subsection*{ Wer unterm Schirm des Höchsten sitzt}

Erstdruck 1653.

\subsection*{ Geduld ist euch vonnöten}

Erstdruck 1661 unter dem Titel \frqq Geduld ist euch noth\flqq .

\subsection*{ Ach Herr, wie lange willst du mein}

Erstdruck im Anhang einer Leichenrede für Chr. Ludw. von Thümen 1660,
Berlin o.J., unter dem Titel \frqq Der 13. Psalm Davids\flqq .

\subsection*{ Herr, was hast du im Sinn?}

Evtl. entstanden 1664 oder 1652 (auf die Erscheinung eines Kometen).
Erstdruck 1666 unter dem Titel \frqq Bey Erscheinung eines Cometen\flqq .

\subsection*{ Gib dich zufrieden und sei stille}

Erstdruck 1666 unter dem Titel \frqq Gib dich zufrieden\flqq .

\subsection*{ Meine Seele ist in der Stille}

Erstdruck 1666 unter dem Titel \frqq Der 62. Psalm Davids\flqq .

\subsection*{ Nun danket all und bringet Ehr}

Erstdruck 1648.

\subsection*{ Wie ist so groß und schwer die Last}

Entstanden vor 1648. Erstdruck 1653 unter dem Titel \frqq Schutz Gottes in
Kriegsläuft\flqq .

\subsection*{ Gott Lob! Nun ist erschollen}

Entstanden 1648 zum Westfälischen Frieden. Erstdruck 1653.

\subsection*{ Sollt ich meinen Gott nicht singen?}

Erstdruck 1653.

\subsection*{ Wer wohlauf ist und gesund}

Erstdruck 1653 unter dem Titel \frqq Danklied für Leibesgesundheit\flqq .

\subsection*{ Ich singe dir mit Herz und Mund}

Erstdruck 1653 ohne Strophe 4, 8, 9, 17. 1666 vollständig.

\subsection*{ Auf den Nebel folgt die Sonne}

Erstdruck 1653 unter dem Titel \frqq Ein schönes Danklied, welches nach
überstandenem Kummer zu singen\flqq .

\subsection*{ Der Herr, der aller Enden}

Erstdruck 1653 unter dem Titel \frqq Psal. 23\flqq .

\subsection*{ Ich preise dich und singe}

Erstdruck 1653 unter dem Titel \frqq Der 30. Psalm Davids\flqq .

\subsection*{ Ich will erhöhen immerfort}

Erstdruck 1653.

\subsection*{ Ich will mit Danken kommen}

Erstdruck 1653 unter dem Titel \frqq Der 111. Psalm\flqq .

\subsection*{ Das ist mir lieb, daß Gott, mein Hort}

Erstdruck 1653 unter dem Titel \frqq Der 116. Psalm Davids\flqq .

\subsection*{ Du meine Seele singe}

Erstdruck 1653.

\subsection*{ Herr, dir trau ich all mein Tage}

Erstdruck 1655 als Anhang einer Leichenpredigt P. Gerhardts auf den
Amtsschreiber Joachim Schröder zu Mittenwalde, unter dem Titel \frqq Der 71.
Psalm, Gesangsweise übersetzet, auff die Meloden: Du o schönes Welt
Gebäude\flqq .

\subsection*{ Wie ist es möglich, höchstes Licht?}

Erstdruck 1667 unter dem Titel \frqq Gott allein die Ehre\flqq .

\subsection*{ Merkt auf, merkt, Himmel, Erde}

Entstanden um 1648. Erstdruck 1666 unter dem Titel \frqq Das Lied Mosis, aus
dem 32. Capitel des fünften Buchs Mose\flqq .

\subsection*{ Ich, der ich oft in tiefes Leid}

Erstdruck 1666 unter dem Titel \frqq Der 145. Psalm Davids\flqq .

Amazon.de Widgets 

\subsection*{ Ich danke dir mit Freuden}

Erstdruck 1666/67 unter dem Titel \frqq Dank- Gebetlein Sirachs aus dem 51.
Cap.\flqq 

\subsection*{ Mein Gott, ich habe mir}

Erstdruck 1648 unter dem Titel \frqq Der 39. Psalm Davids\flqq 

\subsection*{ O Tod, o Tod, du greulichs Bild}

Erstdruck 1667 unter dem Titel \frqq Freudige Empfahung des Todes\flqq .
Bearbeitung von Paul Röbers Lied \frqq O Tod, o Tod, schreckliches Bild\flqq .

\subsection*{ Mein herzer Vater, weint ihr noch?}

Erstdruck 1650 mit der Leichenpredigt auf den Sohn des Rektors Adam
Spengler.

\subsection*{ Du bist zwar mein und bleibest mein}

Erstdruck 1650 im Anhang der Leichenpredigt von Georg Lilie auf
Constantin Andreas Berkow unter dem Titel \frqq Der betrübte Vater tröstet
sich über seinem nunmehr seligen Sohn\flqq .

\subsection*{ Nun, du lebest, unsre Krone}

Erstdruck 1650 im Anhang der Leichenpredigt auf den Hofkammergerichtsrat
Petrus Fritz (gest. 1648), unter dem Titel \frqq Trostgesang derer, so über
den Hintritt des sel. Herrn D. Fritzens betrübet worden\flqq .

\subsection*{ Erhebe dich, betrübtes Herz}

Erstdruck 1650 im Anhang der Leichenpredigt auf den Hofkammergerichtsrat
Petrus Fritz (gest. 1648), unter dem Titel \frqq Trostgesang derer, so über
den Hintritt des sel. Herrn D. Fritzens betrübet worden\flqq .

\subsection*{ Die Zeit ist nunmehr nah}

Entstanden im 30-jährigen Krieg oder anläßlich des Erscheinens eines
Kometen 1652. Erstdruck 1653.

\subsection*{ Leid ist mirs in meinem Herzen}

Erstdruck 1659 im Anhang der Leichenrede von Georg Lilie auf die Tochter
des Diakons Georg Heintzelmann, unter dem Titel \frqq Auf das zwar
frühzeitige aber dennoch selige Abscheiden des Tugend und Gottliebenden
Jungfräuleins Elisabeth Heintzelmans\flqq .

\subsection*{ Herr Lindholtz legt sich hin}

Erstdruck 1659 mit der Leichenpredigt von Chr. Nikolai auf Chr.
Lindholtz unter dem Titel \frqq Auff das selige Absterben Herrn Chr.
Lindholtzes\flqq .

\subsection*{ Liebes Kind, wenn ich bei mir}

Erstdruck 1660 mit dem Leich-Sermon P. Gerhardts unter dem Titel \frqq Auff
das frühzeitige doch wohlselige Absterben deß bald zur Vollkommenheit
gelangten Knabens Friedrich Ludowig Zarlanges\flqq .

\subsection*{ O, wie so ein großes Gut}

Erstdruck 1661 mit der Leichenrede Joh. Rosners auf Frau U. von der
Linden unter dem Titel \frqq Dem Herrn Land-Rentmeister Herrn Chr. von der
Linden. Als desselben hertzgeliebte Haus-Ehre Fr. Ursula Monsin selig im
Herrn entschlaffen\flqq .

\subsection*{ Nun sei getrost und unbetrübt}

Erstdruck 1664 mit der Leichenpredigt Joh. Leißners auf Frau Regina
Leyser, geb. Calow, unter dem Titel \frqq Fröhliche Ergebung zu einem seligen
Abschiede aus dieser müheseligen Welt\flqq .

\subsection*{ Hörst du hier die Ewigkeit?}

Erstdruck 1664 am Schluß von J. Paulis \frqq Vorschmack der Traurigen und
frölichen Ewigkeit\flqq .

\subsection*{ Herr Gott, du bist ja für und für}

Erstdruck 1667 unter dem Titel \frqq Vom Tod und Sterben. Aus den 90. Psal.
Davids\flqq .

\subsection*{ Ich bin ein Gast auf Erden}

Erstdruck 1666 unter dem Titel \frqq Auß dem 119. Psalm Davids\flqq .

\subsection*{ Was trauerst du, mein Angesicht}

Erstdruck 1666 unter dem Titel \frqq Christliche Todes-Freude\flqq .

\subsection*{ Ich weiß, daß mein Erlöser lebt}

Erstdruck 1667.

\subsection*{ Weint, und weint gleichwohl nicht zu sehr}

Erstdruck 1667 mit der Leichenpredigt Gerhardts für M. Zarlang mit der
Überschrift \frqq Bey frühzeitigem Absterben des frommen hertz-lieben
Töchterleins, Margritgen Zarlanges, an desselben hertzlichgekränckte und
hoch-betrübte Eltern\flqq .

\subsection*{ So geht der alte liebe Herr nun auch dahin}

Erstdruck 1667 unter dem Titel \frqq Auff das selige Absterben und christl.
Beerdigung des umb diese gantze Stadt viel Jahr lang wolverdienten Herrn
Bürgermeisters, Herrn Benedicti Reichardts\flqq .

\subsection*{ Wer selig stirbt, stirbt nicht}

Erstdruck 1668 mit der Leichenrede Gerhardts auf Joh. Adam Preunel.

\subsection*{ Johannes sahe durch Gesicht}

Erstdruck 1668 mit der Leichenrede Gerhardts auf Joh. Adam Preunel.

