%StartInfo%%%%%%%%%%%%%%%%%%%%%%%%%%%%%%%%%%%%%%%%%%%%%%%%%%%%%%%%%%%%%%%%%%%%
%  Autor:
%  Titel:
%  File:
%  Ref:
%  Mod:
%EndInfo%%%%%%%%%%%%%%%%%%%%%%%%%%%%%%%%%%%%%%%%%%%%%%%%%%%%%%%%%%%%%%%%%%%%%%
%\poemtitle{Du liebe Unschuld du}
\begin{multicols}{2}
\settowidth{\versewidth}{Du strafst der Bösen Werk}
\begin{verse}[\versewidth]

  \flagverse{1.} Du liebe Unschuld du,\\
  wie schlecht wirst du geacht't!\\
  Wie oftmals wird dein Tun\\
  von aller Welt verlacht!\\
  Du dienest deinem Gott,\\
  hältst dich nach seinen Worten,\\
  darüber höhnt man dich\\
  und drückt dich aller Orten.

  \flagverse{2.} Du gehst geraden Weg,\\
  fleuchst von der krummen Bahn,\\
  ein andrer tut sich zu\\
  und wird ein reicher Mann,\\
  vermehrt sein kleines Gut,\\
  füllt Kästen, Böden, Scheunen;\\
  du bleibst ein armer Tropf\\
  und darbest mit den Deinen.

  \flagverse{3.} Du strafst der Bösen Werk\\
  und sagst, was Unrecht sei.\\
  Ein Andrer braucht die Kunst\\
  der süßen Heuchelei;\\
  die bringt ihm Lieb und Huld\\
  und hebt ihn auf die Höhen,\\
  du aber bleibst zurück\\
  und mußt da unten stehen.

  \flagverse{4.} Du sprichst, die Tugend sei\\
  der Christen schönste Kron;\\
  hingegen hält die Welt\\
  auf Reputation:\\
  Wer diese haben will, sagt sie,\\
  der muß gar eben\\
  sich schicken in die Zeit\\
  und gleich den andern leben.

  \flagverse{5.} Du rühmest viel von Gott\\
  und streichst gewaltig aus\\
  den Segen, den er schickt\\
  in seiner Kinder Haus.\\
  Ist diesem nun also, spricht man,\\
  so laß doch sehen,\\
  was dir denn ist für Guts,\\
  für Glück und Heil geschehen.

  \flagverse{6.} Halt fest, o frommes Herz,\\
  halt fest und bleib getreu\\
  in Widerwärtigkeit,\\
  denn Gott, der steht dir bei;\\
  laß diesen deine Sach handhaben,\\
  schützen, führen,\\
  so wirst du wohl bestehn\\
  und endlich triumphieren.

\vfill\null
\columnbreak

  \flagverse{7.} Gefällst du Menschen nicht,\\
  das ist ein schlechter Schad;\\
  all gnug ists, wenn du hast\\
  des ewgen Vaters Gnad.\\
  Ein Mensch kann doch nicht mehr\\
  als irren, fehlen, lügen;\\
  Gott aber ist gerecht,\\
  sein Urteil kann nicht trügen.

  \flagverse{8.} Spricht er nun: du bist mein,\\
  dein Tun gefällt mir wohl!\\
  Wohlan, so sei dein Herz\\
  getrost und freudenvoll.\\
  Schlag alles in den Wind,\\
  was böse Leute dichten,\\
  sei still und siehe zu:\\
  Gott wird sie balde richten.

  \flagverse{9.} Stolz, Übermut und Pracht\\
  währt in die Länge nicht;\\
  wanns Glas am hellsten scheint,\\
  fällts auf die Erd und bricht,\\
  und wann des Menschen Glück\\
  am höchsten ist gestiegen,\\
  so stürzt es unter sich\\
  und muß zu Boden liegen.

  \flagverse{10.} Das ungerechte Gut,\\
  wers recht und wohl besieht,\\
  ist lauter Zentnerlast,\\
  die Herz, Sinn und Gemüt\\
  ohn Unterlaß beschwert,\\
  Seel und Gewissen dringet\\
  und aus der sanften Ruh\\
  in schweres Leiden bringet.

  \flagverse{11.} Was hat doch mancher mehr\\
  als armer Leute Schweiß?\\
  Was ißt und trinket er?\\
  Worin besteht sein Preis\\
  als im geraubten Gut\\
  und armer Leute Tränen,\\
  die wie ein dürres Land\\
  sich nach Erquickung sehnen?

  \flagverse{12.} Heißt das nun selig sein?\\
  Ist das nun Herrlichkeit?\\
  O, Welch ein hartes Wort\\
  wird über solche Leut\\
  am Tage des Gerichts\\
  aus Gottes Thron erschallen!\\
  Wie schändlich wird ihr Ruhm\\
  und großes Prahlen fallen!

\vfill\null
\columnbreak

  \flagverse{13.} Du aber, der du Gott\\
  von ganzem Herzen ehrst\\
  und deine Füße nicht\\
  von seinem Wege kehrst,\\
  wirst in der schönen Schar,\\
  die Gott mit Manna weidet,\\
  hergehn, mit Lob und Ehr\\
  als einem Rock gekleidet.

  \flagverse{14.} Drum fasse deine Seel\\
  ein wenig mit Geduld,\\
  fahr immer fort, tu recht,\\
  leb außer Sündenschuld;\\
  halt, daß den höchsten Schatz\\
  dort in dem andern Leben\\
  des Höchsten milde Hand\\
  dir werd aus Gnaden geben.

\end{verse}
\end{multicols}


\begin{center}
\settowidth{\versewidth}{Was hier ist in der Welt,}
\begin{verse}[\versewidth]

  \flagverse{15.} Was hier ist in der Welt,\\
  da sei nur unbemüht,\\
  wird dirs ersprießlich sein,\\
  wies Gott am besten sieht,\\
  so glaube du gewiß,\\
  er wird dir deinen Willen\\
  schon geben und mit Freud\\
  all dein Begehren stillen.
  
\end{verse}
\end{center}


%\attrib{\small{THZE}}
