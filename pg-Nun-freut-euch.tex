%StartInfo%%%%%%%%%%%%%%%%%%%%%%%%%%%%%%%%%%%%%%%%%%%%%%%%%%%%%%%%%%%%%%%%%%%%
%  Autor:
%  Titel:
%  File:
%  Ref:
%  Mod:
%EndInfo%%%%%%%%%%%%%%%%%%%%%%%%%%%%%%%%%%%%%%%%%%%%%%%%%%%%%%%%%%%%%%%%%%%%%%
%\poemtitle{pt}
\begin{multicols}{2}
\settowidth{\versewidth}{der sprach: Habt Freud und Trost und seid}
\begin{verse}[\versewidth]

\flagverse{1.} Nun freut euch hier und überall,\\
ihr Christen, lieben Brüder!\\
Das Heil, das durch den Todesfall\\
gesunken, stehet wieder.\\
Des Lebens Leben lebet noch,\\
sein Arm hat aller Feinde Joch\\
mit aller Macht zerbrochen.

\flagverse{2.} Der Held, der alles hält, er lag\\
im Grab als überwunden,\\
er lag, bis daß der dritte Tag\\
sich in die Welt gefunden;\\
da dieser kam, kam auch die Zeit,\\
da, der uns in dem Tod erfreut,\\
sich aus dem Tod erhube.

\flagverse{3.} Die Morgenröte war noch nicht\\
mit ihrem Licht vorhanden,\\
und siehe, da war schon das Licht,\\
das ewig leucht', erstanden;\\
die Sonne war noch nicht erwacht,\\
da wacht und ging in voller Macht\\
die unerschaffne Sonne.

\flagverse{4.} Das wußte nicht die fromme Schar,\\
die Christo angehangen,\\
drum als nunmehr der Sabbat war\\
zum End hinabgegangen,\\
begunnt Maria Magdalen\\
und andre mit ihr auszugehn\\
und Spezerei zu kaufen.

\flagverse{5.} Ihr Herz und Hand ist hoch bemüht,\\
ein Salböl darzugeben\\
für Jesu, dessen teure Güt\\
uns salbt zum ewgen Leben.\\
Ach, liebes Herz, der seinen Geist\\
vom Himmel in die Herzen geußt,\\
darf keines Öls noch Salben.

\flagverse{6.} Ja du, o heilger Jungfrausohn,\\
bist schon gnug balsamieret\\
als König, der im Himmelsthron\\
und überall regieret!\\
Dein Balsam ist die ewge Kraft,\\
dadurch Gott Erd und Himmel schafft,\\
die läßt dich nicht verwesen.

\flagverse{7.} Doch geht die fromme Einfalt hin\\
bald in dem frühsten Morgen,\\
sie gehn, und plötzlich wird ihr Sinn\\
voll großer schwerer Sorgen.\\
Ei, sprechen sie, wer wälzt den Stein\\
vons Grabes Tür und läßt uns ein\\
zum Leichnam unsres Herren? –

\flagverse{8.} So sorgten sie zur selben Zeit\\
für das, was schon bestellet,\\
es war der Stein ja allbereit\\
erhoben und gefället\\
durch einen, der des Erdreichs Wucht\\
erbeben macht und in die Flucht\\
des Grabes Hüter jagte.

\flagverse{9.} Das war ein Diener aus der Höh,\\
von denen, die uns schützen,\\
sein Kleid war weißer als der Schnee,\\
sein Ansehn gleich den Blitzen,\\
der hat das fest verschlossne Grab\\
eröffnet und den Stein herab\\
vons Grabes Tür gewälzet.

\flagverse{10.} Das Weiberhäuflein kam und ging\\
hinein ohn alle Mühe.\\
Hör aber, was für Wunderding\\
sich da begab! Denn siehe,\\
das, was sie suchten, findt sich nicht\\
und wo ihr Herz nicht hingericht,\\
das ist allda zur Stelle.

\flagverse{11.} Sie suchten ihrer Seelen Hort\\
und finden sein Gesinde,\\
sie hören aus der Engel Wort\\
wies gar viel anders stünde,\\
als ihr betrübtes Herz gemeint:\\
Daß billig wer bisher geweint,\\
nun jauchzen soll und lachen.

\flagverse{12.} Sie sehn das Grab entledigt stehn,\\
und als sie das gesehen,\\
da läuft Maria Magdalen,\\
zu sagen, was geschehen.\\
Die andre Schar ist Kummers voll\\
und weiß nicht, was sie machen soll,\\
verharret bei dem Grabe.

\flagverse{13.} Da stellen sich in heller Zier\\
zween edle Himmelsboten,\\
die sprechen: Ei, was suchet ihr\\
das Leben bei den Toten?\\
Der Heiland lebt! Er ist nicht hie!\\
Heut ist er, glaubt uns, heute früh\\
ist er vom Tod erstanden.

\flagverse{14.} Gedenkt und sinnt ein wenig nach\\
den Reden, die er triebe,\\
da er so klar und deutlich sprach,\\
wie er zwar würd aus Liebe\\
den Tod ausstehn und große Plag,\\
jedennoch an dem dritten Tag\\
er herrlich triumphieren.

\flagverse{15.} Da dachten sie an Christi Wort\\
und gingen von dem Grabe\\
hin zu der elf Apostel Ort\\
und sagten, was sich habe\\
erzeigt in ihrem Angesicht;\\
man hielt es aber anders nicht,\\
als ob es Märlein wären.

\flagverse{16.} Maria, die betrübt', sich gibt\\
in schnelles Abescheiden,\\
findt Petrum und den Jesus liebt,\\
erzählet allen beiden:\\
Ach, spricht sie, unser Herr ist hin,\\
und niemand ist, der, wo man ihn\\
hab hingelegt, will wissen.

\flagverse{17.} Der Hochgeliebte läuft geschwind\\
und kommt zuerst zum Grabe;\\
er guckt, und da er nichts mehr findt\\
als Leinen, weicht er abe.\\
Da aber Simon Petrus kömmt,\\
geht er ins Grab hinein und nimmt\\
das Werk recht in die Augen.

\flagverse{18.} Er sieht die Leinen für sich dar,\\
zu voraus, wie mit Fleiße\\
gelegt und eingewickelt war\\
das Haupttuch zu dem Schweiße:\\
Da ging auch, der am ersten kam,\\
hinein, wie Petrus tät, und nahm,\\
was er da sah ins Herze.

\flagverse{19.} Da glauben sie nun dem Bericht,\\
weil sie mit Augen schauen,\\
was sie zuvor als ein Gedicht\\
gehöret von den Frauen;\\
doch werden sie Verwunderns voll,\\
denn keiner weiß, daß Christus soll\\
von Toten auferwachen.

\flagverse{20.} Maria steht vorm Grab und weint,\\
und plötzlich wird sie innen,\\
daß zween in weißen Kleidern seind\\
vor ihr im Grabe drinnen,\\
die sprechen: Weib, was weinest du?\\
Sie haben meines Herzens Ruh,\\
sprach sie, hinweggenommen.

\flagverse{21.} Mein Herr ist weg, und ich weiß nicht,\\
wo ich soll suchen gehen.\\
Indessen wendt sie ihr Gesicht\\
und siehet Jesum stehen.\\
Der spricht: O Weib, was fehlet dir?\\
Was weinest du, was suchst du hier? –\\
sie meint, der Gärtner rede.

\flagverse{22.} Ach, spricht sie, Herr, hast du's getan,\\
so sag es unverhohlen,\\
wo liegt mein Herr? Wo komm ich an?\\
So will ich mir ihn holen.\\
Der Herr spricht mit gewohnter Stimm:\\
Maria! – Da wendt sie sich um\\
und spricht: Sieh da, Rabbuni!

\flagverse{23.} Rühr mich nicht an! Ich bin noch nicht\\
zum Vater aufgefahren,\\
geh aber hin, sprach unser Licht,\\
sags meiner Brüder Scharen:\\
Ich fahr als eures Todes Tod\\
zu meinem und zu eurem Gott\\
und unser aller Vater.

\flagverse{24.} Maria ist das arme Weib,\\
von welcher unser Meister,\\
der starke Helfer, vormals treib\\
auf einmal sieben Geister.\\
Die, die ists, welcher Jesus Christ\\
am ersten Mal erschienen ist\\
am heilgen Ostertage.

% weiter in der anderen Spalte:
\vfill\null
\columnbreak
 
\flagverse{25.} Nun, sie ging hin, täts denen kund,\\
die mit ihr Jesum liebten\\
und über ihn von Herzensgrund\\
sich grämten und betrübten.\\
Kein einzger aber fiel ihr bei,\\
ein jeder hielts für Fantasei,\\
und wollt es niemand glauben.

\flagverse{26.} Es gingen auch ins Grab hinein\\
die andre Schar der Frauen,\\
da gab sich ihrem Augenschein\\
ein Jüngling anzuschauen\\
in einem langen weißen Kleid,\\
der sprach: Habt Freud und Trost und seid\\
ohn alle Furcht und Schrecken.

\flagverse{27.} Ihr sucht den Held von Nazareth,\\
der doch hie nicht vorhanden;\\
seht, das ist seines Lagers Stätt,\\
von der er auferstanden.\\
Geht schnell, sagts Petro und der Zahl\\
der andern Jünger allzumal:\\
Ihr Herr und Meister lebe.

\flagverse{28.} Die Weiber eilen schnell davon,\\
den Jüngern Post zu bringen,\\
und siehe da, die Freudensonn,\\
nach der sie alle gingen,\\
die geht daher, und sehen sie\\
im Leben, den sie also früh\\
als einen Toten suchten.

\flagverse{29.} Sein süßer Mund macht all ihr Leid\\
mit seinem Grüßen süße,\\
sie treten zu mit großer Freud\\
und greifen seine Füße.\\
Er aber spricht: Seid guten Muts!\\
Geht hin, sagt meinen Brüdern Guts,\\
verrichtet, was ihr sahet.

\flagverse{30.} Sprecht, daß sie nunmehr also fort\\
in Galiläum gehen,\\
allda will ich, kraft meiner Wort,\\
vor ihren Augen stehen. –\\
und hiemit schloß er sein Gebot.\\
Die Weiber gehn und loben Gott,\\
verrichten, was befohlen.

\flagverse{31.} O Lebensfürst, o starker Leu\\
aus Judä Stamm erstanden,\\
so bist du nun wahrhaftig frei\\
von Todes Strick und Banden.\\
Du hast gesiegt und trägst zu Lohn\\
ein allzeit unverwelkte Kron\\
als Herr all deiner Feinde.

\flagverse{32.} Was fragst du nach des Teufels Spott\\
und ungereimten Klagen!\\
Man hat, spricht er und seine Rott,\\
ihn heimlich weggetragen.\\
Die Jünger haben ihn bei Nacht\\
gestohlen und bei Seit gebracht,\\
indem wir feste schliefen.

% weiter in der anderen Spalte:
\vfill\null
\columnbreak


\flagverse{33.} O Bosheit! War dein Schlaf so fest,\\
wie hast du können sehen?\\
Ist denn dein Auge wach gewest,\\
wie läßt du's so geschehen,\\
daß durch der Jünger schwache Hand\\
der Stein und seines Siegels Band\\
werd auf- und abgelöset?

\flagverse{34.} Es ist dein hart verstockter Sinn,\\
der dich zum Lügen leitet,\\
so fahr auch nun zum Abgrund hin,\\
da dir dein Lohn bereitet!\\
Ich aber will, Herr Jesu Christ,\\
so lang ein Leben in mir ist,\\
bekennen, daß du lebest.

\flagverse{35.} Ich will dich rühmen, wie du seist\\
die Pest und Gift der Höllen,\\
ich will auch, Herr, durch deinen Geist\\
mich dir zur Seiten stellen\\
und mit dir sterben, wie du stirbst,\\
und was du in dem Sieg erwirbst,\\
soll meine Beute bleiben.

\flagverse{36.} Ich will von Sünden auferstehn,\\
wie du vom Grab aufstehest:\\
Ich will zum andern Leben gehn,\\
wie du zum Himmel gehest.\\
Dies Leben ist doch lauter Tod,\\
drum komm und reiß aus aller Not\\
uns in das rechte Leben!

\end{verse}
\end{multicols}
%\attrib{\small{THZE}}
