%StartInfo%%%%%%%%%%%%%%%%%%%%%%%%%%%%%%%%%%%%%%%%%%%%%%%%%%%%%%%%%%%%%%%%%%%%
%  Autor:
%  Titel:
%  File:
%  Ref:
%  Mod:
%EndInfo%%%%%%%%%%%%%%%%%%%%%%%%%%%%%%%%%%%%%%%%%%%%%%%%%%%%%%%%%%%%%%%%%%%%%%
%ANM:\poemtitle{Also hat Gott die Welt geliebt}
\begin{multicols}{2}
\settowidth{\versewidth}{sein Kreuz und Leiden ist mein Schmuck,}
\begin{verse}[\versewidth]
\flagverse{1.} Also hat Gott die Welt geliebt\\
das merke, wer es höret\\
die Welt, die Gott so hoch betrübt,\\
hat Gott so hoch geehret,\\
daß er den eingebornen Sohn,\\
den eingen Schatz, die einge Kron,\\
das einge Herz und Leben\\
mit Willen hingegeben.

\flagverse{2.} Ach wie muß doch ein einges Kind\\
bei uns hier auf der Erden,\\
da man doch nichts als Bosheit findt,\\
so hoch geschonet werden;\\
wie hitzt, wie brennt der Vatersinn,\\
wie gibt und schenkt er alles hin,\\
eh als er an das Schenken\\
des Eingen nur will denken!

\flagverse{3.} Gott aber schenkt, aus freiem Mut\\
und mildem treuem Herzen,\\
sein einges Kind, sein schönstes Gut\\
in mehr als tausend Schmerzen;\\
er gibt ihn in den Tod hinein,\\
ja in die Höll und ewge Pein,\\
zu unerhörtem Leide\\
stößt Gott sein einge Freude!

\flagverse{4.} Warum doch das? Daß du, o Welt,\\
frei wieder möchtest stehen\\
und durch ein teures Lösegeld\\
aus deinem Kerker gehen;\\
denn du weißt wohl, du schnöde Braut,\\
wie, da dich Gott ihm anvertraut,\\
du, wider deinen Orden,\\
ihm allzu untreu worden.

\flagverse{5.} Darüber hat dich Sünd und Tod\\
und Satanas Gesellen\\
zu bittrer Angst und harter Not\\
beschlossen in der Höllen.\\
Und ist hier gar kein andrer Rat\\
als der, den Gott gegeben hat;\\
wer den hat, wird dem Haufen\\
der höllschen Feind entlaufen.

\flagverse{6.} Gott hat uns seinen Sohn verehrt,\\
daß aller Menschen Wesen,\\
so mit dem ewgen Fluch beschwert,\\
durch diesen soll genesen.\\
Wen die Verdammnis hat umschränkt,\\
der soll durch den, den Gott geschenkt,\\
Erlösung, Trost und Gaben\\
des ewgen Lebens haben.

\flagverse{7.} Ach mein Gott, mein Lebens Grund,\\
wo soll ich Worte finden?\\
Mit was für Lobe soll mein Mund\\
dein treues Herz ergründen?\\
Wie ist dir immermehr geschehn?\\
Was hast du an der Welt ersehn,\\
daß, die so hoch dich höhnet,\\
du so gar hoch gekrönet?

\flagverse{8.} Warum behieltst du nicht dein Recht\\
und ließest ewig pressen\\
diejenge, die dein Recht geschwächt\\
und freventlich vergessen?\\
Was hattest du an der für Lust,\\
von welcher dir doch war bewußt,\\
daß sie für dein Verschonen\\
dir schändlich würde lohnen?

\flagverse{9.} Das Herz im Leibe weinet mir\\
vor großem Leid und Grämen,\\
wenn ich bedenke, wie wir dir\\
so gar schlecht uns bequemen.\\
Die Meisten wollen deiner nicht,\\
und was du ihnen zugericht\\
durch deines Sohnes Büßen,\\
das treten sie mit Füßen.

\flagverse{10.} Du, frommer Vater, meinst es gut\\
mit allen Menschenkindern,\\
du ordnest deines Sohnes Blut\\
und reichst es allen Sündern,\\
willst, daß sie mit der Glaubenshand\\
das, was du ihnen zugewandt,\\
sich völlig zu erquicken,\\
fest in ihr Herze drücken.

\flagverse{11.} Sieh aber, ist nicht immerfort\\
dir alle Welt zuwider?\\
Du bauest hier, du bauest dort,\\
die Welt schlägt alles nieder.\\
Darum erlangt sie auch kein Heil,\\
sie bleibt im Tod und hat kein Teil\\
am Reiche, da die Frommen,\\
die Gott gefolgt, hinkommen.

\flagverse{12.} An dir, o Gott, ist keine Schuld,\\
du, du hast nichts verschlafen:\\
Der Feind und Hasser deiner Huld\\
ist Ursach deiner Strafen,\\
weil er den Sohn, der ihm so klar\\
und nah ans Herz gestellet war,\\
auch einzig helfen sollte,\\
durchaus nicht haben wollte.

\flagverse{13.} So fahre hin, du tolle Schar!\\
Ich bleibe bei dem Sohne.\\
Dem geb ich mich, des bin ich gar,\\
und er ist meine Krone.\\
Hab ich den Sohn, so hab ich gnug,\\
sein Kreuz und Leiden ist mein Schmuck,\\
sein Angst ist meine Freude,\\
sein Sterben meine Weide.

\flagverse{14.} Ich freue mich, so oft und viel\\
ich dieses Sohns gedenke.\\
Dies ist mein Lied und Saitenspiel,\\
wann ich mich heimlich kränke,\\
wann meine Sünd und Missetat\\
will größer sein als Gottes Gnad,\\
und wann mir meinen Glauben\\
mein eigen Herz will rauben.

\flagverse{15.} Ei, sprech ich, war mein Gott geneigt,\\
da wir noch Feinde waren,\\
so wird er ja, der kein Recht beugt,\\
nicht feindlich mit mir fahren\\
anitzo, da ich ihm versühnt,\\
da, was ich Böses je verdient,\\
sein Sohn, der nichts verschuldet,\\
so wohl für mich erduldet.

\flagverse{16.} Fehlts hier und dar? Ei unverzagt!\\
Laß Sorg und Kummer schwinden!\\
Der mir das Größte nicht versagt,\\
wird Rat zum Kleinern finden.\\
Hat Gott mir seinen Sohn geschenkt\\
und für mich in den Tod gesenkt:\\
Wie sollt er, laßt uns denken,\\
nicht alles mit ihm schenken!


\end{verse}
\end{multicols}

\begin{center}
\settowidth{\versewidth}{Ich bins gewiß und sterbe drauf:}
\begin{verse}[\versewidth]

\flagverse{17.} Ich bins gewiß und sterbe drauf:\\
Nach meines Gottes Willen\\
mein Kreuz und ganzer Lebenslauf\\
wird sich noch fröhlich stillen.\\
Hier hab ich Gott und Gottes Sohn,\\
und dort bei Gottes Stuhl und Thron:\\
Da wird fürwahr mein Leben.
  
\end{verse}
\end{center}

%\attrib{\small{THZE}}
