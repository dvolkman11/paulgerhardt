%StartInfo%%%%%%%%%%%%%%%%%%%%%%%%%%%%%%%%%%%%%%%%%%%%%%%%%%%%%%%%%%%%%%%%%%%%
%  Autor:
%  Titel:
%  File:
%  Ref:
%  Mod:
%EndInfo%%%%%%%%%%%%%%%%%%%%%%%%%%%%%%%%%%%%%%%%%%%%%%%%%%%%%%%%%%%%%%%%%%%%%%
%\poemtitle{Der aller Herz und Willen lenkt}
\begin{multicols}{2}
\settowidth{\versewidth}{Wie Gott will, brennen auf der Erd}
\begin{verse}[\versewidth]
% Der aller Herz und Willen lenkt
% Hochzeitsgedicht für Joachim Fromm und Sabina Barthold 1643

\flagverse{1.} Der aller Herz und Willen lenkt\\
und wie er will regieret,\\
der ist's, der euch, Herr Bräutgam, schenkt\\
die man euch hier zuführet.\\
Glück zu, Glück zu, ruft jedermann,\\
Gott gebe, daß es sei getan\\
zu beider Wohlergehen!

\flagverse{2.} Wie sollte nicht sein wohlgetan,\\
was Gott denkt zu vollbringen?\\
Sein Will und Rat nicht fehlen kann,\\
es wird ihm nichts mißlingen.\\
Er regt den Mund und spricht ein Wort,\\
so geht das Werk und dringet fort,\\
muß alles wohl geraten.

\flagverse{3.} Wie Gott will, brennen auf der Erd\\
die ehelichen Flammen.\\
Wie eins dem andern ist beschert,\\
so kommen sie zusammen.\\
Im Himmel wird der Schluß gemacht,\\
auf Erden wird das Werk vollbracht:\\
Das gibt ein schönes Leben.

\flagverse{4.} Ein Leben, daß sehr hoch beliebt\\
dem, der es hat erfunden,\\
da er auch seinen Segen gibt\\
und mehret alle Stunden.\\
Das ist und bleibet sein Gebrauch:\\
Was er gestift't, das hält er auch\\
und lässet es nicht fallen.

\flagverse{5.} Die Bäumlein, die man fortgesetzt\\
in wohlbestallten Garten,\\
die pfleget man zur Erst und Letzt\\
vor allem wohl zu warten.\\
Ihr Bäumlein Gottes, freuet euch!\\
Der Gärtner ist von Liebe reich,\\
der ihm euch hat erwählet.

\flagverse{6.} Was er gepflanzt mit seiner Hand,\\
hält er in großen Ehren;\\
sein Sinn und Aug ist stets gewandt,\\
dasselbe zu vermehren,\\
kommt oft und sieht aus reiner Treu,\\
was seines Gartens Zustand sei,\\
was seine Reislein machen.

\flagverse{7.} Und wenn denn unterweilen will\\
ein rauhes Lüftlein wehen,\\
ist er bald da, setzt Maß und Ziel,\\
läßts eilend übergehen.\\
Wenn er betrübt, ists gut gemeint,\\
er stellt sich hart und ist doch Freund\\
voll süßer Gnad und Hulde.

\flagverse{8.} O selig der, wenns Gott gefällt,\\
ein Wölklein einzuführen,\\
ein treues, fröhlich Herz behält,\\
läßt keinen Unmut spüren!\\
Ein Wölklein geht ja bald vorbei,\\
es währt ein Stündlein oder zwei,\\
so kommt die Sonne wieder.

\flagverse{9.} Ein Schifflein, das im Meere läuft,\\
muß manchen Sturm erfahren,\\
und bleibet dennoch überhäuft\\
mit edlem Gut und Waren;\\
es streicht dahin, und Gottes Hand,\\
die führt und bringt es an das Land\\
bei gutem Wind und Wetter.

\flagverse{10.} Ein Röslein, wenns im Lenzen lacht\\
und in den Farben pranget,\\
wird oft vom Regen matt gemacht,\\
daß es sein Köpflein hanget,\\
doch wenn die Sonne leucht herfür,\\
siehts wieder auf und bleibt die Zier\\
und Fürstin aller Blumen.

\flagverse{11.} Wohlan, laß Regen, Reif und Wind\\
bald oder lang ansetzen,\\
wer Gott liebt, bleibet Gottes Kind,\\
kein Fall wird ihn verletzen.\\
Er sitzet in des Vaters Arm,\\
er gibt ihm Schutz, der hält ihn warm,\\
und spricht: Sei unerschrocken!

\flagverse{12.} Wer fromm ist, hat schon großen Teil\\
der Wohlfahrt in den Händen,\\
Gott gönnt ihm Guts und kann sein Heil\\
von ihme nicht abwenden.\\
Herr Fromm ist fromm, das weiß man wohl,\\
drum er nichts anders haben soll\\
als lauter Glück und Freude.

\flagverse{13.} Die auch, die ihm zur Seiten geht\\
und die Gott selbst gezieret,\\
was Menschenseelen wohl ansteht\\
und Himmelsgunst gebieret;\\
was Tugend bringt, was Tugend heißt,\\
was Tugend auch selbst lobt und preist,\\
das findt sich hier beisammen:

\flagverse{14.} Ein züchtig Herz, ein reiner Mut,\\
von denen angeboren,\\
die ihnen Gottesfurcht zum Gut\\
und Schätzen auserkoren.\\
Was ist doch gut ohn diesem Gut?\\
Wenn dies Gut nicht im Herzen ruht,\\
ist alles Gut verworfen.

\flagverse{15.} Die Augen Gottes sehen bald,\\
die ihm sein Herz erfreuen,\\
wen er nun findet recht gestalt,\\
dem gibt er sein Gedeihen,\\
ja schütts mit vollen Händen aus,\\
da wird denn ein gesegntes Haus,\\
dems nicht kann übel gehen.

\flagverse{16.} Und dieses wird o edles Paar,\\
euch beiden auch geschehen!\\
Was Gott Verspricht, ist ja und wahr,\\
man wirds mit Augen sehen.\\
Es fehlt ihm nicht an Gütigkeit,\\
auch fehlts ihm nicht an Möglichkeit,\\
wie sollt er Guts versagen?

\flagverse{17.} So gehet nun mit Freuden ein\\
zu eurem Stand und Orden!\\
Der Weg Wird ohne Schaden sein,\\
der euch gezeuget worden:\\
Es geht ein Englein vornen an,\\
und wo es geht, bestreuts die Bahn\\
mit Rosen und Violen.

\flagverse{18.} Ein einzig Wunsch vermag den Saal\\
des Himmels durch zu dringen,\\
hier gehn die Wünsch in voller Zahl,\\
sie werden Gutes bringen:\\
Der Frommen Lohn, der euch bereit,\\
euch, die ihr tragt die Frömmigkeit\\
im Herzen und im Namen.

\end{verse}
\end{multicols}
%\attrib{\small{THZE}}
