%StartInfo%%%%%%%%%%%%%%%%%%%%%%%%%%%%%%%%%%%%%%%%%%%%%%%%%%%%%%%%%%%%%%%%%%%%
%  Autor:
%  Titel:
%  File:
%  Ref:
%  Mod:
%EndInfo%%%%%%%%%%%%%%%%%%%%%%%%%%%%%%%%%%%%%%%%%%%%%%%%%%%%%%%%%%%%%%%%%%%%%%
%\poemtitle{pt}
\begin{multicols}{2}
\settowidth{\versewidth}{Ich habs verdient, was will ich doch}
\begin{verse}[\versewidth]
%ich habs verdient, was will ich doch mich wider Gott viel sperren?\\
%(Micha 7)

\flagverse{1.} Ich habs verdient, was will ich doch\\
mich wider Gott viel sperren?\\
Komm immer her, du Kreuzesjoch\\
und bittrer Kelch des Herren!\\
Ohn Angst und Pein\\
mag der nicht sein,\\
der wider Gott gehandelt,\\
wie ich getan,\\
da ich die Bahn\\
der schnöden Welt gewandelt.

\flagverse{2.} Ich will des Herren Straf und Zorn\\
mit willgem Herzen tragen,\\
in Sünden bin ich ja geborn,\\
hab auch im Sündenwagen\\
mit eitler Freud\\
oft meine Zeit\\
ganz liederlich verzehret,\\
gott, meinen Hort,\\
in seinem Wort\\
nicht, wie ich soll, gehöret.

\flagverse{3.} Ich habe den gebahnten Steg\\
verlassen und geliebet\\
den gottvergessnen Irreweg;\\
drum wird auch nun betrübet\\
mein Herz und Mut\\
durch Gottes Rut;\\
er hält ein recht Gerichte\\
vor seinem Thron,\\
gibt Sold und Lohn\\
mit völligem Gewichte.

\flagverse{4.} Gott ist gerecht, doch auch dabei\\
sehr fromm und voller Güte,\\
die Vaterlieb und Muttertreu,\\
die wohnt ihm im Gemüte.\\
Gott zürnet nicht,\\
wie wohl geschicht\\
bei uns hier auf der Erden,\\
da mancher Mann\\
nicht wieder kann\\
zur Sühn erweichet werden.

\flagverse{5.} Nein, traun! Das ist nicht Gottes Sinn,\\
sein Zorn der hat ein Ende,\\
wann wir uns bessern, fällt er hin\\
und macht die strengen Hände\\
sanft und gelind,\\
hört auf, die Sünd\\
hier bei uns heimzusuchen;\\
gott kehrt den Grimm\\
mit Gnaden üm\\
und segnet nach dem Fluchen.

\flagverse{6.} Das wird fürwahr auch mir geschehn!\\
Es solls ein jeder spüren.\\
Gott wird einmal zum Rechten sehn\\
und meine Sach ausführen.\\
Sein Angesicht\\
wird mich ans Licht\\
aus meiner Höhle bringen,\\
daß seine Treu\\
ich frisch und frei\\
erzählen mög und singen.

\flagverse{7.} Drum freut euch nicht, ihr meine Feind,\\
ob ich darniederliege,\\
denn mein Gott wird, eh ihr vermeint,\\
mir helfen, daß ich siege.\\
Sein heilge Hand\\
wird meinen Stand\\
schon wieder feste gründen;\\
es wird sich Freud\\
und gute Zeit\\
nach trübem Wetter finden.

\flagverse{8.} Ich bin in Not und weiß doch nicht\\
von rechter Not zu sagen,\\
denn Gott ist meines Herzens Licht;\\
wo das ist, muß es tagen\\
auch in der Nacht,\\
da sich die Macht\\
der Finsternis vermehret.\\
Wann dieses Licht\\
mir scheint, so bricht\\
und fällt, was mich beschweret.

\end{verse}
\end{multicols}



\begin{center}
\settowidth{\versewidth}{Der, vor dem die Welt erschrickt,}
\begin{verse}[\versewidth]


\flagverse{9.} Es kommt die Zeit und ist nicht weit,\\
da will ich jubilieren;\\
der aber, der mich jetzt verspeit\\
und pfleget zu vexieren\\
in meiner Not:\\
Wo ist dein Gott?\\
Der wird mit Schanden stehen;\\
er wird mit Hohn,\\
ich mit der Kron\\
der Ehren davon gehen.

\end{verse}
\end{center}

%\attrib{\small{THZE}}
