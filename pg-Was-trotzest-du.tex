%StartInfo%%%%%%%%%%%%%%%%%%%%%%%%%%%%%%%%%%%%%%%%%%%%%%%%%%%%%%%%%%%%%%%%%%%%
%  Autor:
%  Titel:
%  File:
%  Ref:
%  Mod:
%EndInfo%%%%%%%%%%%%%%%%%%%%%%%%%%%%%%%%%%%%%%%%%%%%%%%%%%%%%%%%%%%%%%%%%%%%%%
%\poemtitle{pt}
\begin{multicols}{2}
\settowidth{\versewidth}{Dein Dichten, dein Trachten, dein Tun}
\begin{verse}[\versewidth]
%der 52. Psalm\\
%was trotzest du, stolzer Tyrann?

\flagverse{1.} Was trotzest du, stolzer Tyrann,\\
daß deine verkehrte Gewalt\\
den Armen viel Schaden tun kann?\\
Verkreuch dich und schweige nur bald!\\
Denn Gottes, des Ewigen Güte\\
bleibt immer in voller Blüte\\
und währet noch täglich und stehet,\\
ob alles gleich sonsten vergehet.

\flagverse{2.} Die Zunge, dein schändliches Glied,\\
du falscher verlogener Mund,\\
tut manchen gefährlichen Schnitt,\\
schlägt alles zu Schanden und wund;\\
was unrecht, das sprichst du mit Freuden,\\
was recht ist, das kannst du nicht leiden,\\
die Wahrheit verdrückst du, die Lügen\\
muß Oberhand haben und siegen.

\flagverse{3.} Dein Dichten, dein Trachten, dein Tun\\
ist einzig auf Schaden bedacht;\\
da ist dir unmöglich zu ruhn,\\
du habest denn Böses verbracht;\\
dein Rachen sucht lauter Verderben,\\
und wenn nur viel Fromme ersterben\\
von deiner vergällten Zungen,\\
so meinst du, es sei dir gelungen.

\flagverse{4.} Drum wird dich auch Gottes Gericht\\
zerstören, verheeren im Grimm;\\
die Rechte, die alles zerbricht\\
mit Donner und blitzender Stimm,\\
die wird dich zugrunde zuschlagen\\
und wird dich mit schrecklichen Plagen\\
aus deinem bisherigen Bleiben\\
samt allen den Deinen vertreiben.

\flagverse{5.} Das werden mit Freuden und Lust\\
die Frommen, Gerechten ersehn,\\
die anders bisher nicht gewußt,\\
als ob es nun gänzlich geschehn;\\
die werden mit Schrecken da stehen,\\
wenn jene zugrunde vergehen,\\
und endlich mit heiligem Lachen\\
sich wiederum lustig bei machen.

\flagverse{6.} Ei siehe! wirds heißen, da liegt\\
der prächtige, mächtige Mann,\\
der stetig mit Erden vergnügt,\\
der Himmel beiseite getan;\\
vom Reichtum war immer sein Prangen,\\
und wann er die Unschuld gefangen,\\
so hielt ers für treffliche Taten;\\
ei siehe, wie ists ihm geraten!

\flagverse{7.} Ich hoffe mit freudigem Geist\\
ein anders und besseres Glück,\\
denn was mir mein Vater verheißt,\\
das bleibet doch nimmer zurück.\\
Ich werde des Friedens genießen,\\
auch wird sich der Segen ergießen\\
und mich mit erwünschtem Gedeihen\\
samt allen den meinen erfreuen.

\flagverse{8.} Ich werde nach Weise des Baums,\\
der Öle trägt, grünen und blühn,\\
mich freuen des seligen Raums,\\
den ohne mein eignes Bemühn\\
mein Herrscher, mein Helfer, mein Leben\\
mir selber zu eigen gegeben\\
im Hause, da täglich mit Loben\\
sein Name wird herrlich erhoben.

\end{verse}
\end{multicols}
%\attrib{\small{THZE}}

\begin{center}
\settowidth{\versewidth}{Der, vor dem die Welt erschrickt,}
\begin{verse}[\versewidth]


\flagverse{9.} Trotz sei dir, du trotzender Kot!\\
Ich habe den Höchsten bei mir;\\
wo der ist, da hat es nicht Not,\\
und fürcht ich mich gar nicht vor dir.\\
Du, mein Gott, kannst alles wohl machen,\\
dich setz ich zum Richter der Sachen,\\
und weißt es: es wird sich mein Leiden\\
bald enden in Jauchzen und Freuden.

  
\end{verse}
\end{center}

