%StartInfo%%%%%%%%%%%%%%%%%%%%%%%%%%%%%%%%%%%%%%%%%%%%%%%%%%%%%%%%%%%%%%%%%%%%
%  Autor:
%  Titel:
%  File:
%  Ref:
%  Mod:
%EndInfo%%%%%%%%%%%%%%%%%%%%%%%%%%%%%%%%%%%%%%%%%%%%%%%%%%%%%%%%%%%%%%%%%%%%%%
%\poemtitle{pt}
\begin{multicols}{2}
\settowidth{\versewidth}{Wenn mein Geblüt entbrennt,}
\begin{verse}[\versewidth]
%der 39. Psalm\\
%mein Gott, ich habe mir gar fest gesetzet für

\flagverse{1.} Mein Gott ich habe mir\\
gar fest gesetzet für,\\
ich will mich fleißig hüten,\\
wenn meine Feinde wüten,\\
daß, wenn ich ja was spreche,\\
ich dein Gesetz nicht breche.

\flagverse{2.} Wenn mein Geblüt entbrennt,\\
so hab ich mich gewöhnt,\\
vor deinen Stuhl zu treten,\\
laß Herz und Zunge beten;\\
Herr, zeige deinem Knechte,\\
zu tun nach deinem Rechte.

\flagverse{3.} Herr, lehre mich doch wohl\\
bedenken, daß ich soll\\
einmal von dieser Erden\\
hinweg geraffet werden,\\
und daß mir deine Hände\\
gesetzet Zeit und Ende.

\flagverse{4.} Die Tage meiner Zeit\\
sind eine Hande breit,\\
und wenn man dies mein Bleiben\\
soll recht und wohl beschreiben,\\
so ists ein Nichts und bleibet\\
ein Stäublein, das zerstäubet.

\flagverse{5.} Ach, wie so gar nichts wert\\
sind Menschen auf der Erd,\\
die doch so sicher leben\\
und gar nicht Acht drauf geben,\\
daß all ihr Tun und Glücke\\
verschwind im Augenblicke.

\flagverse{6.} Sie gehen in der Welt\\
und suchen Gut und Geld,\\
der Schatten einen Schemen!\\
Und können nichts mitnehmen,\\
wenn nach der Menschen Weise\\
sie tun des Todes Reise.

\flagverse{7.} Sie schlafen ohne Ruh,\\
arbeiten immerzu,\\
sind Tag und Nacht geflissen,\\
und können doch nicht wissen,\\
wer, wenn sie niederliegen,\\
ihr Erbe werde kriegen.

\flagverse{8.} Nun, Herr, wo soll ich hin?\\
Wer tröstet meinen Sinn?\\
Ich komm an deine Pforten,\\
der du mit Werk und Worten\\
erfreuest, die dich scheuen\\
und dein allein sich freuen.

\flagverse{9.} Wenn sich mein Feind erregt\\
und mir viel Dampfs anlegt,\\
so will ich stille schweigen,\\
mein Herz zur Ruhe neigen;\\
du Richter aller Sachen,\\
du kannst und wirsts wohl machen.

\flagverse{10.} Wenn du dein Hand ausstreckst,\\
des Menschen Herz erschreckst,\\
wenn du die Sünd heimsuchest,\\
den Sünder schiltst und fluchest:\\
So geht in einer Stunde\\
all Herrlichkeit zugrunde.

\flagverse{11.} Der schönen Jugend Kranz,\\
der roten Wangen Glanz\\
wird wie ein Kleid verzehret,\\
so hier die Motten nähret.\\
Ach, wie gar nichts im Leben\\
sind die auf Erden schweben!

\flagverse{12.} Du aber, du mein Hort,\\
du bleibest fort und fort\\
mein Helfer, siehst mein Sehnen,\\
mein Angst und heiße Tränen,\\
erhörest meine Bitte,\\
wenn ich mein Herz ausschütte.

\flagverse{13.} Drum ruhet mein Gemüt\\
allein auf deiner Güt;\\
ich laß dein Herze sorgen,\\
als deme nicht verborgen,\\
wie meiner Feinde Tücke\\
du treiben sollst zurücke.

\flagverse{14.} Ich bin dein Knecht und Kind,\\
dein Erb und Hausgesind,\\
dein Pilgrim und dein Bürger,\\
der, wenn der Menschenwürger\\
mein Leben mir genommen,\\
zu dir gewiß wird kommen.

\flagverse{15.} Zur Welt muß ich hinaus,\\
der Himmel ist mein Haus,\\
da in den Engelscharen\\
mein Eltern und Vorfahren,\\
auch Schwestern, Freund und Brüder\\
jetzt singen ihre Lieder.

\flagverse{16.} Hie ist nur Qual und Pein,\\
dort, dort wird Freude sein!\\
Dahin, wenn es dein Wille,\\
ich fröhlich, sanft und stille\\
aus diesen Jammerjahren\\
zur Ruhe will abfahren.
   
\end{verse}
\end{multicols}
\attrib{\small{THZE}}
