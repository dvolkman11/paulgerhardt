%StartInfo%%%%%%%%%%%%%%%%%%%%%%%%%%%%%%%%%%%%%%%%%%%%%%%%%%%%%%%%%%%%%%%%%%%%
%  Autor:
%  Titel:
%  File:
%  Ref:
%  Mod:
%EndInfo%%%%%%%%%%%%%%%%%%%%%%%%%%%%%%%%%%%%%%%%%%%%%%%%%%%%%%%%%%%%%%%%%%%%%%
%\poemtitle{pt}
\begin{multicols}{2}
\settowidth{\versewidth}{Mein Heiland lebt! Ob ich nun werd}
\begin{verse}[\versewidth]
%ich weiß, daß mein Erlöser lebt

\flagverse{1.} Ich weiß, daß mein Erlöser lebt,\\
das soll mir niemand nehmen!\\
Er lebt, und was ihm widerstrebt,\\
das muß sich endlich schämen.\\
Er lebt fürwahr, der starke Held,\\
sein Arm, der alle Feinde fällt,\\
hat auch den Tod bezwungen.

\flagverse{2.} Des bin ich herzlich hoch erfreut\\
und habe gar kein Scheuen\\
vor dem, der alles Fleisch zerstreut\\
gleich wie der Wind die Spreuen.\\
Nimmt er gleich mich und mein Gebein\\
und scharrt uns in die Gruft hinein,\\
was kann er damit schaden!

\flagverse{3.} Mein Heiland lebt! Ob ich nun werd\\
ins Todes Staub mich strecken,\\
so wird er mich doch aus der Erd\\
hernachmals auferwecken;\\
er wird mich reißen aus dem Grab\\
und aus dem Lager, da ich hab\\
ein kleines ausgeschlafen.

\flagverse{4.} Da werd ich eben diese Haut\\
und eben diese Glieder,\\
die jeder jetzo an mir schaut,\\
auch was sich hin und wieder\\
von Adern und Gelenken findt\\
und meinen Leib zusammenbindt,\\
ganz richtig wieder haben.

\flagverse{5.} Zwar alles, was der Mensche trägt,\\
das Fleisch und seine Knochen,\\
wird, wenn er sich hin sterben legt,\\
zermalmet und zerbrochen\\
von Maden, Motten und was mehr\\
gehöret zu der Würmer Heer;\\
doch solls nicht stets so bleiben.

\flagverse{6.} Es soll doch alles wieder stehn\\
in seinem vorgen Wesen,\\
was niederlag, wird Gott erhöhn,\\
was umkam, wird genesen.\\
Was die Verfaulung hat verheert\\
und die Verwesung hat gezehrt,\\
wird alles wiederkommen.

\flagverse{7.} Das hab ich je und je gegläubt\\
und fass ein fest Vertrauen.\\
Ich werde den, der ewig bleibt,\\
in meinem Fleische schauen;\\
ja, in dem Fleische, das hier stirbt\\
und in dem Stank und Kot verdirbt,\\
da werd ich Gott inn sehen.

\flagverse{8.} Ich selber werd in seinem Licht\\
ihn sehn und mich erquicken,\\
mein Auge wird sein Angesicht\\
mit großer Lust erblicken.\\
Ich werd ihn mir sehn, mir zur Freud,\\
und werd ihm dienen ohne Zeit,\\
ich selber und kein Fremder.

\end{verse}
\end{multicols}

\begin{center}
\settowidth{\versewidth}{Der, vor dem die Welt erschrickt,}
\begin{verse}[\versewidth]


\flagverse{9.} Trotz sei nun allem, was mir will\\
mein Herze blöde machen!\\
Wärs noch so mächtig groß und viel,\\
kann ich doch fröhlich lachen.\\
Man treib und spanne noch so hoch\\
Sarg, Grab und Tod, so bleibet doch\\
Gott, mein Erlöser, leben.
  
\end{verse}
\end{center}
%\attrib{\small{THZE}}
