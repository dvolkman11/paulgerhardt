%StartInfo%%%%%%%%%%%%%%%%%%%%%%%%%%%%%%%%%%%%%%%%%%%%%%%%%%%%%%%%%%%%%%%%%%%%
%  Autor:
%  Titel:
%  File:
%  Ref:
%  Mod:
%EndInfo%%%%%%%%%%%%%%%%%%%%%%%%%%%%%%%%%%%%%%%%%%%%%%%%%%%%%%%%%%%%%%%%%%%%%%
%\poemtitle{Herr Lindholtz legt sich hin}
%\begin{multicols}{2}
\settowidth{\versewidth}{Herr Lindholtz legt sich hin und schläft in Gottes Namen,}
\begin{verse}[\versewidth]
%Herr Lindholtz legt sich hin und schläft in Gottes Namen\\
%Auf den Tod des Kammergerichtsadvokaten in Berlin Christian Lindholtz (1659)

\flagverse{1.} Herr Lindholtz legt sich hin und schläft in Gottes Namen,\\
weiß nichts mehr von dem Leid und von dem großen Gramen,\\
das jetzt die Welt durchstreicht. Sein Grabmal deckt ihn zu;\\
der Himmel ist sein Sitz, die Erdgruft seine Ruh.

\flagverse{2.} O schweigt, o schweigt und ruht, ihr hochgeliebten Seinen!\\
Wer in der Freude lebt, den darf man nicht beweinen.\\
Wir schweben in der See, der Sturm trübt unsern Sinn:\\
Herr Lindholtz ist im Port. Gott helf uns allen hin.

\end{verse} 
%\end{multicols}
%\attrib{\small{THZE}}
