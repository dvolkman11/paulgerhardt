%StartInfo%%%%%%%%%%%%%%%%%%%%%%%%%%%%%%%%%%%%%%%%%%%%%%%%%%%%%%%%%%%%%%%%%%%%
%  Autor:
%  Titel:
%  File:
%  Ref:
%  Mod:
%EndInfo%%%%%%%%%%%%%%%%%%%%%%%%%%%%%%%%%%%%%%%%%%%%%%%%%%%%%%%%%%%%%%%%%%%%%%
%\poemtitle{pt}
\begin{multicols}{2}
\settowidth{\versewidth}{Schaut, schaut, was ist für Wunder dar?}
\begin{verse}[\versewidth]
 
\flagverse{1.} Schaut, schaut, was ist für Wunder dar?\\
Die schwarze Nacht wird hell und klar,\\
ein großes Licht bricht dort herein,\\
ihm weichet aller Sterne Schein.
 
\flagverse{2.} Es ist ein rechtes Wunderlicht\\
und gar die alte Sonne nicht,\\
weils, wider die Natur, die Nacht\\
zu einem hellen Tage macht.
 
\flagverse{3.} Was wird hierdurch uns zeigen an\\
der die Natur so ändern kann?\\
Es muß ein großes Werk geschehn,\\
wie wir aus solchem Zeichen sehn.
 
\flagverse{4.} Sollt auch erscheinen dieser Zeit\\
die Sonne der Gerechtigkeit,\\
der helle Stern aus Jakobs Stamm,\\
der Heiden Licht, des Weibes Sam?
 
\flagverse{5.} Es ist also. Des Himmels Heer,\\
das bringt uns jetzt die Freudenmär,\\
wie sich nunmehr hab eingestellt\\
zu Bethlehem das Heil der Welt.
 
\flagverse{6.} O Gütigkeit! Was lange Jahr\\
ihm hat der frommen Väter Schar\\
gewünscht und sehnlich oft begehrt,\\
des werden wir von Gott gewährt.
 
\flagverse{7.} Drum auf, ihr Menschenkinder, auf!\\
Auf, auf, und nehmet euren Lauf\\
mit mir hin zu der Stell und Ort,\\
davon gemeld't der Engel Wort.
 
\flagverse{8.} Schaut hin, dort liegt im finstern Stall,\\
des Herrschaft gehet überall!\\
Da Speise vormals sucht ein Rind,\\
da ruht jetzt der Jungfrauen Kind.
 
\flagverse{9.} O Menschenkind, betracht es recht\\
und strauchle nicht, dieweil so schlecht,\\
so elend scheint dies Kindelein;\\
es ist und soll auch uns groß sein.
 
\flagverse{10.} Es wird im Fleisch hier vorgestellt,\\
der alles schuf und noch erhält.\\
Das Wort, so bald im Anfang war\\
bei Gott, selbst Gott, das lieget dar.
 
\flagverse{11.} Es ist der eingeborne Sohn\\
des Vaters, unser Gnadenthron,\\
das A und O, der große Gott,\\
der Siegsfürst, der Herr Zebaoth.
 
\flagverse{12.} Denn weil die Zeit nunmehr erfüllt,\\
da Gottes Zorn muß sein gestillt,\\
wird sein Sohn Mensch, trägt unsre Schuld,\\
wirbt uns durch sein Blut Gottes Huld.
 
\flagverse{13.} Dies ist die rechte Freudenzeit.\\
Weg Trauern, weg, weg alles Leid!\\
Trotz dem, der ferner uns verhöhnt!\\
Gott selbst ist Mensch. Wir sind versöhnt.
 
\flagverse{14.} Der Sünden Büßer ist nun hier,\\
den Schlangentreter haben wir,\\
der Höllen Pest, des Todes Gift,\\
des Lebens Fürsten man hier trifft.
 
\flagverse{15.} Es hat mit uns nun keine Not,\\
weil Sünde, Teufel, Höll und Tod\\
zu Spott und Schanden sind gemacht\\
in dieser großen Wundernacht.
 
\flagverse{16.} O selig, selig alle Welt,\\
die sich an dieses Kindlein hält!\\
Wohl dem, der dieses recht erkennt\\
und gläubig seinen Heiland nennt!
 
\flagverse{17.} Es danke Gott, wer danken kann,\\
der unser sich so hoch nimmt an\\
und sendet aus des Himmels Thron\\
uns, seinen Feinden, seinen Sohn.
 
\flagverse{18.} Drum stimmt an mit der Engel Heer:\\
Gott in der Höhe sei nun Ehr!\\
Auf Erden Frieden jederzeit!\\
Den Menschen Wonn und Fröhlichkeit!

\end{verse}
\end{multicols}
%\attrib{\small{THZE}}
