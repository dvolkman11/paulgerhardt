%StartInfo%%%%%%%%%%%%%%%%%%%%%%%%%%%%%%%%%%%%%%%%%%%%%%%%%%%%%%%%%%%%%%%%%%%%
%  Autor:
%  Titel:
%  File:
%  Ref:
%  Mod:
%EndInfo%%%%%%%%%%%%%%%%%%%%%%%%%%%%%%%%%%%%%%%%%%%%%%%%%%%%%%%%%%%%%%%%%%%%%%
%ANM:\poemtitle{Auf auf, mein Herz, mit Freuden}

\begin{multicols}{2}
\settowidth{\versewidth}{Ich hang und bleib auch hangen}
\begin{verse}[\versewidth]
\flagverse{1.} Auf auf, mein Herz, mit Freuden\\
nimm wahr, was heut geschicht!\\
Wie kommt nach großem Leiden\\
nun ein so großes Licht!\\
Mein Heiland war gelegt\\
da, wo man uns hinträgt,\\
wenn von uns unser Geist\\
gen Himmel ist gereist.

\flagverse{2.} Er war ins Grab gesenket,\\
der Feind trieb groß Geschrei.\\
Eh ers vermeint und denket,\\
ist Christus wieder frei\\
und ruft Victoria!\\
Schwingt fröhlich hie und da\\
sein Fähnlein als ein Held,\\
der Feld und Mut behält.

\flagverse{3.} Der Held steht auf dem Grabe\\
und sieht sich munter um,\\
der Feind liegt und legt abe\\
Gift, Gall und Ungestüm,\\
er wirft zu Christi Fuß\\
sein Höllenreich und muß\\
selbst in des Siegers Band\\
ergeben Fuß und Hand.

\flagverse{4.} Das ist mir anzuschauen\\
ein rechtes Freudenspiel,\\
nun soll mir nicht mehr grauen\\
vor allem, was mir will\\
entnehmen meinen Mut\\
zusamt dem edlen Gut,\\
so mir durch Jesum Christ\\
aus Lieb erworben ist.

\flagverse{5.} Die Höll und ihre Rotten,\\
die krümmen mir kein Haar,\\
der Sünden kann ich spotten,\\
bleib allzeit ohn Gefahr.\\
Der Tod mit seiner Macht\\
wird nichts bei mir geacht't,\\
er bleibt ein totes Bild,\\
und wär er noch so wild.

\flagverse{6.} Die Welt ist mir ein Lachen\\
mit ihrem großen Zorn,\\
sie zürnt und kann nichts machen,\\
all Arbeit ist verlorn.\\
Die Trübsal trübt mir nicht\\
mein Herz und Angesicht,\\
das Unglück ist mein Glück,\\
die Nacht mein Sonnenblick.

\flagverse{7.} Ich hang und bleib auch hangen\\
an Christo als ein Glied,\\
wo mein Haupt durch ist gangen,\\
da nimmt er mich auch mit.\\
Er reißet durch den Tod,\\
durch Welt, durch Sünd, durch Not,\\
er reißet durch die Höll:\\
Ich bin stets sein Gesell.

\flagverse{8.} Er dringt zum Saal der Ehren,\\
ich folg ihm immer nach\\
und darf mich gar nicht kehren\\
an einzig Ungemach.\\
Es tobe, was da kann,\\
mein Haupt nimmt sich mein an,\\
mein Heiland ist mein Schild,\\
der alles Toben stillt.

\end{verse}
\end{multicols}

\begin{center}
\settowidth{\versewidth}{Ich hang und bleib auch hangen}
\begin{verse}[\versewidth]
  
\flagverse{9.} Er bringt mich an die Pforten,\\
die in den Himmel führt,\\
daran mit güldnen Worten\\
der Reim gelesen wird:\\
Wer dort wird mit verhöhnt,\\
wird hier auch mit gekrönt,\\
wer dort mit sterben geht,\\
wird hier auch mit erhöht.

\end{verse}
\end{center}
%\attrib{\small{THZE}}
