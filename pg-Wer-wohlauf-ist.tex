%StartInfo%%%%%%%%%%%%%%%%%%%%%%%%%%%%%%%%%%%%%%%%%%%%%%%%%%%%%%%%%%%%%%%%%%%%
%  Autor:
%  Titel:
%  File:
%  Ref:
%  Mod:
%EndInfo%%%%%%%%%%%%%%%%%%%%%%%%%%%%%%%%%%%%%%%%%%%%%%%%%%%%%%%%%%%%%%%%%%%%%%
%\poemtitle{pt}
\begin{multicols}{2}
\settowidth{\versewidth}{Wär ich gleich wie Krösus reich,}
\begin{verse}[\versewidth]
%wer wohlauf ist und gesund\\
%danklied für Gesundheit des Leibes

\flagverse{1.} Wer wohlauf ist und gesund,\\
hebe sein Gemüte\\
und erhöhe seinen Mund\\
zu des Höchsten Güte.\\
Laßt uns danken Tag und Nacht\\
mit gesunden Liedern\\
unserm Gott, der uns bedacht\\
mit gesunden Gliedern.

\flagverse{2.} Ein gesundes frisches Blut\\
hat ein fröhlich Leben;\\
gibt uns Gott dies einzge Gut,\\
ist uns gnug gegeben\\
hier in dieser armen Welt,\\
da die schönsten Gaben\\
und des güldnen Himmels Zelt\\
wir noch künftig haben.

\flagverse{3.} Wär ich gleich wie Krösus reich,\\
hätte Barschaft liegen,\\
wär ich Alexandern gleich\\
an Triumph und Siegen;\\
müßte gleichwohl siech und schwach\\
Pfühl und Betten drücken:\\
Würd auch mich im Ungemach\\
all mein Gut erquicken?

\flagverse{4.} Stünde gleich mein ganzer Tisch\\
voller Lust und Freude,\\
hätt ich Wildbret, Wein und Fisch\\
und die ganze Weide,\\
die den Hals und Schmack ergötzt:\\
Wozu würd es nützen,\\
wenn ich dennoch ausgesetzt\\
müßt in Schmerzen sitzen?

\flagverse{5.} Hätt ich aller Ehren Pracht,\\
säß im höchsten Stande,\\
wär ich mächtig aller Macht\\
und ein Herr im Lande;\\
mein Leib aber hätte doch\\
auf- und angenommen\\
der betrübten Krankheit Joch:\\
Was hätt ich für Frommen?

\flagverse{6.} Ich erwähl ein Stücklein Brot,\\
das mir wohl gedeihet,\\
vor des roten Goldes Kot,\\
da man Ach bei schreiet;\\
schmeckt mir Speis und Mahlzeit wohl\\
und darf mein nicht schonen,\\
halt ich ein Gerichtlein Kohl\\
höher als Melonen.

\vfill\null
\columnbreak

\flagverse{7.} Samt und Purpur hilft mir nicht\\
mein Elende tragen,\\
wenn mich Hauptweh, Stein und Gicht\\
und die Schwindsucht plagen.\\
Lieber will ich fröhlich gehn\\
im geringen Kleide,\\
als mit Leid und Ängsten stehn\\
in der schönsten Seide.

\flagverse{8.} Sollt ich stumm und sprachlos sein\\
oder lahm an Füßen,\\
sollt ich nicht des Tages Schein\\
sehen und genießen;\\
sollt ich gehen spat und früh\\
mit verschlossnen Ohren:\\
Würd ich wünschen, daß ich nie\\
wär ein Mensch geboren.

\flagverse{9.} Lebt ich ohne Rat und Witz,\\
wär im Haupt verirret,\\
hätte meiner Seelen Sitz,\\
mein Herz, sich verwirret;\\
wäre mir mein Mut und Sinn\\
niemals guter Dinge:\\
Wär es besser, daß ich hin,\\
wo ich her bin, ginge.

\flagverse{10.} Aber nun gebricht mir nichts\\
an erzählten Stücken,\\
ich erfreue mich des Lichts\\
und der Sonnen Blicken,\\
mein Gesichte sieht sich üm,\\
mein Gehöre höret,\\
wie der Vöglein süße Stimm\\
ihren Schöpfer ehret.

\flagverse{11.} Händ und Füße, Herz und Geist\\
sind bei guten Kräften,\\
alle mein Vermögen fleußt\\
und geht in Geschäften,\\
die mein Herrscher hat gestellt\\
hier in meinem Bleiben,\\
alsolang es ihm gefällt,\\
in der Welt zu treiben.

\flagverse{12.} Ist es Tag, so mach und tu\\
ich, was mir gebühret,\\
kommt die Nacht und süße Ruh,\\
die zum Schlafen führet,\\
schlaf und ruh ich unbewegt,\\
bis die Sonne wieder\\
mit den hellen Strahlen regt\\
meine Augenlider.

\vfill\null
\columnbreak

\flagverse{13.} Habe Dank, du milde Hand,\\
die du aus dem Throne\\
deines Himmels mir gesandt\\
diese schöne Krone\\
deiner Gnad und großen Huld,\\
die ich all mein Tage\\
niemals hab um dich verschuldt\\
und doch an mir trage.

\flagverse{14.} Gib, so lang ich bei mir hab\\
ein lebendges Hauchen,\\
daß ich solche teure Gab\\
auch wohl möge brauchen;\\
hilf, daß mein gesunder Mund\\
und erfreute Sinnen\\
dir zu aller Zeit und Stund\\
alles Liebs beginnen!

\end{verse}
\end{multicols}

\begin{verbatim}


\end{verbatim}

\begin{center}
\settowidth{\versewidth}{Der, vor dem die Welt erschrickt,}
\begin{verse}[\versewidth]



\flagverse{15.} Halte mich bei Stärk und Kraft,\\
wenn ich nun alt werde,\\
bis mein Stündlein hin mich rafft\\
in das Grab und Erde;\\
gib mir meine Lebenszeit\\
ohne sonderm Leide,\\
und dort in der Ewigkeit\\
die vollkommne Freude!

  
\end{verse}
\end{center}


%\attrib{\small{THZE}}
