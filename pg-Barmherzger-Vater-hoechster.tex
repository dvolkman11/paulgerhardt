%StartInfo%%%%%%%%%%%%%%%%%%%%%%%%%%%%%%%%%%%%%%%%%%%%%%%%%%%%%%%%%%%%%%%%%%%%
%  Autor:
%  Titel:
%  File:
%  Ref:
%  Mod:
%EndInfo%%%%%%%%%%%%%%%%%%%%%%%%%%%%%%%%%%%%%%%%%%%%%%%%%%%%%%%%%%%%%%%%%%%%%%
%\poemtitle{Barmherzger Vater, höchster Gott, gedenk an deine Worte!}
\begin{multicols}{2}
\settowidth{\versewidth}{Ach, süßer Hort, wie tröstlich klingt,}
\begin{verse}[\versewidth]
%Barmherzger Vater, höchster Gott, gedenk an deine Worte!\\
%Nach Johann Arnds »ParadiesgäRtlein« III, 26\\
\flagverse{1.} Barmherzger Vater, höchster Gott,\\
gedenk an deine Worte!\\
Du sprichst: Ruf mich an in der Not\\
und klopf an meine Pforte,\\
so will ich dir\\
Errettung hier\\
nach deinem Wunsch erweisen,\\
daß du mit Mund\\
und Herzensgrund\\
in Freuden mich sollst preisen.

\flagverse{2.} Befiehl dem Herren früh und spat\\
all deine Weg und Sachen,\\
er weiß zu geben Rat und Tat,\\
kann alles richtig machen.\\
Wirf auf ihn hin,\\
was dir im Sinn\\
liegt und dein Herz betrübet,\\
er ist dein Hirt,\\
der wissen wird\\
zu schützen, was er liebet.

\flagverse{3.} Der fromme Vater wird sein Kind\\
in seine Arme fassen\\
und, die gerecht und gläubig sind,\\
nicht stets in Unruh lassen.\\
Drum, liebe Leut,\\
hofft allezeit\\
auf den, der völlig labet;\\
dem schüttet aus,\\
was ihr im Haus\\
und auf dem Herzen habet.

\flagverse{4.} Ach, süßer Hort, wie tröstlich klingt,\\
was du versprichst den Frommen:\\
Ich will, wann Trübsal einher dringt,\\
ihm selbst zu Hilfe kommen,\\
er liebet mich,\\
drum will auch ich\\
ihn lieben und beschützen,\\
er soll bei mir\\
im Schoße hier\\
frei aller Sorgen sitzen.

\flagverse{5.} Der Herr ist allen denen nah,\\
die sich zu ihme finden,\\
wann sie ihn rufen, steht er da,\\
hilft fröhlich überwinden\\
all Angst und Weh,\\
hebt in die Höh\\
die schon darnieder liegen;\\
er macht und schafft,\\
daß sie viel Kraft\\
und große Stärke kriegen.

\begin{verbatim}







\end{verbatim}

\flagverse{6.} Fürwahr, wer meinen Namen ehrt,\\
spricht Christus, und fest gläubet,\\
des Bitte wird von Gott erhört,\\
sein Herzenswunsch bekleibet.\\
So tret heran\\
ein jedermann!\\
Wer bittet, wird empfangen,\\
und wer da sucht,\\
der wird die Frucht\\
mit großem Nutz erlangen.

\flagverse{7.} Hört, was dort jener Richter sagt:\\
Ich muß die Witwe hören,\\
dieweil sie mich so treibt und plagt.\\
Sollt denn sich Gott nicht kehren\\
zu seiner Schar,\\
die hier und dar\\
bei Nacht und Tage schreien?\\
Ich sag und halt:\\
Er wird sie bald\\
aus aller Angst befreien.

\flagverse{8.} Wann der Gerecht in Nöten weint,\\
will Gott ihn fröhlich machen;\\
und die zerbrochnes Herzens seind,\\
die sollen wieder lachen.\\
Wer fromm will sein,\\
muß in der Pein\\
und Jammerstraße wallen;\\
doch steht ihm bei\\
des Höchsten Treu\\
und hift ihm aus dem allen.

\flagverse{9.} Ich habe dich ein'n Augenblick,\\
o liebes Kind, verlassen,\\
sieh aber, sieh, mit großem Glück\\
und Trost ohn alle Maßen\\
will ich dir schon\\
die Freudenkron\\
aufsetzen und verehren;\\
dein kurzes Leid\\
soll sich in Freud\\
und ewges Heil verkehren.

\flagverse{10.} Ach lieber Gott, ach Vaterherz,\\
mein Trost von so viel Jahren,\\
wie läßt du mich so manchen Schmerz\\
und große Angst erfahren!\\
Mein Herze schmacht,\\
mein Auge wacht\\
und weint sich krank und trübe;\\
mein Angesicht\\
verliert sein Licht\\
vom Seufzen, das ich übe.

\begin{verbatim}







\end{verbatim}

\flagverse{11.} Ach Herr, wie lange willst du mein\\
so ganz und gar vergessen?\\
Wie lange soll ich traurig sein\\
und mein Leid in mich fressen?\\
Wie lang ergrimmt\\
dein Herz und nimmt\\
dein Antlitz meiner Seelen?\\
Wie lange soll\\
ich sorgenvoll\\
mein Herz im Leibe quälen?

\flagverse{12.} Willst du verstoßen ewiglich\\
und kein Guts mehr erzeigen?\\
Soll dein Wort und Verheißung sich\\
nun ganz zu Grunde neigen?\\
Zürnst du so sehr,\\
daß du nicht mehr\\
dein Heil magst zu mir senden?\\
Doch Herr, ich will\\
dir halten still,\\
dein Hand kann alles wenden.

\flagverse{13.} Nach dir, o Herr, verlanget mich\\
im Jammer dieser Erden.\\
Mein Gott, ich harr und hoff auf dich,\\
laß nicht zuschanden werden,\\
Herr, deinen Freund,\\
daß nicht mein Feind\\
sich freu und jubiliere,\\
gib mir vielmehr,\\
daß ich zur Ehr\\
ersteig und triumphiere.

\flagverse{14.} Ach Herr, du bist und bleibst auch wohl\\
getreu in deinem Sinne,\\
darum, wann ich ja kämpfen soll,\\
so gib, daß ich gewinne.\\
Leg auf die Last,\\
die du mir hast\\
beschlossen aufzulegen,\\
leg auf, doch daß\\
auch nicht das Maß\\
sei über mein Vermögen!

\flagverse{15.} Du bist ja ungebundner Kraft\\
ein Held, der alles stürzet,\\
du hast ein Hand, die alles schafft,\\
die ist noch unverkürzet.\\
Herr Zebaoth\\
wirst du, mein Gott,\\
genannt zu deinen Ehren,\\
bist groß von Rat,\\
und deiner Tat\\
kann keine Stärke wehren.

\flagverse{16.} Du bist der Tröster Israel\\
und Retter aus Trübsalen,\\
wie kommts denn, daß du meine Seel\\
jetzt sinken läßt und fallen?\\
Du stellst und hast\\
dich als ein Gast,\\
der fremd ist in dem Lande,\\
und wie ein Held,\\
dems Herz entfällt\\
mit Schimpf und großer Schande.

\flagverse{17.} Nein Herr, ein solcher bist du nicht,\\
des ist mein Herz gegründet,\\
du stehest fest, der du dein Licht\\
hier bei uns angezündet.\\
Ja hier hältst du,\\
Herr, deine Ruh\\
bei uns, die nach dir heißen,\\
und bist bereit,\\
zu rechter Zeit\\
uns aus der Not zu reißen.

\flagverse{18.} Nun, Herr, nach aller dieser Zahl\\
der jetzt erzählten Worten\\
hilf mir, der ich so manchesmal\\
geklopft an deine Pforten!\\
Hilf, Helfer, mir,\\
so will ich hier\\
dir Freudenopfer bringen,\\
auch nachmals dort\\
dir fort und fort\\
im Himmel herrlich singen.

\end{verse}
\end{multicols}
%\attrib{\small{THZE}}
