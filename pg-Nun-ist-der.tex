%StartInfo%%%%%%%%%%%%%%%%%%%%%%%%%%%%%%%%%%%%%%%%%%%%%%%%%%%%%%%%%%%%%%%%%%%%
%  Autor:
%  Titel:
%  File:
%  Ref:
%  Mod:
%EndInfo%%%%%%%%%%%%%%%%%%%%%%%%%%%%%%%%%%%%%%%%%%%%%%%%%%%%%%%%%%%%%%%%%%%%%%
%\poemtitle{pt}
\begin{multicols}{2}
\settowidth{\versewidth}{Sein Zorn war sehr entbrannt}
\begin{verse}[\versewidth]
%nun ist der Regen hin\\
%danklied vor einem gnädigen Sonnenschein

\flagverse{1.} Nun ist der Regen hin;\\
wohlauf, mein Herz und Sinn,\\
sing nach betrübtem Leiden\\
Gott, deinem Herrn, mit Freuden!\\
Gott hat sein Herz gekehret\\
und unser Bitt erhöret.

\flagverse{2.} Sein Zorn war sehr entbrannt\\
auf uns und unser Land;\\
er sprach: Ihr Menschenkinder,\\
geht, seid und bleibet Sünder,\\
wollt von der Bosheit Straßen\\
euch gar nicht wenden lassen.

\flagverse{3.} Drum soll mein Himmelslicht\\
sein klares Angesicht\\
in schwarze trübe Decken\\
und dunkle Wolken stecken\\
und für das helle Scheinen\\
nur immer zu euch weinen.

\flagverse{4.} Bald aber fiel sein Grimm\\
durch unsers Seufzens Stimm;\\
das ewige Gemüte\\
dacht an sein ewge Güte\\
und ließ auf unser Schreien\\
ihm seinen Zorn gereuen.

\flagverse{5.} Die Wolken flohen weg,\\
der feuchten Winde Steg,\\
daher die Wasser flossen,\\
nahm ab und ward verschlossen;\\
des hohen Himmels Tiefen,\\
die hörten auf zu triefen.

\flagverse{6.} Steh auf, du mattes Feld,\\
aus deinem Trauerzelt,\\
steh auf und laß nun wieder\\
die süßen Sommerlieder\\
zu deines Schöpfers Ehren\\
mit Lust und Freuden hören.

\flagverse{7.} Sie hie, der Sonnen Zier\\
geht wieder schön herfür,\\
bringt nach dem Schlag und Regen\\
den lieben warmen Segen\\
und wirkt auf Berg und Talen\\
mit wunderlichen Strahlen.

\flagverse{8.} Die Erde wird erquickt,\\
und was durch Näß erstickt,\\
das wird nun wieder leben\\
und reife Früchte geben:\\
Die Äcker gut Getreide,\\
die Wiesen Gras und Weide.

\flagverse{9.} Die Bäume werden schön\\
in ihrer Fülle stehn,\\
die Berge werden fließen\\
und Wein und Öle gießen,\\
das Bienlein wird wohl tragen\\
bei guten warmen Tagen.

\flagverse{10.} Davon wird unser Teil\\
das ewge Gut und Heil\\
uns allensamt zumessen,\\
wir werdens sehn und essen\\
und mit dem Gut der Erden\\
zur Gnüg ersättigt werden.

\flagverse{11.} Nun, Gott ist fromm und treu,\\
sein Huld ist immer neu\\
und läßt sich leicht versühnen,\\
gibt, was wir nicht verdienen,\\
läßt gnädiglich sich finden\\
und nicht nach unsern Sünden.

\flagverse{12.} Darum so richte nun,\\
o Mensch, auch du dein Tun\\
zu Gottes Lob und Liebe,\\
daß dein Herz nicht betrübe\\
mit mehrem Zorn und Schmerze\\
das allerfrömmste Herze.

\end{verse}
\end{multicols}
%\attrib{\small{THZE}}
