%StartInfo%%%%%%%%%%%%%%%%%%%%%%%%%%%%%%%%%%%%%%%%%%%%%%%%%%%%%%%%%%%%%%%%%%%%
%  Autor:
%  Titel:
%  File:
%  Ref:
%  Mod:
%EndInfo%%%%%%%%%%%%%%%%%%%%%%%%%%%%%%%%%%%%%%%%%%%%%%%%%%%%%%%%%%%%%%%%%%%%%%
%\poemtitle{Gegrüßet seist du, meine Kron}
\begin{multicols}{2}
\settowidth{\versewidth}{Was soll ich dir doch immermehr,}
\begin{verse}[\versewidth]

%{2.} An die Knie (Salve Jesu, rex sanctorum) Gegrüßet seist du, meine Kron

\flagverse{1.} Gegrüßet seist du, meine Kron\\
und König aller Frommen,\\
der du zum Trost von deinem Thron\\
uns armen Sündern kommen!\\
O wahrer Mensch, o wahrer Gott,\\
o Helfer, voller Hohn und Spott,\\
den du doch nicht verschuldest!\\
Ach, wie so arm, wie nackt und bloß\\
hängst du am Kreuz,\\
wie schwer und groß\\
ist dein Schmerz, den du duldest.

\flagverse{2.} Es fleußet deines Blutes Bach\\
mit ganzem vollem Haufen,\\
dein Leib ist auch mit Ungemach\\
ganz durch und durch belaufen.\\
O ungeschränkte Majestät,\\
wie kommts, daß dirs so kläglich geht?\\
Das macht dein Huld und Treue.\\
Wer dankt dir des? Wo ist der Mann,\\
der sich, wie du für uns getan,\\
für dich zu sterben freue?

\flagverse{3.} Was soll ich dir doch immermehr,\\
o Liebster, dafür geben,\\
daß dein Herz sich so hoch und sehr\\
bemüht hat um mein Leben?\\
Du rettest mich durch deinen Tod\\
von mehr als eines Todes Not\\
und machst mich sicher wohnen.\\
Laß HöLl und Teufel böse sein,\\
was schad'ts? Sie müssen dennoch mein\\
und meiner Seele schonen.

\flagverse{4.} Vor großer Lieb und heilger Lust,\\
damit du mich erfüllet,\\
so wird mein Leid gestillet,\\
drück ich dich an mein Herz und Brust,\\
das deinen Augen wohlbekannt.\\
Und das ist dir ja keine Schand,\\
ein krankes Herz zu laben.\\
Ach bleib mir hold und gutes Muts,\\
bis mich die Ströme deines Bluts\\
ganz rein gewaschen haben.

\end{verse}
\end{multicols}

\begin{center}
\settowidth{\versewidth}{Sei du mein Schatz und höchste Freud,}
\begin{verse}[\versewidth]




\flagverse{5.} Sei du mein Schatz und höchste Freud,\\
ich will dein Diener bleiben,\\
und deines Kreuzes Herzeleid\\
will ich in mein Herz schreiben.\\
Verleihe Du nur Kraft und Macht,\\
damit, was ich bei mir bedacht,\\
ich mög ins Werk auch setzen;\\
so wirst du, Schönster, meinen Sinn\\
und alles, was ich hab und bin,\\
ohn Unterlaß ergötzen.

  
\end{verse}
\end{center}

%\attrib{\small{THZE}}
