%StartInfo%%%%%%%%%%%%%%%%%%%%%%%%%%%%%%%%%%%%%%%%%%%%%%%%%%%%%%%%%%%%%%%%%%%%
%  Autor:
%  Titel:
%  File:
%  Ref:
%  Mod:
%EndInfo%%%%%%%%%%%%%%%%%%%%%%%%%%%%%%%%%%%%%%%%%%%%%%%%%%%%%%%%%%%%%%%%%%%%%%
%\poemtitle{pt}
\begin{multicols}{2}
\settowidth{\versewidth}{Ach, wie bezwang und drang dich doch}
\begin{verse}[\versewidth]

%{6.} An das Herz (Summi regis cor aveto) O Herz des Königs aller Welt

\flagverse{1.} O Herz des Königs aller Welt,\\
des Herrschers in dem Himmelszelt,\\
dich grüßt mein Herz in Freuden.\\
Mein Herze, wie dir wohl bewußt,\\
hat seine größt und höchste Lust\\
an dir und deinem Leiden.\\
Ach, wie bezwang und drang dich doch\\
dein edle Lieb, ins bittre Joch\\
der Schmerzen dich zu geben,\\
da du dich neigtest in den Tod,\\
zu retten aus der Todesnot\\
mich und mein armes Leben.

\flagverse{2.} O Tod, du fremder Erdengast,\\
wie warst du so ein herbe Last\\
dem allersüß'sten Herzen!\\
Dich hat ein Weib der Welt gebracht,\\
und machst dem, der die Welt gemacht,\\
so unerhörte Schmerzen!\\
Du meines Herzens Herz und Sinn,\\
du brichst und fällst und stirbst dahin,\\
wollst mir ein Wort gewähren:\\
Ergreif mein Herz und schleuß es ein\\
in dir und deiner Liebe Schrein.\\
Mehr will ich nicht begehren.

\flagverse{3.} Mein Herz ist kalt, hart und betört\\
von allem, was zur Welt gehört,\\
fragt nur nach eitlen Sachen,\\
drum, herzes Herze, bitt ich dich,\\
du wollest dies mein Herz und mich\\
warm, weich und sauber machen.\\
Laß deine Flamm und starke Glut\\
durch all mein Herze, Geist und Mut\\
mit allen Kräften dringen;\\
laß deine Lieb und Freundlichkeit\\
zur Gegenlieb und Dankbarkeit\\
mich armen Sünder bringen.

\flagverse{4.} Erweitre dich, mach alles voll!\\
Sei meine Ros und riech mir wohl,\\
bring Herz und Herz zusammen,\\
entzünde mich durch dich und laß\\
mein Herz ohn End und alle Maß\\
in deiner Liebe flammen!\\
Wer dieses hat, wie wohl ist dem;\\
in dir beruhn ist angenehm,\\
ach, niemand kanns gnug sagen.\\
Wer dich recht liebt, ergibt sich frei,\\
in deiner Lieb und süßen Treu\\
auch wohl den Tod zu tragen.

\flagverse{5.} Ich ruf aus aller Herzensmacht\\
dich, Herz, in dem mein Herz erwacht,\\
ach laß dich doch errufen!\\
Komm, beug und neige dich zu mir\\
an meines Herzens arme Tür\\
und zeuch mich auf die Stufen\\
der Andacht und der Freudigkeit,\\
gib, daß mein Herz in Lieb und Leid\\
dein eigen sei und bleibe,\\
daß dir es dien an allem Ort\\
und dir zu Ehren immerfort\\
all seine Zeit vertreibe.

\flagverse{6.} O Herzensros', o schönste Blum!\\
Ach, wie so köstlich ist dein Ruhm,\\
du bist nicht auszupreisen.\\
Eröffne dich, laß deinen Saft\\
und des Geruchs erhöhte Kraft\\
mein Herz und Seele speisen!\\
Dein Herz, Herr Jesu, ist verwundt,\\
ach tritt zu mir in meinen Bund\\
und gib mir deinen Orden!\\
Verwund auch mich, o süßes Heil,\\
und triff mein Herz mit deinem Pfeil,\\
wie du verwundet worden.

\end{verse}
\end{multicols}

\begin{center}
\settowidth{\versewidth}{Nimm mein Herz, o mein höchstes Gut,}
\begin{verse}[\versewidth]

\flagverse{7.} Nimm mein Herz, o mein höchstes Gut,\\
und leg es hin, wo dein Herz ruht,\\
da ists wohl aufgehoben.\\
Da gehts mit dir gleich als zum Tanz,\\
da lobt es deines Hauses Glanz\\
und kanns doch nicht gnug loben.\\
Hier setzt sichs, hier gefällts ihm wohl,\\
hier freut sichs, daß es bleiben soll.\\
Erfüll, Herr, meinen Willen!\\
Und weil mein Herz dein Herze liebt,\\
so laß auch, wie dein Recht es gibt,


%\attrib{\small{THZE}}
\end{verse}
\end{center}






