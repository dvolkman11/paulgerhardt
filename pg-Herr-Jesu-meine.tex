%StartInfo%%%%%%%%%%%%%%%%%%%%%%%%%%%%%%%%%%%%%%%%%%%%%%%%%%%%%%%%%%%%%%%%%%%%
%  Autor:
%  Titel:
%  File:
%  Ref:
%  Mod:
%EndInfo%%%%%%%%%%%%%%%%%%%%%%%%%%%%%%%%%%%%%%%%%%%%%%%%%%%%%%%%%%%%%%%%%%%%%%
%\poemtitle{Herr Jesu, meine Liebe}
\begin{multicols}{2}
\settowidth{\versewidth}{Nun weißt du meine Plagen}
\begin{verse}[\versewidth]


\flagverse{1.} Herr Jesu, meine Liebe,\\
ich hätte nimmer Ruh und Rast,\\
wo nicht fest in mir bliebe\\
was du für mich geleistet hast;\\
es müßt in meinen Sünden,\\
die sich sehr hoch erhöhn,\\
all meine Kraft verschwinden\\
und wie ein Rauch vergehn,\\
wenn sich mein Herz nicht hielte\\
zu dir und deinem Tod,\\
und ich nicht stets mich kühlte\\
an deines Leidens Not.

\flagverse{2.} Nun weißt du meine Plagen\\
und Satans, meines Feindes, List.\\
Wenn Meinen Geist zu nagen,\\
er emsig und bemühet ist,\\
da hat er tausend Künste,\\
von dir mich abzuziehn:\\
Bald treibt er mir die Dünste\\
des Zweifels in den Sinn,\\
bald nimmt er mir dein Meinen\\
und Wollen aus der Acht\\
und lehrt mich ganz verneinen,\\
was du doch fest gemacht.

\flagverse{3.} Solch Unheil abzuweisen,\\
hast du, Herr, deinen Tisch gesetzt,\\
da lässest du mich speisen,\\
so daß sich Mark und Bein ergötzt.\\
Du reichst mir zu genießen\\
dein teures Fleisch und Blut\\
und lässet Worte fließen,\\
da all mein Herz auf ruht.\\
Komm, sprichst du, komm und nahe\\
dich ungescheut zu mir,\\
was ich dir geb, empfahe\\
und nimms getrost zu dir.

\flagverse{4.} Hier ist beim Brot vorhanden\\
mein Leib, der dargegeben wird\\
zum Tod- und Kreuzesbanden\\
für dich, der sich von mir verirrt.\\
Beim Wein ist, was geflossen\\
zu Tilgung deiner Schuld,\\
mein Blut, das ich vergossen\\
in Sanftmut und Geduld.\\
Nimms beides mit dem Munde\\
und denk auch mit darbei,\\
wie fromm im Herzensgrunde\\
ich, dein Erlöser, sei.

\flagverse{5.} Herr, ich will dein gedenken,\\
so lang ich Luft und Leben hab,\\
und bis man mich wird senken\\
an meinem End ins finstre Grab.\\
Ich sehe dein Verlangen\\
nach einem ewgen Heil,\\
am Holz bist du gehangen\\
und hast so manchen Pfeil\\
des Trübsals lassen dringen\\
in dein unschuldigs Herz,\\
auf daß ich möcht entspringen\\
des Todes Pein und Schmerz.

\flagverse{6.} So hast du auch befohlen,\\
daß, was den Glauben stärken kann,\\
ich bei dir solle holen,\\
und soll doch ja nicht zweifeln dran,\\
du habst für alle Sünden,\\
die in der ganzen Welt\\
bei Menschen je zu finden,\\
ein völligs Lösegeld\\
und Opfer, das bestehet\\
vor dem, der alles trägt,\\
in dem auch alles gehet,\\
bezahlet und erlegt.

\flagverse{7.} Und daß ja mein Gedanke,\\
der voller Falschheit und Betrug,\\
nicht im geringsten wanke,\\
als wär es dir nicht Ernst genug:\\
So neigst du dein Gemüte\\
zusamt der rechten Hand\\
und gibst mit großer Güte\\
mir das hochwerte Pfand\\
zu essen und zu trinken.\\
Ist das nicht Trost und Licht\\
dem, der sich läßt bedünken,\\
du wollest seiner nicht?

\flagverse{8.} Ach Herr, du willst uns alle,\\
das sagt uns unser Herze zu,\\
die, so der Feind zu Falle\\
gebracht, rufst du zu deiner Ruh.\\
Ach hilf, Herr, hilf uns eilen\\
zu dir, der jederzeit\\
uns allesamt zu heilen\\
geneigt ist und bereit!\\
Gib Lust und heilges Dürsten\\
nach deinem Abendmahl,\\
und dort mach uns zu Fürsten\\
im güldnen Himmelssaal.

\end{verse}
\end{multicols}
%\attrib{\small{THZE}}
