%StartInfo%%%%%%%%%%%%%%%%%%%%%%%%%%%%%%%%%%%%%%%%%%%%%%%%%%%%%%%%%%%%%%%%%%%%
%  Autor:
%  Titel:
%  File:
%  Ref:
%  Mod:
%EndInfo%%%%%%%%%%%%%%%%%%%%%%%%%%%%%%%%%%%%%%%%%%%%%%%%%%%%%%%%%%%%%%%%%%%%%%
%\poemtitle{pt}
\begin{multicols}{2}
\settowidth{\versewidth}{Du wirst aus des Himmels Throne}
\begin{verse}[\versewidth]

\flagverse{1.} O du allersüß'ste Freude!\\
O du allerschönstes Licht!\\
Der du uns in Lieb und Leide\\
unbesuchet lässest nicht,\\
Geist des Höchsten! Höchster Fürst,\\
der du hältst und halten wirst\\
ohn Aufhören alle Dinge,\\
höre, höre, was ich singe!

\flagverse{2.} Du bist ja die beste Gabe,\\
die ein Mensche nennen kann;\\
wenn ich dich erwünsch und habe,\\
geb ich alles Wünschen an.\\
Ach, ergib dich, komm zu mir\\
in mein Herze, das du dir,\\
da ich in die Welt geboren,\\
selbst zum Tempel auserkoren.

\flagverse{3.} Du wirst aus des Himmels Throne\\
wie ein Regen ausgeschütt,\\
bringst vom Vater und vom Sohne\\
nichts als lauter Segen mit;\\
laß doch, o du werter Gast,\\
Gottes Segen, den du hast\\
und verwaltst nach deinem Willen,\\
mich an Leib und Seele füllen.

\flagverse{4.} Du bist weis und voll Verstandes,\\
was geheim ist, ist dir kund,\\
zählst den Staub des kleinen Sandes,\\
gründst des tiefen Meeres Grund.\\
Nun, du weißt auch zweifelsfrei,\\
wie verderbt und blind ich sei;\\
drum gib Weisheit und vor allen,\\
wie ich möge Gott gefallen.

\flagverse{5.} Du bist heilig, läßt dich finden,\\
wo man rein und sauber ist,\\
fleuchst hingegen Schand und Sünden,\\
wie die Tauben Stank und Mist.\\
Mache mich, o Gnadenquell,\\
durch dein Waschen rein und hell;\\
laß mich fliehen, was du fliehest,\\
gib mir, was du gerne siehest.

\flagverse{6.} Du bist, wie ein Schäflein pfleget,\\
frommes Herzens, sanftes Muts,\\
bleibst im Lieben unbeweget,\\
tust uns Bösen alles Guts.\\
Ach, verleih und gib mir auch\\
diesen edlen Sinn und Brauch,\\
daß ich Freund und Feinde liebe,\\
keinen, den du liebst, betrübe.

\flagverse{7.} Mein Hort, ich bin wohl zufrieden,\\
wenn du mich nur nicht verstößt,\\
bleib ich von dir ungeschieden,\\
ei, so bin ich gnug getröst.\\
Laß mich sein dein Eigentum,\\
ich versprech hinwiederum,\\
hier und dort all mein Vermögen\\
dir zu Ehren anzulegen.

\flagverse{8.} Ich entsage alle deme,\\
was dir deinen Ruhm benimmt,\\
ich will, daß mein Herz annehme\\
nun allein, was von dir kömmt.\\
Was der Satan will und sucht,\\
will ich halten als verflucht,\\
ich will seinen schnöden Wegen\\
mich mit Ernst zuwiderlegen.

\flagverse{9.} Nur allein daß du mich stärkest\\
und mir treulich stehest bei;\\
hilf, mein Helfer, wo du merkest,\\
daß mir Hilfe nötig sei.\\
Brich des bösen Fleisches Sinn,\\
nimm den alten Willen hin,\\
mach ihn allerdinge neue,\\
daß sich mein Gott meiner freue.

\flagverse{10.} Sei mein Retter! Halt mich eben;\\
wenn ich sinke, sei mein Stab!\\
Wenn ich sterbe, sein mein Leben,\\
wenn ich liege, sei mein Grab!\\
Wenn ich wieder aufersteh,\\
ei, so hilf mir, daß ich geh\\
hin, da du in ewgen Freuden\\
wirst dein' Auserwählten weiden.

\end{verse}
\end{multicols}
%\attrib{\small{THZE}}
