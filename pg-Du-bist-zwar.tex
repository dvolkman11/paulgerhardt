%StartInfo%%%%%%%%%%%%%%%%%%%%%%%%%%%%%%%%%%%%%%%%%%%%%%%%%%%%%%%%%%%%%%%%%%%%
%  Autor:
%  Titel:
%  File:
%  Ref:
%  Mod:
%EndInfo%%%%%%%%%%%%%%%%%%%%%%%%%%%%%%%%%%%%%%%%%%%%%%%%%%%%%%%%%%%%%%%%%%%%%%
%\poemtitle{Du bist zwar mein und bleibest mein}
\begin{multicols}{2}
\settowidth{\versewidth}{Du bist zwar mein und bleibest mein}
\begin{verse}[\versewidth]

%Aauf den Tod des Sohnes des Archidiakonus Johann Berkow in Berlin (1650)

\flagverse{1.} Du bist zwar mein und bleibest mein\\
(Wer will mir anders sagen?),\\
doch bist du nicht nur mein allein;\\
der Herr von ewgen Tagen,\\
der hat das meiste Recht an dir,\\
der fordert und erhebt von mir\\
dich, o mein Sohn, mein Wille,\\
mein Herz und Wunsches Fülle.

\flagverse{2.} Ach, gült es Wünschens, wollt ich dich,\\
du Sternlein meiner Seelen,\\
vor allem Weltgut williglich\\
mir wünschen und erwählen;\\
ich wollte sagen: Bleib bei mir!\\
Du sollst sein meins Hauses Zier;\\
an dir will ich mein Lieben\\
bis in mein Sterben üben.

\flagverse{3.} So sagt mein Herz und meint es gut.\\
Gott aber meints noch besser.\\
Groß ist die Lieb in meinem Mut,\\
in Gott ist sie noch größer.\\
Ich bin ein Vater und nichts mehr,\\
Gott ist der Väter Haupt und Ehr,\\
ein Quell, da Alt und Jungen\\
in aller Welt entsprungen.

\flagverse{4.} Ich sehne mich nach meinem Sohn,\\
und der mir ihn gegeben\\
will, daß er nah an seinem Thron\\
im Himmel solle leben.\\
Ich sprech: Ach weh, mein Licht verschwindt!\\
Gott spricht: Willkommn, du liebes Kind,\\
dich will ich bei mir haben\\
und ewig reichlich laben.

\flagverse{5.} O süßer Rat, o schönes Wort\\
und heilger, als wir denken!\\
Bei Gott ist ja kein böser Ort,\\
kein Unglück und kein Kränken,\\
kein Angst, kein Mangel, kein Versehn,\\
bei Gott kann keinem Leid geschehn;\\
wen Gott versorgt und liebet,\\
wird nimmermehr betrübet.

\flagverse{6.} Wir Menschen sind ja auch bedacht,\\
die Unsrigen zu zieren;\\
wir gehn und sorgen Tag und Nacht,\\
wie wir sie wollen führen\\
in einen feinen selgen Stand,\\
und ist doch selten so bewandt\\
mit dem, wohin sie kommen,\\
als wir uns vorgenommen.

\flagverse{7.} Wie manches junges fromme Blut\\
wird jämmerlich verführet\\
durch bös Exempel, daß es tut,\\
was Christen nicht gebühret.\\
Da hats denn Gottes Zorn zu Lohn,\\
auf Erden nichts als Spott und Hohn,\\
der Vater muß mit Grämen\\
sich seines Kindes schämen.

\flagverse{8.} Ein solches darf ja ich nun nicht\\
an meinen Sohn erwarten;\\
der steht vor Gottes Angesicht\\
und geht in Christi Garten,\\
hat Freude, die ihn recht erfreut,\\
und ruht von allem Herzeleid;\\
er sieht und hört die Scharen,\\
die uns allhier bewahren.

\flagverse{9.} Er sieht und hört der Engel Mund,\\
sein Mündlein hilft selbst singen;\\
Weiß alle Weisheit aus dem Grund\\
und redt von solchen Dingen,\\
die unser keiner noch nicht weiß,\\
die auch durch unsern Fleiß und Schweiß\\
wir, weil wir sind auf Erden,\\
nicht ausstudieren werden.

\flagverse{10.} Ach, sollt ich doch von fernen stehn\\
und nur ein wenig hören,\\
wenn deine Sinnen sich erhöhn\\
und Gottes Namen ehren,\\
der heilig, heilig, heilig ist,\\
durch den du auch geheiligt bist:\\
Ich weiß, ich würde müssen\\
vor Freuden Tränen gießen.

\flagverse{11.} Ich würde sprechen: Bleib allhier!\\
Nun will ich nicht mehr klagen:\\
Ach, mein Sohn, wärst du noch bei mir!\\
Nein; sondern: Komm du Wagen\\
Eliä, hole mich geschwind\\
und bring mich dahin, da mein Kind\\
und so viel liebe Seelen\\
so schöne Ding erzählen.

\flagverse{12.} Nun, es sei ja und bleib also,\\
ich will dich nicht mehr weinen.\\
Du lebst und bist von Herzen froh,\\
siehst lauter Sonnen scheinen,\\
die Sonnen ewger Freud und Ruh;\\
hier leb und bleib nun immerzu,\\
ich will, wills Gott, mit andern\\
auch bald hernacher wandern.

\end{verse}
\end{multicols}
%\attrib{\small{THZE}}
