%StartInfo%%%%%%%%%%%%%%%%%%%%%%%%%%%%%%%%%%%%%%%%%%%%%%%%%%%%%%%%%%%%%%%%%%%%
%  Autor:
%  Titel:
%  File:
%  Ref:
%  Mod:
%EndInfo%%%%%%%%%%%%%%%%%%%%%%%%%%%%%%%%%%%%%%%%%%%%%%%%%%%%%%%%%%%%%%%%%%%%%%
%\poemtitle{Herr, was hast du im Sinn?}
\begin{multicols}{2}
\settowidth{\versewidth}{Kein Mensche hört fast mehr,}
\begin{verse}[\versewidth]
%Herr, was hast du im Sinn?\\
%Gedichtet Auf die Erscheinung des Kometen von 1664 (nicht 1652, vgl. P. Anm. 378)

\flagverse{1.} Herr, was hast du im Sinn?\\
Wo denkt dein Eifer hin?\\
Von was für neuen Plagen\\
soll uns der Himmel sagen?\\
Was soll uns armen Leuten\\
der neue Stern bedeuten?

\flagverse{2.} Die Zeichen in der Höh\\
erwecken Ach und Weh,\\
es hats in nächsten Jahren\\
die ganze Welt erfahren:\\
Die brennenden Kometen\\
sind traurige Propheten.

\flagverse{3.} Sie brennen in der Luft,\\
und unsers Herzens Kluft\\
ist blind und kalt zum Guten,\\
erkennet nicht die Ruten,\\
die uns zu unsern Wunden\\
des Höchsten Hand gebunden.

\flagverse{4.} Kein Mensche hört fast mehr,\\
was Gottes Geist uns lehr\\
in seinen heilgen Worten;\\
drum muß an so viel Orten\\
von großem Zorn und Dräuen\\
das Sternenland selbst schreien.

\flagverse{5.} Die Welt hält keine Zucht,\\
der Glaub ist in der Flucht,\\
die Treu ist hart gebunden,\\
die Wahrheit ist verschwunden\\
barmherzig sein und lieben,\\
das sieht man selten üben.

\flagverse{6.} Daher wächst Gottes Grimm\\
und dringt mit Ungestüm\\
aus seines Eifers Kammer\\
und will mit großem Jammer,\\
wo wir uns nicht bekehren,\\
uns allesamt verheeren.

\flagverse{7.} Und das will der Prophet,\\
der in der Luft da steht,\\
uns, die wir sicher leben,\\
klar zu verstehen geben\\
mit seinem hellen Lichte\\
und klarem Angesichte.

\flagverse{8.} Sein Lauf ist gar geschwind.\\
Ach, Gott, Laß unsre Sünd\\
uns nicht geschwind hinrücken\\
und eilends unterdrücken;\\
laß uns der Strafen Haufen\\
nicht plötzlich überlaufen!

\flagverse{9.} Sein Strahl ist breit und lang,\\
macht uns fast angst und bang,\\
ach, Jesu, hilf uns allen,\\
auf das nicht auf uns fallen\\
die hochbetrübten Zahlen\\
der letzten Zornesschalen.

\flagverse{10.} Erhalt uns unsern Herrn,\\
den schönen edlen Stern,\\
laß uns sein Licht beleuchten,\\
laß seinen Tau uns feuchten,\\
daß wir uns seiner freuen\\
und unter ihm gedeihen.

\flagverse{11.} Laß auch noch immerfort\\
dein liebes wertes Wort\\
in unserm Land und Grenzen\\
schön rein und helle glänzen;\\
wenn dein Wort uns nur blicket,\\
so sind wir gnug erquicket.

\flagverse{12.} Gedenk an deine Güt\\
und laß doch dein Gemüt\\
erweichen von uns Armen!\\
Regier uns mit Erbarmen,\\
damit die bösen Zeichen\\
ein gutes End erreichen!
 
\end{verse}
\end{multicols}
%\attrib{\small{THZE}}
