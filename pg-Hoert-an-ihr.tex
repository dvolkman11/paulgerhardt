%StartInfo%%%%%%%%%%%%%%%%%%%%%%%%%%%%%%%%%%%%%%%%%%%%%%%%%%%%%%%%%%%%%%%%%%%%
%  Autor:
%  Titel:
%  File:
%  Ref:
%  Mod:
%EndInfo%%%%%%%%%%%%%%%%%%%%%%%%%%%%%%%%%%%%%%%%%%%%%%%%%%%%%%%%%%%%%%%%%%%%%%
%\poemtitle{pt}
\begin{multicols}{2}
\settowidth{\versewidth}{Was hilft ihm all sein Hab und Gut,}
\begin{verse}[\versewidth]
%der 49. Psalm\\
%hört an, ihr Völker, hört doch an

\flagverse{1.} Hört an, ihr Völker, hört doch an,\\
hört alle, die ihr lebet,\\
arm, reich, Herr, Diener, Frau und Mann\\
und was auf Erden schwebet:\\
Mein Mund soll reden von Verstand\\
und rechte Weisheit lehren;\\
wir wollen, was mein Herz erfand,\\
ein fein Gedichte hören\\
und spielen auf der Harfen.

\flagverse{2.} Was sollt ich fürchten meinen Feind\\
in meinen bösen Tagen,\\
da mich, ders böse mit mir meint,\\
umgibt mit vielen Plagen,\\
wann mich mein Untertreter drückt\\
mit seinen Missetaten\\
und sich, weil ihm sein Tun geglückt\\
und alles wohl geraten,\\
erhebet, pocht und prahlet?

\flagverse{3.} Was hilft ihm all sein Hab und Gut,\\
wann sich der Tod herfindet?\\
Da gilt kein Geld, kein hoher Mut,\\
all Hilf und Rat verschwindet.\\
Und wenn auch gleich sein Bruder wollt\\
ihm an die Seite treten,\\
doch kann ihn weder rotes Gold\\
noch Bruders Blut erbeten,\\
er muß dem Tod herhalten.

\flagverse{4.} Der Tod ist gar ein treuer Mann,\\
fragt nichts nach gutem Willen;\\
wann einer gleich gibt, was er kann,\\
noch läßt er sich nicht stillen.\\
Und sieht er auch schon manchem zu,\\
läßt ihn viel Jahr erlangen,\\
doch bricht er endlich solche Ruh,\\
er kommt einmal gegangen\\
und holt die alten Greisen.

\flagverse{5.} Denn solche Weisen müssen doch\\
sowohl als wie die Narren\\
sich lassen in des Grabes Loch\\
versenken und verscharren;\\
da kommt denn, was sie an sich bracht,\\
in andrer Leute Hände,\\
und also gehet ihre Pracht\\
und Herrlichkeit zu Ende,\\
viel anders, als sie wünschen.

\flagverse{6.} Dies ist ihr Herz, das ist ihr Sinn,\\
daß ihr Haus ewig bleibe,\\
ihr Ehr und Würd auch immerhin\\
sich mehr und wohl erkleibe;\\
noch dennoch aber können sie\\
nichts überall erhalten,\\
sie müssen fort und wie ein Vieh\\
hinunter und erkalten.\\
Das ist ein töricht Wesen.

\flagverse{7.} Doch gleichwohl wird es hoch gerühmt\\
mit Lippen der Nachkommen\\
und gar nicht, wie es sich geziemt,\\
zur Beßrung angenommen.\\
Sie liegen in der Höllen Grund\\
in einem bösen Schlafe,\\
der Tod, der nagt sie wie ein Hund\\
und wie ein Wolf die Schafe,\\
die keine Hilfe haben.

\flagverse{8.} Die Bösen sind des Teufels Beut\\
und müssen Marter leiden,\\
die Frommen wird der Herr mit Freud\\
im Himmelsreiche weiden.\\
Der Trotz der unverschämten Rott\\
muß brechen und vergehen,\\
wer aber treu bleibt seinem Gott,\\
der soll dort ewig stehen\\
im Chor der Auserwählten.

\flagverse{9.} Darum, mein allerliebstes Kind,\\
laß dich nicht irre machen,\\
ob einer reich wird und mit Sünd\\
erlangt viel teure Sachen;\\
denn wann er stirbt, bleibt alles hier,\\
er kann nichts mit sich nehmen.\\
Sein Herrlichkeit, sein Ehr und Zier\\
verschwindet wie ein Schemen\\
und will ihm nicht nachfolgen.

\flagverse{10.} Die Welt liebt ihren Kot und Stank,\\
hält viel von schnöden Dingen.\\
Und also gehn sie auch den Gang,\\
den ihre Väter gingen,\\
und sehen hinfort nimmermehr\\
das Licht, das uns ernähret;\\
kurz: Wann ein Mensch hat Würd und Ehr\\
und ist nicht fromm, so fähret\\
er wie ein Vieh von hinnen.

\end{verse}
\end{multicols}
%\attrib{\small{THZE}}
