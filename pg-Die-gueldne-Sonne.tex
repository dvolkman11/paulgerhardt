%StartInfo%%%%%%%%%%%%%%%%%%%%%%%%%%%%%%%%%%%%%%%%%%%%%%%%%%%%%%%%%%%%%%%%%%%%
%  Autor:
%  Titel:
%  File:
%  Ref:
%  Mod:
%EndInfo%%%%%%%%%%%%%%%%%%%%%%%%%%%%%%%%%%%%%%%%%%%%%%%%%%%%%%%%%%%%%%%%%%%%%%
%\poemtitle{Die güldne Sonne}
\begin{multicols}{2}
\settowidth{\versewidth}{ein herzerquickendes liebliches Licht.}
\begin{verse}[\versewidth]

\flagverse{1.} Die güldne Sonne\\
voll Freud und Wonne\\
bringt unsern Grenzen\\
mit ihrem Glänzen\\
ein herzerquickendes liebliches Licht.\\
Mein Haupt Und Glieder,\\
die lagen darnieder,\\
aber nun steh ich,\\
bin munter und fröhlich,\\
schaue den Himmel mit meinem Gesicht.

\flagverse{2.} Mein Auge schauet,\\
was Gott gebauet\\
zu seinen Ehren\\
und uns zu lehren,\\
wie sein Vermögen sei mächtig und groß,\\
und wo die Frommen\\
dann sollen hinkommen,\\
wenn sie mit Frieden\\
von hinnen geschieden\\
aus dieser Erden vergänglichem Schoß.

\flagverse{3.} Lasset uns singen,\\
dem Schöpfer bringen\\
Güter und Gaben;\\
was wir nur haben,\\
alles sei Gotte zum Opfer gesetzt.\\
Die besten Güter\\
sind unsre Gemüter;\\
dankbare Lieder\\
sind Weihrauch und Widder,\\
an welchen er sich am meisten ergötzt.

\flagverse{4.} Abend und Morgen\\
sind seine Sorgen;\\
segnen und mehren,\\
Unglück verwehren\\
sind seine Werke und Taten allein.\\
Wann wir uns legen,\\
so ist er zugegen,\\
wann wir aufstehen,\\
so läßt er aufgehen\\
über uns seiner Barmherzigkeit Schein.

\flagverse{5.} Ich hab erhoben\\
zu dir hoch droben\\
all meine Sinnen;\\
laß mein Beginnen\\
ohn allen Anstoß und glücklich ergehn!\\
Laster und Schande,\\
des Luzifers Bande,\\
Fallen und Tücke\\
treib ferne zurücke,\\
laß mich auf deinen Geboten bestehn!

\vfill\null
\columnbreak

\flagverse{6.} Laß mich mit Freuden\\
ohn alles Neiden\\
sehen den Segen,\\
den du wirst legen\\
in meines Bruders und Nähesten Haus;\\
geiziges Brennen,\\
unchristliches Rennen\\
nach Gut mit Sünde,\\
das tilge geschwinde\\
von meinem Herzen und wirf es hinaus!

\flagverse{7.} Menschliches Wesen,\\
was ist's? Gewesen.\\
In einer Stunde\\
geht es zu Grunde,\\
sobald das Lüftlein des Todes drein bläst.\\
Alles in allen\\
muß brechen und fallen,\\
Himmel und Erden\\
die müssen das werden,\\
was sie vor ihrer Erschöpfung gewest.

\flagverse{8.} Alles vergehet,\\
Gott aber stehet\\
ohn alles Wanken;\\
seine Gedanken,\\
sein Wort und Willen hat ewigen Grund,\\
sein Heil und Gnaden,\\
die nehmen nicht Schaden,\\
heilen im Herzen\\
die tödlichen Schmerzen,\\
halten uns zeitlich und ewig gesund.

\flagverse{9.} Gott, meine Krone,\\
vergib und schone;\\
laß meine Schulden\\
in Gnad und Hulden\\
aus deinen Augen sein abegewandt.\\
Sonsten regiere,\\
mich lenke und führe,\\
wie dirs gefället.\\
Ich habe gestellet\\
alles in deine Beliebung und Hand.

\flagverse{10.} Willst du mir geben,\\
womit mein Leben\\
ich kann ernähren,\\
so laß mich hören\\
allzeit im Herzen dies heilige Wort:\\
Gott ist das Größte,\\
das Schönste und Beste,\\
Gott ist das Süßte\\
und Allergewißte,\\
aus allen Schätzen der edelste Hort.

\vfill\null
\columnbreak

\flagverse{11.} Willst du mich kränken,\\
mit Gallen tränken,\\
und soll von Plagen\\
ich auch was tragen:\\
Wohlan, so mach es, wie dir es beliebt.\\
Was Gut und tüchtig,\\
was schädlich und nichtig\\
meinem Gebeine,\\
das weißt du alleine,\\
hast niemals keinen zu sehre betrübt.

\flagverse{12.} Kreuz und Elende,\\
das nimmt ein Ende;\\
nach Meeresbrausen\\
und Windessausen\\
leuchtet der Sonnen gewünschtes Gesicht.\\
Freude die Fülle\\
und selige Stille\\
hab ich zu warten\\
im himmlischen Garten;\\
dahin sind meine Gedanken gericht.

\end{verse}
\end{multicols}
%\attrib{\small{THZE}}
