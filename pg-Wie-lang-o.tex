%StartInfo%%%%%%%%%%%%%%%%%%%%%%%%%%%%%%%%%%%%%%%%%%%%%%%%%%%%%%%%%%%%%%%%%%%%
%  Autor:
%  Titel:
%  File:
%  Ref:
%  Mod:
%EndInfo%%%%%%%%%%%%%%%%%%%%%%%%%%%%%%%%%%%%%%%%%%%%%%%%%%%%%%%%%%%%%%%%%%%%%%
%\poemtitle{pt}
\begin{multicols}{2}
\settowidth{\versewidth}{Wie lang, o Herr, wie lange soll}
\begin{verse}[\versewidth]
%der 13. Psalm\\
%wie lang, o Herr, wie lange soll dein Herze mein vergessen?

\flagverse{1.} Wie lang, o Herr, wie lange soll\\
dein Herze mein vergessen?\\
Wie lange soll ich Jammers voll\\
mein Brot mit Tränen essen?\\
Wie lange willst du nicht\\
mir dein Angesicht\\
zu schauen reichen dar?\\
Willst du denn ganz und gar\\
dich nun von mir verbergen?

\flagverse{2.} Wie lange soll die Trauerhöhl\\
in Sorgen ich besitzen?\\
Wie lange soll mein arme Seel\\
in diesem Bade schwitzen?\\
Soll ich denn alle Tag\\
immer lauter Plag,\\
die Welt im Gegenteil\\
nur immer lauter Heil\\
nach ihrem Wunsche haben?

\flagverse{3.} Ach, schaue doch von deinem Saal\\
und siehe, wie ich leide!\\
Mein Herzensweh und große Qual\\
ist meiner Feinde Freude.\\
Herr, mein getreuer Hort,\\
hör an meine Wort,\\
die ich, durch Trübsal hier\\
gepresset, schütt herfür;\\
laß dein Gemüt erweichen!

\flagverse{4.} Erleuchte meiner Augen Licht,\\
mit deinem Gnadenwinke,\\
damit ich in dem Tode nicht\\
enschlafe noch versinke!\\
Gib, daß die böse Rott\\
nicht treib ihren Spott\\
aus mir und meinem Fall,\\
als hätt ich überall\\
verspielet und verloren.

\flagverse{5.} Ich steh und hoffe steif und fest\\
darauf, daß du die Deinen\\
nicht endlich untergehen läßt.\\
Kannsts auch nicht böse meinen;\\
obs gleich bisweilen scheint,\\
als wärst du uns feind\\
und gänzlich abgewendt,\\
so find sich doch behend\\
dein Vaterherze wieder.

\flagverse{6.} Mein Herze lacht vor großer Freud,\\
wann ich bei mir bedenke,\\
wie herzlich gern in böser Zeit\\
dein Herz sich zu uns lenke.\\
Der Herr ist frommes Muts,\\
tut uns nichts als Guts.\\
Das ist mein Lobgesang,\\
den ihm zum Ehrendank\\
ich hier und dort will singen.

\end{verse}
\end{multicols}
%\attrib{\small{THZE}}
