%StartInfo%%%%%%%%%%%%%%%%%%%%%%%%%%%%%%%%%%%%%%%%%%%%%%%%%%%%%%%%%%%%%%%%%%%%
%  Autor:
%  Titel:
%  File:
%  Ref:
%  Mod:
%EndInfo%%%%%%%%%%%%%%%%%%%%%%%%%%%%%%%%%%%%%%%%%%%%%%%%%%%%%%%%%%%%%%%%%%%%%%
%\poemtitle{pt}
\begin{multicols}{2}
\settowidth{\versewidth}{Mein Trost, mein Schatz, mein Licht und Heil,}
\begin{verse}[\versewidth]
%o Jesu Christ, mein schönstes Licht\\
%nach Johann Arnds »Paradiesgärtlein«, Goslar 1621, II, 5: »Gebet um die Liebe Christi«

\flagverse{1.} O Jesu Christ, mein schönstes Licht,\\
der du in deiner Seelen\\
so hoch mich liebst, daß ich es nicht\\
aussprechen kann noch zählen:\\
Gib, daß mein Herz dich wiederum\\
mit Lieben und Verlangen\\
mög umfangen\\
und als dein Eigentum\\
nur einzig an dir hangen!

\flagverse{2.} Gib, daß sonst nichts in meiner Seel\\
als deine Liebe wohne,\\
gib, daß ich deine Lieb erwähl\\
als meinen Schatz und Krone;\\
stoß alles aus, nimm alles hin,\\
was mich und dich will trennen\\
und nicht gönnen,\\
daß all mein Mut und Sinn\\
in deiner Liebe brennen!

\flagverse{3.} Wie freundlich, selig, süß und schön\\
ist, Jesu, deine Liebe!\\
Wann diese steht, kann nichts entstehn,\\
das meinen Geist betrübe.\\
Drum laß nichts anders denken mich,\\
nichts sehen, fühlen, hören,\\
lieben, ehren\\
als deine Lieb und dich,\\
der du sie kannst vermehren.

\flagverse{4.} O, daß ich dieses hohe Gut\\
möcht ewiglich besitzen!\\
O, daß in mir dies' edle Glut\\
ohn Ende möchte hitzen!\\
Ach, hilf mir wachen Tag und Nacht\\
und diesen Schatz bewahren\\
vor den Scharen,\\
die wider uns mit Macht\\
aus Satans Reiche fahren!

\flagverse{5.} Mein Heiland, du bist mir zulieb\\
in Not und Tod gegangen\\
und hast am Kreuz als wie ein Dieb\\
und Mörder da gehangen,\\
verhöhnt, verspeit und sehr verwundt;\\
ach, laß mich deine Wunden\\
alle Stunden\\
mit Lieb im Herzensgrund\\
auch ritzen und verwunden.

\flagverse{6.} Dein Blut, daß dir vergossen ward,\\
ist köstlich, gut und reine,\\
mein Herz hingegen böser Art\\
und hart gleich einem Steine.\\
O laß doch deines Blutes Kraft\\
mein hartes Herze zwingen,\\
wohl durchdringen\\
und diesen Lebenssaft\\
mir deine Liebe bringen!

\vfill\null
\columnbreak

\flagverse{7.} O daß mein Herze offen stünd\\
und fleißig möcht auffangen\\
die Tröpflein Bluts, die meine Sünd\\
im Garten dir abdrangen!\\
Ach daß sich meiner Augen Brunn\\
auftät und mit Stöhnen\\
heiße Tränen\\
vergösse, wie die tun,\\
die sich in Liebe sehnen.

\flagverse{8.} O daß ich wie ein kleines Kind\\
mit Weinen dir nachginge\\
so lange, bis dein Herz entzündt\\
mit Armen mich umfinge\\
und deine Seel in mein Gemüt\\
in voller süßer Liebe\\
sich erhübe\\
und also deiner Güt\\
ich stets vereinigt bliebe!

\flagverse{9.} Ach zeuch, mein Liebster, mich nach dir,\\
so lauf ich mit den Füßen;\\
ich lauf und will dich mit Begier\\
in meinem Herzen küssen.\\
Ich will aus deines Mundes Zier\\
den süßen Trost empfinden,\\
der die Sünden\\
und alles Unglück hier\\
kann leichtlich überwinden.

\flagverse{10.} Mein Trost, mein Schatz, mein Licht und Heil,\\
mein höchstes Gut und Leben,\\
ach nimm mich auf zu deinem Teil,\\
dir hab ich mich ergeben.\\
Denn außer dir ist lauter Pein,\\
ich find hier überalle\\
nichts denn Galle;\\
nichts kann mir tröstlich sein,\\
nichts ist, das mir gefalle.

\flagverse{11.} Du aber bist die rechte Ruh,\\
in dir ist Fried und Freude,\\
gib, Jesu, gib, daß immerzu\\
mein Herz in dir sich weide!\\
Sei meine Flamm und brenn in mir,\\
mein Balsam, wollest eilen,\\
lindern, heilen\\
den Schmerzen, der allhier\\
mich seufzen macht und heulen.

\flagverse{12.} Was ists, o Schönster, das ich nicht\\
in deiner Liebe habe?\\
Sie ist mein Stern, mein Sonnenlicht,\\
mein Quell, da ich mich labe,\\
mein süßer Wein, mein Himmelsbrot,\\
mein Kleid vor Gottes Throne,\\
meine Krone,\\
mein Schutz in aller Not,\\
mein Haus, darin ich wohne.

\vfill\null
\columnbreak

\flagverse{13.} Ach, liebstes Lieb, wann du entweichst,\\
was hilft mir sein geboren?\\
Wann du mir deine Lieb entzeuchst,\\
ist all mein Gut verloren.\\
So gib, daß ich dich, meinen Gast,\\
wohl such und bester Maßen\\
möge fassen\\
und, wenn ich dich gefaßt,\\
in Ewigkeit nicht lassen!

\flagverse{14.} Du hast mich je und je geliebt\\
und auch nach dir gezogen;\\
eh ich noch etwas Guts geübt,\\
warst du mir schon gewogen.\\
Ach, laß doch ferner, edler Hort,\\
mich diese Liebe leiten\\
und begleiten,\\
daß sie mir immerfort\\
beisteh auf allen Seiten!

\flagverse{15.} Laß meinen Stand, darin ich steh,\\
herr, deine Liebe zieren\\
und, wo ich etwa irre geh,\\
alsbald zurechte führen;\\
laß sie mir allzeit guten Rat\\
und gute Werke lehren,\\
steuern, wehren\\
der Sünd, und nach der Tat\\
bald wieder mich bekehren!

\flagverse{16.} Laß sie sein meine Freud im Leid,\\
in Schwachheit mein Vermögen,\\
und wann ich nach vollbrachter Zeit\\
mich soll zur Ruhe legen,\\
alsdann laß deine Liebestreu,\\
herr Jesu, bei mir stehen,\\
luft zuwehen,\\
daß ich getrost und frei\\
mög in dein Reich eingehen!

\end{verse}
\end{multicols}
%\attrib{\small{THZE}}
