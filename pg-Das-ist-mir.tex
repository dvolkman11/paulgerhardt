%StartInfo%%%%%%%%%%%%%%%%%%%%%%%%%%%%%%%%%%%%%%%%%%%%%%%%%%%%%%%%%%%%%%%%%%%%
%  Autor:
%  Titel:
%  File:
%  Ref:
%  Mod:
%EndInfo%%%%%%%%%%%%%%%%%%%%%%%%%%%%%%%%%%%%%%%%%%%%%%%%%%%%%%%%%%%%%%%%%%%%%%
%\poemtitle{Das ist mir lieb, daß Gott, mein Hort, so treulich bei mir stehet}
\begin{multicols}{2}
\settowidth{\versewidth}{Das ist mir lieb, daß Gott, mein Hort,}
\begin{verse}[\versewidth]
%der 116. Psalm\\
%Das ist mir lieb, daß Gott, mein Hort, so treulich bei mir stehet

\flagverse{1.} Das ist mir lieb, daß Gott, mein Hort\\
so treulich bei mir stehet;\\
wann ich ihn bitte, wird kein Wort\\
in meiner Bitt verschmähet.\\
Des schwarzen Todes Hand\\
samt der Höllen Band\\
umfingen überall\\
mein Herz mit Angst und Qual;\\
doch hat mir Gott geholfen.

\flagverse{2.} Ich kam in Jammer und in Not\\
und sank fast gar zugrunde,\\
und da ich sank, rief ich zu Gott\\
mit Herzen und mit Munde:\\
O Herr, ich weiß, du wirst\\
als des Lebens Fürst\\
schon führen meine Sach!\\
And wie ich bat und sprach,\\
so ist's auch nun geschehen.

\flagverse{3.} Sei wieder froh und gutes Muts,\\
mein Herze, sei zufrieden,\\
der Herr, der tut dir alles Guts,\\
durch ihn ist nun geschieden\\
und ferne weggebracht,\\
was mich traurig macht;\\
er hat mich aus dem Loch\\
und schwarzen Todesjoch\\
mit seiner Hand gerissen.

\flagverse{4.} Mein Aug ist nun von Tränen frei,\\
mein Fuß von seinem Gleiten;\\
das will ich sagen ohne Scheu\\
und rühmen bei den Leuten.\\
Was gar kein Mensch nicht kann,\\
das hat Gott getan.\\
Der Mensch ist Lügen voll,\\
Gott aber weiß gar wohl,\\
wie er sein Wort soll halten.

\flagverse{5.} Ich glaube fest in meinem Sinn,\\
und was mein Herze glaubet,\\
das redt mein Mund in Einfalt hin:\\
Wer Gott vertraut, der bleibet.\\
Die Welt und böse Rott\\
lacht des, mir zum Spott,\\
ja plagt mich noch dazu;\\
ich aber steh und ruh\\
auf dir, mein Gott und Helfer.

\flagverse{6.} Du stürzest meiner Feinde Rat\\
und segnest, wenn sie schelten,\\
wie soll ich doch die große Gnad\\
dir immer mehr vergelten?\\
Ich will, Herr, meines Teils\\
den Kelch deines Heils,\\
der voller Bitterkeit,\\
doch mir zu Nutz gedeiht,\\
gehorsamlich annehmen.

\flagverse{7.} Was du mir zugemessen hast,\\
das will ich gerne leiden;\\
wer fröhlich trägt des Kreuzes Last,\\
dem hilfst du aus mit Freuden.\\
Du weißt der Deinen Not\\
und hältst ihren Tod\\
sehr hoch, sehr lieb und wert,\\
auch läßt du auf der Erd\\
ihr Blut nicht ungerochen.

\flagverse{8.} So zürne nun gleich alle Welt\\
mit mir, Herr, deinem Knechte:\\
Du, du deckst mich in deinem Zelt\\
und reichst mir deine Rechte.\\
Darüber will ich dich\\
allstets inniglich,\\
so gut ich immer kann,\\
mit Dank vor jedermann\\
in deinem Hause preisen.

\end{verse}
\end{multicols}
%\attrib{\small{THZE}}
