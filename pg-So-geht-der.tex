%StartInfo%%%%%%%%%%%%%%%%%%%%%%%%%%%%%%%%%%%%%%%%%%%%%%%%%%%%%%%%%%%%%%%%%%%%
%  Autor:
%  Titel:
%  File:
%  Ref:
%  Mod:
%EndInfo%%%%%%%%%%%%%%%%%%%%%%%%%%%%%%%%%%%%%%%%%%%%%%%%%%%%%%%%%%%%%%%%%%%%%%
%\poemtitle{pt}
%\begin{multicols}{2}
\settowidth{\versewidth}{Die Kinder klagen ihn, ach Vater, unser Schutz!}
\begin{verse}[\versewidth]
%so geht der alte liebe Herr nun auch dahin\\
%»Auff das selige Absterben und Christliche Beerdigung des umb diese gantze Stadt
%  viel Jahr lang wolverdienten Herrn Bürgermeisters, Herrn\\
% Benedicti Reichardts« (Berlin 1667)

\flagverse{1.} So geht der alte liebe Herr nun auch dahin:\\
Nachdem er achtzig und was drüber ist erlebet.\\
Er geht zu Gott: Und legt und schlägt aus seinem Sinn\\
das, was noch, wies Gott weiß, uns überm Haupte schwebet.

\flagverse{2.} Die Kinder klagen ihn, ach Vater, unser Schutz!\\
Die Ehgenossin läßt die Tränen häufig fließen.\\
Was Kindeskinder sind, bedenken, was für Nutz\\
sie hiebevor gehabt und nun nicht mehr genießen.

\flagverse{3.} Und weinen bitterlich. Die werte Bürgerschaft\\
folgt ihrem Haupte nach und gibt ihm das Geleite\\
zu seinem Schlafgemach, dahin der Tod ihn rafft\\
gleich wie uns allzumal. Ich aber setz ihm heute

\flagverse{4.} zu Ehren diese Schrift: Ein Mann von alter Treu\\
und deutscher Redlichkeit, ein Mann von vielen Gaben\\
und großer Wissenschaft, ein Mann, der frisch und frei\\
das Recht geschützt, die Stadt regiert, wird jetzt begraben.\\

%»Zur Bezeugung Christlichen Mitleidens Gegen die gesambte Hochbetrübte Leidtragende setzte dieses Paulus Gerhardt.«

\end{verse}
%\end{multicols}
%\attrib{\small{Auff das selige Absterben und Christliche Beerdigung\\
%des umb diese gantze Stadt viel Jahr lang wolverdienten Herrn Bürgermeisters,\\ Herrn Benedicti Reichardts« (Berlin 1667)}}
