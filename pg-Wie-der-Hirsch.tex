%StartInfo%%%%%%%%%%%%%%%%%%%%%%%%%%%%%%%%%%%%%%%%%%%%%%%%%%%%%%%%%%%%%%%%%%%%
%  Autor:
%  Titel:
%  File:
%  Ref:
%  Mod:
%EndInfo%%%%%%%%%%%%%%%%%%%%%%%%%%%%%%%%%%%%%%%%%%%%%%%%%%%%%%%%%%%%%%%%%%%%%%
%\poemtitle{pt}
\begin{multicols}{2}
\settowidth{\versewidth}{Wie der Hirsch im großen Dürsten}
\begin{verse}[\versewidth]
%der 42. Psalm\\
%wie der Hirsch im großen Dürsten schreiet und frisch Wasser sucht

\flagverse{1.} Wie der Hirsch im großen Dürsten\\
schreiet und frisch Wasser sucht,\\
also sucht dich Lebensfürsten\\
meine Seel in ihrer Flucht;\\
meine Seele brennt in mir\\
lechzet, dürstet, trägt Begier\\
nach dir, o du süßes Leben,\\
der mir Leib und Seel gegeben.

\flagverse{2.} Ach, wann werd ich dahin kommen,\\
daß ich Gottes Angesicht,\\
das gewünschte Licht der Frommen,\\
schau mit meiner Augen Licht!\\
Meine Tränen sind mein Brot\\
Tag und Nacht in meiner Not,\\
wann mich schmähen meine Spötter:\\
Wo ist nun dein Gott und Retter?

\flagverse{3.} Wenn ich dann des inne werde,\\
schütt ich mein Herz bei mir aus,\\
wollte gerne mit der Herde\\
deiner Kinder in dein Haus;\\
ja, in dein Haus wollt ich gern\\
gehen und dir, meinem Herrn,\\
in der Schar, die Opfer bringen,\\
mit erhobner Stimme singen.

\flagverse{4.} Was bist du so hoch betrübet\\
und voll Unruh, meine Seel?\\
Harr auf Gott, der herzlich liebet\\
und wohl siehet, was dich quäl.\\
Ei, ich werd ihm dennoch hier\\
fröhlich danken, daß er mir,\\
wann mein Herz ich zu ihm richte,\\
hilft mit seinem Angesichte.

\flagverse{5.} Mein Gott, ich bin voller Schande,\\
meine Seele voller Leid,\\
darum denk ich dein im Lande\\
bei dem Jordan an der Seit,\\
da Hermonim hoch herfür\\
und hingegen meine Zier,\\
zion, ein klein wenig steiget\\
und dir Kron und Zepter neiget.

\flagverse{6.} Deines Zornes Fluten sausen\\
mit Gewalt auf mich daher;\\
dein Gericht und Eifer brausen\\
wie das tiefe weite Meer;\\
deine Wellen heben sich\\
hoch empor und haben mich\\
mit ergrimmten Wasserwogen\\
fast zu Grund hinabgezogen.

\flagverse{7.} Gott der Herr hat mir versprochen,\\
wann es Tag ist, seine Güt,\\
und wann sich die Sonn verkrochen,\\
heb ich zu ihm mein Gemüt,\\
spreche: Du mein Fels und Stein,\\
gegen welchen alles klein,\\
dem ich in dem Schoß gesessen,\\
warum hast du mein vergessen?

\flagverse{8.} Warum muß ich gehn und weinen\\
über meiner Feinde Wort?\\
Es ist mir in meinen Beinen\\
durch und durch als wie ein Mord,\\
wann sie sagen: Wo ist nun\\
dein Gott und sein großes Tun?\\
Davon, wann du sicher lagest,\\
du so viel zu rühmen pflagest.
\end{verse}
\end{multicols}

\begin{center}
\settowidth{\versewidth}{Der, vor dem die Welt erschrickt,}
\begin{verse}[\versewidth]



\flagverse{9.} Was bist du so hoch betrübet\\
und voll Unruh, meine Seel?\\
Harr auf Gott, der herzlich liebet\\
und wohl siehet, was dich quäl!\\
Ei, ich werd ihm dennoch hier\\
fröhlich danken für und für,\\
daß er meinem Angesichte\\
sich selbst gibt zum Heil und Lichte.
  
\end{verse}
\end{center}


%\attrib{\small{THZE}}
