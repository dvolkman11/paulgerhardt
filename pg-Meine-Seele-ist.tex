%StartInfo%%%%%%%%%%%%%%%%%%%%%%%%%%%%%%%%%%%%%%%%%%%%%%%%%%%%%%%%%%%%%%%%%%%%
%  Autor:
%  Titel:
%  File:
%  Ref:
%  Mod:
%EndInfo%%%%%%%%%%%%%%%%%%%%%%%%%%%%%%%%%%%%%%%%%%%%%%%%%%%%%%%%%%%%%%%%%%%%%%
%\poemtitle{pt}
\begin{multicols}{2}
\settowidth{\versewidth}{Meine Hasser, hört! Wie lange}
\begin{verse}[\versewidth]
%der 62. Psalm\\
%meine Seele ist in der Stille

\flagverse{1.} Meine Seel ist in der Stille,\\
tröstet sich des Höchsten Kraft,\\
dessen Rat und heilger Wille\\
mir bald Rat und Hilfe schafft.\\
Der kann mehr als alle Götter,\\
ist mein Hort, mein Heil, mein Retter,\\
daß kein Fall mich stürzen kann,\\
trät er noch so heftig an.

\flagverse{2.} Meine Hasser, hört! Wie lange\\
stellt ihr alle einem nach?\\
Ihr macht meinem Herzen bange,\\
mir zur Ehr und euch zur Schmach,\\
hanget wie zerrißne Mauern\\
und wie Wände, die nicht dauern,\\
über mir und seid bedacht,\\
wie ich werde totgemacht.

\flagverse{3.} Ja fürwahr, daß einge denken,\\
die, so mir zuwider seind,\\
wie sie mir mein Leben senken\\
dahin, wo kein Licht mehr scheint:\\
Darum geht ihr Mund aufs Lügen\\
und das Herz auf lauter Trügen;\\
gute Wort und falsche Tück\\
ist ihr bestes Meisterstück.

\flagverse{4.} Dennoch bleib ich ungeschrecket,\\
und mein Geist ist unverzagt\\
in dem Gotte, der mich decket,\\
wann die arge Welt mich plagt.\\
Auf den harret meine Seele;\\
da ist Trost, den ich erwähle,\\
da ist Schutz, der mir gefällt,\\
und Errettung, die mich hält.

\flagverse{5.} Nimmer, nimmer werd ich fallen,\\
nimmer werd ich untergehn,\\
denn hier ist, der mich vor allen,\\
die mich drücken, kann erhöhn;\\
bei dem ist mein Heil und Ehre,\\
meine Stärke, meine Wehre;\\
meine Freud und Zuversicht\\
ist nur stets auf Gott gericht.

\flagverse{6.} Hoffet allzeit, lieben Leute,\\
hoffet allzeit stark auf ihn.\\
Kommt die Hilfe nicht bald heute,\\
falle doch der Mut nicht hin.\\
Sondern schüttet aus dem Herzen\\
eures Herzens Sorg und Schmerzen,\\
legt sie vor sein Angesicht,\\
traut ihm fest und zweifelt nicht.

\flagverse{7.} Gott kann alles Unglück enden,\\
wirds auch herzlich gerne tun\\
denen, die sich zu ihm wenden\\
und auf seiner Güte ruhn.\\
Aber Menschenhilf ist nichtig,\\
ihr Vermögen ist nicht tüchtig,\\
wär es gleich noch eins so groß,\\
uns zu machen frei und los.

\flagverse{8.} Große Leute, große Toren!\\
Prangen sehr und sind doch Kot,\\
füllen Sinnen, Aug und Ohren:\\
Kommts zur Tat, so sind sie tot;\\
will man ihres Tuns und Sachen\\
eine Prob und Rechnung machen,\\
nach dem Ausschlag des Gewichts\\
sind sie weniger denn nichts.

\flagverse{9.} Laßt sie fahren, liebe Kinder,\\
da ist schlechter Vorteil bei!\\
Habt vor allem, was die Sünder\\
frechlich treiben, Furcht und Scheu!\\
Laßt euch Eitelkeit nicht fangen,\\
nach, was nichts ist, nicht verlangen;\\
käm auch Gut und Reichtum an,\\
ei, so hängt das Herz nicht dran!

\flagverse{10.} Wo das Herz am besten stehe,\\
lehrt am besten Gottes Wort\\
aus der güldnen Himmelshöhe;\\
denn da hör ich fort und fort,\\
daß er groß und reich von Kräften,\\
rein und heilig in Geschäften,\\
gütig dem, der Gutes tut.\\
Nun, der sei mein schönstes Gut.
   
\end{verse}
\end{multicols}
%\attrib{\small{THZE}}
