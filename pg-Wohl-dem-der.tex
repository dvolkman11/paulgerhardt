%StartInfo%%%%%%%%%%%%%%%%%%%%%%%%%%%%%%%%%%%%%%%%%%%%%%%%%%%%%%%%%%%%%%%%%%%%
%  Autor:
%  Titel:
%  File:
%  Ref:
%  Mod:
%EndInfo%%%%%%%%%%%%%%%%%%%%%%%%%%%%%%%%%%%%%%%%%%%%%%%%%%%%%%%%%%%%%%%%%%%%%%
%\poemtitle{pt}
\begin{multicols}{2}
\settowidth{\versewidth}{Wohl dem, der den Herren scheuet}
\begin{verse}[\versewidth]
%der 112. Psalm\\
%wohl dem, der den Herren scheuet

\flagverse{1.} Wohl dem, der den Herren scheuet\\
und sich fürcht't vor seinem Gott,\\
selig, der sich herzlich freuet,\\
zu erfüllen sein Gebot!\\
Wer den Höchsten liebt und ehrt,\\
wird erfahren, wie sich mehrt\\
alles, was in seinem Leben\\
ihm vom Himmel ist gegeben.

\flagverse{2.} Seine Kinder werden stehen\\
wie die Rosen in der Blüt,\\
sein Geschlecht wird einhergehen\\
voller Gnad und Gottes Güt;\\
und was diesen Leib erhält,\\
wird der Herrscher aller Welt\\
reichlich und mit vollen Händen\\
ihnen in die Häuser senden.

\flagverse{3.} Das gerechte Tun der Frommen\\
steht gewiß und wanket nicht;\\
sollt auch gleich ein Wetter kommen,\\
bleibt doch Gott der Herr ihr Licht,\\
tröstet, stärket, schützt und macht,\\
daß nach ausgestandner Nacht\\
und nach hochbetrübtem Weinen\\
freud und Sonne wieder scheinen.

\flagverse{4.} Gottes Gnad, Huld und Erbarmen\\
bleibt den Frommen immer fest.\\
Wohl dem, der die Not der Armen\\
sich zu Herzen gehen läßt\\
und mit Liebe Gutes tut;\\
den wird Gott, das höchste Gut,\\
gnädiglich in seinen Armen\\
als ein liebster Vater wärmen.

\flagverse{5.} Wenn die schwarzen Wolken blitzen\\
vor dem Donner in der Luft,\\
wird er ohne Sorgen sitzen\\
wie ein Vöglein in der Kluft.\\
Er wird bleiben ewiglich,\\
auch wird sein Gedächtnis sich\\
hie und da auf allen Seiten\\
wie die edlen Zweig ausbreiten.

\flagverse{6.} Wenn das Unglück an will kommen,\\
das die rohen Sünder plagt,\\
bleibt der Mut ihm unbenommen\\
und das Herze unverzagt;\\
unverzagt, ohn Angst und Pein\\
bleibt das Herze, das sich fein\\
seinem Gott und Herren ergibet\\
und die, so verlassen, liebet.

\flagverse{7.} Wer betrübte gern erfreuet,\\
wird vom Höchsten wohl ergötzt,\\
was die milde Hand ausstreuet,\\
wird vom Himmel hoch ersetzt;\\
wer viel gibt, erlanget viel.\\
Was sein Herze wünscht und will,\\
das wird Gott mit gutem Willen\\
schon zu rechter Zeit erfüllen.

\flagverse{8.} Aber seines Feindes Freude\\
wird er untergehen sehn;\\
er, der Feind, vor großem Neide\\
wird zerbeißen seine Zähn,\\
er wird knirschen und mit Grimm\\
solches Glück mißgönnen ihm\\
und doch damit gar nichts wehren,\\
sondern sich nur selbst verzehren.

\end{verse}
\end{multicols}
%\attrib{\small{THZE}}
