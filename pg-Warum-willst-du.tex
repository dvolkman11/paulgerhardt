%StartInfo%%%%%%%%%%%%%%%%%%%%%%%%%%%%%%%%%%%%%%%%%%%%%%%%%%%%%%%%%%%%%%%%%%%%
%  Desc:  Warum willst du draußen stehen - Paul Gerhard
%  Desc:  Include 
%  Desc:  Zweispaltiger Satz mit \verse
%  Tags:  GERHARD VERSE EG361 INCLUDE
%  File:  pg-warum-willst-du-ich.tex
%  Autor: dv
%  Ref:   
%  Mod:   15.11.2016/dv/initial
%  Mod:   
%EndInfo%%%%%%%%%%%%%%%%%%%%%%%%%%%%%%%%%%%%%%%%%%%%%%%%%%%%%%%%%%%%%%%%%%%%%%

%\poemtitle{Warum willst du daraußen stehen}
\begin{multicols}{2}
\settowidth{\versewidth}{Warum willst du draußen stehen,}                                                  
\begin{verse}[\versewidth]
  
  \flagverse{1.} Warum willst du draußen stehen,\\
  du Gesegneter des Herrn?\\
  Laß dir bei mir einzugehen\\
  wohl gefallen, du mein Stern!\\
  Du, mein Jesu, meine Freud,\\
  Helfer in der rechten Zeit,\\
  hilf, o Heiland, meinem Herzen\\
  von den Wunden, die mich schmerzen.

  \flagverse{2.} Meine Wunden sind der Jammer,\\
  welchen oftmals Tag und Nacht\\
  des Gesetzes starker Hammer\\
  mir mit seinem Schrecken macht.\\
  O der schweren Donnerstimm,\\
  die mir Gottes Zorn und Grimm\\
  also tief ins Herze schläget,\\
  daß sich all mein Blut beweget.

  \flagverse{3.} Dazu kommt des Teufels Lügen,\\
  der mir alle Gnad absagt,\\
  als müßt ich nun ewig liegen\\
  in der Höllen, die ihn plagt;\\
  ja auch, was noch ärger ist,\\
  so zermartert und zerfrißt\\
  mich mein eigenes Gewissen\\
  mit vergift'ten Schlangenbissen.


  \flagverse{4.} Will ich dann mein Elend lindern\\
  und erleichtern meine Not\\
  bei der Welt und ihren Kindern,\\
  fall ich vollends in den Kot:\\
  Da ist Trost, der mich betrübt,\\
  Freude, die mein Unglück liebt,\\
  Helfer, die mir Herzleid machen,\\
  gute Freunde, die mein lachen.

  \flagverse{5.} In der Welt ist alles nichtig,\\
  nichts ist, das nicht kraftlos wär:\\
  Hab ich Hoheit, die ist flüchtig!\\
  Hab ich Reichtum, was ist's mehr\\
  als ein Stücklein armer Erd?\\
  Hab ich Lust, was ist sie wert?\\
  Was ist's, das mich heut erfreuet,\\
  das mich morgen nicht gereuet?

  \flagverse{6.} Aller Trost und alle Freude\\
  ruht in dir, Herr Jesu Christ;\\
  dein Erfreuen ist die Weide,\\
  da man sich recht fröhlich ißt.\\
  Leuchte mir, o Freudenlicht,\\
  ehe mir mein Herze bricht;\\
  laß mich, Herr, an dir erquicken;\\
  Jesu, komm, laß dich erblicken!

  \flagverse{7.} Freu dich, Herz, du bist erhöret,\\
  jetzo zeucht er bei dir ein,\\
  sein Gang ist zu dir gekehret,\\
  heiß ihn nur willkommen sein\\
  und bereite dich ihm zu,\\
  gib dich ganz zu seiner Ruh,\\
  öffne dein Gemüt und Seele,\\
  klag ihm, was dich drückt und quäle.

  \flagverse{8.} Siehst du, wie sich alles setzet,\\
  was dir vor zuwider stund?\\
  Hörst du, wie er dich ergötzet\\
  mit dem zuckersüßen Mund?\\
  Ei, wie läßt der große Drach\\
  all sein Tun und Toben nach!\\
  Er muß aus dem Vorteil ziehen\\
  und in seinen Abgrund fliehen.

  \flagverse{9.} Nun, du hast ein süßes Leben;\\
  alles, was du willst, ist dein.\\
  Christus, der sich dir ergeben,\\
  legt sein Reichtum bei dir ein.\\
  Seine Gnad ist deine Kron\\
  und du bist sein Hütt' und Thron.\\
  Er hat dich in sich geschlossen,\\
  nennt dich seinen Hausgenossen.

  \flagverse{10.} Seines Himmels güldne Decke\\
  spannt er um dich ringsherum,\\
  daß dich fort nicht mehr erschrecke\\
  deines Feindes Ungestüm.\\
  Seine Engel stellen sich\\
  dir zur Seiten, wann du dich\\
  hier willst oder dorthin wenden,\\
  tragen sie dich auf den Händen.

  \flagverse{11.} Was du Böses hast begangen,\\
  das ist alles abgeschafft.\\
  Gottes Liebe nimmt gefangen\\
  deiner Sünde Macht und Kraft.\\
  Christi Sieg behält das Feld,\\
  und was Böses in der Welt\\
  sich will wider dich erregen,\\
  wird zu lauter Glück und Segen.

  \flagverse{12.} Alles dient zu deinem Frommen,\\
  was dir bös und schädlich scheint,\\
  weil dich Christus angenommen\\
  und es treulich mit dir meint.\\
  Bleibst du deme wieder treu,\\
  ist's gewiß und bleibt dabei,\\
  daß du mit den Engeln droben\\
  ihn dort ewig werdest loben.

\end{verse}
\end{multicols}
%\attrib{\small{1653}}
