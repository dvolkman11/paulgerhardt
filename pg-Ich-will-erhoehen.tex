%StartInfo%%%%%%%%%%%%%%%%%%%%%%%%%%%%%%%%%%%%%%%%%%%%%%%%%%%%%%%%%%%%%%%%%%%%
%  Autor:
%  Titel:
%  File:
%  Ref:
%  Mod:
%EndInfo%%%%%%%%%%%%%%%%%%%%%%%%%%%%%%%%%%%%%%%%%%%%%%%%%%%%%%%%%%%%%%%%%%%%%%
%\poemtitle{pt}
\begin{multicols}{2}
\settowidth{\versewidth}{Gott ist ein Gott, der reichlich tröst't,}
\begin{verse}[\versewidth]
%der 34. Psalm\\

\flagverse{1.} Ich will erhöhen immerfort\\
und preisen meiner Seelen Hort,\\
ich will ihn herzlich ehren.\\
Wer Gott liebt, stimme mit mir ein,\\
laß alle, die betrübet sein,\\
ein Freudenliedlein hören.

\flagverse{2.} Gott ist ein Gott, der reichlich tröst't,\\
wer ihn nur sucht, der wird erlöst,\\
ich hab es selbst erfahren:\\
Sobald ein Ach im Himmel klingt,\\
kommt Heil und was uns Freude bringt\\
vom Himmel ab gefahren.

\flagverse{3.} Der starken Engel Kompanie\\
zieht fröhlich an, macht dort und hie\\
sich selbst zum Wall und Mauern,\\
da weicht und fleucht die böse Rott,\\
der Satan wird zu Hohn und Spott,\\
kein Unglück kann da dauern.

\flagverse{4.} Ach, was ist das für Süßigkeit!\\
Ach, schmecket alle, die ihr seid\\
mit Sinnen wohl begabet!\\
Kein Honig ist mehr auf der Erd\\
hinfort des süßen Namens wert;\\
gott ists, der uns recht labet.

\flagverse{5.} O seligs Herz, o seligs Haus,\\
das alle Lust stößt von sich aus\\
und diese Lust beliebet!\\
All andre Schönheit wird verrückt,\\
der aber bleibet stets geschmückt,\\
wer sich nur Gott ergibet.

\flagverse{6.} Der Kön'ge Gut, der Fürsten Geld\\
ist Kot und bleibet in der Welt,\\
wann die Besitzer sterben.\\
Wie oft verarmt ein reicher Mann!\\
Wer Gott vertraut, bleibt reich und kann\\
die ewgen Schätz ererben.

\flagverse{7.} Kommt her, ihr Kinder, hört mir zu!\\
Ich will euch zeigen, wie ihr Ruh\\
und Wohlfahrt könnt erjagen:\\
Ergebet euch und euren Sinn\\
zu Gottes Wohlgefallen hin\\
in allen euren Tagen!

\flagverse{8.} Bewahrt die Zung! Habt solchen Mut,\\
der Zank, und was zum Zanken tut,\\
nicht reget, sondern stillet:\\
So werden eure Tage sein\\
mit stillem Fried und süßem Schein\\
des Segens überfüllet.

\flagverse{9.} Laß ab vom Bösen, fleuch die Sünd,\\
o Mensch, und halt dich als ein Kind\\
des Vaters in der Höhe!\\
Du wirsts erfahren in der Tat,\\
wies dem, der ihm gefolget hat,\\
so herzlich wohl ergehe.

\flagverse{10.} Den Frommen ist Gott wieder fromm\\
und machet, daß geflossen komm\\
auf uns all sein Gedeihen;\\
sein Aug ist unser Sonnenlicht,\\
sein Ohr ist Tag und Nacht gericht,\\
zu hören unser Schreien.

\flagverse{11.} Zwar, wer Gott dient, muß leiden viel,\\
doch hat sein Leiden Maß und Ziel,\\
gott hilft ihm aus dem allen;\\
er sorgt für alle seine Bein,\\
er hebt sie auf und legt sie ein,\\
kein einzges muß verfallen.

\flagverse{12.} Gott sieht ins Herz und weiß gar wohl,\\
was uns macht Angst und Sorgen voll,\\
kein Tränlein fällt vergebens.\\
Er zählt sie all und legt darvor\\
uns treulich bei im Himmelschor\\
all Ehr des ewgen Lebens.

\end{verse}
\end{multicols}
%\attrib{\small{THZE}}
