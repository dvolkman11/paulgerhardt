%StartInfo%%%%%%%%%%%%%%%%%%%%%%%%%%%%%%%%%%%%%%%%%%%%%%%%%%%%%%%%%%%%%%%%%%%%
%  Autor:
%  Titel:
%  File:
%  Ref:
%  Mod:
%EndInfo%%%%%%%%%%%%%%%%%%%%%%%%%%%%%%%%%%%%%%%%%%%%%%%%%%%%%%%%%%%%%%%%%%%%%%
%\poemtitle{pt}
\begin{multicols}{2}
\settowidth{\versewidth}{Steck ein das Schwert, sprach unser Licht,}
\begin{verse}[\versewidth]
%o Mensch, beweine deine Sünd\\
%die Passion nach Sebaldus Heyd

\flagverse{1.} O Mensch beweine deine Sünd,\\
um welcher willen Gottes Kind\\
ein Mensche mußte werden;\\
er kam von seines Vaters Thron,\\
ward einer armen Jungfrau Sohn,\\
tat große Ding auf Erden.\\
Die Kranken macht er frisch und stark\\
und risse, was schon lag im Sarg,\\
dem Tod aus seinem Rachen;\\
bis daß er selbst durch Feindes Händ\\
am Kreuze seines Lebens End\\
in Schmerzen mußte machen.

\flagverse{2.} Denn als nun wieder Ostern war,\\
nahm er zu sich der Zwölfe Schar\\
und sprach mit treuem Munde:\\
Nach zweien Tagen kommt die Nacht,\\
da man das Osterlämmlein schlacht't;\\
dann ist auch meine Stunde.\\
Da ging die ganze Klerisei\\
zu Rat, wie sie ihm kämen bei,\\
hingegen die ihn liebte,\\
salbt ihn gar schön in Simons Haus,\\
der Herr strich diese Tat heraus,\\
schalt den, der sie betrübte.

\flagverse{3.} Das war der bös Ischarioth,\\
der seinen Herrn der bösen Rott\\
geschworen zu verraten.\\
Das fromme Lamm, der Heiland, kam,\\
aß süßes Brot und Osterlamm,\\
wie andre Juden taten.\\
Drauf stiftet er sein Fleisch und Blut,\\
des Neuen Testamentes Gut,\\
zu trinken und zu essen,\\
und stund hernach von seinem Ort,\\
wusch seine Jünger, redt'te Wort,\\
die nimmer zu vergessen.

\flagverse{4.} Er kam zum heilgen Öleberg;\\
da, da ging an das hohe Werk\\
mit Zittern und mit Zagen.\\
Die Erde nahm den Blutschweiß an,\\
der häufig aus ihm drang und rann,\\
der Himmel hört ihn sagen:\\
O Vaterherz, gefällt es dir,\\
so gehe dieser Kelch von mir;\\
wo nicht, gescheh dein Wille!\\
Und täte das zum dritten Mal,\\
indessen lag der Jünger Zahl\\
in Schlaf und süßer Stille.

\flagverse{5.} Ach, sprach das liebe treue Herz,\\
ihr liegt und schlaft; mich hat\\
der Schmerz und Todesangst umfangen.\\
Ach, wacht und betet, betet, wacht!\\
Damit ihr von des Feindes Macht\\
nicht werdet hintergangen.\\
Nun ist mein Stündlein vor der Tür,\\
steht auf! Da kommet her zu mir\\
mein Jünger und Verräter!\\
Er hatte kaum gehöret auf,\\
umringt ihn Judas und sein Hauf\\
als einen Übeltäter.

\flagverse{6.} Der Führer küßt ihn mit dem Mund,\\
und war doch nichts im Herzensgrund\\
als bittres Gift und Fluchen,\\
doch trat der Heiland frei dahin,\\
sprach klar und deutlich: seht, ich bin,\\
den eure Augen suchen.\\
Sucht ihr denn mich, so lasset gehn,\\
die ihr hier bei mir sehet stehn.\\
Meint hiermit seine Jünger.\\
Und als des Petri strenger Sinn\\
den Malchum schluge, heilt er ihn\\
am Ohr mit seinem Finger.

\flagverse{7.} Steck ein das Schwert, sprach unser Licht,\\
solch Arbeit dienet hieher nicht,\\
mein Kelch muß sein getrunken.\\
Drauf ist der Richter aller Welt\\
den Hohepriestern dargestellt;\\
und da ist auch gesunken\\
des Petri Herz und Leuenmut,\\
nicht zwar durch Schwert und Feuersglut,\\
nur durch ein bloßes Fragen,\\
ob er nicht Jesu Jünger sei?\\
Da fällt sein Glaube, Lieb und Treu,\\
weiß nichts als Nein zu sagen.

\flagverse{8.} Auf diesen Fall kam große Reu,\\
er fing an, da der Hahne schrei,\\
sehr bitterlich zu weinen.\\
Das Auge, das die Herzen sieht,\\
tät einen Blick, ließ Gnad und Güt\\
dem armen Petro scheinen.\\
Die falschen Zeugen traten dar\\
und red'ten viel, so nimmer wahr,\\
auch niemals wird geschehen;\\
drum auch der Herr unnötig schätzt,\\
daß er sein Wort dagegen setzt,\\
läßts durch den Wind zerwehen.

\flagverse{9.} Dem aber, dem er ward verklagt,\\
antwortet er, da er ihn fragt,\\
ob er von Gott geboren:\\
Ja, ich bin Mensch und Gottes Sohn,\\
der Welt zum Heil, zur Freud und Kron\\
vom Vater auserkoren;\\
ihr werdet meine Herrlichkeit\\
zur Rechten Gottes mit der Zeit\\
hoch in den Wolken sehen. –\\
das nennt der Lästrer Lästerwort,\\
da schrie ein jeder: Tod und Mord!\\
Da ging es an ein Schmähen.

\flagverse{10.} Man schlug, man spie ihm ins Gesicht.\\
O Wunder, Wunder, daß hier nicht\\
die Erde sich zerrissen!\\
O Wunder, daß nicht Gottes Grimm\\
mit seiner starken Donnerstimm\\
vom Himmel drein geschmissen!\\
Sie bunden ihm die Augen zu\\
und hatten weder Maß noch Ruh\\
im Höhnen und im Schlagen;\\
denn wenn sie schlugen, fragten sie:\\
Sag an, wer tats? Du kannst es je\\
als ein Prophete sagen!

\flagverse{11.} Und damit war es noch nicht aus.\\
Am Morgen ward er in das Haus\\
Pilati hingeführet.\\
Der Judas dacht den Sachen nach,\\
sein frecher Sinn sank hin und brach,\\
sein Herze ward gerühret;\\
es ward ihm leid, er hatte Reu,\\
weil aber war kein Trost dabei,\\
ging Seel und Leib zugrunde.\\
Er nahm ein grausam schrecklich End,\\
er und sein Name bleibt geschänd't\\
noch bis auf diese Stunde.

\flagverse{12.} Da Jesu vor Pilato stund,\\
war sehr viel Klag und gar kein Grund;\\
das meiste, das man triebe\\
war, daß er nichts mehr tu und lehr,\\
als was die Untertanen kehr\\
vons Kaisers Pflicht und Liebe,\\
dieweil er sich zum Könge macht.\\
Pilatus ward dahin gebracht,\\
daß er den Herren fragte,\\
ob er der Juden König wär?\\
Der Herr sprach: Ja, zu Gottes Ehr,\\
er wäre, was er sagte.

\flagverse{13.} Weil nun Herodes, dessen Hand\\
sonst herrscht im Galiläerland,\\
gleich damals war zugegen,\\
schickt ihm Pilatus Christum hin.\\
Des freut er sich in seinem Sinn,\\
ließ ihn zum Spott anlegen\\
ein weißes Kleid, ein arme Tracht,\\
und da man seiner gnug gelacht,\\
da schickt er ihn zurücke\\
Pilato heim; der ging zu Rat\\
und fand ihn rein von arger Tat,\\
unschuldig aller Tücke.

\flagverse{14.} Er nahm den Mörder Barrabam,\\
dem jedermann sonst war sehr gram,\\
den stellt er in die Mitten:\\
Hier sind der Übeltäter zwei,\\
sprach er zum Volk, es steht euch frei,\\
ihr möget einen bitten. –\\
Halt Jesum, schrie die tolle Schar,\\
laß Barrabam, wie er vor war,\\
frei ledig in das Seine. –\\
Was fang ich denn mit Jesu an? –\\
Ans Kreuz, ans Kreuz mit diesem Mann!\\
Antwortet die Gemeine.

\flagverse{15.} Da gab Pilatus Jesum hin\\
dem Kriegesvolk, das geißelt ihn\\
ohn alle Gnad und Schonen.\\
Der freche Haufe trat zuhauf\\
und setzen unserm Könge auf\\
von Dornen eine Kronen.\\
Er ward gehandelt als ein Tor;\\
sie äfften ihn mit einem Rohr\\
und schlugen ihn nicht wenig.\\
Du bist ein König, sagten sie,\\
drum beugen wir dir unsre Knie,\\
Glück zu, o Judenkönig!

\flagverse{16.} Als er nun übel zugericht't,\\
führt ihn Pilatus ins Gesicht\\
des Volks und sprach darneben:\\
Seht, seht doch, welch ein armer Wurm!\\
Nun wird sich euer Grimm und Sturm\\
einmal zufrieden geben. –\\
Nein, nein, sprach die vergallte Rott,\\
zum Kreuz, zum Kreuz! Nur immer tot! –\\
Pilatus wusch die Hände\\
und wollt im Kote reine sein;\\
dem aber, der in allem rein,\\
bestimmt er Tod und Ende.

\flagverse{17.} Das Leben ging zum bittern Tod\\
und mußte seine letzte Not\\
mit eignen Schultern tragen.\\
Er trug sein Kreuz und unsern Schmerz,\\
darüber führt manch Mutterherz\\
ein hochbetrübtes Klagen.\\
Weint nicht, sprach Christus, über mich,\\
ein jeder weine über sich\\
und über seine Sünde!\\
Es kommt die Zeit, da selig wird\\
gepreiset die, so nicht gebiert\\
und gar nicht weiß vom Kinde. 

\flagverse{18.} Da man nun kam zur Schädelstatt,\\
da ward, ders nicht verdienet hat,\\
bis in den Tod gekränket.\\
Zwar also, daß ein Mörderpaar\\
zur Seiten wurde hier und dar\\
er mitten ein gehenket.\\
Man nahm ihm Leben, Ehr und Blut;\\
den sanften Sinn, den frommen Mut,\\
den mußten sie ihm lassen.\\
Er liebte, die ihm weh getan,\\
rief seinen Vater für die an,\\
die ihm sein Herz zerfraßen.

\flagverse{19.} Pilatus heftet oben an\\
ein Überschrift, die jedermann,\\
der bei dem Kreuz gewesen,\\
Hebräer, Römer, Griechenland\\
und wer Vernunft hat und Verstand,\\
gar wohl hat können lesen.\\
Die Krieger nehmen ihm sein Kleid\\
und teilen sich diese Beut,\\
der Rock bleibt unzerstücket;\\
er wird dem Los anheimgestellt,\\
des soll er sein, wem jenes fällt;\\
laßt sehen, wem es glücket.

\flagverse{20.} Maria voller Lieb und Treu\\
stund an dem Kreuz, und auch dabei,\\
den unser Heiland liebte.\\
Sieh hier, sprach Jesus, Weib, dein Sohn!\\
Und Jünger, siehe deine Kron\\
und Mutter, die betrübte;\\
die laß dir ja befohlen sein! –\\
dies Wort, das drang ins Herz hinein\\
Johanni, dem geliebten.\\
Er nahm die auf und tat ihr wohl,\\
die andern machten Jammers voll\\
durch Bosheit, die sie übten.

\flagverse{21.} Viel Lästrer red'ten böse Ding,\\
auch einer, der zur Seiten hing,\\
goß auf ihn seinen Geifer.\\
Der aber an dem andern Ort\\
straft ihn und seine Lästerwort\\
mit großem Ernst und Eifer,\\
sprach Jesum an: O Himmelsfürst,\\
gedenke meiner, wenn du wirst\\
nun in dein Reich eingehen! –\\
Fürwahr, fürwahr, ich sage dir,\\
sprach Jesus, du wirst heut bei mir\\
im Paradiese stehen.

\flagverse{22.} Der Mittag kam und war doch Nacht,\\
die Sonn, die alles fröhlich macht,\\
war selbst mit Leid erfüllet.\\
Des Lichtes Schöpfer fühlet Pein,\\
drum mußt mit finstern Schatten sein\\
das schönste Licht verhüllet.\\
Eli! Rief Jesus, Gott, mein Gott,\\
wie läßt du mich in meiner Not\\
und Angst so gar alleine?\\
Und bald darauf: Mich dürstet sehr! –\\
das alles hört der Juden Heer\\
und weiß nicht, was er meine.

\flagverse{23.} Sie sind vom Zorne taub und blind,\\
hart wie ein Stein, der nichts empfindt,\\
auch gar nicht zu erweichen.\\
Sie nehmen aus dem Essigfaß\\
und machen einen Schwamm mit naß,\\
den lassen sie ihm reichen.\\
Ihr Herz ist voller Bitterkeit,\\
und damit sind sie auch bereit,\\
den, der jetzt stirbt, zu laben.\\
Viel machen aus dem Ernst ein Spiel\\
und sprechen: Halt, laß sehn, er will\\
Eliä Hilfe haben. –

\flagverse{24.} Er aber sprach: Es ist vollbracht!\\
Und darauf ward er von der Macht\\
des Todes überfallen.\\
Er neigte sich zur sanften Ruh,\\
er schloß die schwachen Augen zu\\
und schrie mit großem Schallen:\\
Nimm auf, nimm auf, Herr, meinen Geist,\\
du, mein herzliebster Vater, weißt,\\
wie du ihn sollst bewahren! –\\
Und also ist der große Held,\\
der Himmel, Erd und alles hält,\\
von dieser Welt gefahren.

\flagverse{25.} Er fuhr dahin. Im Augenblick\\
zerriß der Vorhang in zwei Stück,\\
die Erd erschrak und bebte.\\
Die Felsen sprangen in die Luft,\\
auch öffnet sich der Gräber Gruft\\
und was darinnen lebte.\\
Der Juden Herzen blieben hart,\\
allein der Hauptmann, dem da ward\\
die Wach am Kreuz befohlen,\\
der glaubt, und mit ihm sein Gesind,\\
es wäre Jesus Gottes Kind\\
und sagtens unverhohlen.

\flagverse{26.} Man brach den Schächern ihre Bein,\\
mein und dein Heiland blieb allein\\
an Beinen ungebrochen.\\
Das aber ist wahr und gewiß,\\
daß ein Soldat mit seinem Spieß\\
die Seiten ihm zerstochen,\\
aus welcher Wund ein edle Flut\\
von Blut und Wasser uns zugut\\
alsbald herausgeflossen.\\
Zuletzt ward er vom Kreuz gebracht\\
und, wohl beschickt, noch vor der Nacht\\
in Josephs Grab geschlossen.

\flagverse{27.} Die Juden hatten wohl gehört,\\
er würde, wie er selbst gelehrt,\\
von Toten auferstehen;\\
das halten sie für unwahr sein,\\
sie bilden ihnen aber ein,\\
es möchte List ergehen.\\
Drum siegeln sie des Grabes Tür\\
und legten starke Wache für;\\
umsonst und gar vergebens!\\
Der Herr dringt durch, kein Fels und Stein,\\
kein Wächter mag zu mächtig sein\\
dem Fürsten unsres Lebens.

\flagverse{28.} Nun seh und lern ein jedermann,\\
wie sehr viel Gutes uns getan\\
der Bräutgam unsrer Seelen:\\
Er nahm auf sich all unser Schuld\\
und ließ aus treuer Lieb und Huld\\
sich unserthalben quälen.\\
Zerknirschtes Herz, betrübter Geist,\\
den seine Sünde nagt und beißt,\\
laß Sorg und Kummer fallen,\\
weil unser Heiland Jesus Christ\\
ein Sündenopfer worden ist\\
dir und uns Menschen allen!

\end{verse}
\end{multicols}

\begin{center}
\settowidth{\versewidth}{Der, vor dem die Welt erschrickt,}
\begin{verse}[\versewidth]

\flagverse{29.} Du aber, der du sicher stehst,\\
und ohne Buße täglich gehst\\
in ungescheute Sünden,\\
betrachte, was für Straf und Last,\\
wenn du dein Maß gefüllet hast,\\
dich endlich werde finden!\\
Denn tut man das am grünen Baum,\\
so denke, was für Ort und Raum\\
der dürre werd erlangen.\\
O Jesu, gibt uns deinen Sinn\\
und bring uns alle, wo du hin\\
durch deinen Tod gegangen!
  
\end{verse}
\end{center}




%\attrib{\small{THZE}}
