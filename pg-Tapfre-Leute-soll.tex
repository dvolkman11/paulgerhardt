%StartInfo%%%%%%%%%%%%%%%%%%%%%%%%%%%%%%%%%%%%%%%%%%%%%%%%%%%%%%%%%%%%%%%%%%%%
%  Autor:
%  Titel:
%  File:
%  Ref:
%  Mod:
%EndInfo%%%%%%%%%%%%%%%%%%%%%%%%%%%%%%%%%%%%%%%%%%%%%%%%%%%%%%%%%%%%%%%%%%%%%%
%\poemtitle{pt}
\begin{multicols}{2}
\settowidth{\versewidth}{Er, Herr Sturm, pflanzt Palmenbäume;}
\begin{verse}[\versewidth]
%an den Landphysikus Samuel Sturm in Lukau\\
%tapfere Leute soll man loben\\
%(Aus »Fünfzehn-ästiger Nieder-Lausitzer Palm-Baum« des Samuel Sturm, 1675)

\flagverse{1.} Tapfre Leute soll man loben,\\
und was Tugend hat erhoben,\\
hebt auch billig unser Fleiß.\\
Laß, was schnöd ist, unten liegen,\\
was die Welt hat überstiegen,\\
deme bleibt sein Ruhm und Preis.

\flagverse{2.} Also wer, was andre haben\\
von des edlen Himmels Gaben,\\
weiß gebührlich anzuziehn,\\
dem gebührt vor andern allen,\\
daß zu seinem Wohlgefallen\\
Harf und Saiten sich bemühn.
\end{verse}
\end{multicols}

\begin{center}
\settowidth{\versewidth}{Er, Herr Sturm, pflanzt Palmenbäume;}
\begin{verse}[\versewidth]  
\flagverse{3.} Er, Herr Sturm, pflanzt Palmenbäume;\\
billig, daß hier keiner säume,\\
ihm ein Ehr und Dank zu tun.\\
Ich kann nichts mehr als nur bitten,\\
daß er stets mög in der Mitten\\
aller Tugendpalmen ruhn.
   
\end{verse}
\end{center}
\attrib{\small{An den Landphysikus Samuel Sturm in Lukau - 1675}}
