%StartInfo%%%%%%%%%%%%%%%%%%%%%%%%%%%%%%%%%%%%%%%%%%%%%%%%%%%%%%%%%%%%%%%%%%%%
%  Autor:
%  Titel:
%  File:
%  Ref:
%  Mod:
%EndInfo%%%%%%%%%%%%%%%%%%%%%%%%%%%%%%%%%%%%%%%%%%%%%%%%%%%%%%%%%%%%%%%%%%%%%%
%\poemtitle{pt}
\begin{multicols}{2}
\settowidth{\versewidth}{Wohl dem Menschen, der nicht wandelt}
\begin{verse}[\versewidth]
%der 1. Psalm\\
%wohl dem Menschen, der nicht wandelt

\flagverse{1.} Wohl dem Menschen, der nicht wandelt\\
in gottloser Leute Rat!\\
Wohl dem, der nicht unrecht handelt\\
noch tritt auf der Sünder Pfad;\\
der der Spötter Freundschaft fleucht\\
und von ihren Stühlen weicht,\\
der hingegen herzlich ehret\\
was uns Gott vom Himmel lehret.

\flagverse{2.} Wohl dem, der mit Lust und Freuden\\
das Gesetz des Höchsten treibt\\
und hie, als auf süßer Weiden,\\
tag und Nacht beständig bleibt;\\
dessen Segen wächst und blüht\\
wie ein Palmbaum, den man sieht\\
bei den Flüssen an der Seiten\\
seine frischen Zweig ausbreiten.

\flagverse{3.} Also, sag ich, wird auch grünen,\\
wer in Gottes Wort sich übt,\\
luft und Sonne wird ihm dienen,\\
bis er reiche Früchte gibt.\\
Seine Blätter werden alt\\
und doch niemals ungestalt.\\
Gott gibt Glück zu seinen Taten,\\
was er macht, muß wohl geraten.

\flagverse{4.} Aber wen die Sünd erfreuet,\\
mit dem gehts viel anders zu:\\
Er wird wie die Spreu zerstreuet\\
von dem Wind im schnellen Nu.\\
Wo der Herr sein Häuflein richt't,\\
da bleibt kein Gottloser nicht.\\
Summa: Gott liebt alle Frommen,\\
und wer bös ist, muß umkommen.

\end{verse}
\end{multicols}
%\attrib{\small{THZE}}
