%StartInfo%%%%%%%%%%%%%%%%%%%%%%%%%%%%%%%%%%%%%%%%%%%%%%%%%%%%%%%%%%%%%%%%%%%%
%  Autor:
%  Titel:
%  File:
%  Ref:
%  Mod:
%EndInfo%%%%%%%%%%%%%%%%%%%%%%%%%%%%%%%%%%%%%%%%%%%%%%%%%%%%%%%%%%%%%%%%%%%%%%
%\poemtitle{pt}
\begin{multicols}{2}
\settowidth{\versewidth}{Zeuch ein, laß mich empfinden}
\begin{verse}[\versewidth]

\flagverse{1.} Zeuch ein zu deinen Toren,\\
sei meines Herzens Gast,\\
der du, da ich geboren,\\
mich neu geboren hast,\\
o hochgeliebter Geist\\
des Vaters und des Sohnes,\\
mit beiden gleichen Thrones,\\
mit beiden gleich gespeist.

\flagverse{2.} Zeuch ein, laß mich empfinden\\
und schmecken deine Kraft,\\
die Kraft, die uns von Sünden\\
hilf und Errettung schafft.\\
Entsündge meinen Sinn,\\
daß ich mit reinem Geiste\\
dir Ehr und Dienste leiste,\\
die ich dir schuldig bin.

\flagverse{3.} Ich war ein wilder Reben,\\
du hast mich gut gemacht,\\
der Tod durchdrang mein Leben,\\
du hast ihn umgebracht\\
und in der Tauf erstickt,\\
als wie in einer Flute,\\
mit dessen Tod und Blute,\\
der uns im Tod erquickt.

\flagverse{4.} Du bist das heilig Öle,\\
dadurch gesalbet ist\\
mein Leib und meine Seele\\
dem Herren Jesu Christ\\
zum wahren Eigentum,\\
zum Priester und Propheten,\\
zum Könge, den in Nöten\\
Gott schützt vom Heiligtum.

\flagverse{5.} Du bist ein Geist, der lehret,\\
wie man recht beten soll,\\
dein Beten wird erhöret,\\
dein Singen klinget wohl.\\
Es steigt zum Himmel an,\\
es steigt und läßt nicht abe,\\
bis der geholfen habe,\\
der allen helfen kann.

\flagverse{6.} Du bist ein Geist der Freuden,\\
von Trauern hältst du nicht,\\
erleuchtest uns im Leiden\\
mit deines Trostes Licht.\\
Ach ja, wie manchesmal\\
hast du mit süßen Worten\\
mir aufgetan die Pforten\\
zum güldnen Freudensaal.

\flagverse{7.} Du bist ein Geist der Liebe,\\
ein Freund der Freundlichkeit,\\
willst nicht, daß uns betrübe\\
Zorn, Zank, Haß, Neid und Streit;\\
der Feindschaft bist du feind,\\
willst, daß durch Liebesflammen\\
sich wieder tun zusammen\\
die voller Zwietracht seind.

\flagverse{8.} Du, Herr, hast selbst in Händen\\
die ganze weite Welt,\\
kannst Menschenherzen wenden,\\
wie es dir wohlgefällt:\\
So gib doch deine Gnad\\
zum Fried und Liebesbanden,\\
verknüpf in allen Landen,\\
was sich getrennet hat.

\flagverse{9.} Ach, edle Friedensquelle,\\
schleuß deinen Abgrund auf\\
und gib dem Frieden schnelle\\
hier wieder seinen Lauf.\\
Halt ein die große Flut,\\
die Flut, die eingerissen\\
so, daß man siehet fließen,\\
wie Wasser, Menschenblut.

\flagverse{10.} Laß deinem Volk erkennen\\
die Vielheit seiner Sünd,\\
auch Gottes Grimm so brennen,\\
daß er bei uns entzünd\\
den ernsten bittern Schmerz\\
und Buße, die bereuet,\\
des sich zuerst gefreuet\\
ein weltergebnes Herz.

\flagverse{11.} Auf Buße folgt der Gnaden,\\
auf Reu der Freuden Blick,\\
sich bessern heilt den Schaden,\\
fromm werden bringet Glück.\\
Herr, tus zu deiner Ehr,\\
erweiche Stahl und Steine,\\
auf daß das Herze weine,\\
das böse sich bekehr.

\flagverse{12.} Erhebe dich und steure\\
dem Herzleid auf der Erd,\\
bring wieder und erneuere\\
die Wohlfahrt deiner Herd!\\
Laß blühen wie zuvorn\\
die Länder, so verheeret,\\
die Kirchen, so zerstöret\\
durch Krieg und Feuerszorn.

\flagverse{13.} Beschirm die Polizeien,\\
bau unsers Fürsten Thron,\\
daß er und wir gedeihen,\\
schmück, als mit einer Kron,\\
die Alten mit Verstand,\\
mit Frömmigkeit die Jugend,\\
mit Gottesfurcht und Tugend\\
das Volk im ganzen Land.

\flagverse{14.} Erfülle die Gemüter\\
mit reiner Glaubenszier,\\
die Häuser und die Güter\\
mit Segen für und für.\\
Vertreib den bösen Geist,\\
der dir sich widersetzet\\
und was dein Herz ergötzet,\\
aus unserm Herzen reißt.

\flagverse{15.} Gib Freudigkeit und Stärke,\\
zu stehen in dem Streit,\\
den Satans Reich und Werke\\
uns täglich anerbeut,\\
hilf kämpfen ritterlich,\\
damit wir überwinden\\
und ja zum Dienst der Sünden\\
kein Christ ergebe sich.

\flagverse{16.} Richt unser ganzes Leben\\
allzeit nach deinem Sinn,\\
und wenn wirs sollen geben\\
in Todes Rachen hin,\\
wenns mit uns hie wird aus,\\
so hilf uns fröhlich sterben\\
und nach dem Tod ererben\\
des ew'gen Lebens Haus.

\end{verse}
\end{multicols}
%\attrib{\small{THZE}}
