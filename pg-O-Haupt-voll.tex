%StartInfo%%%%%%%%%%%%%%%%%%%%%%%%%%%%%%%%%%%%%%%%%%%%%%%%%%%%%%%%%%%%%%%%%%%%
%  Autor:
%  Titel:
%  File:
%  Ref:
%  Mod:
%EndInfo%%%%%%%%%%%%%%%%%%%%%%%%%%%%%%%%%%%%%%%%%%%%%%%%%%%%%%%%%%%%%%%%%%%%%%
%\poemtitle{pt}
\begin{multicols}{2}
\settowidth{\versewidth}{O Haupt voll Blut und Wunden,}
\begin{verse}[\versewidth]

%\{7.} An das Angesicht (Salve caput cruentatum) O Haupt voll Blut und Wunden

\flagverse{1.} O Haupt voll Blut und Wunden,\\
voll Schmerz und voller Hohn,\\
o Haupt, zu Spott gebunden\\
mit einer Dornenkron!\\
O Haupt, sonst schön gezieret\\
mit höchster Ehr und Zier,\\
jetzt aber hoch schimpfieret,\\
gegrüßet seist du mir!

\flagverse{2.} Du edles Angesichte,\\
davor sonst schrickt und scheut\\
das große Weltgewichte,\\
wie bist du so bespeit,\\
wie bist du so erbleichet,\\
wer hat dein Augenlicht,\\
dem sonst kein Licht nicht gleichet,\\
so schändlich zugericht?

\flagverse{3.} Die Farbe deiner Wangen,\\
der roten Lippen Pracht\\
ist hin und ganz vergangen,\\
des blassen Todes Macht\\
hat alles hingenommen,\\
hat alles hingerafft,\\
und daher bist du kommen\\
von deines Leibes Kraft.

\flagverse{4.} Nun, was du, Herr, erduldet,\\
ist alles meine Last,\\
ich hab es selbst verschuldet,\\
was du getragen hast!\\
Schau her, hier steh ich Armer,\\
der Zorn verdienet hat,\\
gib mir, o mein Erbarmer,\\
den Anblick deiner Gnad.

\flagverse{5.} Erkenne mich, mein Hüter;\\
mein Hirte, nimm mich an!\\
Von dir, Quell aller Güter,\\
ist mir viel Guts getan.\\
Dein Mund hat mich gelabet\\
mit Milch und süßer Kost;\\
dein Geist hat mich begabet\\
mit mancher Himmelslust.

\flagverse{6.} Ich will hier bei dir stehen,\\
verachte mich doch nicht!\\
Von dir will ich nicht gehen,\\
wann dir dein Herze bricht.\\
Wann dein Haupt wird erblassen\\
im letzten Todesstoß,\\
alsdann will ich dich fassen\\
in meinen Arm und Schoß.

\flagverse{7.} Es dient zu meinen Freuden\\
und kommt mir herzlich wohl,\\
wenn ich in deinem Leiden,\\
mein Heil, mich finden soll.\\
Ach, möcht ich, o mein Leben,\\
an deinem Kreuze hier\\
mein Leben von mir geben,\\
wie wohl geschähe mir!

\flagverse{8.} Ich danke dir von Herzen,\\
o Jesu, liebster Freund,\\
für deines Todes Schmerzen,\\
da du's so gut gemeint.\\
Ach gib, daß ich mich halte\\
zu dir und deiner Treu\\
und, wenn ich nun erkalte,\\
in dir mein Ende sei.

\flagverse{9.} Wenn ich einmal soll scheiden,\\
so scheide nicht von mir;\\
wenn ich den Tod soll leiden,\\
so tritt du dann herfür.\\
Wenn mir am allerbängsten\\
wird um das Herze sein,\\
so reiß mich aus den Ängsten\\
kraft deiner Angst und Pein.

\flagverse{10.} Erscheine mir zum Schilde,\\
zum Trost in meinem Tod\\
und laß mich sehn dein Bilde\\
in deiner Kreuzesnot.\\
Da will ich nach dir blicken,\\
da will ich glaubensvoll\\
dich fest an mein Herz drücken:\\
Wer so stirbt, der stirbt wohl.
      
\end{verse}
\end{multicols}
%\attrib{\small{THZE}}
