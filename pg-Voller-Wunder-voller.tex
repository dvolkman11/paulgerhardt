%StartInfo%%%%%%%%%%%%%%%%%%%%%%%%%%%%%%%%%%%%%%%%%%%%%%%%%%%%%%%%%%%%%%%%%%%%
%  Autor:
%  Titel:
%  File:
%  Ref:
%  Mod:
%EndInfo%%%%%%%%%%%%%%%%%%%%%%%%%%%%%%%%%%%%%%%%%%%%%%%%%%%%%%%%%%%%%%%%%%%%%%
%\poemtitle{pt}
\begin{multicols}{2}
\settowidth{\versewidth}{Hier wächst ein geschickter Sohn,}
\begin{verse}[\versewidth]
%der wundervolle Ehestand\\
%voller Wunder, voller Kunst

\flagverse{1.} Voller Wunder, voller Kunst,\\
voller Weisheit, voller Kraft,\\
voller Hulde, Gnad und Gunst,\\
voller Labsal, Trost und Saft,\\
voller Wunder, sag ich noch,\\
ist der keuschen Liebe Joch.

\flagverse{2.} Die sich nach dem Angesicht\\
niemals hiebevor gekannt,\\
auch sonst im geringsten nicht\\
mit Gedanken zugewandt,\\
derer Herzen, derer Hand\\
knüpft Gott in ein Liebesband.

\flagverse{3.} Dieser Vater zeucht sein Kind,\\
jener seins dagegen auf,\\
beide treibt ihr sonder Wind,\\
ihre sondre Bahn und Lauf.\\
Aber wenn die Zeit nun dar,\\
wirds ein wohlgeratnes Paar.

\flagverse{4.} Hier wächst ein geschickter Sohn,\\
dort ein edle Tochter zu,\\
eines ist des andern Kron,\\
eines ist des andern Ruh,\\
eines ist des andern Licht,\\
wissens aber beide nicht.

\flagverse{5.} Bis solang es dem beliebt,\\
der die Welt im Schoße hält,\\
und zur rechten Stunde gibt\\
jedem, der ihm wohlgefällt;\\
da erscheint im Werk und Tat\\
der so tief verborgne Rat.

\flagverse{6.} Da wählt Ahasverus Blick\\
ihm die stille Esther aus,\\
den Tobias führt das Glück\\
in der frommen Sara Haus,\\
Davids bald gewandter Will\\
holt die klug Abigail.

\flagverse{7.} Jakob fleucht vor Esaus Schwert\\
und trifft seine Rahel an,\\
Joseph dient auf fremder Erd\\
und wird Asnath Herr und Mann,\\
Mose spricht bei Jethro ein,\\
da wird die Zipora sein.

\flagverse{8.} Jeder findet, jeder nimmt,\\
was der Höchst ihm ausersehn,\\
was im Himmel ist bestimmt,\\
pflegt auf Erden zu geschehn,\\
und was denn nun so geschicht,\\
das ist sehr wohl ausgericht.

\flagverse{9.} Öfters denkt man dies und dies\\
hätte können besser sein,\\
aber wie die Finsternis\\
nicht erreicht der Sonnen Schein,\\
also geht auch Menschensinn\\
hinter Gottes Weisheit hin.

\flagverse{10.} Laßt zusammen, was Gott fügt,\\
der weiß, wies am besten sei,\\
unser Denken fehlt und trügt,\\
sein Gedank ist mangelfrei.\\
Gottes Werk hat festen Fuß,\\
wann sonst alles fallen muß.

\flagverse{11.} Siehe frommen Kindern zu,\\
die im heilgen Stande stehn,\\
wie so wohl Gott ihnen tu,\\
wie so schön er lasse gehn\\
alle Taten ihrer Händ\\
auf ein gutes selges End.

\flagverse{12.} Ihrer Tugend werter Ruhm\\
steht in steter voller Blüt,\\
wann sonst aller Liebe Blum,\\
als ein Schatten, sich verzieht;\\
und wann aufhört alle Treu,\\
ist doch ihre Treue neu.

\flagverse{13.} Ihre Lieb ist immer frisch\\
und verjüngt sich fort und fort,\\
Liebe zieret ihren Tisch\\
und verzuckert alle Wort;\\
Liebe gibt dem Herzen Rast\\
in der Müh- und Sorgenlast.

\flagverse{14.} Gehts nicht allzeit wie es soll,\\
ist doch diese Liebe still,\\
hält sich in dem Kreuze wohl,\\
denkt, es sei des Herren Will,\\
und versichert sich mit Freud\\
einer künftig bessern Zeit.

\flagverse{15.} Unterdessen geht und fleußt\\
Gottes reicher Segenbach,\\
speist die Leiber, tränkt den Geist,\\
stärkt des Hauses Grund und Dach,\\
und was klein, gering und bloß,\\
macht er mächtig, viel und groß.

\flagverse{16.} Endlich wenn nun ganz vollbracht,\\
was Gott hier in dieser Welt\\
frommen Kindern zugedacht,\\
nimmt er sie ins Himmelszelt\\
und drückt sie mit großer Lust\\
selbst an seinen Mund und Brust.



\end{verse}
\end{multicols}


\begin{center}
\settowidth{\versewidth}{Der, vor dem die Welt erschrickt,}
\begin{verse}[\versewidth]

\flagverse{17.} Nun so bleibt ja voller Gunst,\\
voller Labsal, Trost und Saft,\\
voller Wunder, voller Kunst,\\
voller Weisheit, voller Kraft,\\
voller Wunder, sag ich noch,\\
bleibt der keuschen Liebe Joch.
  
\end{verse}
\end{center}



%\attrib{\small{THZE}}
