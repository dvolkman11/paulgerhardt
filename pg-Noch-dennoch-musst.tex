%StartInfo%%%%%%%%%%%%%%%%%%%%%%%%%%%%%%%%%%%%%%%%%%%%%%%%%%%%%%%%%%%%%%%%%%%%
%  Autor:
%  Titel:
%  File:
%  Ref:
%  Mod:
%EndInfo%%%%%%%%%%%%%%%%%%%%%%%%%%%%%%%%%%%%%%%%%%%%%%%%%%%%%%%%%%%%%%%%%%%%%%
%\poemtitle{pt}
\begin{multicols}{2}
\settowidth{\versewidth}{Noch dennoch mußt du drum nicht ganz}
\begin{verse}[\versewidth]
%noch dennoch mußt du drum nicht ganz in Traurigkeit versinken

\flagverse{1.} Noch dennoch mußt du drum nicht ganz\\
in Traurigkeit versinken,\\
gott wird des süßen Trostes Glanz\\
schon wieder lassen blinken.\\
Steh in Geduld, wart in der Still\\
und laß Gott machen, wie er will,\\
er kanns nicht böse machen.

\flagverse{2.} Ist denn dies unser erstes Mal,\\
daß wir betrübet werden?\\
Was haben wir als Angst und Qual\\
bisher gehabt auf Erden?\\
Wir sind wohl mehr so hoch gekränkt,\\
und hat doch Gott uns drauf geschenkt\\
ein Stündlein voller Freuden.

\flagverse{3.} So ist auch Gottes Meinung nicht,\\
wenn er uns Unglück sendet,\\
als sollt darum sein Angesicht\\
ganz von uns sein gewendet;\\
nein, sondern dieses ist sein Rat,\\
daß der, so ihn verlassen hat,\\
durchs Unglück wiederkehre.

\flagverse{4.} Denn das ist unser Fleisches Mut,\\
wenn wir in Freuden leben,\\
daß wir dann unserm höchsten Gut\\
am ersten Urlaub geben,\\
wir sind von Erd und halten wert\\
viel mehr, was hier ist auf der Erd\\
als was im Himmel wohnet.

\flagverse{5.} Drum fährt uns Gott durch unsern Sinn\\
und läßt uns Weh geschehen;\\
er nimmt oft, was uns lieb, dahin,\\
damit wir aufwärts sehen\\
und uns zu seiner Güt und Macht,\\
die wir bisher nicht groß geacht,\\
als Kinder wiederfinden.

\flagverse{6.} Tun wir nun das, ist er bereit,\\
uns wieder anzunehmen,\\
macht aus dem Leide lauter Freud\\
und Lachen aus dem Grämen,\\
und ist ihm das gar schlichte Kunst;\\
wen er umfängt mit Lieb und Gunst,\\
dem ist geschwind geholfen.

\flagverse{7.} Drum falle, du betrübtes Heer,\\
in Demut vor ihm nieder;\\
sprich: Herr, wir geben dir die Ehr,\\
ach, nimm uns Sünder wieder\\
in deine Gnade! Reiß die Last,\\
die du uns aufgeleget hast,\\
hinweg, heil unsern Schaden!

\flagverse{8.} Denn Gnade gehet doch vor Recht,\\
zorn muß der Liebe weichen,\\
wenn wir erliegen, muß uns schlecht\\
gott sein Erbarmen reichen;\\
dies ist die Hand, die uns erhält,\\
wo wir die lassen, bricht und fällt\\
all unser Tun in Haufen.

\flagverse{9.} Auf Gottes Liebe mußt du stehn\\
und dich nicht lassen fällen,\\
wenn auch der Himmel ein wollt gehn\\
und alle Welt zerschellen;\\
gott hat uns Gnade zugesagt,\\
sein Wort ist klar, wer sich drauf wagt,\\
dem kann es nimmer fehlen.

\flagverse{10.} So darfst du auch an seiner Kraft\\
gar keinen Zweifel haben.\\
Wer ists, der alle Dinge schafft?\\
Wer teilt aus alle Gaben?\\
Gott tuts! Und das ist auch der Mann,\\
der Rat und Tat erfinden kann,\\
wann jedermann verzaget.

\flagverse{11.} Deucht dir die Hilf unmöglich sein,\\
so sollst du gleichwohl wissen:\\
Gott räumt uns dieses nimmer ein,\\
daß er sich laß einschließen\\
in unsers Sinnes engen Stall;\\
sein Arm ist frei, tut überall\\
viel mehr als wir verstehen.

\flagverse{12.} Was ist sein ganzes wertes Reich\\
als lauter Wundersachen?\\
Er hilft und baut, wann wir uns gleich\\
des gar kein Hoffnung machen,\\
und das ist seines Namens Ruhm,\\
den du, wann du sein Heiligtum\\
willst sehen, ihm mußt geben.

\end{verse}
\end{multicols}
%\attrib{\small{THZE}}
