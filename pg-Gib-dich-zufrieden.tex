%StartInfo%%%%%%%%%%%%%%%%%%%%%%%%%%%%%%%%%%%%%%%%%%%%%%%%%%%%%%%%%%%%%%%%%%%%
%  Autor:
%  Titel:
%  File:
%  Ref:
%  Mod:
%EndInfo%%%%%%%%%%%%%%%%%%%%%%%%%%%%%%%%%%%%%%%%%%%%%%%%%%%%%%%%%%%%%%%%%%%%%%
%\poemtitle{Gib dich zufrieden und sei stille}
\begin{multicols}{2}
\settowidth{\versewidth}{Er ist voll Lichtes, Trosts und Gnaden,}
\begin{verse}[\versewidth]
%der 37. Psalm (Vers 7)

\flagverse{1.} Gib dich zufrieden und sei stille\\
in dem Gotte deines Lebens;\\
in ihm ruht aller Freuden Fülle,\\
ohn ihn mühst du dich vergebens.\\
Er ist dein Quell\\
und deine Sonne,\\
scheint täglich hell\\
zu deiner Wonne.\\
Gib dich zufrieden!

\flagverse{2.} Er ist voll Lichtes, Trosts und Gnaden,\\
ungefärbten treuen Herzens;\\
wo er steht, tut dir keinen Schaden\\
auch die Pein des größten Schmerzens;\\
Kreuz, Angst und Not\\
kann er bald wenden,\\
ja auch den Tod\\
hat er in Händen.\\
Gib dich zufrieden!

\flagverse{3.} Wie dirs und andern oft ergehe,\\
ist ihm wahrlich nicht verborgen,\\
er sieht und kennet aus der Höhe\\
der betrübten Herzen Sorgen.\\
Er zählt den Lauf\\
der heißen Tränen\\
und faßt zuhauf\\
all unser Sehnen.\\
Gib dich zufrieden!

\flagverse{4.} Wenn gar kein einzger mehr auf Erden,\\
dessen Treue darfst du trauen,\\
alsdann will er dein Treuster werden\\
und zu deinem Besten schauen.\\
Er weiß dein Leid\\
und heimlich Grämen,\\
auch weiß er Zeit,\\
dich zu benehmen.\\
Gib dich zufrieden!

\flagverse{5.} Er hört die Seufzer deiner Seelen\\
und des Herzens stilles Klagen,\\
und was du keinem darfst erzählen,\\
magst du Gott gar kühnlich sagen,\\
er ist nicht fern,\\
steht in der Mitten,\\
hört bald und gern\\
der Armen Bitten.\\
Gib dich zufrieden!

\flagverse{6.} Laß dich dein Elend nicht bezwingen,\\
halt an Gott, so wirst du siegen;\\
ob alle Fluten einher gingen,\\
dennoch mußt du oben liegen.\\
Denn wenn du wirst\\
zu hoch beschweret,\\
hat Gott, dein Fürst,\\
dich schon erhöret.\\
Gib dich zufrieden!

\begin{verbatim}



\end{verbatim}

\flagverse{7.} Was sorgst du für dein armes Leben,\\
wie du's halten wollst und nähren?\\
Der dir das Leben hat gegeben,\\
wird auch Unterhalt bescheren.\\
Er hat ein Hand\\
voll aller Gaben,\\
da See und Land\\
sich muß von laben.\\
Gib dich zufrieden!

\flagverse{8.} Der allen Vöglein in den Wäldern\\
ihr bescheidnes Körnlein weiset,\\
der Schaf und Rinder in den Feldern\\
alle Tage tränkt und speiset,\\
der wird ja auch\\
dich eingen füllen\\
und deinen Bauch\\
zur Notdurft stillen.\\
Gib dich zufrieden!

\flagverse{9.} Sprich nicht: Ich sehe keine Mittel;\\
wo ich such, ist nichts zum Besten;\\
denn das ist Gottes Ehrentitel:\\
Helfen, wann die Not am größten.\\
Wenn ich und du\\
ihn nicht mehr spüren,\\
da schickt er zu,\\
uns wohl zu führen.\\
Gib dich zufrieden!

\flagverse{10.} Bleibt gleich die Hilf in etwas lange,\\
wird sie dennoch endlich kommen,\\
macht dir das Harren angst und bange,\\
glaube mir, es ist dein Frommen.\\
Was langsam schleicht,\\
faßt man gewisser,\\
und was verzeucht,\\
ist desto süßer.\\
Gib dich zufrieden!

\flagverse{11.} Nimm nicht zu Herzen, was die Rotten\\
deiner Feinde von dir dichten,\\
laß sie nur immer weidlich spotten,\\
Gott wirds hören und recht richten.\\
Ist Gott dein Freund\\
und deiner Sachen,\\
was kann dein Feind,\\
der Mensch, groß machen!\\
Gib dich zufrieden!

\flagverse{12.} Hat er doch selbst auch wohl das Seine,\\
wenn ers sehen könnt und wollte.\\
Wo ist ein Glück so klar und reine,\\
dem nicht etwas fehlen sollte?\\
Wo ist ein Haus,\\
das könnte sagen:\\
Ich weiß durchaus\\
von keinen Plagen?\\
Gib dich zufrieden!

\begin{verbatim}




\end{verbatim}

\flagverse{13.} Es kann und mag nicht anders werden,\\
alle Menschen müssen leiden;\\
was webt und lebet auf der Erden,\\
kann das Unglück nicht vermeiden.\\
Des Kreuzes Stab\\
schlägt unsre Lenden\\
bis in das Grab:\\
Da wird sichs enden.\\
Gib dich zufrieden!

\flagverse{14.} Es ist ein Ruhetag vorhanden,\\
da uns unser Gott wird lösen,\\
er wird uns reißen aus den Banden\\
dieses Leibs und allem Bösen.\\
Es wird einmal\\
der Tod herspringen\\
und aus der Qual\\
uns sämtlich bringen.\\
Gib dich zufrieden!
\end{verse}
\end{multicols}

\begin{center}
\settowidth{\versewidth}{Der, vor dem die Welt erschrickt,}
\begin{verse}[\versewidth]

\flagverse{15.} Er wird uns bringen zu den Scharen\\
der Erwählten und Getreuen,\\
die hier mit Frieden abgefahren,\\
sich auch nun im Frieden freuen,\\
da sie den Grund,\\
der nicht kann brechen,\\
den ewgen Mund\\
selbst hören sprechen:\\
Gib dich zufrieden!

\end{verse}
\end{center}

%\attrib{\small{THZE}}
