%StartInfo%%%%%%%%%%%%%%%%%%%%%%%%%%%%%%%%%%%%%%%%%%%%%%%%%%%%%%%%%%%%%%%%%%%%
%  Autor:
%  Titel:
%  File:
%  Ref:
%  Mod:
%EndInfo%%%%%%%%%%%%%%%%%%%%%%%%%%%%%%%%%%%%%%%%%%%%%%%%%%%%%%%%%%%%%%%%%%%%%%
%\poemtitle{Der Herr, der aller Enden regiert, der ist mein Hirt und Hüter}
\begin{multicols}{2}
\settowidth{\versewidth}{Der Herr, der aller Enden}
\begin{verse}[\versewidth]
%Der 23. Psalm\\
%Der Herr, der aller Enden regiert, der ist mein Hirt und Hüter

\flagverse{1.} Der Herr, der aller Enden\\
regiert mit seinen Händen,\\
der Brunn der ewgen Güter,\\
der ist mein Hirt und Hüter.

\flagverse{2.} So lang ich diesen habe,\\
fehlt mirs an keiner Gabe,\\
der Reichtum seiner Fülle\\
gibt mir die Füll und Hülle.

\flagverse{3.} Er lässet mich mit Freuden\\
auf grüner Aue weiden,\\
führt mich zu frischen Quellen,\\
schafft Rat in schweren Fällen.

\flagverse{4.} Wann meine Seele zaget\\
und sich mit Sorgen plaget,\\
weiß er sie zu erquicken,\\
aus aller Not zu rücken.

\flagverse{5.} Er lehrt mich tun und lassen,\\
führt mich auf rechter Straßen,\\
läßt Furcht und Angst sich stillen\\
um seines Namens willen.

\flagverse{6.} Und ob ich gleich vor andern\\
im finstern Tal muß wandern,\\
fürcht ich doch keine Tücke,\\
bin frei vom Ungelücke.

\flagverse{7.} Denn du stehst mir zur Seiten,\\
schützst mich vor bösen Leuten,\\
dein Stab, Herr, und dein Stecken\\
benimmt mir all mein Schrecken.

\flagverse{8.} Du setzest mich zu Tische,\\
machst, daß ich mich erfrische,\\
wann mir mein Feind viel Schmerzen\\
erweckt in meinem Herzen.

\flagverse{9.} Du salbst mein Haupt mit Öle\\
und füllest meine Seele,\\
die leer und durstig saße,\\
mit vollgeschenktem Maße.

\flagverse{10.} Barmherzigkeit und Gutes\\
wird mein Herz gutes Mutes,\\
voll Lust, voll Freud, voll Lachen,\\
so lang ich lebe, machen.

\flagverse{11.} Ich will dein Diener bleiben\\
und dein Lob herrlich treiben\\
im Hause, da du wohnest\\
und Frommsein wohl belohnest.

\flagverse{12.} Ich will dich hier auf Erden\\
und dort, da wir dich werden\\
selbst schaun, im Himmel droben\\
hoch rühmen, singn und loben.

\end{verse}
\end{multicols}
%\attrib{\small{THZE}}
