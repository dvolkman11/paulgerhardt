%StartInfo%%%%%%%%%%%%%%%%%%%%%%%%%%%%%%%%%%%%%%%%%%%%%%%%%%%%%%%%%%%%%%%%%%%%
%  Autor:
%  Titel:
%  File:
%  Ref:
%  Mod:
%EndInfo%%%%%%%%%%%%%%%%%%%%%%%%%%%%%%%%%%%%%%%%%%%%%%%%%%%%%%%%%%%%%%%%%%%%%%
%\poemtitle{pt}
\begin{multicols}{2}
\settowidth{\versewidth}{Weint, und weint gleichwohl nicht zu sehr,}
\begin{verse}[\versewidth]
%weint, und weint gleichwohl nicht zu sehr\\
%auf den Tod der kleinen Margaretha Zarlang\\
%an die Eltern (1667)

\flagverse{1.} Weint, und weint gleichwohl nicht zu sehr,\\
denn was euch abgestorben,\\
ist wohl daran und hat nunmehr\\
das beste Teil erworben!\\
Es ist hindurch ins Vaterland,\\
nachdem der harte schwere Stand,\\
der hier war, überstanden.

\flagverse{2.} Hier sind wir auf der wilden See\\
im Sturm und tiefen Fluten,\\
da gehts uns, daß vor Ach und Weh\\
das Herze möchte bluten.\\
Sobald der Mensch ins Leben tritt,\\
sobald kommt auch die Trübsal mit\\
und folgt ihm auf dem Fuße.

\flagverse{3.} Da ist kein Kind so zart und klein,\\
es muß sein Leiden tragen;\\
ein jedes hat sein Angst und Pein,\\
kanns oft nicht von sich sagen;\\
und wenns auch gleich noch etwas spricht,\\
so bleibt doch drum das Elend nicht\\
von seines Leibes Gliedern.

\flagverse{4.} Kommts auf die Bein und wächst herzu,\\
lernt schwarz und weiß verstehen,\\
so merkts, was man auf Erden tu,\\
wie Menschenwerke gehen,\\
sieht lauter Böses, gar nichts Guts,\\
darüber wird betrübtes Muts\\
und fängt sich an zu grämen.

\flagverse{5.} Hilft endlich Gott zur vollen Kraft\\
und reifen Mannesjahren,\\
tritts in den Stand, da man was schafft,\\
da kanns denn recht erfahren,\\
wie alles so voll Mühe sei;\\
und hat doch selten mehr dabei\\
als wenig gute Stunden.

\flagverse{6.} Das alles sieht der Vater an,\\
die Mutter nimmts zu Herzen,\\
und niemand ist, der helfen kann;\\
da kommen denn die Schmerzen,\\
die häufen sich ohn Unterlaß\\
und halten stets die Augen naß\\
bei Eltern und bei Kindern.

\flagverse{7.} Drum laßts Gott machen, wie er will!\\
Er weiß die besten Weisen.\\
Wer balde kommt zu seinem Ziel,\\
der darf nicht ferne reisen;\\
und wer bei Zeit wird ausgespannt,\\
der darf des Jammers schweren Stand\\
nicht allzu lange ziehen.

\flagverse{8.} Was unser Welt ist zugedacht,\\
darf euer Kind nicht schmecken;\\
es schläft und ruht, bis Gottes Macht\\
es wieder wird erwecken.\\
Und wann ihr kommt ins Himmels Saal,\\
so wird euch eurer Kinder Zahl\\
mit großer Lust empfangen.
\end{verse}
\end{multicols}

\begin{center}
\settowidth{\versewidth}{Der, vor dem die Welt erschrickt,}
\begin{verse}[\versewidth]


\flagverse{9.} So schlaf nun wohl, du herzes Kind,\\
doch tröste Gott die Deinen,\\
wann jetzt ihr Herz und Auge rinnt,\\
und kehr ihr bittres Weinen\\
zu seiner Zeit, die er bestellt,\\
auf Weis und Art, die ihm gefällt,\\
in Freud und süßes Singen.

  
\end{verse}
\end{center}

%\attrib{\small{THZE}}
