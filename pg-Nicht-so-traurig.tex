%StartInfo%%%%%%%%%%%%%%%%%%%%%%%%%%%%%%%%%%%%%%%%%%%%%%%%%%%%%%%%%%%%%%%%%%%%
%  Autor:
%  Titel:
%  File:
%  Ref:
%  Mod:
%EndInfo%%%%%%%%%%%%%%%%%%%%%%%%%%%%%%%%%%%%%%%%%%%%%%%%%%%%%%%%%%%%%%%%%%%%%%
%\poemtitle{pt}
\begin{multicols}{2}
\settowidth{\versewidth}{Nicht so traurig, nicht so sehr,}
\begin{verse}[\versewidth]
%nicht so traurig, nicht so sehr, meine Seele, sei betrübt\\
%(1. Timoth. 6, 6 ff.)

\flagverse{1.} Nicht so traurig, nicht so sehr,\\
meine Seele, sei betrübt,\\
daß dir Gott Glück, Gut und Ehr\\
nicht so viel wie andern gibt!\\
Nimm vorlieb mit deinem Gott!\\
Hast du Gott, so hats nicht not.

\flagverse{2.} Du noch einzig Menschenkind\\
habt ein Recht in dieser Welt;\\
alle, die geschaffen sind,\\
sind nur Gäst im fremden Zelt;\\
Gott ist Herr in seinem Haus,\\
wie er will, so teilt er aus.

\flagverse{3.} Bist du doch darum nicht hier,\\
daß du Erden haben sollt,\\
schau den Himmel über dir,\\
da, da ist dein edles Gold,\\
da ist Ehre, da ist Freud,\\
Freud ohn End, Ehr ohne Neid.

\flagverse{4.} Der ist albern, der sich kränkt\\
um ein Hand voll Eitelkeit,\\
wenn ihm Gott dagegen schenkt\\
schätze der Beständigkeit;\\
bleibt der Zentner dein Gewinn,\\
fahr der Heller immer hin!

\flagverse{5.} Schaue alle Güter an,\\
die dein Herz für Güter hält,\\
keines mit dir gehen kann,\\
wann du gehest aus der Welt;\\
alles bleibet hinter dir,\\
wann du trittst ins Grabes Tür.

\flagverse{6.} Aber was die Seele nährt,\\
Gottes Huld und Christi Blut,\\
wird von keiner Zeit verzehrt,\\
ist und bleibet allzeit gut;\\
Erdengut zerfällt und bricht,\\
Seelengut das schwindet nicht.

\flagverse{7.} Ach, wie bist du doch so blind\\
und im Denken unbedacht!\\
Augen hast du, Menschenkind,\\
und hast doch noch nie betracht\\
deiner Augen helles Glas:\\
Siehe, welch ein Schatz ist das!

\flagverse{8.} Zähle deine Finger her\\
und der andern Glieder Zahl;\\
keins ist, das dir unwert wär,\\
ehrst und liebst sie allzumal;\\
keines gäbst du weg um Gold,\\
wenn man dirs abnehmen wollt.

\flagverse{9.} Nun, so gehe in den Grund\\
deines Herzens, das dich lehrt,\\
wie viel Gutes alle Stund\\
dir von oben wird beschert:\\
Du hast mehr als Sand am Meer,\\
und willst doch noch immer mehr.

\flagverse{10.} Wüßte, der im Himmel lebt,\\
daß dir wäre nütz und gut,\\
wonach so begierig strebt\\
dein verblendet Fleisch und Blut,\\
würde seine Frömmigkeit\\
dich nicht lassen unerfreut.

\flagverse{11.} Gott ist deiner Liebe voll\\
und von ganzem Herzen treu;\\
wenn du wünschest, prüft er wohl,\\
wie dein Wunsch beschaffen sei;\\
ist dirs gut, so geht ers ein,\\
ists dein Schade, spricht er: Nein.

\flagverse{12.} Unterdessen trägt sein Geist\\
dir in deines Herzens Haus\\
Manna, das die Engel speist,\\
ziert und schmückt es herrlich aus,\\
ja erwählet, dir zum Heil,\\
dich zu seinem Gut und Teil.

\flagverse{13.} Ei, so richte dich empor,\\
du betrübtes Angesicht!\\
Laß das Seufzen, nimm hervor\\
deines Glaubens Freudenlicht;\\
das behalt, wenn dich die Nacht\\
deines Kummers traurig macht.

\flagverse{14.} Setze als ein Himmelssohn\\
deinem Willen Maß und Ziel,\\
rühre stets vor Gottes Thron\\
deines Dankens Saitenspiel,\\
weil dir schon gegeben ist\\
mehres als du würdig bist.

\end{verse}
\end{multicols}
%\attrib{\small{THZE}}

\begin{center}
\settowidth{\versewidth}{Der, vor dem die Welt erschrickt,}
\begin{verse}[\versewidth]

\flagverse{15.} Führe deines Lebens Lauf\\
allzeit Gottes eingedenk.\\
Wie es kommt, nimm alles auf\\
als ein wohlbedacht Geschenk.\\
Geht dirs widrig, laß es gehn!\\
Gott und Himmel bleibt dir stehn.
  
\end{verse}
\end{center}



