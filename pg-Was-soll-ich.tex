%StartInfo%%%%%%%%%%%%%%%%%%%%%%%%%%%%%%%%%%%%%%%%%%%%%%%%%%%%%%%%%%%%%%%%%%%%
%  Autor:
%  Titel:
%  File:
%  Ref:
%  Mod:
%EndInfo%%%%%%%%%%%%%%%%%%%%%%%%%%%%%%%%%%%%%%%%%%%%%%%%%%%%%%%%%%%%%%%%%%%%%%
%\poemtitle{pt}
\begin{multicols}{2}
\settowidth{\versewidth}{Sollt ich nicht billig deiner Tat}
\begin{verse}[\versewidth]
%was soll ich doch, o Ephraim, was soll ich aus dir machen?\\
%(Hosea 11, 8/9)

\flagverse{1.} Was soll ich doch, o Ephraim,\\
was soll ich aus dir machen?\\
Der du so oftmals meinen Grimm\\
hast pflegen zu verlachen?\\
Soll ich dich schützen, Israel?\\
Soll ich dir deine freche Seel\\
hinfürder noch bewahren?\\
Aus welcher doch von Jugend auf\\
ein solcher großer Sündenhauf\\
ohn alle Scheu gefahren.

\flagverse{2.} Sollt ich nicht billig deiner Tat\\
und Leben gleich mich stellen?\\
Und dich wie Sodom ohne Gnad\\
und wie Adama fällen?\\
Sollt ich nicht billig meine Glut\\
auf dein verfluchtes Gut und Blut\\
wie auf Zeboim schütten?\\
Dieweil du ja mein Wort und Bahn\\
fast ärger noch, als sie getan,\\
bis hieher überschritten.

\flagverse{3.} Ja, billig sollt ich dich dahin\\
in alles Herzleid senken,\\
allein es will mir nicht zu Sinn,\\
ich hab ein andres Denken;\\
mein Herze will durchaus nicht dran,\\
daß es dir tu, wie du getan,\\
es brennt für Gnad und Liebe;\\
mich jammert dein von Herzen sehr\\
und kann nicht sehen, daß das Heer\\
der Höllen dich betrübe.

\flagverse{4.} Ich kann und mag nicht, wie du wohl\\
verdienet, dich verderben;\\
ich bin und bleib Erbarmens voll\\
und halte nichts vom Sterben;\\
denn ich bin Gott, der treue Gott,\\
mitnichten einer aus der Rott\\
der bösen Adamskinder,\\
die ohne Treu und Glauben seind\\
und werden ihren Feinden feind\\
und täglich größre Sünder.

\flagverse{5.} So bin ich nicht, das glaube mir,\\
und nimms recht zu Gemüte,\\
ich bin der Heilge unter dir,\\
der ich aus lauter Güte\\
für meine Feinde in den Tod\\
und in des bittern Kreuzes Not\\
mich als ein Lamm begeben;\\
ich, ich will tragen deine Last,\\
die du dir, Mensch, gehäufet hast,\\
auf daß du mögest leben.

\flagverse{6.} O heilger Herr, o ewges Heil,\\
versöhner meiner Sünden,\\
ach, heilge mich und laß mich teil\\
in, bei und an dir finden!\\
Erwecke mich zur wahren Reu\\
und gib, daß ich dein edle Treu\\
im festen Glauben fasse;\\
auch töte mich durch deinen Tod,\\
damit ich allen Sündenkot\\
hinfort von Herzen hasse.

\end{verse}
\end{multicols}
%\attrib{\small{THZE}}
