%StartInfo%%%%%%%%%%%%%%%%%%%%%%%%%%%%%%%%%%%%%%%%%%%%%%%%%%%%%%%%%%%%%%%%%%%%
%  Autor:
%  Titel:
%  File:
%  Ref:
%  Mod:
%EndInfo%%%%%%%%%%%%%%%%%%%%%%%%%%%%%%%%%%%%%%%%%%%%%%%%%%%%%%%%%%%%%%%%%%%%%%
%\poemtitle{pt}
\begin{multicols}{2}
\settowidth{\versewidth}{Groß ist der Herr und mächtig}
\begin{verse}[\versewidth]
%der 111. Psalm\\
%ich will mit Danken kommen

\flagverse{1.} Ich will mit Danken kommen\\
in den gemeinen Rat\\
der rechten wahren Frommen,\\
die Gottes Rat und Tat\\
mit süßem Lohn erhöhn;\\
zu denen will ich treten,\\
und soll mein Dank und Beten\\
von ganzem Herzen gehn.

\flagverse{2.} Groß ist der Herr und mächtig,\\
groß ist auch, was er macht.\\
Wer aufmerkt und andächtig\\
nimmt seine Werk in Acht,\\
hat eitel Lust daran.\\
Was seine Weisheit setzet\\
und ordnet, das ergötzet\\
und ist sehr wohl getan.

\flagverse{3.} Sein Heil und große Güte\\
steht fest und unbewegt,\\
damit auch dem Gemüte,\\
das uns im Herzen schlägt,\\
dieselbe nicht entweich,\\
hat er zum Glaubenszunder\\
ein Denkmal seiner Wunder\\
gestift't in seinem Reich.

\flagverse{4.} Gott ist voll Gnad und Gaben,\\
gibt Speis aus milder Hand,\\
die Seinen wohl zu laben,\\
die ihm allein bekannt;\\
denkt stets an seinen Bund,\\
gibt denen, die er weiden\\
will mit dem Erb der Heiden,\\
all seine Taten kund.

\flagverse{5.} Das Wirken seiner Hände\\
und was er uns gebeut,\\
das hat ein gutes Ende,\\
bringt reichen Trost und Freud\\
und Wahrheit, die nicht treugt.\\
Gott leitet seine Knechte\\
in dem rechtschaffnen Rechte,\\
das sich zum Leben neigt.

\flagverse{6.} Sein Herz läßt ihm nicht reuen,\\
was uns sein Mund verspricht,\\
gibt redlich und mit Treuen,\\
was unser Unglück bricht;\\
ist freudig, unverzagt,\\
uns alle zu erlösen\\
vom Kreuz und allem Bösen,\\
das seine Kinder plagt.

\flagverse{7.} Sein Wort ist wohl gegründet,\\
sein Mund ist rein und klar,\\
wozu er sich verbindet,\\
das macht er fest und wahr\\
und wird ihm gar nicht schwer.\\
Seine Name, den er führet,\\
ist heilig und gezieret\\
mit großer Pracht und Ehr.

\flagverse{8.} Die Furcht des Herren gibet\\
den ersten besten Grund\\
zur Weisheit, die Gott liebet\\
und rühmt mit seinem Mund.\\
O, wie klug ist der Sinn,\\
der diesen Weg verstehet\\
und fleißig darauf gehet!\\
Des Lob fällt nimmer hin.

\end{verse}
\end{multicols}
%\attrib{\small{THZE}}
