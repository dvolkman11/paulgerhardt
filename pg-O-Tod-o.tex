%StartInfo%%%%%%%%%%%%%%%%%%%%%%%%%%%%%%%%%%%%%%%%%%%%%%%%%%%%%%%%%%%%%%%%%%%%
%  Autor:
%  Titel:
%  File:
%  Ref:
%  Mod:
%EndInfo%%%%%%%%%%%%%%%%%%%%%%%%%%%%%%%%%%%%%%%%%%%%%%%%%%%%%%%%%%%%%%%%%%%%%%
%\poemtitle{pt}
\begin{multicols}{2}
\settowidth{\versewidth}{Ich weiß, daß dir zerschlagen ist}
\begin{verse}[\versewidth]
%o Tod, o Tod, du greulichs Bild\\
%nach Paul Röbers »O Tod, o Tod, schreckliches Bild«

\flagverse{1.} O Tod, o Tod, du greulichs Bild\\
und Feind voll Zorns und Blitzen,\\
wie machst du dich so groß und wild\\
mit deiner Pfeile Spitzen?\\
Hier ist ein Herz, das dich nicht acht\\
und spottet deiner schnöden Macht\\
und der zerbrochnen Pfeile.

\flagverse{2.} Komm nur mit deinem Bogen bald\\
und ziele mir zum Herzen;\\
in deiner seltsamen Gestalt\\
versuchs mit Pein und Schmerzen:\\
Was wirst du damit richten aus?\\
Ich werde dir doch aus dem Haus\\
einmal gewiß entlaufen.

\flagverse{3.} Ich weiß, daß dir zerschlagen ist\\
dein Schloß und seine Riegel\\
durch meinen Heiland Jesum Christ;\\
der brach des Grabes Siegel\\
und führte dich zum Siegesschau,\\
auf daß uns nicht mehr vor dir grau;\\
ein Spott ist aus dir worden.

\flagverse{4.} Besiehe deinen Palast wohl\\
und deines Reiches Wesen,\\
obs noch anitzo sei so voll\\
als es zuvor gewesen:\\
Ist Moses nicht aus deiner Hand\\
entwischt und im gelobten Land\\
auf Tabor schön erschienen?

\flagverse{5.} Wo ist der alten Heilgen Zahl,\\
die auch daselbst begraben?\\
Sie sind erhöht im Himmelssaal,\\
da sie sich ewig laben.\\
Des starken Jesus Heldenhand\\
hat dir zersprengt all deine Band,\\
als er dein Kämpfer wurde.

\flagverse{6.} Was solls denn nun, o Jesu, sein,\\
daß mich der Tod so schrecket?\\
Hat doch Elisa Totenbein,\\
was tot war, auferwecket:\\
Viel mehr wirst du, den Trost hab ich,\\
zum Leben kräftig rüsten mich,\\
drum schlaf ich ein mit Freuden.
   
\end{verse}
\end{multicols}
%\attrib{\small{THZE}}
