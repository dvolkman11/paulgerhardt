%StartInfo%%%%%%%%%%%%%%%%%%%%%%%%%%%%%%%%%%%%%%%%%%%%%%%%%%%%%%%%%%%%%%%%%%%%
%  Autor:
%  Titel:
%  File:
%  Ref:
%  Mod:
%EndInfo%%%%%%%%%%%%%%%%%%%%%%%%%%%%%%%%%%%%%%%%%%%%%%%%%%%%%%%%%%%%%%%%%%%%%%
%\poemtitle{pt}
\begin{multicols}{2}
\settowidth{\versewidth}{Nun, der Gott, der sie gekränkt,}
\begin{verse}[\versewidth]
%o, wie so ein großes Gut ist es doch, im Frieden scheiden\\
%auf den Tod der Frau Ursula von der Linden (1661)

\flagverse{1.} O, wie so ein großes Gut\\
ist es doch, im Frieden scheiden\\
und mit wohlvergnügtem Mut\\
in Geduld den Tod erleiden!\\
Lasset uns loben, was jeder nur weiß:\\
Seliges Sterben hat dennoch den Preis.

\flagverse{2.} Dieses Gut, das herrlich prangt,\\
hat aus Gottes Hand und Throne,\\
mein Herr Linde, wohl erlangt\\
eures Hauses Ehr und Krone.\\
Ihre Begierde nach himmlischer Au\\
ist ihr erfüllet, der seligen Frau.

\flagverse{3.} Sie hat ja des Kreuzes Joch\\
auch zuweilen wohl genossen:\\
Wie gekränket war sie doch,\\
da ihr Berkow ward erschossen,\\
berkow, das feine, geschickte Gemüt,\\
dessen Gedächtnis noch immerzu blüht!

\flagverse{4.} Nun, der Gott, der sie gekränkt,\\
hat sie wieder auch erfreuet\\
und euch ihr zum Mann geschenkt,\\
welches euch noch nie gereuet.\\
Jetzo genießt sie der ewigen Ehr\\
in Gottes Reiche. Was will sie doch mehr?

\end{verse}
\end{multicols}
%\attrib{\small{THZE}}
